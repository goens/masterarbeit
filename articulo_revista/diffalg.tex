\documentclass[12pt,a4paper,BCOR15mm,twoside,DIV12]{article}
\documentclass{article}
%\usepackage[paper=a4paper,left=20mm,right=20mm,top=25mm,bottom=25mm]{geometry}
\usepackage[spanish]{babel}
\usepackage[utf8]{inputenc}
\usepackage{amsmath}
\usepackage{color}
\usepackage{amssymb}
\usepackage{amsfonts}
\usepackage{amsthm}
\usepackage{hyperref}
\usepackage{graphicx, float,epsfig}
\usepackage[nottoc,numbib]{tocbibind}

\def\P{\mathcal{P}}
\def\R{\mathbb{R}} 
\def\E{\mathcal{E}} 
\def\N{\mathbb{N}} 
\def\Z{\mathbb{Z}} 
\def\Q{\mathbb{Q}} 
\def\F{\mathbb{F}}
\def\C{\mathbb{C}}
\def\U{\mathcal{U}}
\def\GL{\text{GL}}
\def\supp{\text{Supp}}
\def\id{\text{id}}
\def\n{\underline{n}}
\def\Gram{\text{Gram}}
\def\diag{\text{diag}}
\def\End{\text{End}}
\def\Hom{\text{Hom}}
\def\fa{\text{ para todo }}
\def\Tr{\text{Tr}}
\def\Id{\text{Id}}
\def\Sym{\text{Sym}}
\def\H{\mathcal{H}}
\def\wt{\text{wt}}
\def\Perf{\text{Perf}}


\renewcommand{\labelenumi}{\alph{enumi})}
%\renewcommand{\P}{\textfrak{P}}
\newcommand{\cupdot}{\mathop{\mathaccent\cdot\cup}}
\newenvironment{bew}{\begin{proof}[Proof]}{\end{proof}}
\theoremstyle{definition}
\newtheorem{Satz}{Satz}[section]
\newtheorem{theorem}[Satz]{Teorema}
\newtheorem{ex}[Satz]{Ejemplo}
\newtheorem{cor}[Satz]{Corolar}
\newtheorem{algorithm}[Satz]{Algoritmo}
\newtheorem{prop}[Satz]{Proposici\'{o}n}
\newtheorem{rem}[Satz]{Comentario}
\newtheorem{defn}[Satz]{Definición}
\newtheorem{lem}[Satz]{Lemma}

\title{Álgebra de Diferencias}
\author{Andr\'{e}s Goens}
\date{\today}
\begin{document}
\maketitle
\section{Introducción}

Acerca mio: Yo soy un estudiante salvadoreño de matemática. Tengo una licenciatura en física y otra en matemática de la universidad de Aquisgrán ``RWTH Aachen University''.
Ahora estoy por terminar la maestría en matemática en la misma universidad, y escribo mi tésis en el área de álgebra de diferencias, las bases de la cual quiero compartir en este artículo.

El álgebra de diferencias es una rama pequeña de las matemáticas, 
cuyo origen está cercanamente relacionado con el del álgebra diferencial, 
rama más grande con la que comparte mucha similitud. Es una rama relativamente nueva a su vez, [falta reseña histórica]

Para dar una primera idea del objeto de estudio de álgebra de diferencias, veremos un par de ejemplos de equaciones de diferencias.
Probablemente uno de los ejemplos mejor conocidos es la secuencia de Fibonacci $1,1,2,3,5,8,13,\ldots$ que se puede ver como una solución
de la siguiente ecuación recursiva: 
\begin{align}
a_0 = 1,  a_1 = 1 \\ a_n = a_{n-1} + a_{n-2}, n\geq 2
\end{align}

Otro ejemplo que probablemente también conozca cualquier matemático o físico es la ecuación funcional de la función Gamma:

\begin{align}
\Gamma(x+1) = x \Gamma(x)
\end{align}

Es un resultado clásico del análisis complejo que cualquier función que cumpla esta ecuación es un múltiplo de la función $\Gamma$,
a la que se le considera una generalización del factorial:
\begin{align*}
\Gamma(x) = \int_0^\infty{\frac{t^x}{t} e^{-t} dt}
\end{align*}

Estos dos son ejemplos notables de ecuaciones de diferencias. En el álgebra de diferencias no se busca, sin embargo,
encontrar soluciones de estas ecuaciones de manera ``explicita'', como lo son los números explicitos en la secuencia de Fibonacci,
o la representación integral de la función $\Gamma$. El álgebra de diferencias investiga, más bien, la estructura que tienen dichas
ecuaciones, y en sí, la existencia de tales. 

\subsection{Requisitos para este artículo}

Este artículo pretende presentar lo más básico de la teoría del álgebra de diferencias, la cual en muchos aspectos se basa en geometría algebráica. Aunque conocimiento de geometría algebráica sería de mucha ayuda para entender este artículo, traté de escribirlo
de tal manera que no sea necesaria. Sin embargo, conocimientos básicos de álgebra, como anillos, ideales, etc. y cierta madurez matemática probablemente sean verdaderamente indispensables para comprender este artículo. 
Para aquel lector que tenga interés en leerlo, pero tenga alguna laguna de conocimiento en el álgebra básica, recomiendo consultar el texto de Lang: \cite{lang}. Para aquellos que quieran buscar algo más específico de álgebra comutativa
o geometría algebráica, recomiendo los textos de Eisenbud \cite{eisenbud} o Hartshorne \cite{hartshorne}.

\section{Bases del Álgebra de Diferencias}
\begin{defn}
Sea $R$ un anillo comutativo (en este artículo, todos los anillos van a ser asociativos y unitales), y  sea $
\sigma: R \rightarrow R$ un endomorfismo de anillos en $R$. Entonces llamamos al par $(R,\sigma)$ un anillo de diferencias,
o un anillo $\sigma$. Por ``abuso de notación'' diremos que $R$ es un anillo $\sigma$ para referirnos al par, y si $R'$ es otro anillo $\sigma$
usaremos siempre la notación $\sigma$ para el endomorfismo de $R'$; esto no debería causar confusión, puesto que se podrá inferir del contexto de 
que endomorfísmo se esta hablando
\end{defn}

\begin{defn}
Sean $R, R'$ anillos $\sigma$ y sea $\varphi: R \rightarrow R'$ un morfismo de anillos. Decimos que $\varphi$ es un morfismo de anillos $\sigma$ si 
\begin{align*}
\sigma(\varphi(r)) = \varphi(\sigma(r)) \fa r \in R
\end{align*}
\end{defn}

\begin{ex} Hay un sinfin de ejemplos de anillos $\sigma$. Un par de los más importantes son los siguientes:

\begin{itemize}
\item Cualquier anillo $R$ es un anillo $\sigma$ con $\sigma = \Id_R$, a este le llamamos un anillo $\sigma$ constante
\item El cuerpo de las funciones meromórfas $\C \rightarrow \C$, que denotaremos con $\mathcal{M}$,
 es un anillo $\sigma$ con $\sigma(f)(x) = f(x+1) \forall x \in \C$.
\item Las secuencias de números en $\Z$, que denominaremos $\text{Seq}(\Z)$ forman un anillo $\sigma$ con la operación de ``correr'' sus términos hacia la izquierda:
\begin{align*} \sigma: (a_n)_{n \in \N} \mapsto (a_{n+1})_{n \in \N} \end{align*}
\end{itemize}
\end{ex}

\begin{defn}
Sea $R$ un anillo $\sigma$. Si $R$ es un cuerpo, llamamos al par $(R,\sigma)$ también un cuerpo $\sigma$. 
Si $k$ es un cuerpo $\sigma$, $A$ una $k$-álgebra que (como anillo) es un anillo $\sigma$ y cumple:
$\sigma(ar) = \sigma(a) \sigma(r)$, entonces llamamos a $A$ un álgebra $k-\sigma$.
\end{defn}

\begin{ex}
Un ejemplo adicional, que por su generalidad es de especial importancia, son los llamados anillos polinomiales $\sigma$:
Sea $k$ un cuerpo. Consideremos el anillo de polynomios $R:= k[y,\sigma(y),\sigma^2(y),\ldots]$,
 dónde $y,\sigma(y),\sigma^2(y),\ldots$ son por el momento simplemente nombres de variables (álgebraicamente independientes).
A este anillo lo podemos hacer un álgebra $k-\sigma$ definiendo 
\begin{align*} 
\sigma:  R \rightarrow R, y \mapsto \sigma(y), \sigma^{n-1}(y) \mapsto \sigma^{n}(y) \fa n > 1 
\end{align*}
Extendendiolo de la manera obvia $k$-linearmente. Denotamos este anillo de polinomios $\sigma$ con $k\{y\}$. De forma análoga podemos definir un anillo de polinomio $\sigma$ sobre muchas variables:
$k\{y_1, \ldots, y_n \}$.
\end{ex}

\begin{defn}
\begin{itemize}
\item Si $S$ es un anillo $\sigma$ y $R \leq S$ un subanillo de $S$, decimos que $R$ es un subanillo $\sigma$, si $(R,\sigma_{|R})$ es un anillo $\sigma$,
es decir, si la imagen de $\sigma_{|R}$ esta contenida en $R$.
\item Si $I \unlhd R$ es un ideal y subanillo $\sigma$ se le llama ideal $\sigma$. No es dificil convencerse que esto define una estructura $\sigma$ del anillo factorial canonica:
\begin{align*} \sigma: R/I \rightarrow R/I, a + I \mapsto \sigma(a) + I \end{align*}
\end{itemize}
\end{defn}

\begin{defn}
A un álgebra $k-\sigma$ $A$ se le llama finitamente $\sigma$ generada, si existen elementos $f_1, \ldots, f_n$ tales, que $A = k[f_1,f_2,\ldots,f_n,\sigma(f_1),\ldots,\sigma(f_n),\sigma^2(f_1),\ldots]$.
\end{defn}

\begin{rem}\label{epipoli}
Si $A$ es un álgebra $k-\sigma$, $\sigma$ generada por $f_1, \ldots, f_n$, entonces tenemos un epimorfismo de 
álgebras $k-\sigma$ canónico del anillo de polinomios $\sigma$, $k\{y_1, \ldots, y_n \}$ a $A$: $y_i \mapsto f_i, i = 1, \ldots, n$.
Vemos entonces que los anillos de polinomios $\sigma$ son objetos libres en la categoría de álgebras $k-\sigma$. Denotamos por esto al
álgebra $\sigma$ generada por $f_1, \ldots, f_n$ con $k\{f_1, \ldots, f_n\}$.
\end{rem}

\begin{defn}
Si el Núcleo (Kernel) $I$ del epimorfismo mencionado en el Comentario \ref{epipoli} es a su vez finitamente $\sigma$ generada, por $r_1, \ldots, r_m$ digamos, entonces decimos que el álgebra $A$ es finitamente $\sigma$ presentada
\end{defn}

\begin{rem}
En las condiciones de la definicion anterior, por el teorema de homomorfismos tenemos $A \cong k\{y_1, \ldots, y_n}/(r_1,\ldots,r_m)$. Nótese también que los conceptos finitamente $\sigma$ generado y finitamente $\sigma$ presentado son
verdaderamente diferentes: a diferencia del caso de anillos polinomiales comunes, no tenemos el teorema de bases de Hilbert, lo que quiere decir que $k\{y_1, \ldots, y_n\}$ no es Noetheriano. En particular, a pesar de ser 
finitamente $\sigma$ generada el álgebra, el ideal $I$ puede no serlo.
\end{rem}

\begin{ex}
[FIXME:] Ejemplo de lo anterior
\end{ex}

\begin{rem}
Así como las ecuaciones algebráicas regulares se ``reducen'' al buscar soluciones de un polinomio, 
las ecuaciones de diferencias las podemos expresar como la búsqueda de soluciones para  polinomios $\sigma$. Entiéndase aquí un homomorfismo como el del Comentario \ref{epipoli}.
\end{rem}

\begin{ex}
Ahora podemos expresar los dos ejemplos de la introducción en el lenguaje del álgebra de diferencias:
La secuencia de Fibonacci es una solución del polinomio $\sigma^2(y) + \sigma(y) - y$ en el anillo $\sigma$ $\text{Seq}(\Z)$: esta es precisamente la relación de recurrencia $a_{n+2} + a_{n+1} = a_n$.
De la misma manera, la función $\Gamma$ es la solución del polinomio $\sigma(y) - zy$, adonde el coeficiente $z$ es la función $z \mapsto z$, un elemento del cuerpo $\C(x)$ de funciones racionales.
\end{ex}

\section{Ideales de Diferencias y la topología de Cohn}

\begin{defn} perfectos, wm, sigma primo, primo sigma
\end{defn}

bleh... hasta: topología Cohn


\end{document}

%\documentclass[12pt,a4paper,BCOR15mm,twoside,DIV12]{article}
\documentclass{article}
\usepackage[paper=a4paper,left=20mm,right=20mm,top=25mm,bottom=25mm]{geometry}
\usepackage[english]{babel}
\usepackage[utf8]{inputenc}
\usepackage{amsmath}
\usepackage{color}
\usepackage{amssymb}
\usepackage{amsfonts}
\usepackage{amsthm}
\usepackage{hyperref}
\usepackage{makeidx}
\usepackage{graphicx, float,epsfig}
\usepackage[nottoc,numbib]{tocbibind}
\usepackage[arrow, matrix, curve]{xy}

\newcommand{\properideal}{%
  \mathrel{\ooalign{$\lneq$\cr\raise.22ex\hbox{$\lhd$}\cr}}}

\def\P{\mathcal{P}}
\def\I{\mathbb{I}}
\def\R{\mathbb{R}} 
\def\E{\mathcal{E}} 
\def\NE{\mathbb{N}_{\geq1}} 
\def\N{\mathbb{N}} 
\def\Z{\mathbb{Z}} 
\def\Q{\mathbb{Q}} 
\def\F{\mathbb{F}}
\def\Vm{\mathcal{V}_m}
\def\V{\mathcal{V}}
\def\VV{\mathbb{V}}
\def\C{\mathbb{C}}
\def\U{\mathcal{U}}
\def\a{\mathfrak{a}}
\def\b{\mathfrak{b}}
\def\p{\mathfrak{p}}
\def\q{\mathfrak{q}}
\def\s{\sigma}
\def\si{\unlhd_{\sigma}}
\def\GL{\text{GL}}
\def\supp{\text{Supp}}
\def\id{\text{id}}
\def\n{\underline{n}}
\def\Spec{\text{Spec}}
\def\sSpec{\sigma\text{-Spec}}
\def\diag{\text{diag}}
\def\End{\text{End}}
\def\Hom{\text{Hom}}
\def\fa{\text{ for all }}
\def\Tr{\text{Tr}}
\def\Id{\text{Id}}
\def\ker{\text{ker}}
\def\H{\mathcal{H}}
\def\trdeg{\text{trdeg}}
\def\sdim{\sigma\text{-dim}}


\renewcommand{\labelenumi}{\alph{enumi})}
%\renewcommand{\P}{\textfrak{P}}
\newcommand{\cupdot}{\mathop{\mathaccent\cdot\cup}}
\newenvironment{bew}{\begin{proof}[Proof]}{\end{proof}}
\theoremstyle{definition}
\newtheorem{Satz}{Satz}[section]
\newtheorem{theorem}[Satz]{Theorem}
\newtheorem{ex}[Satz]{Example}
\newtheorem{cor}[Satz]{Corollary}
\newtheorem{algorithm}[Satz]{Algorithm}
\newtheorem{prop}[Satz]{Proposition}
\newtheorem{rem}[Satz]{Remark}
\newtheorem{defn}[Satz]{Definition}
\newtheorem{lem}[Satz]{Lemma}


\makeindex
\title{Difference Kernels}
\author{Andr\'{e}s Goens}
\date{\today}

\begin{document}

\begin{prop}
Let $R \subseteq S$ be integral domains. Let $I \unlhd R[y]$ be an ideal in the polynomial ring over $R$ in $n$ variables and assume that
$I \cap R = 0$ holds. Consider the ideal generated by $I$ in $S[y]$, which we will denote by $(I)$. 
Then it holds that $(I) \cap S = 0$.
\begin{bew}
Let $R':= R[y]/I$ and $S' := S[y]/(I)$. 
Our assumption $R \cap I = 0$ can then be rewritten as the equivalent condition that the mapping 
$$  R \rightarrow R', r \mapsto r + I $$
is injective, and similarly, we want to show that the analogous mapping $S \rightarrow S', s \mapsto s + (I)$ is injective, 
which is equivalent to the assertion that $(I) \cap S = 0$.

As a first step, we will show that 
$$ S' = S[y]/(I) \cong R' \otimes_R S.$$
For this we consider the following short exact sequence:
\begin{equation}\label{sec1}
0 \rightarrow I \rightarrow R[y] \rightarrow R[y]/I \rightarrow 0
\end{equation}
Since the tensor product is always right exact, if we tensor the Sequence (\ref{sec1}) over $R$ with $S$ we get the exact sequence
\[
\begin{xy}
\xymatrixcolsep{3.2pc}
 \xymatrix{
I \otimes_R S \ar[r]^{\operatorname{Id}_I \otimes \operatorname{Id}_S} & \underbrace{ R[y] \otimes_R S}_{\cong S[y]} \ar[r]^\varphi & (R[y]/I) \otimes_R S  \ar[r] &  0 }
\end{xy}
\]
Where $S \otimes_R R[y] \cong S[y]$ follows from the fact that $y = y_1,\ldots,y_n$ are algebraically independent.  By the fundamental theorem on homomorphisms we thus get that $$(R[y]/I) \otimes_R S \cong \underbrace{ S \otimes_R R[y]}_{\cong S[y]}/\operatorname{Ker}(\varphi).$$
But since the sequence is exact, it holds that 
$$\operatorname{Ker}(\varphi) = \operatorname{Im}(\operatorname{Id}_I \otimes \operatorname{Id}_S) \cong \{ \sum_{i = 1}^k s_i f_i \mid s_i \in S, f_i \in I, k \geq 0 \} \cong  \{ \sum_{i = 1}^k g_i f_i \mid g_i \in S[y], f_i \in I, k \geq 0 \} \cong (I) \unlhd S[y]$$
from which the statement follows.


Now we will show our main assertion. Let $E := \operatorname{Quot}(S)$ and $F :=  \operatorname{Quot}(R)$ be the quotient fields of the integral domains $S$ and $R$ respectively.
Since $E$ is a field, $E$ is a flat $F$ module. Similarly, since $F$ as the quotent field of $R$ is a localization of the former, $F$ is a flat $R$ module (see Theorem 1 in  Ch. II § 2.4 of \cite{burbaki}).
Together these two facts imply that $E$ is a flat $R$ module as well. To see this, let $J \unlhd R$ be finitely generated. Since $F$ is a flat $R$ module, the mapping $J \otimes_R F \hookrightarrow R \otimes_R F \cong F$ is injective.
Further, since $E$ is a flat $F$ module, we can tensor this mapping with $E$ over $F$ and get the injective mapping 
$$J \otimes_R F \otimes _F E \cong J \otimes_R E \hookrightarrow (R \otimes_R F) \otimes_F E\cong F \otimes_F E \cong E $$
This is exactly the flatness of $E$ as an $R$ module. We can then conclude that if we take the tensor product over $R$ with $E$ from the mapping  $ R \rightarrow R'$
it will remain injective: 
$$  E \otimes_R R \cong E \hookrightarrow E \otimes_R R' , e \otimes_R r \mapsto e \otimes_R (r + I)$$ is injective. 
Furthermore, the canonical embedding $S \hookrightarrow E = \operatorname{Quot}(S)$ is also injective. 
So that we get an injective mapping from the composition:
$$S \hookrightarrow E \otimes_R R', s \mapsto s \otimes_R (1 + I)$$
Additionally, it holds that 
$$ E \otimes_R R' \cong (E \otimes_S S) \otimes_R R' \cong E \otimes_S \underbrace{(S \otimes_R R')}_{\cong S' \text{, see above}} \cong E \otimes_S S' $$
The composition of this isomorphism with the mapping above yields the injectivity of the mapping
$$S \hookrightarrow E \otimes_S S', s \mapsto s \otimes_S (1 +(I)) = 1 \otimes_S s (1 + (I)) = 1 \otimes_S (s + (I))$$
This implies that the following diagram is commutative:
\[
\begin{xy}
\xymatrixrowsep{3.5pc}

 \xymatrix{
      S \ar@^{(->}[r] \ar[rd]^-{ s \mapsto s + (I)} &E \ar@^{(->}[r] & E \otimes_R R' \ar[r]^\sim & E \otimes_S S' \\ 
      & S' \ar[urr]_-{~~~~~ s + (I) \mapsto 1 \otimes (s + (I)) }}
\end{xy}
\]
Since the mapping $S \hookrightarrow E \otimes_S S'$ is injective, it means in particular that the mapping $S \rightarrow S'$ is injective as well, which is what we wanted to show.
\end{bew}
\end{prop}



\end{document}


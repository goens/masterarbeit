%\documentclass[12pt,a4paper,BCOR15mm,twoside,DIV12]{article}
\documentclass{article}
%\usepackage[paper=a4paper,left=20mm,right=20mm,top=25mm,bottom=25mm]{geometry}
\usepackage[english]{babel}
\usepackage[utf8]{inputenc}
\usepackage{amsmath}
\usepackage{color}
\usepackage{amssymb}
\usepackage{amsfonts}
\usepackage{amsthm}
\usepackage{hyperref}
\usepackage{makeidx}
\usepackage{graphicx, float,epsfig}
\usepackage[nottoc,numbib]{tocbibind}


\newcommand{\properideal}{%
  \mathrel{\ooalign{$\lneq$\cr\raise.22ex\hbox{$\lhd$}\cr}}}

\def\P{\mathcal{P}}
\def\I{\mathbb{I}}
\def\R{\mathbb{R}} 
\def\E{\mathcal{E}} 
\def\NE{\mathbb{N}_{\geq1}} 
\def\N{\mathbb{N}} 
\def\Z{\mathbb{Z}} 
\def\Q{\mathbb{Q}} 
\def\F{\mathbb{F}}
\def\Vm{\mathcal{V}_m}
\def\V{\mathcal{V}}
\def\VV{\mathbb{V}}
\def\C{\mathbb{C}}
\def\U{\mathcal{U}}
\def\a{\mathfrak{a}}
\def\b{\mathfrak{b}}
\def\p{\mathfrak{p}}
\def\q{\mathfrak{q}}
\def\s{\sigma}
\def\si{\unlhd_{\sigma}}
\def\GL{\text{GL}}
\def\supp{\text{Supp}}
\def\id{\text{id}}
\def\n{\underline{n}}
\def\Spec{\text{Spec}}
\def\sSpec{\sigma\text{-Spec}}
\def\diag{\text{diag}}
\def\End{\text{End}}
\def\Hom{\text{Hom}}
\def\fa{\text{ for all }}
\def\Tr{\text{Tr}}
\def\Id{\text{Id}}
\def\ker{\text{ker}}
\def\H{\mathcal{H}}
\def\trdeg{\text{trdeg}}
\def\sdim{\sigma\text{-dim}}


\renewcommand{\labelenumi}{\alph{enumi})}
%\renewcommand{\P}{\textfrak{P}}
\newcommand{\cupdot}{\mathop{\mathaccent\cdot\cup}}
\newenvironment{bew}{\begin{proof}[Proof]}{\end{proof}}
\theoremstyle{definition}
\newtheorem{Satz}{Satz}[section]
\newtheorem{theorem}[Satz]{Theorem}
\newtheorem{ex}[Satz]{Example}
\newtheorem{cor}[Satz]{Corollary}
\newtheorem{algorithm}[Satz]{Algorithm}
\newtheorem{prop}[Satz]{Proposition}
\newtheorem{rem}[Satz]{Remark}
\newtheorem{defn}[Satz]{Definition}
\newtheorem{lem}[Satz]{Lemma}


\makeindex
\title{Difference Kernels}
\author{Andr\'{e}s Goens}
\date{\today}

\begin{document}

\begin{defn}
 Let $R$ be an integral domain and $R \subseteq L$ be a further integer domain, finitely generated over $R$.
Let $\phi : R \rightarrow L$ be a morphism of rings. We call the morphism $\phi$ extendable, if there exists a generating set ${a_1,\ldots,a_n}$ of $L$ over $R$, i.e. $L = R[a_1,\ldots,a_n]$,
such that the following is true: $a_1,\ldots,a_{n_1}$ is algebraically independant and maximal as such, and for every $n_1 < i \leq n$ it holds for the minimal polynomial of $a_i$ over $R[a_1,\ldots,a_{i-1}]$, which we denote by 
$p_i(X) \in R[a_1,\ldots,a_{i_1}][X]$, that the following is true: If we write $p_i(X) = \sum_{k=0}^r f_k x^k$ with $f_k \in R[a_1,\ldots,a_{i-1}]$ then there exists an $f_k, k > 0$ such that $\phi^s(f_k) \neq 0$ for all $s \geq 0$ for which it is defined.
\end{defn} %% is this enough???

\begin{lem} Let $\phi R \rightarrow L$ be an extendable morphism. Then there exists an overring $L_1 \supseteq L$ and a morphism $\phi': L \rightarrow L_1$ which extends $\phi$, i.e., $\phi'_{|R}=\phi$.
Is $\phi'$ still extendable???
\begin{proof}
We can assume without loss of generality that the algebraically independant set $\{a_1,\ldots,a_{n_1}\}$ is empty: 
We define $R':= R[a_1,\ldots,a_{n_1}]$ and $L':=L[y_1,\ldots,y_{n_1}]$ and set $\phi'(a_i) := y_i$ for all $1 \leq i \leq n_1$. 
It is easy to see that this is an extension of $\phi$ and that $\phi':R' \rightarrow L'$ remains extendable.
We first look at the case where there is only one generator $a$ of $L$ (algebraical over $R$):
Let $a \in L$ such that $L = R[a]$ and let $p(X)= \sum_{k=0}^t f_k X^k \in R[X]$ be the minimal polynomial of $a$. Consider the polynomial $p'(X) = \sum_{k=0}^t \phi(f_k) X^k \in \phi(R)[X] \subseteq L[X]$.
Since $\phi$ is extendable, by definition, it holds that the polynomial $p'$ is at least of degree $1$, i.e., $p'(X) \in L[X] \backslash L$. Let $M$ be the splitting field of $p'(X)$ and let $a' \in M$ be a zero of $p'$.
Then the ring $L_1:= L[a']$ is a subring of the field $M$, and thus also an integral domain. We can thus extend $\phi$ by $\phi': L \rightarrow L_1, a \mapsto a'$, since $p'(X)$ is the minimal polynomial of $a'$. 
It holds that $p'(X) = \sum_{k=0}^t \phi(f_k) X^k$. For a $k > 0$ assume that $\phi(f_k) \in L \backslash R$. Then it has to be that $\phi(f_k) = \sum_{s=1}^m g_s a^s$ for some $g_s \in R$ and of degree at least $1$ in $a$.
So we know that $\phi'(\phi(f_k)) = \sum_{s=1}^m \phi(g_s)a'^s$ Why should this never be $0$??

\end{proof}
\end{lem}

\end{document}


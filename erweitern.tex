%\documentclass[12pt,a4paper,BCOR15mm,twoside,DIV12]{article}
\documentclass{article}
%\usepackage[paper=a4paper,left=20mm,right=20mm,top=25mm,bottom=25mm]{geometry}
\usepackage[english]{babel}
\usepackage[utf8]{inputenc}
\usepackage{amsmath}
\usepackage{color}
\usepackage{enumerate}
\usepackage{amssymb}
\usepackage{amsfonts}
\usepackage{amsthm}
\usepackage{hyperref}
\usepackage{makeidx}
\usepackage{graphicx, float,epsfig}
\usepackage[nottoc,numbib]{tocbibind}
\usepackage[arrow, matrix, curve]{xy}

\newcommand{\properideal}{%
  \mathrel{\ooalign{$\lneq$\cr\raise.22ex\hbox{$\lhd$}\cr}}}

\def\P{\mathcal{P}}
\def\I{\mathbb{I}}
\def\R{\mathbb{R}} 
\def\E{\mathcal{E}} 
\def\NE{\mathbb{N}_{\geq1}} 
\def\N{\mathbb{N}} 
\def\Z{\mathbb{Z}} 
\def\Q{\mathbb{Q}} 
\def\F{\mathbb{F}}
\def\Vm{\mathcal{V}_m}
\def\V{\mathcal{V}}
\def\VV{\mathbb{V}}
\def\C{\mathbb{C}}
\def\U{\mathcal{U}}
\def\a{\mathfrak{a}}
\def\b{\mathfrak{b}}
\def\p{\mathfrak{p}}
\def\q{\mathfrak{q}}
\def\s{\sigma}
\def\si{\unlhd_{\sigma}}
\def\sD{\s\text{-}\operatorname{D}}
\def\GL{\text{GL}}
\def\supp{\text{Supp}}
\def\id{\text{id}}
\def\n{\underline{n}}
\def\Spec{\operatorname{Spec}}
\def\sSpec{\sigma\operatorname{-Spec}}
\def\diag{\text{diag}}
\def\End{\text{End}}
\def\Hom{\text{Hom}}
\def\fa{\text{ for all }}
\def\Tr{\text{Tr}}
\def\Id{\text{Id}}
\def\Sym{\text{Sym}}
\def\H{\mathcal{H}}
\def\wt{\text{wt}}
\def\Perf{\text{Perf}}


\renewcommand{\labelenumi}{\alph{enumi})}
%\renewcommand{\P}{\textfrak{P}}
\newcommand{\cupdot}{\mathop{\mathaccent\cdot\cup}}
\newcommand{\textsim}{\mathord{\sim}}
\newcommand{\catname}[1]{{\normalfont\textbf{#1}}}
\newcommand{\Set}{\catname{Set}}
\newcommand{\Top}{\catname{Top}}
\newcommand{\sintk}{\s\text{\catname{-int}}_k}
\newcommand{\sringk}{\s\text{\catname{-ring}}_k}
\newenvironment{bew}{\begin{proof}[Proof]}{\end{proof}}
\theoremstyle{plain}
\newtheorem{Satz}{Satz}[section]
\newtheorem{theorem}[Satz]{Theorem}
\newtheorem{ex}[Satz]{Example}
\newtheorem{cor}[Satz]{Corollary}
\newtheorem{algorithm}[Satz]{Algorithm}
\newtheorem{prop}[Satz]{Proposition}
\newtheorem{lem}[Satz]{Lemma}
\newtheorem{defn}[Satz]{Definition}
\theoremstyle{definition}
\newtheorem{rem}[Satz]{Remark}


\begin{document}

\section*{Extensions of kernels}

\begin{lem}\label{primeoverp1}
Let R be an integral domain and $I \unlhd R[y]$ be an ideal of $R[y] = R[y_1,\ldots,y_n]$ which satisfies that $I \cap R = \{ 0 \}$.
Then there exists a prime ideal $P, I \subseteq P \subseteq R[y] $ with $P \cap R = \{0\}$.
\begin{proof}
We can assume without loss of generality that I is radical:
Namely, if $f \in \sqrt{I} \cap R$, then there exists an $m \in \N$ such that $f^m \in I \cap R = \{0\}$, and since $R$ is an integral domain this already means that $f = 0$.
We then note that for two sets $A,B \subseteq R[y]$ it holds that $\sqrt{A}\sqrt{B} \subseteq \sqrt{AB}$: Consider $f \in \sqrt{A}, g \in \sqrt{B}$. Then there exist $m, \tilde m \in \N$ such that $f^m \in (A), g^{\tilde m} \in (B)$;
 assume without loss of generality that $m > \tilde m$, then $(fg)^m \in (A)(B)$, which implies $fg \in \sqrt{(A)(B)} = \sqrt{AB}$.
Now, for the proof, consider the set of all radical ideals $J$ contaning $I$ which satisify $J \cap R = \{0\}$. This set is not empty and is inductively ordered by inclusion.
By Zorn's lemma this means that there is a maximal element $P$ of this set. This ideal $P$ is prime, then: assume there exist $f,g \notin P$ with $fg \in P$. 
Then the radical ideals $\sqrt{P \cup \{f\}}$, $\sqrt{P \cup \{f\}}$ strictly include $P$, and by the maximality of $P$ it means there exist $t_1, t_2 \in R\backslash\{0\}$ such that
$t_1 \in \sqrt{P \cup \{f\}}$, $t_2 \in \sqrt{P \cup \{g\}}$. But in particular, because $R$ is free of zero divisors, this implies that
 \[0 \neq t_1t_2 \in \sqrt{P \cup \{f\}}\sqrt{P \cup \{g\}} \subseteq \sqrt{ \underbrace{(P \cup \{f\})(P \cup \{g\})}_{=P\text{, since }fg \in P}} = P\]
A contradiction.
\end{proof}
\end{lem}


\begin{lem}\label{idealstill0}
Let $R \subseteq S$ be two rings, and let $I \unlhd R[y]$ be an ideal in the polynomial ring $R[y]$ with $I \cap R = 0$. 
Then it holds for the ideal $(I) \unlhd S[y]$, that $(I) \cap S = 0$.
\begin{proof}
Since $S$ is an $R$ module, we know that $S \cong R \otimes_R S$, from which it easily follows that $S[y] \cong R[y] \otimes_R S$, and similarly, that $I \otimes_R S \cong (I) \unlhd S[y]$.
Together, these two imply that as well $R[y]/I \otimes_R S \cong S[y]/(I)$. FIXME: check this!
The condition $R \cap I = 0$ is equivalent to the mapping $R \rightarrow R[y]/I, r \mapsto r + I$ being injective. We can express this as the exactness of the following sequence:
\[ 0 \rightarrow R \rightarrow R[y]/I \]
Since the tensor product functor is exact, we know that tensoring over $R$ with $S$ yields an exact sequence:
\[ 0 \rightarrow R \otimes_R S \cong S \rightarrow R[y]/I \otimes_R S \cong S[y]/(I) \]
This, in turn, is equivalent to $S \cap (I) = 0$ by going the arguments above backwards.

\end{proof}
\end{lem}


\begin{prop}
Let $\a \subseteq k[y,\ldots,\s^d(y)]$ be a prime difference kernel of length $d$ and let $k[y,\ldots,\s^d(y)]/\a =: k[a,\s(a),\ldots,\s^d(a)]$. Consider the mapping 
\[ \s: k[a,\ldots,\s^{d-1}(a)] \rightarrow k[a,\ldots,\s^d(a)]. \]
Assume that for $(\operatorname{ker}(\s)) \subseteq k[a,\ldots,\s^d(a)]$ it holds that $(\operatorname{ker}(\s)) \cap k[a,\ldots,\s^{d-1}(a)] = \operatorname{ker}(\s)$. 
Then there exists a prime difference kernel $\tilde \a \subseteq k[y,\ldots,\s^{d+1}(y)]$ of length $d+1$ with $\tilde \a \cap k[y,\ldots,\s^d(y)] = \a$
\begin{bew}
Consider the surjective mapping 
\[ k[a,\ldots,\s^{d-1}(a)][\s^d(y)] \rightarrow k[a,\ldots,\s^d(a)], \s^d(y) \mapsto \s^d(a) \]
Let $\p_1$ be the kernel of this mapping. Since $k[a,\ldots,\s^d(a)]$ is a domain, we know by the fundamental theorem on homomorphisms that $\p_1$ has to be prime. 
First, we will show that our assumption on $\operatorname{ker}(\s)$ is equivalent to:
\begin{equation}\label{skercapsk} \s(\p_1) \cap \s(k)[\s(a),\ldots,\s^d(a)] = 0 \end{equation}
To see this consider the following commutative diagram:
\[
\begin{xy}
 \xymatrix{
      k[a,\ldots,\s^{d-1}(a)][\s^d(y)] \ar[rr]^\s \ar@{->>}[rd]^\s  &     &  k[a,\ldots,\s^d(a)][\s^{d+1}(y)]   \\
      &  \s(k)[\s(a),\ldots,\s^d(a)][\s^{d+1}(y)] \ar@^{(->}[ur] &  }
\end{xy}
\]

We can factor out $\p_1$ and its image, $\s(\p_1)$. Since $k[a,\ldots,\s^{d-1}(a)][\s^d(y)] / \p_1 \cong k[a,\ldots, \s^d(a)]$,
we get a surjectve mapping $k[a,\ldots, \s^d(a)] \twoheadrightarrow \s(k)[\s(a),\ldots,\s^d(a)][\s^{d+1}(y)]/\s(\p_1)$.
If we factor out the kernel, we get an isomorphism:
\[ k[a,\ldots, \s^d(a)]/(\operatorname{ker}(\s)) \cong \s(k)[\s(a),\ldots,\s^d(a)][\s^{d+1}(y)]/\s(\p_1)\]
On the other hand, we have the canonical embedding $k[a,\ldots,\s^{d-1}(a)] \hookrightarrow k[a,\ldots,\s^{d}(a)]$.
This mapping stays injective after factoring out the kernel of $\s$ on both sides, 
i.e. $k[a,\ldots,\s^{d-1}(a)]/\operatorname{ker}(\s) \hookrightarrow k[a,\ldots,\s^{d}(a)]/(\operatorname{ker}(\s))$ is injective, if and only if our assumption on $\ker(\s)$ holds.
In this case, by composition with the isomorphism above, we get an embedding:
\[ k[a,\ldots,\s^{d-1}(a)]/\operatorname{ker}(\s) \hookrightarrow \s(k)[\s(a),\ldots,\s^d(a)][\s^{d+1}(y)]/\s(\p_1) \]
This is injective if and only if Equation \ref{skercapsk} holds.
Now, by Lemma \ref{idealstill0} Equation \ref{skercapsk} implies that
\begin{equation}\label{skercapk}
(\s(\p_1)) \cap k[a,\s(a),\ldots,\s^d(a)][\s^{d+1}(y)] = 0
\end{equation}
We will now construct a prime difference kernel $\tilde \a$ using Equation \ref{skercapsk}:

By Lemma \ref{primeoverp1} there exists a minimal prime ideal $\p_2 \supset (\s(\p_1))$ of $k[a,\ldots,\s^d(a)][\s^{d+1}(y)]$ containing $\s(p_1)$ with $\p_2 \cap k[a,\ldots,\s^d(a)] = \{0\}$. 
We thus get a well-defined maping
\[ \s: k[a,\ldots,\s^{d-1}(a)][\s^d(y)]/\p_1 \rightarrow k[a,\ldots, \s^d(a)][\s^{d+1}(y)]/\p_2 \]
We define $R_2:= k[a,\ldots,\s^d(a)][\s^{d+1}(y)]/\p_2 =: k[a,\ldots,\s^d(a),\s^{d+1}(a)]$, which is an integer domain since $\p_2$ is prime. Since $\p_2 \cap k[a,\ldots,\s^d(a)] = 0$ we can use this notation unambiguosly:
this guarantees namely that for $a, \ldots, \s^d(a)$ we have the same residue classes modulo $\p_2$ as we had modulo $\a$.
The kernel $\tilde \a$ of the natural epimorphism $k[y,\ldots,\s^{d+1}(y)] \rightarrow R_2$ is thus a prime ideal.
Further we have $\a \subseteq \tilde \a$ by construction (as $\a = 0 \subset R_2$). In fact, it holds that $\tilde \a[d] = \p$ since: 
\begin{align*}
\tilde \a[d] = \{ f \in k[y,\ldots,\s^d(y)] \mid f(a) = 0 \} = \ker( k\{y\}[d] \rightarrow k[a,\ldots,\s^{d}(a)]) = \a
\end{align*}
where the first equality uses the fact that $\p_2 \cap k[a,\ldots,\s^d(a)] = 0$, as noted by the use of the notation explained above, and last equality holds by definition of $a \in k[a,\ldots,\s^d(a)]$. This means, that $\tilde \a$ is a prolongation of $\a$. 
\end{bew}
\end{prop}

\end{document}

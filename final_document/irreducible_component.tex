\subsection{A Geometric Application of Difference Kernels}

There is an application of Difference Kernels that is of geometric nature. We will just quote a result for the case of reflexive prime kernels here,
and do a partial generalization for the case that arises for some non-necesarilly-reflexive prime $\s$-ideals.

\begin{theorem}\label{di=d(i+1)+e}
Let $k$ be a $\s$-field and let $\p \in k\{y\} = k\{y_1,\ldots,y_n\}$ be a prime $\s$-ideal of $k\{y\}$. For $i \in \N$ set $$d_i := \dim(k\{y\}[i]/\p[i]).$$
Then there exists integers $d, e \in \N$ such that $d_i = d(i+1) + e$ for $i \gg 0$. Moreover, $d = \s\operatorname{-trdeg}(k\{y\}/\p^*)$.
\begin{bew}
See Theorem 5.1 of \cite{wibmer}.
\end{bew}
\end{theorem}

\begin{defn}
Let $k$ be a $\s$-field and $\p \in k\{y_1,\ldots,y_n\}$ be a prime $\s$-ideal. Further let $d, e \in \N$ as in Theorem \ref{di=d(i+1)+e}. We call $d$ the $\s$-dimension of $\p$, 
or the $\s$-dimension of the irreducible $\s$-variety $X:= \VV(\p)$ and denote it by $\s$-$\dim(\p)$ and $\s$-$\dim(X)$ respectively.
\end{defn}

Recall from the introductory Chapter, Theorem \ref{irredcomp}:
\begin{reptheorem}{irredcomp}
Let $k$ be a $\s$-field and $f \in k\{y_1,\ldots,y_n\}, f \notin k$ an irreducible $\s$-polynomial such that $\operatorname{Eord}(f) = \operatorname{Ord}(f)$. Then $\VV_{\operatorname{Field}}(f) \subset \mathbb{A}_k^n$ has an irreducible component $X$ such that $\s$-$\dim(X) = n-1$ and $\s$-$\operatorname{deg}(X) = \operatorname{Ord}(f)$.
\end{reptheorem}

\begin{cor}\label{corfinal}
Let $k$ be a $\s$-field and $f \in k\{y_1,\ldots,y_n\}, f \notin k$ an irreducible $\s$-polynomial. 
Then $\VV(f) \subset \mathbb{A}_k^n$ has an irreducible component $X$ such that $\s$-$\dim(X) = n-1$ and $\s$-$\operatorname{deg}(X) = \operatorname{Ord}(f) + (n-1)(\operatorname{Ord}(f) - \operatorname{Eord}(f))$.
\begin{bew}
Consider the $\s$-polynomial ring $k\{\s^d(y)\} =: k\{z\}$, where $d$ is maximal, such that $f \in k\{\s^d(y)\}$ (i.e., $d = \operatorname{Ord}(f) - \operatorname{Eord}(f)$). We let $f'\in k\{z\}$ be the corresponding polynomial to $f$. 
Since $f$ is irreducible, $f'$ has to be as well. Additionally, by definition of $d$, it has to hold that $\operatorname{Eord}_z(f') = \operatorname{Ord}_z(f')$, where the order is considered with respect to $z$,
as indicated by the subscript. 

By Theorem \ref{irredcomp} we know that the $\s$-variety (in the sense of $\s\text{\catname{-VarField}}_k$) $ \VV_{\operatorname{Field}}(f')$ has an irreducbile component (also in the sense of $\s\text{\catname{-VarField}}_k$) $X' = \VV(\p')$ of $\s$-dimension $n-1$ and $\s$-$\operatorname{deg} = \operatorname{Ord}(f')$. This means that there exists a $\s$-prime difference ideal $\p \si k\{z\}$ minimal over $[f'] \si k\{z\}$ with
$\s$-$\dim(\p) = n-1$ and $\s$-$\operatorname{deg}(\p) = \operatorname{Ord}_z(f')$. 

Now, we assert that the ideal $(\p)$ generated by $\p$ in $k\{y\}$ is a prime difference ideal. 
To prove this, let $f = \sum_{i=1}^r f_i p_i \in (\p)$, with $p_i \in \p; f_i \in k\{y\}$ for all $i$. Then, since $\p$ is a difference ideal, it has to be that $\s(p_i) \in \p$ for all $i$.
This, in turn, implies that $\s(f) =  \sum_{i=1}^r \s(f_i) \s(p_i) \in (\p)$. Thus, $(\p)$ is a difference ideal in $k\{y\}$. 

Now we can show the primality of $(\p)$. As rings, it holds that $$k\{y\} \cong k[y,\ldots,\s^{d-1}(y)] \otimes_k k\{z\}.$$
From this it follows that 
\begin{equation}\label{isomoduloideal} (k[y,\ldots,\s^{d-1}(y)] \otimes_k k\{z\}) / (\p) \cong k\{z\} / \p \otimes_k k[y,\ldots,\s^{d-1}(y)]. \end{equation}
This last statement is proven in detail in the proof of Proposition \ref{idealzeroabove} (NOTE: I would split that off as a lemma and make a reference to the lemma, not to the proof of the proposition).
Since $\p$ is prime, $k\{z\}/\p$ is an integral domain, and thus $k\{z\}/\p \otimes k[y,\ldots,\s^{d-1}(y)]$ as well.
Equation \ref{isomoduloideal} then implies that $(\p) \si k\{y\}$ is prime as well. 

We only have to compare the dimension polynomials of $\p$ (over $k\{z\}$) and $(\p)$ over $k\{y\}$.
For this, we first consider the isomorphism
\begin{align*} \varphi: k\{y\} \xrightarrow{\sim} k[y,\ldots,\s^{d-1}(y)] \otimes_k k\{z\}, \\ (y,\ldots,\s^{d-1}(y)) \mapsto (y,\ldots,\s^{d-1}(y)); \s^d(y) \mapsto z. \end{align*}
For any $f \in k\{\s^{d}(y)\}$ we have the property that $\s$-$\operatorname{dim}_y(f) = \s$-$\operatorname{dim}_z(\varphi(f)) - d$.
In particular, this means that $\varphi$ commutes with the opreation of restricting to the $i$-th power of $\s$, in the following sense:
For a subset $F \subseteq k\{y\}$ and $i > d$ it holds that
\begin{align*} \varphi(F[i]) = \varphi(k\{y\}[i] \cap F) = \varphi(k\{y\}[i]) \cap \varphi(F) \\ = k[y,\ldots,\s^{d-1}(y)] \otimes_k k\{z\}[i-d] \cap \varphi(F) .\end{align*}
If we apply this to $\p$, we get that for $i\geq 0$, $\p[i] = (\p)[i+d]$, and thus $$\operatorname{dim}(k\{y\}[i+d]/(\p)[i+d]) = \operatorname{dim}(k\{z\}[i]/\p[i]).$$
By definition of the $\s$-$\operatorname{deg}$ and $\s$-$\operatorname{dim}$ this implies that
$$\s\text{-}\operatorname{dim}((\p))= \s\text{-}\operatorname{dim}(\p) = n-1 $$
as well as
$$\s\text{-}\operatorname{deg}((\p))= \s\text{-}\operatorname{deg}(\p) + \s\text{-}\operatorname{dim}(\p) \cdot d = \operatorname{Ord}(f) + (n-1) \cdot (\operatorname{Ord}(f) - \operatorname{Eord}(f))$$

\end{bew}
\end{cor}

This corollary shows an application of the theory developed for mixed difference ideals, where we obtained a simple generalization of the result obtained by the standard methods. 
The following simple example is not covered by the assertion of Theorem \ref{irredcomp}, and shows the additional factor obtained in the general case of Corollary \ref{corfinal}


\begin{ex}
$$(\s^2(y_1 y_2 y_3) + 1)$$
\end{ex}

%% \begin{lem}
%% Let $k$ be a $\s$-field and $f \in k\{y_1,\ldots,y_n\}, f \notin k$ an irreducible $\s$-polynomial.  
%% Then there exists a prime difference ideal $\q \si k\{y_1,\ldots,y_n\}$, minimal containing $f$, and a $d \in \N$, such that 
%% $$\dim(k\{y_1,\ldots,y_n\}[i]/\q[i]) = \left\{ \begin{array}{lr} i, i < \operatorname{Ord}(f) \\ d(i-\operatorname{Ord}(f)) +\ operatorname{Ord}(f),
%%  i \geq \operatorname{Ord}(f) \end{array} \right.$$
%% \end{lem}


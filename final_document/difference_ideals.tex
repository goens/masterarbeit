In this section we will study difference ideals more closely. A special emphasis will be given to certain difference ideals which mimic in many ways the properties of and relationship between radical and prime ideals on (general) rings.
Much of the following is based on Section 1.2 of the lecture notes of M. Wibmer \cite{wibmer}, where most of it is worked out for an analogous case. \\

\subsection{Difference Ideals}
\indent There are numerous properties in which we are interested for studying difference ideals. We will start by defining some of them:
\begin{defn}
Let  $\a \si R$ be a $\s$-ideal of $R$. 
\begin{itemize}
\item Then $\a$ is called a \emph{mixed $\s$-ideal} if for any $f,g \in R$ with $fg \in \a$ it follows that $f \sigma(g) \in \a$. \index{mixed $\s$-ideal}
\item $\a$ is called \emph{perfect}, if $\sigma^{i_1}(f) \cdots \sigma^{i_n}(f) \in \a$ implies that $f \in \a$, where $n \in \NE, i_j \in \N \fa j \in \n$. \index{perfect $\s$-ideal}
\item $\a$ is called \emph{reflexive}, if $\s(f) \in \a$ implies $f \in \a$. \index{reflexive $\s$-ideal}
\item $\a$ is called \emph{$\s$-prime}, if $\a$ is a prime, reflexive $\s$-ideal. \index{$\s$-prime}
\item The ring $R$ is called \emph{well-mixed}, if the zero ideal $[0]$ is mixed, and \emph{perfectly $\s$-reduced}, if it is perfect. \index{well-mixed $\s$-ring}\index{perfectly $\s$-reduced $\s$-ring}
\end{itemize}
\end{defn}

\begin{rem}
It is easy to see from the definitions that $\s$-prime ideals are perfect, and that perfect $\s$-ideals are mixed, radical and reflexive. Prime $\s$-ideals are also mixed, but not necessarily perfect. Note that there is a difference between a prime $\s$-ideal, and a $\s$-prime ideal:
the former does not necessarily have to be reflexive, as is the case with the latter. Both, prime $\s$-ideals and $\s$-prime ideals will be very important throughout this thesis. It is for that reason that the distinction between both is of utmost importance.
\end{rem}

The above properties behave well with respect to morphisms of $\s$-rings in the sense of the following Lemma (see Exercise 1.2.7 of \cite{wibmer}):
\begin{lem}\label{bijmapping}
Let $\varphi: R \rightarrow S$ be a morphism of $\s$-rings and $\a \si S$ a $\s$-ideal of $S$. Then $\varphi^{-1}(\a) \si R$ is a $\s$-ideal of $R$. Similarly, if $\a$ is a mixed $\s$-ideal, then so is $\varphi^{-1}(\a)$. The same is true for perfect and for reflexive $\s$-ideals.
\begin{bew}
Since $\a \unlhd S$ is an ideal, so is $\b := \varphi^{-1}(\a) \unlhd R$. Let $b \in \b$. Then $\varphi(b) =: a \in \a$ by definition. Since $\a \si S$ is a $\s$-ideal, $\s(a) \in \a$, and since $\varphi$ is a morphism of $\s$-rings
it follows that $\sigma(a) = \sigma(\varphi(b)) = \varphi (\s (b)) \in \a$. Hence, $\s(b) \in \b$ which implies that $\b$ is a $\s$-ideal. \\
\indent Now let $\a$ be mixed and $fg \in \b$. This means by definition of $\b$, 
that $\varphi(fg) = \varphi(f) \varphi(g) \in \a$. Since $\a$ is mixed, this in turn implies that $$\varphi(f) \s( \varphi(g)) = \varphi(f) \varphi(\s(g)) = \varphi(f\s(g)) \in \a,$$ which yields $f\s(g) \in \b$, so that $\b$ is also mixed. 
The proof for perfect and for reflexive difference ideals is analogous.
\end{bew}
\end{lem}

\begin{rem}
Let $R$ be a $\s$-ring and $\a \si R$ a $\s$-ideal. We can define a canonical $\s$-ring structure on the quotient ring $R/\a$ via $\s(r+\a):= \s(r) + \a$. 
This is well defined and in particular makes the canonical epimorphism $\tau: R \twoheadrightarrow R/\a$ a morphism of $\s$-rings.
\end{rem}

\begin{prop}\label{bijideals}
Let $R$ be a $\s$-ring and $\a \si R$ a $\s$-ideal. The canonical epimorphism $\tau: R \twoheadrightarrow R/\a$ induces, in the sense of Lemma \ref{bijmapping}, a bijection between the sets $\{ \b \si R/\a \}$ and $\{ \a \si \b \si R \}$. The same holds true if we restrict both sets to prime, radical and mixed, $\s$-prime or perfect $\s$-ideals.
\begin{bew}
See Proposition 1.2.8 of \cite{wibmer}.
\end{bew}
\end{prop}

\begin{rem}\label{wmwelldef}
Let $R$ be a $\s$-ring, and $F \subseteq R$ be a subset of $R$. Any intersection of mixed, radical $\s$-ideals containing $F$ is also a mixed, radical $\s$-ideal, which of course contains $F$. 
This means that there is a smallest (with respect to inclusion) mixed, radical $\s$-ideal $\a$ containing $F$; namely, the intersection of all such $\s$-ideals:
\begin{align*} \a = \bigcap_{\substack{ \b \si R, \\ \b \text{ radical and mixed}}} \b. \end{align*}
\begin{proof}
Let $I$ be an index set and $\a_i \si R \fa i \in I$ be mixed, radical $\s$-ideals. Further let $\b := \bigcap_{i \in I} \a_i$ be the intersection of these. Obviously, $\b$ is (algebraically) an ideal of $R$. We will show that it is also a $\s$-ideal, radical and mixed.
If $a \in \a_i \fa i \in I$, then $\s(a) \in \a_i \fa i \in I$, since each $\a_i$ is a $\s$-ideal.
It follows that $\s(a) \in \b$. \\
\indent Similarly, if $aa' \in \a_i \fa i \in I$, then $a \s(a') \in \a_i \fa i \in I$, since each $\a_i$ is mixed, which implies that $a \s(a') \in \b$.  \\
\indent Finally, if $a \in \sqrt(\b)$ there exists an $n \in \N$ such that $a^n \in \b$. This means that $a^n \in \a_i \fa i \in I$, which implies that $a \in \sqrt(\a_i) \fa i \in I$. Since every $\a_i, i \in I$ is radical, this means that $a \in \a_i \fa i \in I$,
and thus $a \in \b$.
\end{proof}
\end{rem}

\begin{defn}
The $\s$-ideal $\a$ from Remark \ref{wmwelldef} is called the radical, mixed closure of $F$, and we will denote it by $\{F\}_{m}$.
\end{defn}


\begin{lem}\label{sqrtmixed}
Let $R$ be a $\s$-ring and $\a \si R$ be a mixed $\s$-ideal. Then the radical of $\a$, $\sqrt{\a}$, is also mixed.
\begin{bew}
Let $f,g \in R$ be such that $fg \in \sqrt \a$. By definition there exists an $n \in \NE$ such that $f^n g^n = (fg)^n \in \a$. Since $\a$ is mixed, this implies that $f^n \s(g^n) = f^n \s(g)^n = (f\s(g))^n \in \a$. 
But this in turn implies that $f\s(g) \in \sqrt \a$, which is what we wanted to show.
\end{bew}
\end{lem}


\begin{ex}\label{nombasisex}
Let $k$ be a constant $\s$-field and let $R:= k\{y_1,y_2\}$. Consider the difference ideal $\a:= [y_1y_2] \si R$. We can inductively define a chain \begin{align*}\a^{\{0\}}:= \a,~ \a^{\{m+1\}}:= [\{ f \s(g) \mid f,g \in \a^{\{m\}}\}] \\ = [\s^k(y_1)\s^l(y_2) \mid k,l = 0,1,\ldots,m+1], \fa m \in \NE.\end{align*}
This is an infinite properly ascending chain of difference ideals. 
%% \begin{bew}
%% We will show that $\a^{\{m\}} = [\s^k(y_1)\s^l(y_2) \mid k,l = 0,1,\ldots,m]$ by induction. The assertion that the $\a^{\{m\}}$ form an infinite chain of properly ascending $\s$-ideals is obvious from this.\\
%% \indent The inclusion $ [\s^k(y_1)\s^l(y_2) \mid k,l = 0,1,\ldots,m] \subseteq \a^{\{m\}}$ follows from an induction on $m$ directly from the definition of $\a^{\{m\}}$. \\ 
%% \indent  To show the other inclusion, ``$\supseteq$'', we do it also inductively. For $m=0$ it holds that $\a^{\{0\}} = \a = [y_1y_2]$. Let thus $m \in \N$ and $a \in \a^{\{m+1\}}$.
%% By definition, there exist $k \in \NE, f_i, g_i, r_i \in R$ with $f_ig_i \in \a^{\{m\}}, i = 1,\ldots,k$ such that $a = \sum_{i=1}^k r_i f_i \s(g_i)$. 

%% FIXME: FINISH THIS!!!!!!!

%%  To prove that the chain is properly ascending, it is enough to show that $y_1\s^m(y_2) \in \a^{\{m\}}$, but $y_1\s^{m+1}(y_2) \notin \a^{\{m\}}$, since $\a^{\{m\}} \subseteq \a^{\{m+1\}}$ is obvious by definition.
%% This can be shown by induction: For $m = 0$, it is clear that $y_1y_2 \in [y_1y_2]$, but $y_1 \s(y_2) \notin [y_1y_2]$. Now, if $y_1 \s^m(y_2) \in \a^{\{m\}}$, then by definition, $y_1 \s^{m+1}(y_2) \in \a^{\{m+1\}}$.
%% On the other hand, if $\y_1 \s^{m+1}(y_2) \notin \a^{\{m\}}$, then it is clear that $\y_1 \s^{m+2}(y_2) \notin \{ f\s(g) \mid f,g \in \a^{\{m\}}\} =: M$. We only need to show that $y_1\s^{m+2}(y_2) \notin [M]$.

%% \end{bew}
\end{ex}


To try to find $\{\a\}_m$ for a $\s$-ideal $\a \si R$ in a difference ring $R$, it might be tempting to consider $\a':= \{ f\s(g) \mid fg \in \a \}$, or to ensure that is a difference ideal rather, $[\a']$. The example above shows that this is not enough,
as the ideal $[\a']$ does not have to be mixed in general. However, by iteratively repeating this process and taking the union of $\s$-ideals obtained this way, we do get a mixed $\s$-ideal, as we will see in the following Lemma:
\begin{lem}\label{lemshuffling}
Let $R$ be a $\s$-ring and $F \subseteq R$. Further let $F' := \{f\s(g) \mid fg \in F \}$, and set $F^{\{1\}}:= [F']$, $F^{\{m\}}:= [{F^{\{m-1\}}}'] \fa m \in \NE$. Then
\begin{align} \{F\}_m = \sqrt{\bigcup_{n=1}^{\infty} F^{\{n\}}}. \end{align}
This way of obtaining $\{F\}_m$ is called a shuffling processes and has an analog for perfect $\s$-ideals (see for example \cite{levin}, p. 121f.) \index{Shuffling process}
\begin{proof}
Let $\a:= \bigcup_{n=1}^{\infty} F^{\{n\}}$. It is obvious from the construction that $F \subseteq \a$. It also holds that $\a$ is a mixed $\s$-ideal, since for any $f,g \in \a$ there exists an $m \in \NE$ such that $f,g \in F^{\{m\}}$.
And hence $f + g, \s(f) \in F^{\{m\}} \subseteq \a$, as well as $fh \in F^{\{m\}} \subseteq \a$ for any $h \in R$. Furthermore, for $f, g \in R$ with $fg \in \a$ there also exists an $m \in \NE$ such that $fg \in F^{\{m\}}$. 
Then we have $f\s(g) \in F^{\{m+1\}} \subseteq \a$. \\
\indent On the other hand, by induction on the iterative steps $F^{\{n\}}$ it follows that for every mixed $\s$-ideal $\b$ which contains $F$, $F^{\{n\}} \subseteq \b$. Hence, $\a$ is the smallest mixed $\s$-ideal containing $F$. \\
\indent By Lemma \ref{sqrtmixed} we know that $\sqrt{\a}$ is mixed. This actually shows that $\sqrt a$ is indeed the smallest mixed, radical $\s$-ideal of $R$ containing $F$, since every such ideal has to contain $\a$, and thus $\sqrt{\a}$ as well.

% To see this, assume there exists a radical, mixed $\s$-ideal $\b \supseteq F$
%such that $\b \subsetneqq \sqrt{\a}$. Then we have $F \subseteq \a \cap \b \subsetneqq \a$, a contradiction to the minimality of $\a$ (since the intersection of mixed $\s$-ideals is mixed). 
\end{proof}
\end{lem}


\begin{ex} %%not *radical*! 
Let $k$ be a $\s$-field, and consider $R = k\{y_1\}$. Then the $\s$-ideal $[y_1] \si R$ is mixed, hence equal to its mixed closure.
The mixed closure of $[y_1] \cdot [y_1]$ is $[ y_1 \s^i(y_1) \mid i \in \N ] \not \ni y_1$.
One could have expected, perhaps, to get an analog of the statement in algebraic geometry that $\sqrt{F_1  F_2 } = \sqrt{F_1} \cap \sqrt{F_2}$, but this example shows it is not in general so for mixed ideals.
It is however very noteworthy that the ideal $[ y_1 \s^i(y_1) \mid i \in \N ] \si R$ is not radical. For radical, mixed difference ideals we will in fact get such a statement later (Corollary \ref{prod=cap}).
\end{ex}

A very important result in commutative algebra is the fact that every radical ideal is the intersection of prime ideals. This has an analogue for perfect $\s$-ideals, as well as for mixed $\s$-ideals. 
We will prove the latter, but for this we need a few additional tools. We will first prove a weaker version of the statement, for which we need a few results from commutative algebra:

\begin{lem}\label{commalg}
Let $R$ be a ring. 
\begin{enumerate}[(a)]
\item If $S \geq R$ is an overring of $R$, and $\p$ is a minimal prime ideal of $R$, then there exists a minimal prime ideal $\q$ of $S$ such that $\p = \q \cap R$.
\item Every radical ideal of $R$ is the intersection of prime ideals. If $R$ is Noetherian, then every radical ideal of $R$ is the intersection of finitely many prime ideals.
\item If $R$ is Noetherian and $\p \unlhd R$ is a minimal prime ideal of $R$, then there exists an element $a \in R$ such that $\p$ is the annihilator ideal of $a$, i.e. $\p = \text{Ann}(a) = \{ r \in R \mid ra = 0 \}$.
\end{enumerate}
\begin{bew} $~$
\begin{enumerate}[(a)]
\item See Remark 2.9 of \cite{hrushovski}
\item See \cite{bourbaki} Ch. 2, \S 2.6, Corollary 2 to Proposition 13 and Ch. 2, \S 4.3, Corollary 3 to Proposition 14.
\item This is a special case of Theorem 3.1 of \cite{eisenbud} for $R$ as an $R$-module.
\end{enumerate}
\end{bew}
\end{lem}


The following is adapted from the proof given in \cite{wibmer} of Proposition 1.2.28 (here the upcoming Theorem \ref{intersectionprimes}). It has been worked out in more detail and divided further in a way to try and make this more detailed version of the proof easier to read.

\begin{defn}
Let $R$ be a difference ring. We say $R$ is \emph{finitely $\s$-generated over $\Z$} if there exists a finite set $A \subseteq R$ so that every $f \in R$ can be written as a finite $\Z$-linear combination of $\s$-powers of elements in $A$. In other words,
for every $f \in R$ there exists an $n \in \NE$ so that $f \in \Z[A,\sigma(A),\ldots,\s^n(A)]$. \\ 
\indent For any subset $F \subseteq R$ we denote by $$\Z\{F\} = \{ f \in R \mid \text{ there exists an } n \in \N: f \in \Z[F, \s(F), \ldots, \s^n(F)] \}$$ the set of all elements $\s$-generated by $F$ over $\Z$.
\end{defn}\index{finitely $\s$-generated over $\Z$}

\begin{prop}\label{mixedintersectionprimesfinite}
Let $R$ be a $\s$-ring finitely $\s$-generated over $\Z$. Then, every radical, mixed $\s$-ideal of $R$ is the intersection of prime $\s$-ideals.
\begin{bew}
Let $\a \si R$ be a mixed, radical $\s$-ideal. By Proposition \ref{bijideals} there is a bijection between the prime $\s$-ideals of $R$ containing $\a$ and those of $R/\a$. Hence, we can assume without loss of generality that $\a = [0] \si R$,
 by replacing $R$ with $R/\a$. This means that we only have to show that the zero ideal $[0]$ of a well-mixed, reduced $\s$-ring $R$ is the intersection of all its prime $\s$-ideals. Note that this does not change the fact
that $R$ is finitely $\s$-generated over $\Z$. \\
\indent Let thus $f \in R$ be such that $f \in \q \fa \q \si R$ prime. We assert that $f$ then has to be $0$. Assume this is not the case, i.e., $f \neq 0$. Then by assumption on $R$ there is an $n \in \N$ such that $f \in \Z[A,\s(A),\ldots,\s^n(A)]$.
We now use the special case for (algebraic) ideals: since $\Z[A,\s(A),\ldots,\s^n(A)]$ is Noetherian and reduced, $(0) \unlhd \Z[A,\ldots,\s^n(A)]$ is the intersection of all prime ideals of $R$. In particular, there exist prime ideals which do not contain $f$.
Let $\q_0 \unlhd \Z[A,\ldots,\s^n(A)]$ be a minimal such prime ideal, i.e., with $f \notin \q_0$. Since $f \in \Z[A,\s(A),\ldots,\s^n(A)] \subset \Z[A,\s(A),\ldots,\s^{n+1}(A)]$, again by Lemma \ref{commalg}, we can find a minimal prime ideal $\q_1 \unlhd \Z[A,\s(A),\ldots,\s^{n+1}(A)]$
such that $\q_1 \cap \Z[A,\s(A),\ldots,\s^{n}(A)] = \q_0$. \\
\indent Inductively we find a chain of minimal prime ideals $\q_i, i \in \N$, $\q_i \unlhd \Z[A,\s(A),\ldots,\s^{n+i}(A)]$, with $\q_{i+1} \cap \Z[A,\s(A),\ldots,\s^{n+i}(A)] = \q_i$ for all $i \in \N$.
Then $\q := \bigcup_{i=0}^{\infty} \q_i$ is a prime ideal of $R$, with $f \notin \q$. In fact, $\q$ is a $\s$-ideal of $R$: Let $a \in \q$. We want to show that $\s(a) \in \q$. By construction of $\q$ there exists an $i \in \N$ such,
that $a \in \q_{i-1} \subseteq \Z[A,\s(A),\ldots,\s^{n+i-1}(A)]$, which implies that $\s(a) \in \Z[A,\s(A),\ldots,\s^{n+i}(A)]$. Lemma \ref{commalg} states then, that there is an $h \in \Z[A,\s(A),\ldots,\s^{n+i}(A)]$ such that $ \q_i = \text{Ann}(h)$.
It follows that $ah = 0$, and since $R$ is well-mixed, this implies that $\s(a)h = 0$, hence, $\s(a) \in \q_i \subseteq \q$. This means that $\q$ is a prime $\s$-ideal of $R$ not containing $f$, which contradicts the assumption on $f$, so that $f = 0$ has to follow.
\end{bew}
\end{prop}

For the general case we need yet another tool, the concept of filters:

\begin{defn}\index{filter}\index{ultrafilter}
Let $U$ be a set, and let $F \subseteq \text{Pot}(U)$, where $\text{Pot}(U)$ denotes the power set on $U$. Then $F$ is called a \emph{filter} if it satisfies the following axioms: 
\begin{itemize}
\item  $U \in F$ and $\emptyset \notin F$.
\item If $V,W \subseteq U$ with $V \subseteq W \text{ and }V  \in F $ it holds that $W \in F$.
\item For $V_1, \ldots, V_n \in F$ it holds that \[ \bigcap_{i = 1}^n V_i \in F. \]
\end{itemize}
A filter $F$ is called an \emph{ultrafilter}, if for any $V \subseteq U$ it holds that $V \in F$ or $U \setminus V \in F$. Note that the first and third axioms together imply that at most one of $V$ and $U \setminus V$ can be in $F$.
\end{defn}



\begin{rem}
Let $U$ be a set. Then, the set of filters on $U$ is inductively ordered by inclusion. By Zorn's lemma, for every filter $F$ on $U$ there must exist a maximal filter $G$ with respect to inclusion such that $F \subseteq G$.
The maximality of the filter implies that $G$ will be an ultrafilter, since we could otherwise find a new filter $G'$ where $G$ is properly included by adding one of the sets which contradict the ultrafilter property and considering the smallest filter containing this set.
\end{rem}

The reason why this concept is useful in our context is the following:

\begin{lem}\label{lemmafilters}
Let $R$ be a $\s$-ring, and let $M$ be the set of all $\s$-subrings of $R$ which are finitely $\s$-generated over $\Z$. For any fixed subset $F \subseteq R$, consider the set $M_F:= \{ T \subseteq M \mid \{S \in M \mid F \subseteq S \} \subseteq T \} \subseteq \text{Pot}(M)$. 
Then, \[ \mathcal{F}:= \bigcup_{ F \subseteq R \text{ finite} } M_F \]
 defines a filter on $M$. If $\mathcal{G}$ is an ultrafilter containing $\mathcal{F}$, and $P:= \prod_{S \in M} S$ with component-wise operations,
 then the ultrafilter $\mathcal{G}$ defines an equivalence relation on $P$ via $(g_S)_{S \in M} \sim (h_S)_{S \in M} : \Leftrightarrow \{ S \in M \mid g_S = h_S \} \in \mathcal{G}$. 
The set of equivalence classes $P/\mathcal{G}:= P/\textsim$ has a natural $\s$-ring structure and is called an \emph{ultraproduct}. \index{ultraproduct} %%fixme: of what?
\begin{proof}
Let us first show that $\mathcal{F}$ is a filter. For $F \subseteq R$ finite we have $\Z\{F\} \in \{ S \in M \mid F \subseteq S \} \neq \emptyset$, and since $T \supseteq \{ S \in M \mid F \subseteq S \} \fa T \in M_F$, $\emptyset \notin M_F$ (Note that $\Z\{\emptyset\} = (0)$, so it also holds for $F = \emptyset$).
  That $M \in M_F$ for any $F \subseteq R$ is obvious, as well as that for $T \subseteq U, T \in M_F$ it holds that $U \in M_F$. \\ 
\indent We only need to show that $U,T \in \mathcal{F}$ implies that $U \cap T \in \mathcal{F}$.
  Let $\hat U, \hat T \subseteq R$ be finite, such that $U \in M_{\hat U}, T \in M_{\hat T}$. $\hat U \cup \hat T \subseteq R$ is also finite and it holds that  $\{ S \in M \mid \hat U \cup \hat T \subseteq S \} \subseteq \{ S \in M \mid \hat U \subseteq S \} \subseteq U$,
 and similarly for $T$. This means that $U \cap T \in M_{\hat U \cup \hat T} \subseteq \mathcal{F}$, which finishes the proof that $\mathcal{F}$ is a filter. \\

 Now, consider an ultrafilter $\mathcal{G} \supseteq \mathcal{F}$ and define $\sim$ on $P$ as above. This is an equivalence relation: Let $f \sim g, g \sim h$ for $f,g,h \in P$. 
 This means that $\{ S \in M \mid f_S = g_S \} \in \mathcal{G}, \{ S \in M \mid g_S = h_S \} \in \mathcal{G}$. But then $$\{ S \in M \mid f_S = g_S \} \cap \{ S \in M \mid g_S = h_S \} \subseteq \{ S \in M \mid f_S = h_S \} \in \mathcal{G},$$
 since $\mathcal{G}$ is a filter.
 Reflexivity follows from the fact that $M \in \mathcal{G}$, and symmetry is obvious. \\
\indent  We now only need to show that we have a well-defined $\s$-ring structure on $P/\textsim$.
 Consider $f,f' \in P$ with $f \sim f'$. We have that for all $S \in M$ with $f_S = f'_S$ it holds that $\sigma(f)_S = \sigma(f')_S$. 
 But then $\{ S \in M \mid \s(f)_S = \s(f')_S \} \supseteq \{ S \in M \mid f_S = f'_S \} \in \mathcal{G}$ by assumption, and since $\mathcal{G}$ is a filter, this means that $\{ S \in M \mid \s(f)_S = \s(f')_S \} \in \mathcal{G}$,
 hence $\s(f) \sim \s(f')$. That $+$ and $\cdot$ are also well-defined can be proven in an analogous fashion.
\end{proof}
\end{lem}

We can now turn our attention to the generalization of Proposition \ref{mixedintersectionprimesfinite}. 


\begin{theorem}\label{intersectionprimes}
Let $R$ be a $\s$-ring and $F \subseteq R$ be a subset of $R$. Then, 
\begin{align*} \{F\}_m = \bigcap_{\substack{F \subseteq \p \si R \\ \p \text{ prime}}} \p \end{align*}
In particular, every radical, mixed $\s$-ideal of $R$ is the intersection of prime $\s$-ideals.
\begin{bew}
It suffices to show that every radical, mixed $\s$-ideal of $R$ is the intersection of prime $\s$-ideals.
Indeed, since prime $\s$-ideals are radical and mixed, it is clear that $\{F\}_m \subseteq \p$ for every prime $\p \si R$ with $F \subseteq \p$, which together with the fact that every radical, mixed $\s$-ideal of $R$ is the intersection of prime $\s$-ideals gives the representation 
\begin{align*} \{F\}_m = \bigcap_{\substack{F \subseteq \p \si R \\ \p \text{ prime}}} \p. \end{align*}
Now, by the same argument as in the beginning of the proof of Proposition \ref{mixedintersectionprimesfinite}, it is enough to prove in the case that $R$ is well-mixed and reduced, that the intersection of all prime $\s$-ideals of $R$ is $[0]$.
Let $0 \neq f \in R$. We will construct a prime $\s$-ideal $\q$ of $R$ which does not contain $f$: 

Let $P/\mathcal{G}$ be the difference ring as in Lemma \ref{lemmafilters}. Consider the mapping $\varphi: R \rightarrow P/\mathcal{G}, g \mapsto (g_S)_{S \in M}$ with $(g_S) = g \fa S \in M$ with $g \in S$ and $(g_S) = 0$ for $g \notin S$. 
It is in fact $\{ S \in M \mid g \in S \} \in M_{\{g\}}$ (with $M_{\{g\}}$ as in Lemma \ref{lemmafilters}). It follows from this that the image of the mapping onto $P/\mathcal{G}$ is in fact independent of the $(g_S)$ for $g \notin S$, as any other choice of these would be in the same $\sim$ class as the image described above.
It follows that $\varphi$ is a well-defined morphism of $\s$-rings. \\
\indent From Proposition \ref{mixedintersectionprimesfinite} we know that for every $S \in M$, there exists a prime $\s$-ideal $\p_S \si S$ such that $f \notin \p_S$. 
We define $\p \subseteq P/\mathcal{G}$ as the set of all equivalence classes of elements $(g_S)_{S \in M}$ such that $\{ S \in M \mid g_S \in \p_S \} \in \mathcal{G}$. 
For $[(g_S)_{S \in M}]_{\sim}, [(h_s)_{S \in M}]_{\sim} \in \p$ we have $$ \mathcal{G} \ni \{ S \in M \mid  g_S \in \p_S \} \cap  \{ S \in M \mid  h_S \in \p_S \} \subseteq \{ S \in M \mid  g_S + h_S \in \p_S \} \in \mathcal{G},$$
since $\mathcal{G}$ is a filter. Similar arguments for $\s(g), gh$ for $h \in P/\mathcal{G}$ show that $\p$ is indeed a $\s$-ideal. $\p$ is also prime since $\mathcal{G}$ is an ultrafilter:
Let $g,h \in P$ with $\{ S \in M \mid g_Sh_S \in \p_S \} \in \mathcal{G}$. If $[g]_\sim \notin \p$, then $V:= \{ S \in M \mid g_S \in \p_S \} \notin \mathcal{G}$. Since $\mathcal{G}$ is an ultrafilter, 
this means that $M \setminus V \in \mathcal{G}$. But $$\mathcal{G} \ni (M \setminus V) \cap \{ S \in M \mid g_S h_S \in \p_S \} \subseteq \{ S \in M \mid h_S \in \p_S \} \in \mathcal{G},$$
which means that $[h]_\sim \in \p$. The preimage of a prime $\s$-ideal, $\q := \varphi^{-1}(\p) \si R$ is also prime. By construction, $[\varphi(f)]_\sim \notin \p$, which means that $f \notin \varphi^{-1}(\p)$, as desired. 

\end{bew}
\end{theorem}

\subsection{An Analog of the Cohn Topology}

\begin{defn}
Let $R$ be a $\s$-ring. We denote the set of all prime $\s$-ideals of $R$ by $\sSpec(R):= \{ \p \si R \mid \p \text{ prime }\}$. Similarly, we denote the set of $\s$-prime ideals by $\Spec^\s(R):= \{ \p \si R \mid \p ~ \s\text{-prime }\} \subseteq \sSpec(R)$.
\index{$\sSpec$} \index{$\Spec^\s$}
\end{defn}


\begin{rem}
As is the case with $\Spec^\s(R)$, it can be the case that $\sSpec(R)= \emptyset$. For example, let $R$ be a $\s$-ring, and consider the $\s$-ring $R \oplus R$, with $\s( (r,s)):= (\s(s),\s(r))$. 
We will show that this ring has no prime $\s$-ideals. Let $\p \unlhd R$ prime. Then $0 = (1,0)(0,1) \in \p$, which means that either $(1,0) \in \p$ or $(0,1) \in \p$. But then $R \oplus 0 \subseteq \p$ or $0 \oplus R \subseteq \p$. If we assume that $\p$ is a $\s$-ideal then
 this implies that $R \oplus R \subseteq \p$, which cannot be, by definition.
\end{rem}

In algebraic geometry, one usually considers $\Spec(R)$ as a topological space with a topology called the Zariski topology. This has an analog for $\Spec^\s(R)$, usually called the Cohn topology. Here we will develop a further analog of both,
 which we will define on $\sSpec(R)$, and will be closely related to radical, mixed $\s$-ideals, as we shall see by its many properties.

\begin{defn}
Let $R$ be a $\s$-ring and $F \subseteq R$ be a subset of $R$. We set $\Vm (F):= \{ \p \in \sSpec(R) \mid F \subseteq \p \}$. 
If $F$ has only one element $f$, we write $\Vm(f)$ for $\Vm(F)$. \index{$\Vm(F)$}
\end{defn}

\begin{lem}\label{topologywelldef}
Let $R$ be a $\s$-ring. Then we have:
\begin{enumerate}[(a)]
\item $\Vm((0)) = \sSpec(R)$, and $\Vm(R) = \emptyset$.
\item For any two ideals $\a,\b \unlhd R$ we have $\Vm(\a) \cup \V(\b) = \Vm(\a \cap \b).$
\item For any family of ideals $(\a_i)_{i \in I}$ for an index set $I$, we have $$\bigcap_{i \in I} \Vm(\a_i) = \Vm(\sum_{i \in I} \a_i).$$ \label{vmintersectionideals}
\end{enumerate}
\begin{bew} $~$
\begin{enumerate}[(a)]
\item We have $(0) \subseteq \p \fa \p \in \sSpec(R)$, as well as $R \not\subseteq \p \fa \p \in \sSpec(R)$.
\item Let $\a, \b \unlhd R$ be two ideals in $R$. Then $\Vm(\a) \cup \Vm(\b) \subseteq \Vm(\a \cap \b)$, since for $\p \si R$ prime, $\a \subseteq \p$ it follows that $\a \cap \b \subseteq \p$, and similarly for $\b$.
On the other hand, let $\p \si R$ prime with $\a \cap \b \subseteq \p$, and $\a \not\subseteq \p$ (otherwise $\p \in \Vm(\a)$ and we are done). Then there exists an $f \in \a$, $f \notin \p$. 
For any $g \in \b$, it follows that $fg \in \a \cap \b \subseteq \p$. Since $\p$ is prime, this implies that $g \in \p$. Hence, $\b \subseteq \p$, which concludes the proof.
\item Let $(\a_i)_{i \in I}$ be a family  of ideals of $R$. Then $$\p \in \bigcap_{i \in I} \Vm(\a_i) \Leftrightarrow \a_i \subseteq \p \fa i \in I \Leftrightarrow \p \in \Vm(\sum_{i \in I} \a_i).$$
\end{enumerate}
\end{bew}
\end{lem}


\begin{rem}\label{vmsequal}
Since for a $\s$-ring $R$ any prime $\s$-ideal of $R$ is radical and mixed, it holds that for any $F \subseteq R$, and any prime $\s$-ideal $\p \si R$ with $F \subseteq \p$ we have
$(F) \subseteq [F] \subseteq \{ F \}_m \subseteq \p$. In particular, this means that $\Vm(F) = \Vm((F)) = \Vm([F]) = \Vm(\{F\}_m)$. 
\end{rem}

\begin{defn}\label{deftop}
Let $R$ be a $\s$-ring. We define a topology on $\sSpec(R)$ by setting $A \subseteq \sSpec(R)$ closed if $A = \Vm(\a)$ for an ideal $\a \unlhd R$, or equivalently,
 by defining a set to be open, if it is a complement of such a $\Vm(\a)$. This is a well-defined topology thanks to Lemma \ref{topologywelldef}.
For $f \in R$ we set $$\sD(f):= \sSpec(R) \setminus \Vm(f).$$ $\sD(f)$ is the complement of a closed set, and hence, open. 
We call the sets of the form $\sD(f) \subseteq \sSpec(R)$ \emph{basic open subsets} of $\sSpec(R)$. \index{basic open subsets}
\end{defn}

From here on, if not explicitly stated otherwise, when referring to topological concepts on $\sSpec(R)$ we will be referring to the topology just defined.

\begin{rem}
From its definition it is clear that $\sSpec(R) \subseteq \Spec(R):= \{ I \unlhd R \mid I \text{ prime} \}$. Since Lemma \ref{topologywelldef} does not require the ideals to be $\s$-ideals, 
it is easy to conclude that in fact the topology on $\sSpec(R)$ is just the topology induced by restriction of the Zariski topology to $\sSpec(R)$. The same argument can be made to see that the Cohn topology in turn,
defined on $\Spec^\s(R) = \{ \p \si R \mid $ $\p$ $ \s$-prime $\} \subseteq \sSpec(R)$, is also the restriction of the topology defined on $\sSpec(R)$. 
\end{rem}

\begin{defn}
Let $X$ be a topological space.
\begin{enumerate}[(a)]
\item  We say that $X$ is \emph{irreducible} if $X = X_1 \cup X_2$ with $X_1, X_2$ closed implies that $X = X_1$ or $X = X_2$. 
$X_1 \subseteq X$ is called \emph{irreducible} if it is an irreducible topological space with the topology induced by the restriction to $X_1$.\index{irreducible topological space}
\item Let $Y \subseteq X$ be closed. We say that a point $f \in Y$ is a \emph{generic point} of $Y$, if $\overline{\{  f \} } = Y$, where for $A \subseteq X$, $\overline{A}$ denotes the closure of $A$. \index{generic point}
\end{enumerate}
\end{defn}

\begin{prop}
Let $R$ be a $\s$-ring. We have:
\begin{enumerate}[(a)]
\item \label{vmbijection} The mapping 
$$\{ \a \si R \mid \a\text{ mixed and radical }\} \rightarrow \{ A \subseteq \sSpec(R) \mid A \text{ closed }\}, \a \mapsto \Vm(\a)$$
 is bijective and order-reversing.
\item \label{irred=prime} For $F \subseteq R$ it holds that $\Vm(F)$ is irreducible if and only if $\{F\}_m$ is prime.
\item $\sSpec(R)$ is quasi-compact.
\item The basic open sets $\{ \sD(f) \mid f \in R \}$ form a basis for the topology on $\sSpec(R)$.
\item Every irreducible closed subset $Y$ of $\sSpec(R)$ has a unique generic point $y$.
\end{enumerate}
\clearpage %%FIXME: remove, just so a page is not left basically completely blank
\begin{bew} $~$
\begin{enumerate}[(a)]
\item \label{orderreversingbij} That the mapping is order-reversing is obvious. The injectivity follows from the fact that by Theorem \ref{intersectionprimes} $\a = \bigcap_{\a \subseteq \p \in \sSpec(R)} \p$. By Remark \ref{vmsequal} we obtain the surjectivity,
 since $\Vm(\a) = \Vm(\{\a\}_m)$.
\item Since $\Vm(F) = \Vm(\{F\}_m)$, we can assume without loss of generality, that $F \si R$ is a radical, mixed $\s$-ideal.
For the first implication, ``$\Leftarrow$'', let $F \si R$ be prime, and $\Vm(F) = \Vm(\a) \cup \Vm(\b)$ with radical, mixed $\s$-ideals $\a, \b$. Assume that $\Vm(F) \not\subseteq \Vm(\a)$. Then by (\ref{orderreversingbij}), $\a \not \subseteq F$, so there exists an $a \in \a$, with $a \notin F$.
For any $b \in \b$ we then have $ab \in \p \fa \p \in \V(F) = \Vm(\a) \cup \Vm(\b)$, and with Theorem \ref{intersectionprimes} we get $ab \in F = \bigcap_{\p \in \Vm(F)}\p$. By assumption, $F$ is prime and $a \notin F$, which implies
 that  $b \in F$. But this means that $\b \subseteq F$, and thus $\Vm(F) \subseteq \Vm(\b)$, which shows the irreducibility. \\
\indent Now, for the other implication, ``$\Rightarrow$'', assume that $\Vm(F)$ is irreducible, and let $a,b \in R$ with $ab \in F$. Consider $F \subseteq \p \in \Vm(F)$. Then $ab \in \p$, 
which means that $a \in \p$ or $b \in \p$, since $\p$ is prime. This implies that $\p \in \Vm(\{a\}_m) \cup \Vm(\{b\}_m)$, which in turn implies that $\Vm(F) \subseteq \Vm(\{a\}_{m}) \cup \Vm(\{b\}_{m})$.
Now, by assumption, $\Vm(F)$ is irreducible, and thus it has to be that $\Vm(F) \subseteq \Vm(\{a\}_{m})$ or $\Vm(F) \subseteq \Vm(\{b\}_m)$. By the bijectivity of the mapping in (\ref{orderreversingbij}) this means that $a \in F$ or $b \in F$.
\item Let $\Vm(\a_i)_{i \in I}$ be a family of closed sets, $\a_i \si R$ mixed, radical for all $i \in I$, satisfying that 
$\bigcap_{i \in J} \Vm( a_i) \neq \emptyset$ for every finite $J \subseteq I$. By  going to the complement of open sets, quasi-compactness is equivalent to the implication that $\bigcap_{i \in I} \Vm(a_i) \neq \emptyset$.
By Lemma \ref{topologywelldef} we see that $\bigcap_{i \in I} \Vm( \a_i) = \Vm ( \sum_{i \in I} \a_i)$. Assume that $ \Vm ( \sum_{i \in I} \a_i) = \emptyset$. 
By Theorem \ref{intersectionprimes} this means that $\{ \sum_{i \in I} \a_i \}_m = R$. In particular, $1 \in \{ \sum_{i \in I} \a_i \}_m$. By the construction in Lemma \ref{lemshuffling} (and with the notation used there), this means that there has to be an $n \in \NE$,
so that $1 \in (\sum_{i \in I} \a_i )^{\{n\}}$. In particular, this means that $1$ can be written as a finite $R$-linear combination $\sum_{k=1}^l r_k a_k$ with $a_k \in (\sum_{i \in I} \a_i )^{\{n\}}, r_k \in R, k = 1,\ldots,l$. In particular, there exists a $J \subseteq I$ finite,
such that $a_k \in (\sum_{i \in J} \a_i )^{\{n\}}$ for all $k \in \{1, \ldots, l \}$. But this implies that $1 \in (\sum_{i \in J} \a_i)^{\{n\}}$, meaning that $\Vm(\sum_{i \in J} \a_i) = \cap_{i \in J} \Vm(\a_i) = \emptyset$, a contradiction. 
\item For an open subset $U \subseteq \sSpec(R)$ there exists by definition an $\a \si R$ such that $U = \sSpec(R) \setminus \Vm(\a)$. We can then write $U$ as a union of basic open sets as follows: $$U = \bigcup_{a \in \a} \sD(a).$$
\item By (\ref{irred=prime}), an irreducible closed subset $A$ of $\sSpec(R)$ has the form $A = \Vm(\p)$, for $\p \si R$ prime. This prime $\s$-ideal $\p$ is the unique generic point of $A$.
To see this, consider the closure of $\p$: $$\overline{\{\p\}} = \bigcap_{\{\p\} \subseteq \Vm(F)}\Vm(F).$$ From the definition of $\Vm(F)$ it holds that $\{\p\} \subseteq \V_m(F)$ if and only if $F \subseteq \p$. By Lemma \ref{topologywelldef} (\ref{vmintersectionideals}), and the obvious fact that we can restrict the intersection to $\s$-ideals, we get thus
\[ \overline{\{\p\}} = \bigcap_{F \subseteq \p}\Vm(F) = \Vm(\sum_{F \subseteq \p, F \si R} F) = \Vm(\p) = A \]
\end{enumerate}
\end{bew}
\end{prop}

\clearpage

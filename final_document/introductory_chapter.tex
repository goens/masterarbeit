
\subsection{Introduction to Difference Algebra} 

Difference algebra is a small branch of mathematics, whose origin is closely related to that of differential algebra; a larger field with which it bears great similarity. It is also a realitively new field:
it could be argued that it was born as a branch of mathematics around the 1930s, through a series of articles published by J. Ritt between 1929 and 1939. However, it was not until the 1950s thanks to R. Cohn 
that difference algebra reached levels of development comparable to those of differential algebra.  Since then it has enjoyed a satisfactory growth thanks to a large number of mathematicians, and although it remains a small field today,
it still has many important results and a mature structure. A more detailed historical review can be found in the preface of \cite{levin}, where this one is based off.


To give a first idea of the object of study of difference algebra, we will start off with a few examples of difference equations. Probably one of the best known examples is the Fibonacci sequence, $1,1,2,3,5,8,13,\ldots$, which can be seen as a solution of the folowing recursive equation:
\begin{align*}
a_0 = 1,  a_1 = 1 \\ a_n = a_{n-1} + a_{n-2}, n\geq 2
\end{align*}

Another example that probably any mathematician of physicist will know is the functional equation of the Gamma function:

\begin{align*}
\Gamma(x+1) = x \Gamma(x)
\end{align*}

A classical result in complex analysis states that any function which satisfies this equation is a multiple of the $\Gamma$ function,
which is considered a generalization of the factorial:
\begin{align*}
\Gamma(x) = \int_0^\infty{\frac{t^x}{t} e^{-t} dt}
\end{align*}

These are two notable examples of difference equations, as we will soon see. In difference algebra however, we do not seek to find 'explicit' solutions of these equations,
 as are the numbers in the Fibonacci sequence, or the integral representation of the $\Gamma$ function. We will seek to study the structure of these equations and tackle problems like the existence of solutions in a more abstract sense.

\subsection{Basics of Difference Algebra}\label{fundamentos}
\begin{defn}
Let  $R$ be a ring (in this thesis all rings will be commutative and unital), and let
 $\sigma: R \rightarrow R$ be an endomorphism of rings in $R$. Then we call the tuple $(R,\sigma)$ a \emph{difference ring}, or $\sigma$\emph{-ring}. \index{$\s$-ring}
By abuse of notation we will say that $R$ is a $\sigma$-ring  to refer to the pair, and if $R'$ is a further $\sigma$-ring, we will also use the symbol $\sigma$ for the endomorphism on $R'$; This should not lead to confusion, since it can be infered from the context which endomorphism is meant. 
\end{defn}

\begin{defn}
Let $R, R'$ be  $\sigma$-rings and let $\varphi: R \rightarrow R'$ be a morphism of rings. We say that $\varphi$ is a \emph{morphism of $\sigma$-rings}  if \index{moprhism of $\sigma$-rings}
\begin{align*}
\sigma(\varphi(r)) = \varphi(\sigma(r)) \fa r \in R
\end{align*}
\end{defn}

\begin{ex} A few of the more notable examples of $\s$-rings are the following:

\begin{itemize}
\item Every ring $R$ is a $\sigma$-ring with $\sigma = \Id_R$. We call this a \emph{constant $\s$-ring}.  \index{constant $\s$-ring}
\item The field of meromorphic functions $\C \rightarrow \C$, which we will denote by $\mathcal{M}$,
is a $\sigma$-ring with $\sigma(f)(x) = f(x+1)$ for every $x \in \C$.
\item The sequences of integers, which we will denote by $\text{Seq}(\Z)$, build a $\sigma$-ring with the operation of shifting its terms to the left:
\begin{align*} \sigma: (a_n)_{n \in \N} \mapsto (a_{n+1})_{n \in \N}. \end{align*}
\end{itemize}
\end{ex}

\begin{defn}
Let $R$ be a $\sigma$-ring. If $R$ is also a field, we call the pair $(R,\sigma)$ a $\sigma$\emph{-field}. \index{$\s$-field} 
If $k$ is a $\sigma$-field, $A$ a $k$-algebra which (as a ring) is a $\sigma$-ring and it it holds that 
$\sigma(ra) = \sigma(r) \sigma(a), r \in k, a \in A$, then we call $A$ a  $k$-$\sigma$\emph{-algebra}. \index{$k$-$\sigma$-algebra}
\end{defn}

\begin{ex}
An example of special important is that of $\sigma$\emph{-polynomial-rings}. \index{$\sigma$-polynomial-rings}
Let $k$ be a field. We consider the polynomial ring in infinitely many variables $R:= k[y,\sigma(y),\sigma^2(y),\ldots]$,
 where $y,\sigma(y),\sigma^2(y),\ldots$ are, for the moment at least, simply the names of (algebraically independent) variables.
This ring we can turn into a $k$-$\sigma$-algebra by defining:
\begin{align*} 
\sigma:  R \rightarrow R, y \mapsto \sigma(y), \sigma^{n-1}(y) \mapsto \sigma^{n}(y) \fa n > 1 
\end{align*}
and extending this mapping $k$-linearly in the obvious way. We denote this $\sigma$-polynomial-ring by $k\{y\}$. In an analogous fashion we can define $\sigma$-polynomial-rings $k\{y_1, \ldots, y_n \}$ in many variables. 
\end{ex}

\begin{defn} $\phantom{}$
\begin{itemize}
\item Given a $\sigma$-ring $S$ and a subring $R \leq S$, we say that $R$ is a $\sigma$\emph{-subring} \index{$\s$-subring} of $S$ if $(R,\sigma|_{R})$ is a $\sigma$-ring,
i.e. , if the image of $\sigma|_{R}$ is contained in $R$.
\item A $\sigma$\emph{-ideal} \index{$\s$-ideal} is an ideal $I \unlhd R$ which is also a $\sigma$-subring of $R$; we denote this by $I \si R$. In this case, there exists a canonical $\sigma$-ring structure on the cocient ring $R/I$:
\begin{align*} \sigma: R/I \rightarrow R/I, a + I \mapsto \sigma(a) + I. \end{align*}
\end{itemize}
\end{defn}

\begin{defn}
Let $R$ be a $\s$-ring and let  $I \si R$ be a $\s$-ideal of $R$. For elements $a_1, \ldots, a_k \in R$ we denote by $[a_1, \ldots, a_k] \si R$ the $\s$-ideal minimal (with respect to inclusion) of $R$ which contains $a_1,\ldots,a_k$. 
In fact, it holds that \[[a_1,\ldots,a_k] = \{ \sum_{i=1}^n \s^{j_i}(x_i) \mid n \geq 1, j_i \geq 0, x_i \in \{a_1,\ldots,a_k\}, i=1,\ldots,n \} \]. If there exist $b_1,\ldots,b_r \in I$ such that $I = [b_1,\ldots,b_r]$,
 we say that $I$ is \emph{finitely $\s$-generated} as a $\s$-ideal \index{finitely $\s$-generated $\s$-ideal}
\end{defn}

\begin{defn}
A $k$-$\sigma$-algebra  $A$ is \emph{finitely $\sigma$-generada} if there exist elements $f_1, \ldots, f_n$ such that $$A = k[f_1,f_2,\ldots,f_n,\sigma(f_1),\ldots,\sigma(f_n),\sigma^2(f_1),\ldots].$$
\end{defn}

\begin{rem}\label{epipoli}
If $A$ is a $k$-$\sigma$-algebra , $\sigma$-generated by $f_1, \ldots, f_n$, then we have a canonical epimorphism of $k$-$\sigma$-algebras from the $\sigma$-polynomial-ring $k\{y_1, \ldots, y_n \}$ to $A$: $y_i \mapsto f_i, i = 1, \ldots, n$. Vemos entonces que los $\sigma$-anillos de polinomios son objetos libres en la categor\'ia de $k$-$\sigma$-\'algebras finitamente generadas. Por ello, denotamos a la $\sigma$-\'algebra  generada por $f_1, \ldots, f_n$ por $k\{f_1, \ldots, f_n\}$.
\end{rem}

\begin{defn}
If the kernel $I$ of the epimorphism mentioned on Remark \ref{epipoli} is finitely $\s$-generated as well, say, by $r_1, \ldots, r_m$, then we call the algebra $A$ \emph{finitamentely $\sigma$-presented}. \index{finitamentely $\sigma$-presented}
\end{defn}

\begin{rem}
In the conditions of the former definition, by the fundamental theorem on homomorphisms we have $A \cong k\{y_1, \ldots, y_n\}/(r_1,\ldots,r_m)$. Note as well that the concepts ``finitely $\sigma$-generated'' and ``finitamente $\sigma$-presented are truly different, contrary to the case of polynomial rings where we can argue with the Hilbert basis theorem, as shows the next example:
\end{rem}

\begin{ex}
Let $k$ be a $\sigma$-field and let $I \si k\{y\} $ the $\sigma$-ideal $\s$-generated by $y\s(y), y\s^2(y), y\s^3(y), \ldots$, i.e., $I = [y \s^i(y) \mid i\geq 1]$. Then the $\sigma$-ring $R := k\{y\}/I$ (where $\s (r + I) := \s(r) + I)$ is 
finitely $\sigma$-generated, $R = k\{ y + I \}$, but not finitely $\sigma$-presented, since $I$ is not finitely $\sigma$-generated.
\end{ex}


Just as the usual algebraic equations are reduced to finding solutions of a polynomial, difference equations can be expressed as the search for solutions of $\s$-polynomials. This can be understood as a homomorphism as in Remark \ref{epipoli}.

\begin{ex}
Now we can express both examples from the introduction in the language of difference algebra:
The Fibonacci sequence is a solution to the $\s$-polynomial $\sigma^2(y) + \sigma(y) - y$ in the $\sigma$-ring  $\text{Seq}(\Z)$; This is precisely the recurrence relation $a_{n+2} + a_{n+1} = a_n$.
In the same way, the $\Gamma$ function is the solution to the $\s$-polynomial $\sigma(y) - zy$, where $z \in \C(z)$ denotes the identity function $z \mapsto z$.
\end{ex}

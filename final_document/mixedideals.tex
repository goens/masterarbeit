


\documentclass[12pt,a4paper,BCOR15mm,twoside,DIV12]{article}
%\documentclass{article}
%\usepackage[paper=a4paper,left=20mm,right=20mm,top=25mm,bottom=25mm]{geometry}
\usepackage[english]{babel}
\usepackage[utf8]{inputenc}
\usepackage{amsmath}
\usepackage{color}
\usepackage{enumerate}
\usepackage[titletoc]{appendix}
\usepackage{amssymb}
\usepackage{amsfonts}
\usepackage{amsthm}
\usepackage{hyperref}
\usepackage{makeidx}
\usepackage{graphicx, float,epsfig}
\usepackage[nottoc,numbib]{tocbibind}
\usepackage{listings}

\newcommand{\properideal}{%
  \mathrel{\ooalign{$\lneq$\cr\raise.22ex\hbox{$\lhd$}\cr}}}

\def\P{\mathcal{P}}
\def\I{\mathbb{I}}
\def\R{\mathbb{R}} 
\def\E{\mathcal{E}} 
\def\NE{\mathbb{N}_{\geq1}} 
\def\N{\mathbb{N}} 
\def\Z{\mathbb{Z}} 
\def\Q{\mathbb{Q}} 
\def\F{\mathbb{F}}
\def\Vm{\mathcal{V}_m}
\def\V{\mathcal{V}}
\def\VV{\mathbb{V}}
\def\C{\mathbb{C}}
\def\U{\mathcal{U}}
\def\a{\mathfrak{a}}
\def\b{\mathfrak{b}}
\def\p{\mathfrak{p}}
\def\q{\mathfrak{q}}
\def\s{\sigma}
\def\si{\unlhd_{\sigma}}
\def\sD{\s\text{-}\operatorname{D}}
\def\GL{\text{GL}}
\def\supp{\text{Supp}}
\def\id{\text{id}}
\def\n{\underline{n}}
\def\Spec{\operatorname{Spec}}
\def\sSpec{\sigma\operatorname{-Spec}}
\def\diag{\text{diag}}
\def\End{\text{End}}
\def\Hom{\text{Hom}}
\def\fa{\text{ for all }}
\def\Tr{\text{Tr}}
\def\Id{\text{Id}}
\def\Sym{\text{Sym}}
\def\H{\mathcal{H}}
\def\wt{\text{wt}}
\def\Perf{\text{Perf}}
\def\sdim{\sigma\operatorname{-dim}}
\def\trdeg{\operatorname{trdeg}}

\renewcommand{\labelenumi}{\alph{enumi})}
%\renewcommand{\P}{\textfrak{P}}
\newcommand{\cupdot}{\mathop{\mathaccent\cdot\cup}}
\newcommand{\textsim}{\mathord{\sim}}
\newcommand{\catname}[1]{{\normalfont\textbf{#1}}}
\newcommand{\Set}{\catname{Set}}
\newcommand{\Top}{\catname{Top}}
\newcommand{\sintk}{\s\text{\catname{-int}}_k}
\newcommand{\sringk}{\s\text{\catname{-ring}}_k}
\newenvironment{bew}{\begin{proof}[Proof]}{\end{proof}}
\theoremstyle{plain}
\newtheorem{Satz}{Satz}[section]
\newtheorem{theorem}[Satz]{Theorem}
\newtheorem{ex}[Satz]{Example}
\newtheorem{cor}[Satz]{Corollary}
\newtheorem{algorithm}[Satz]{Algorithm}
\newtheorem{prop}[Satz]{Proposition}
\newtheorem{conj}[Satz]{Conjecture}
\newtheorem{lem}[Satz]{Lemma}
\newtheorem{defn}[Satz]{Definition}
\theoremstyle{definition}
\newtheorem{rem}[Satz]{Remark}

%%begin restatement stuff
\makeatletter
\newtheorem*{rep@theorem}{\rep@title}
\newcommand{\newreptheorem}[2]{%
\newenvironment{rep#1}[1]{%
 \def\rep@title{#2 \ref{##1}}%
 \begin{rep@theorem}}%
 {\end{rep@theorem}}}
\makeatother


\newreptheorem{theorem}{Theorem}
%%end restatement stuff

\usepackage[arrow, matrix, curve]{xy}

\makeindex
\title{Geometric Aspects of Mixed Difference Ideals}
\author{Andr\'{e}s Goens}
\date{\today}
\begin{document}
\setlength{\parindent}{1.5em}

\maketitle

\clearpage
\section*{Introduction}
Over a field $k$, the polynomial $x^n$ has only a single solution, namely $0$. 
There is a difference, however, between $x$ and $x^2$ for example, which we cannot establish by simply looking at the solutions of these polynomials over a field.
In particular, the solution $0$ has a different 'multiplicity' for different $n$.  A way we can distinguish $x$ and $x^2$ is by looking for solutions elsewhere, not only in fields.
Over a general commutative ring, for example, the sets of solutions of the equations $x = 0$ and $x^2 = 0$ are different. \\


On a related note, the ideals generated by the two polynomials, $(x)$ and $(x^2) \unlhd k[x]$ are different. However, the radicals of these ideals are equal: 
$$ \sqrt{ (x) } = (x) = \sqrt{(x^2)}, $$ which means we cannot distinguish the polynomials like that either.
Instead of considering the coordinate ring of the varieties, we could capture this 'multiplicity', for example, by studying the ring $k[x]/(x^n)$.
The $k$-dimension of this ring is exactly $n$, $\operatorname{dim}_k( k[x]/(x^n)) = n$, what we would like to define as the multiplicity. \\

A similar problem arises in difference algebra, where the fields and polynomial rings also have a \emph{difference operator} $\s$.
A difference operator is basically a ring(field) endomorphism, which in the polynomial ring can also be applied to the variables.
Here, instead of the degree of the polynomial, we are interested in the \emph{order} of the difference equation.
In analogy to the equation $x^n = 0$, we can consider the difference equation $\s^n(x) = 0$. This is an equation of order $n$.
However, in a difference field $k$ there is only one solution to $\s^n(x) = 0$: since $\s$ is an endomorphism of the field $k$,
it has to be injective. Thus, $\s^n(x) = 0$ implies that $x = 0$. \\

We would like to study difference equations in a way we can distinguish the difference equations $\s(x) = 0$ and $\s^2(x) = 0$. Just as in the analogy for the polynomials $x$ and $x^2$,
there seems to be a sort of 'multiplicity' of the solution $0$ in the difference equation $\s^2(x) = 0$. We would like to have a way of finding this 'multiplicity' of the solutions too.

As suggested by the analogy above, we achieve this by changing the type of structure where we look for solutions of the difference equations. Instead of looking for solutions in difference fields,
we consider integral domains with an endomorphism. This endomorphism does not have to be injective anymore, and thus, the two difference equations can have different sets of solutions.
In difference polynomial rings we are interested in so-called \emph{difference ideals}, which are ideals that are invariant under $\s$. 
As in algebraic geometry, to a set of difference equations $F$ we will define a sort of difference algebraic sets $\VV_m(F)$ that take this additional solutions into account.
This new approach will be seen to be closely related to a class of difference ideals called \emph{mixed difference ideals}.
We also achieve a way of quantifying the 'multiplicity' of the solutions to the difference equations. The following theorem does this, and it is one of the main results of this thesis

\begin{theorem*}
Let $k$ be a $\s$-field and $f \in k\{y_1,\ldots,y_n\}, f \notin k$ an irreducible $\s$-polynomial. 
Then $\VV_m(f)$ has an irreducible closed subset $X$ such that $\s$-$\dim(X) = n-1$ and $\s$-$\operatorname{deg}(X) = \operatorname{Ord}(f)$.
\end{theorem*}

The concepts of $\s$-$\operatorname{deg}$ and $\s$-$\dim$ are of course also appropriately defined as part of the thesis. \\

To prove the theorem above, we consider so-called \emph{difference kernels}.
In a difference polynomial ring, even if we only have one '\emph{difference variable}' we have an infinite number of algebraically-independent variables when considered as a polynomial ring. 
These variables come from all the powers of $\s$ applied to the difference variables. Since such a ring is not Noetherian, there is a lot of difficulty that arises when attempting to study it. 
The idea behind difference kernels is the following.
In order to study a difference ideal $\a$ in a difference polynomial ring $k[y,\s(y),\s^2(y),\ldots]$,
we study the intersection of the ideal with a finite number of variables, and call it a \emph{difference kernel} $$\a[d] := \a \cap k[y,\s(y),\ldots,\s^d(y)].$$
We can then study the properties $\a[d]$ has depending on the properties of the difference ideal $\a$. At least as interesting is the question how it works the other way around.
What properties of an ideal $I \unlhd k[y,\s(y),\ldots,\s^d(y)]$ ensure that there is a \emph{realization} of this ideal: a difference ideal $\a$ in the difference polynomial ring such that $I = \a[d]$?
 And how do the properties of $I$ affect those of $\a$?
For a class of difference ideals called \emph{$\s$-prime difference ideals} the answers to the former questions are known. 
There is a characterization of the ideals $I \unlhd k[y,\s(y),\ldots,\s^d(y)]$ where there exists a realization as a $\s$-prime difference ideal. 
For the concepts developed in this thesis, however, a larger class of difference ideals is interesting. We are interested in prime difference ideals. 

Unfortunately, a full characterization as the one just described was not found. However, some results in the direction of one were achieved. A main result of this thesis is a characterization of the difference kernels that 
can be extended from $k[y,\ldots,\s^d(y)]$ to $k[y,\ldots,\s^{d+1}(y)]$. Also, a characterization of the ideals of $k[y,\ldots,\s^d(y)]$
which have a realization as a prime difference ideal is conjectured. \\


This thesis is divided into four sections. Section 1 will introduce the subject of difference algebra, presenting the most basic definitions and some of the more known results concerning the standard theory of perfect difference ideals and their geometric aspects.
The rest of the thesis will then deal with the results presented in Section 1 and, to the extent possible, generalize them to the case of mixed difference ideals. 
Section 2 begins with a more in-depth study of difference ideals, with a particular emphasis on mixed difference ideals. It also touches on some geometrical subjects, defining a topological space similar to the Zariski topology on a special class of ideals of a difference ring.
Section 3 continues with a more geometric nature, presenting a definition of mixed difference varieties as well as some basic properties of these. 
Finally, Section 4 deals with difference kernels. The two main results and the conjecture stated above are presented here.

\clearpage
\section*{Acknowledgments}
I would like to thank Michael Wibmer, for his patience, ideas and support throughout the work on this thesis. Writing this thesis would certainly not have been possible if it was not for his mentoring.
In the time spent working on this thesis, I did not only learn about difference algebra, but also how to do mathematics in general, and I have Michael Wibmer to thank for this.
I would also like to thank Julia Harmann for her patience and help in writing this thesis.
Finally, I would like to thank Daniel Rettstadt, Annette Maier and Stefan Ernst for always offering advice the numerous times I had smaller and not-so-small problems and/or questions.

\clearpage

\tableofcontents

\clearpage 


\section{Difference Algebra}

\subsection{Introduction to Difference Algebra} 

Difference algebra is a small branch of mathematics, whose origin is closely related to that of differential algebra; a larger field with which it bears great similarity. It is also a realitively new field:
it could be argued that it was born as a branch of mathematics around the 1930s through a series of articles published by J. Ritt between 1929 and 1939. However, it was not until the 1950s thanks to R. Cohn 
that difference algebra reached levels of development comparable to those of differential algebra.  Since then it has enjoyed a satisfactory growth thanks to a large number of mathematicians, and although it remains a small field today,
it still has many important results and a mature structure. A more detailed historical review can be found in the preface of \cite{levin}, where this one is based off.


To give a first idea of the object of study of difference algebra, we will start off with a few examples of difference equations. Probably one of the best known examples is the Fibonacci sequence: $1,1,2,3,5,8,13,\ldots$, which can be seen as a solution of the folowing recursive equation:
\begin{align*}
a_0 = 1,  a_1 = 1 \\ a_n = a_{n-1} + a_{n-2}, n\geq 2
\end{align*}

Another example that probably any mathematician of physicist will know is the functional equation of the Gamma function:

\begin{align*}
\Gamma(x+1) = x \Gamma(x)
\end{align*}

A classical result in complex analysis states that any function which satisfies this equation is a multiple of the $\Gamma$ function,
which is considered a generalization of the factorial:
\begin{align*}
\Gamma(x) = \int_0^\infty{\frac{t^x}{t} e^{-t} dt}
\end{align*}

These are two notable examples of difference equations, as we will soon see. In difference algebra however, we do not seek to find 'explicit' solutions of these equations,
 as are the numbers in the Fibonacci sequence, or the integral representation of the $\Gamma$ function. We will seek to study the structure of these equations and tackle problems like the existence of solutions in a more abstract sense.

\subsection{Basics of Difference Algebra}\label{fundamentos}
\begin{defn}
Let  $R$ be a ring (in this thesis all rings will be commutative and unital), and let
 $\sigma: R \rightarrow R$ be an endomorphism of rings in $R$. Then we call the tuple $(R,\sigma)$ a \emph{difference ring}, or $\sigma$\emph{-ring}. \index{$\s$-ring}
By abuse of notation we will say that $R$ is a $\sigma$-ring  to refer to the pair, and if $R'$ is a further $\sigma$-ring, we will also use the symbol $\sigma$ for the endomorphism on $R'$; This should not lead to confusion, since it can be infered from the context which endomorphism is meant. 
\end{defn}

\begin{defn}
Let $R, R'$ be  $\sigma$-rings and let $\varphi: R \rightarrow R'$ be a morphism of rings. We say that $\varphi$ is a \emph{morphism of $\sigma$-rings}  if \index{moprhism of $\sigma$-rings}
\begin{align*}
\sigma(\varphi(r)) = \varphi(\sigma(r)) \fa r \in R
\end{align*}
\end{defn}

\begin{ex} A few of the more notable examples of $\s$-rings are the following:

\begin{itemize}
\item Every ring $R$ is a $\sigma$-ring with $\sigma = \Id_R$. We call this a \emph{constant $\s$-ring}.  \index{constant $\s$-ring}
\item The field of meromorphic functions $\C \rightarrow \C$, which we will denote by $\mathcal{M}$,
is a $\sigma$-ring with $\sigma(f)(x) = f(x+1)$ for every $x \in \C$.
\item The sequences of integers, which we will denote by $\text{Seq}(\Z)$, build a $\sigma$-ring with the operation of shifting its terms to the left:
\begin{align*} \sigma: (a_n)_{n \in \N} \mapsto (a_{n+1})_{n \in \N}. \end{align*}
\end{itemize}
\end{ex}

\begin{defn}
Let $R$ be a $\sigma$-ring. If $R$ is also a field, we call the pair $(R,\sigma)$ a $\sigma$\emph{-field}. \index{$\s$-field} 
If $k$ is a $\sigma$-field, $A$ a $k$-algebra which (as a ring) is a $\sigma$-ring and it it holds that 
$\sigma(ra) = \sigma(r) \sigma(a), r \in k, a \in A$, then we call $A$ a  $k$-$\sigma$\emph{-algebra}. \index{$k$-$\sigma$-algebra}
\end{defn}

\begin{ex}
An example of special importance is that of $\sigma$\emph{-polynomial-rings}. \index{$\sigma$-polynomial-rings}
Let $k$ be a field. We consider the polynomial ring in infinitely many variables $R:= k[y,\sigma(y),\sigma^2(y),\ldots]$,
 where $y,\sigma(y),\sigma^2(y),\ldots$ are, for the moment at least, simply the names of (algebraically independent) variables.
This ring we can turn into a $k$-$\sigma$-algebra by defining:
\begin{align*} 
\sigma:  R \rightarrow R, y \mapsto \sigma(y), \sigma^{n-1}(y) \mapsto \sigma^{n}(y) \fa n > 1 
\end{align*}
and extending this mapping $k$-linearly in the obvious way. We denote this $\sigma$-polynomial-ring by $k\{y\}$. In an analogous fashion we can define $\sigma$-polynomial-rings $k\{y_1, \ldots, y_n \}$ in many variables. 
\end{ex}

\begin{defn} $\phantom{}$
\begin{itemize}
\item Given a $\sigma$-ring $S$ and a subring $R \leq S$, we say that $R$ is a $\sigma$\emph{-subring} \index{$\s$-subring} of $S$ if $(R,\sigma|_{R})$ is a $\sigma$-ring,
i.e. , if the image of $\sigma|_{R}$ is contained in $R$.
\item A $\sigma$\emph{-ideal} \index{$\s$-ideal} is an ideal $I \unlhd R$ which is also a $\sigma$-subring of $R$; we denote this by $I \si R$. In this case, there exists a canonical $\sigma$-ring structure on the cocient ring $R/I$:
\begin{align*} \sigma: R/I \rightarrow R/I, a + I \mapsto \sigma(a) + I. \end{align*}
\end{itemize}
\end{defn}

\begin{defn}
Let $R$ be a $\s$-ring and let  $I \si R$ be a $\s$-ideal of $R$. For elements $a_1, \ldots, a_k \in R$ we denote by $[a_1, \ldots, a_k] \si R$ the $\s$-ideal minimal (with respect to inclusion) of $R$ which contains $a_1,\ldots,a_k$. 
In fact, it holds that \[[a_1,\ldots,a_k] = \{ \sum_{i=1}^n \s^{j_i}(x_i) \mid n \geq 1, j_i \geq 0, x_i \in \{a_1,\ldots,a_k\}, i=1,\ldots,n \}. \] If there exist $b_1,\ldots,b_r \in I$ such that $I = [b_1,\ldots,b_r]$,
 we say that $I$ is \emph{finitely $\s$-generated} as a $\s$-ideal \index{finitely $\s$-generated $\s$-ideal}
\end{defn}

\begin{defn}
A $k$-$\sigma$-algebra  $A$ is \emph{finitely $\sigma$-generada} if there exist elements $f_1, \ldots, f_n$ such that $$A = k[f_1,f_2,\ldots,f_n,\sigma(f_1),\ldots,\sigma(f_n),\sigma^2(f_1),\ldots].$$
\end{defn}

\begin{rem}\label{epipoli}
If $A$ is a $k$-$\sigma$-algebra, $\sigma$-generated by $f_1, \ldots, f_n$, then we have a canonical epimorphism of $k$-$\sigma$-algebras from the $\sigma$-polynomial-ring $k\{y_1, \ldots, y_n \}$ to $A$: $y_i \mapsto f_i, i = 1, \ldots, n$. We thus see that $\sigma$-polynomial-rings are free objects in the category of finitely generated $k$-$\sigma$-algebras. We write  $k\{f_1, \ldots, f_n\}$ to denote the  $\sigma$-algebra generated by $f_1, \ldots, f_n$.
\end{rem}

\begin{defn}
If the kernel $I$ of the epimorphism mentioned on Remark \ref{epipoli} is finitely $\s$-generated as well, say, by $r_1, \ldots, r_m$, then we call the algebra $A$ \emph{finitely $\sigma$-presented}. \index{finitamentely $\sigma$-presented}
\end{defn}

\begin{rem}
In the conditions of the former definition, by the fundamental theorem on homomorphisms we have $A \cong k\{y_1, \ldots, y_n\}/[r_1,\ldots,r_m]$. Note as well that the concepts ``finitely $\sigma$-generated'' and ``finitely $\sigma$-presented'' are truly different, contrary to the case of polynomial rings where we can argue with the Hilbert basis theorem, as shows the next example:
\end{rem}

\begin{ex}
Let $k$ be a $\sigma$-field and let $I \si k\{y\} $ be the $\sigma$-ideal $\s$-generated by $y\s(y), y\s^2(y), y\s^3(y), \ldots$, i.e., $I = [y \s^i(y) \mid i\geq 1]$. Then the $\sigma$-ring $R := k\{y\}/I$ (where $\s (r + I) := \s(r) + I)$ is 
finitely $\sigma$-generated, $R = k\{ y + I \}$, but not finitely $\sigma$-presented, since $I$ is not finitely $\sigma$-generated.
\end{ex}


Just as the usual approach with algebraic equations is to consider them as solutions of a polynomial, difference equations can be expressed as the search for solutions of $\s$-polynomials. To see what is meant with this, we will return to the examples of the introduction,
 as we now have the necessary concepts to formulate these in the language of difference algebra. 

\begin{ex}
The Fibonacci sequence is a solution to the $\s$-polynomial $\sigma^2(y) + \sigma(y) - y$ in the $\sigma$-ring  $\text{Seq}(\Z)$; This is precisely the recurrence relation $a_{n+2} + a_{n+1} = a_n$.
In the same way, the $\Gamma$ function is the solution to the $\s$-polynomial $\sigma(y) - zy$, where $z \in \C(z)$ denotes the identity function $z \mapsto z$.
\end{ex}

\clearpage 
\section{Difference Ideals}

In this section we study difference ideals more closely. A special focus will be put on mixed and prime difference ideals, which in many ways mimic the properties of and relationship between radical and prime ideals on commutative rings.
Much of the following is based on Section 1.2 of the lecture notes of M. Wibmer \cite{wibmer}, where most of this has been worked out for the analogous case of perfect and $\s$-prime difference ideals. \\

\subsection{Difference Ideals}

We recall Definition \ref{idealprops}, and consider an additional type of difference ideal here.

\begin{defn}
Let  $\a \si R$ be a $\s$-ideal of $R$. 
\begin{itemize}
\item Then $\a$ is called a \emph{mixed $\s$-ideal} if for any $f,g \in R$ with $fg \in \a$ it follows that $f \sigma(g) \in \a$.  \index{mixed $\s$-ideal}
\item The ring $R$ is called \emph{well-mixed}, if the zero ideal $[0]$ is mixed. \index{well-mixed $\s$-ring}
\end{itemize}
\end{defn}

\begin{rem}\label{rempropideals}
It is easy to see from the definitions that $\s$-prime $\s$-ideals are perfect, and that perfect $\s$-ideals are mixed, radical and reflexive. Prime $\s$-ideals are also mixed, but not necessarily perfect. Note that there is a difference between a prime $\s$-ideal and a $\s$-prime $\s$-ideal:
the former does not necessarily have to be reflexive, while the latter does. Both, prime $\s$-ideals and $\s$-prime ideals will be very important throughout this thesis. It is for this reason that the distinction between both is of utmost importance.
\end{rem}

The above properties behave well with respect to morphisms of $\s$-rings in the sense of the following lemma (see Exercise 1.2.7 of \cite{wibmer}):
\begin{lem}\label{bijmapping}
Let $\varphi: R \rightarrow S$ be a morphism of $\s$-rings and $\a \si S$ a $\s$-ideal of $S$. Then $\varphi^{-1}(\a) \si R$ is a $\s$-ideal of $R$. Similarly, if $\a$ is a mixed $\s$-ideal, then so is $\varphi^{-1}(\a)$. The same is true for perfect, prime and for reflexive $\s$-ideals.
\begin{bew}
Since $\a \unlhd S$ is an ideal, so is $\b := \varphi^{-1}(\a) \unlhd R$. Let $b \in \b$. Then $\varphi(b) =: a \in \a$ by definition. Since $\a \si S$ is a $\s$-ideal, $\s(a) \in \a$, and since $\varphi$ is a morphism of $\s$-rings
it follows that $\sigma(a) = \sigma(\varphi(b)) = \varphi (\s (b)) \in \a$. Hence, $\s(b) \in \b$ which implies that $\b$ is a $\s$-ideal. \\
\indent Now let $\a$ be mixed and $fg \in \b$. This means by definition of $\b$, 
that $\varphi(fg) = \varphi(f) \varphi(g) \in \a$. Since $\a$ is mixed, this in turn implies that $$\varphi(f) \s( \varphi(g)) = \varphi(f) \varphi(\s(g)) = \varphi(f\s(g)) \in \a,$$ which yields $f\s(g) \in \b$, so that $\b$ is also mixed. 
The proof for perfect and for reflexive difference ideals is analogous.
\end{bew}
\end{lem}

\begin{rem}
Let $R$ be a $\s$-ring and $\a \si R$ a $\s$-ideal. We can define a canonical $\s$-ring structure on the quotient ring $R/\a$, via $\s(r+\a):=~\s(r)~+~\a$. 
This is well defined and in particular makes the canonical epimorphism $\tau:~R~\twoheadrightarrow~R/\a$ a morphism of $\s$-rings.
\end{rem}

\begin{prop}\label{bijideals}
Let $R$ be a $\s$-ring and $\a \si R$ a $\s$-ideal. The canonical epimorphism $\tau: R \twoheadrightarrow R/\a$ induces, in the sense of Lemma \ref{bijmapping}, a bijection between the sets $\{ \b \si R/\a \}$ and $\{ \b \mid \a \si \b \si R \}$. The same holds true, if we restrict both sets to prime, radical and mixed, $\s$-prime or perfect $\s$-ideals.
\begin{bew}
See Proposition 1.2.8 of \cite{wibmer}.
\end{bew}
\end{prop}

\begin{rem}\label{wmwelldef}
Let $R$ be a $\s$-ring, and $F \subseteq R$ be a subset of $R$. Any intersection of mixed, radical $\s$-ideals containing $F$ is also a mixed, radical $\s$-ideal, which of course contains $F$. 
This means that there is a smallest (with respect to inclusion) mixed, radical $\s$-ideal $\a$ containing $F$, namely, the intersection of all such $\s$-ideals:
\begin{align*} \a = \bigcap_{\substack{ \b \si R, \\ \b \text{ radical and mixed}}} \b. \end{align*}
\begin{proof}
Let $I$ be an index set and $\b_i \si R \fa i \in I$ be mixed, radical $\s$-ideals. Further, let $\a := \bigcap_{i \in I} \b_i$ be the intersection of these. Obviously, $\a$ is an ideal of $R$. We will show that it is also a $\s$-ideal, radical and mixed.
If $b \in \b_i \fa i \in I$, then $\s(b) \in \b_i \fa i \in I$, since each $\b_i$ is a $\s$-ideal.
It follows that $\s(b) \in \a$. \\
\indent Similarly, if $bb' \in \b_i \fa i \in I$, then $b \s(b') \in \b_i \fa i \in I$, since each $\b_i$ is mixed, which implies that $b \s(b') \in \a$. Hence, $\a$ is mixed.  \\
\indent Finally, if $a \in \sqrt(\a)$ there exists an $n \in \N$, such that $a^n \in \a$. This means that $a^n \in \b_i \fa i \in I$, which implies that $a \in \sqrt(\b_i) \fa i \in I$. Since every $\b_i, i \in I$ is radical, this means that $a \in \b_i \fa i \in I$,
and thus $a \in \a$. This implies that $\a$ is radical.
\end{proof}
\end{rem}

\begin{defn}
The $\s$-ideal $\a$ from Remark \ref{wmwelldef} is called the radical, mixed closure of $F$. We will denote it by $\{F\}_{m}$.
\end{defn}


\begin{lem}\label{sqrtmixed}
Let $R$ be a $\s$-ring and $\a \si R$ be a mixed $\s$-ideal. Then the radical of $\a$, $\sqrt{\a}$, is also mixed.
\begin{bew}
Let $f,g \in R$ be such that $fg \in \sqrt \a$. By definition there exists an $n \in \NE$ such that $f^n g^n = (fg)^n \in \a$. Since $\a$ is mixed, this implies that $f^n \s(g^n) = f^n \s(g)^n = (f\s(g))^n \in \a$. 
However, this in turn implies that $f\s(g)~\in~\sqrt \a$, which is what we wanted to show.
\end{bew}
\end{lem}


\begin{ex}\label{nombasisex}
Let $\Q$ be the field of rational numbers considered as a constant $\s$-field and let $R:= \Q\{y_1,y_2\}$. Consider the difference ideal $\a:= [y_1y_2] \si R$. We can inductively define a chain \begin{align*}\a^{\{0\}}:= \a,~ \a^{\{m+1\}}:= [\{ f \s(g) \mid f,g \in \a^{\{m\}}\}] \\ = [\s^k(y_1)\s^l(y_2) \mid k,l = 0,1,\ldots,m+1], \fa m \in \NE.\end{align*}
This is an infinite properly ascending chain of difference ideals. 

\begin{bew}
It is enough to only consider the $\s$-generators of the ideal. We can show the assertion by an induction on $m$. For $m=0$ it is obvious. \\
Assume then that $$\a^{\{m\}} = [\s^k(y_1)\s^l(y_2) \mid k,l = 0,1,\ldots,m].$$
For $k = 0,\ldots,m$, we know that $\s^k(y_1)\s^m(y_2) \in \a^{\{m\}}$. This implies that $\s^k(y_1)\s^{m+1}(y_2) \in \a^{\{m+1\}}$.
Similarly, $\s^m(y_1)\s^l(y_2) = \s^l(y_2)\s^m(y_1) \in \a^{\{m\}}$ implies that $ \s^l(y_2)\s^{m+1}(y_1) = \s^{m+1}(y_1)\s^l(y_2) \in \a^{\{m\}}$, for $l = 0,\ldots,m$.
Since these are the only new generators we get through this process, we also get the other inclusion.
\end{bew}
%% \begin{bew}
%% We will show that $\a^{\{m\}} = [\s^k(y_1)\s^l(y_2) \mid k,l = 0,1,\ldots,m]$ by induction. The assertion that the $\a^{\{m\}}$ form an infinite chain of properly ascending $\s$-ideals is obvious from this.\\
%% \indent The inclusion $ [\s^k(y_1)\s^l(y_2) \mid k,l = 0,1,\ldots,m] \subseteq \a^{\{m\}}$ follows from an induction on $m$ directly from the definition of $\a^{\{m\}}$. \\ 
%% \indent  To show the other inclusion, ``$\supseteq$'', we do it also inductively. For $m=0$ it holds that $\a^{\{0\}} = \a = [y_1y_2]$. Let thus $m \in \N$ and $a \in \a^{\{m+1\}}$.
%% By definition, there exist $k \in \NE, f_i, g_i, r_i \in R$ with $f_ig_i \in \a^{\{m\}}, i = 1,\ldots,k$ such that $a = \sum_{i=1}^k r_i f_i \s(g_i)$. 

%% FIXME: FINISH THIS!!!!!!!

%%  To prove that the chain is properly ascending, it is enough to show that $y_1\s^m(y_2) \in \a^{\{m\}}$, but $y_1\s^{m+1}(y_2) \notin \a^{\{m\}}$, since $\a^{\{m\}} \subseteq \a^{\{m+1\}}$ is obvious by definition.
%% This can be shown by induction: For $m = 0$, it is clear that $y_1y_2 \in [y_1y_2]$, but $y_1 \s(y_2) \notin [y_1y_2]$. Now, if $y_1 \s^m(y_2) \in \a^{\{m\}}$, then by definition, $y_1 \s^{m+1}(y_2) \in \a^{\{m+1\}}$.
%% On the other hand, if $\y_1 \s^{m+1}(y_2) \notin \a^{\{m\}}$, then it is clear that $\y_1 \s^{m+2}(y_2) \notin \{ f\s(g) \mid f,g \in \a^{\{m\}}\} =: M$. We only need to show that $y_1\s^{m+2}(y_2) \notin [M]$.

%% \end{bew}
\end{ex}


When trying to find $\{\a\}_m$ for a $\s$-ideal $\a \si R$ in a difference ring $R$, it might be tempting to consider $\a':= \{ f\s(g) \mid fg \in \a \}$, or to ensure it is a difference ideal rather, $[\a']$. The example above shows that this is not enough,
as the ideal $[\a']$ does not have to be mixed in general. However, by iteratively repeating this process and taking the union of $\s$-ideals obtained this way, we do get a mixed $\s$-ideal. We will see this in the following lemma, 
which is very similar to Lemma 3.1 of \cite{levinmixed}, where the mixed closure is obtained this way.
\begin{lem}\label{lemshuffling}
Let $R$ be a $\s$-ring and $F \subseteq R$. Further, let \\ $F'~:=~\{f\s(g)~\mid~fg~\in~F~\}$, and set $F^{\{1\}}:= [F']$, $F^{\{n\}}:= [{F^{\{n-1\}}}']$ for all $n \in \NE$. Then
\begin{align} \{F\}_m = \sqrt{\bigcup_{n=1}^{\infty} F^{\{n\}}}. \end{align}
This way of obtaining $\{F\}_m$ is called a shuffling processes and has an analogy for perfect $\s$-ideals (see for example \cite{levin}, p. 121f.) \index{Shuffling process}
\begin{proof}
Let $\a:= \bigcup_{n=1}^{\infty} F^{\{n\}}$. It is obvious from the construction that $F \subseteq \a$. It also holds that $\a$ is a mixed $\s$-ideal, since for any $f,g \in \a$ there exists an $n \in \NE$ such that $f,g \in F^{\{n\}}$.
Hence, $f + g, \s(f) \in F^{\{n\}} \subseteq \a$, as well as $fh \in F^{\{n\}} \subseteq \a$ for any $h \in R$. Furthermore, for $f, g \in R$ with $fg \in \a$ there also exists an $n \in \NE$ such that $fg \in F^{\{n\}}$. 
Then we have $f\s(g) \in F^{\{n+1\}} \subseteq \a$. \\
\indent On the other hand, by induction on the iterative steps $F^{\{n\}}$, it follows that for every mixed $\s$-ideal $\b$, which contains $F$, $F^{\{n\}} \subseteq \b$. Hence, $\a$ is the smallest mixed $\s$-ideal containing $F$. \\
\indent By Lemma \ref{sqrtmixed} we know that $\sqrt{\a}$ is mixed. This actually shows that $\sqrt a$ is indeed the smallest mixed, radical $\s$-ideal of $R$ containing $F$, since every such ideal has to contain $\a$, and thus $\sqrt{\a}$ as well.

% To see this, assume there exists a radical, mixed $\s$-ideal $\b \supseteq F$
%such that $\b \subsetneqq \sqrt{\a}$. Then we have $F \subseteq \a \cap \b \subsetneqq \a$, a contradiction to the minimality of $\a$ (since the intersection of mixed $\s$-ideals is mixed). 
\end{proof}
\end{lem}


\begin{ex} %%not *radical*! 
Let $k$ be a $\s$-field, and consider $R = k\{y_1\}$. Then the $\s$-ideal $[y_1] \si R$ is mixed, hence equal to its mixed closure (similar to the radical, mixed closure, the smallest mixed $\s$-ideal containing $y_1$).
The mixed closure of $[y_1] \cdot [y_1]$ is $[ y_1 \s^i(y_1) \mid i \in \N ] \not \ni y_1$.
One could have expected to get an analogy of the statement in algebraic geometry that $\sqrt{F_1  F_2 } = \sqrt{F_1} \cap \sqrt{F_2}$. However, this example shows this is not the case for mixed ideals in general.
It is however very noteworthy that the ideal $[ y_1 \s^i(y_1) \mid i \in \N ] \si R$ is not radical. For radical, mixed difference ideals we will get such a statement later (Corollary \ref{prod=cap}).
\end{ex}

A very important result in commutative algebra is the fact that every radical ideal is the intersection of prime ideals. This has an analog for perfect $\s$-ideals (see Theorem \ref{intersectionperfect}), as well as for mixed $\s$-ideals. 
We will prove the latter, but for this we need a few additional tools. We will first prove a weaker version of the statement, for which we need a few results from commutative algebra:

\begin{lem}\label{commalg}
Let $R$ be a ring. 
\begin{enumerate}[(a)]
\item If $S \geq R$ is an overring of $R$, and $\p$ is a minimal prime ideal of $R$, then there exists a minimal prime ideal $\q$ of $S$, such that $\p = \q \cap R$.
\item Every radical ideal of $R$ is the intersection of prime ideals. If $R$ is Noetherian, then every radical ideal of $R$ is the intersection of finitely many prime ideals.
\item If $R$ is Noetherian and $\p \unlhd R$ is a minimal prime ideal of $R$, then there exists an element $a \in R$, such that $\p$ is the annihilator ideal of $a$, i.e. $\p = \text{Ann}(a) = \{ r \in R \mid ra = 0 \}$.
\end{enumerate}
\begin{bew} $~$
\begin{enumerate}[(a)]
\item See \cite{bourbaki} Ch. 2, \S 2.6, Proposition 16.
\item See \cite{bourbaki} Ch. 2, \S 2.6, Corollary 2 to Proposition 13 and Ch. 2, \S 4.3, Corollary 3 to Proposition 14.
\item This is a special case of Theorem 3.1 of \cite{eisenbud} for $R$ as an $R$-module.
\end{enumerate}
\end{bew}
\end{lem}


The following is adapted from the proof given in \cite{wibmer} of Proposition 1.2.28 (here the upcoming Theorem \ref{intersectionprimes}). It has been worked out in more detail and divided further in an attempt to make this more detailed version of the proof easier to read.

\begin{defn}
Let $R$ be a difference ring. We say $R$ is \emph{finitely $\s$-generated over $\Z$}, if there exists a finite set $A \subseteq R$, so that every $f \in R$ can be written as a finite $\Z$-linear combination of $\s$-powers of elements in $A$. In other words,
for every $f \in R$ there exists an $n \in \NE$, so that $f \in \Z[A,\sigma(A),\ldots,\s^n(A)]$. \\ 
\indent For any subset $F \subseteq R$ we denote by $$\Z\{F\} = \{ f \in R \mid \text{ there exists an } n \in \N: f \in \Z[F, \s(F), \ldots, \s^n(F)] \},$$ the set of all elements $\s$-generated by $F$ over $\Z$.
\end{defn}\index{finitely $\s$-generated over $\Z$}

\begin{prop}\label{mixedintersectionprimesfinite}
Let $R$ be a $\s$-ring finitely $\s$-generated over $\Z$. Then, every radical, mixed $\s$-ideal of $R$ is the intersection of prime $\s$-ideals.
\begin{bew}
Let $\a \si R$ be a mixed, radical $\s$-ideal. By Proposition \ref{bijideals} there is a bijection between the prime $\s$-ideals of $R$ containing $\a$ and those of $R/\a$. Hence, we can assume without loss of generality that $\a = [0] \si R$,
 by replacing $R$ with $R/\a$. This means that we only have to show that the zero ideal $[0]$ of a well-mixed, reduced $\s$-ring $R$ is the intersection of all its prime $\s$-ideals. Note that this does not change the fact
that $R$ is finitely $\s$-generated over $\Z$. \\

\indent Let thus $f \in R$ be such that $f \in \q \fa \q \si R$ prime. We assert that $f$ then has to be $0$. Assume this is not the case, i.e. $f \neq 0$. Then, by assumption on $R$, there is an $n \in \N$, such that $f \in \Z[A,\s(A),\ldots,\s^n(A)]$.
We can now argue with the usual case of commutative algebra (Lemma \ref{commalg}): since $\Z[A,\s(A),\ldots,\s^n(A)]$ is Noetherian and reduced, $(0) \unlhd \Z[A,\ldots,\s^n(A)]$ is the intersection of all prime ideals of $R$. In particular, there exist prime ideals which do not contain $f$.
Let $\q_0 \unlhd \Z[A,\ldots,\s^n(A)]$ be a minimal prime ideal with $f \notin \q_0$. Since $f \in \Z[A,\s(A),\ldots,\s^n(A)] \subset \Z[A,\s(A),\ldots,\s^{n+1}(A)]$, by Lemma \ref{commalg}, we can find a minimal prime ideal $\q_1 \unlhd \Z[A,\s(A),\ldots,\s^{n+1}(A)]$
such that $\q_1 \cap \Z[A,\s(A),\ldots,\s^{n}(A)] = \q_0$. \\

\indent Inductively we find a chain of minimal prime ideals $\q_i, i \in \N$, $\q_i \unlhd \Z[A,\s(A),\ldots,\s^{n+i}(A)]$, with $\q_{i+1} \cap \Z[A,\s(A),\ldots,\s^{n+i}(A)] = \q_i$ for all \\ $i \in \N$.
Then $\q := \bigcup_{i=0}^{\infty} \q_i$ is a prime ideal of $R$, with $f \notin \q$. In fact, $\q$ is a $\s$-ideal of $R$: Let $a \in \q$. We want to show that $\s(a) \in \q$. By construction of $\q$ there exists an $i \in \N$, such
that $a \in \q_{i-1} \subseteq \Z[A,\s(A),\ldots,\s^{n+i-1}(A)]$, which implies that $\s(a) \in \Z[A,\s(A),\ldots,\s^{n+i}(A)]$. Lemma \ref{commalg} then states that there is an $h \in \Z[A,\s(A),\ldots,\s^{n+i}(A)]$, such that $ \q_i = \text{Ann}(h)$.
It follows that $ah = 0$, and since $R$ is well-mixed, this implies that $\s(a)h = 0$. Hence, $\s(a) \in \q_i \subseteq \q$. This means that $\q$ is a prime $\s$-ideal of $R$ not containing $f$, which contradicts the assumption on $f$, so that $f = 0$ has to follow.
\end{bew}
\end{prop}

For the general case we need yet another tool, the concept of filters:

\begin{defn}\index{filter}\index{ultrafilter}
Let $U$ be a set, and let $F \subseteq \text{Pot}(U)$, where $\text{Pot}(U)$ denotes the power set on $U$. Then $F$ is called a \emph{filter} if it satisfies the following axioms: 
\begin{itemize}
\item  $U \in F$ and $\emptyset \notin F$.
\item If $V,W \subseteq U$ with $V \subseteq W \text{ and }V  \in F $, then $W \in F$.
\item For $V_1, \ldots, V_n \in F$ we have \[ \bigcap_{i = 1}^n V_i \in F. \]
\end{itemize}
A filter $F$ is called an \emph{ultrafilter}, if for any $V \subseteq U$ we have $V \in F$ or $U \setminus V \in F$. Note that the first and third axioms together imply that at most one of $V$ and $U \setminus V$ can be in $F$.
\end{defn}



\begin{rem}
Let $U$ be a set. Then, the set of filters on $U$ is inductively ordered by inclusion. By Zorn's lemma, for every filter $F$ on $U$ there must exist a maximal filter $G$ with respect to inclusion, such that $F \subseteq G$.
The maximality of the filter implies that $G$ will be an ultrafilter, since we could otherwise find a new filter $G'$, where $G$ is properly included by adding one of the sets that contradict the ultrafilter property and considering the smallest filter containing this set.
\end{rem}

The reason why this concept is useful in our context is the following:

\begin{lem}\label{lemmafilters}
Let $R$ be a $\s$-ring, and let $M$ be the set of all $\s$-subrings of $R$ which are finitely $\s$-generated over $\Z$. For any fixed subset $F \subseteq R$, consider the set $M_F:= \{ T \subseteq M \mid \{S \in M \mid F \subseteq S \} \subseteq T \} \subseteq \text{Pot}(M)$. 
Then, \[ \mathcal{F}:= \bigcup_{ F \subseteq R \text{ finite} } M_F \]
 defines a filter on $M$. If $\mathcal{G}$ is an ultrafilter containing $\mathcal{F}$, and $P:= \prod_{S \in M} S$ with component-wise operations,
 then the ultrafilter $\mathcal{G}$ defines an equivalence relation on $P$ via $(g_S)_{S \in M} \sim (h_S)_{S \in M} : \Leftrightarrow \{ S \in M \mid g_S = h_S \} \in \mathcal{G}$. 
The set of equivalence classes $P/\mathcal{G}:= P/\textsim$ has a natural $\s$-ring structure. %% and is called an \emph{ultraproduct}. \index{ultraproduct} %%fixme: of what?
\begin{proof}
Let us first show that $\mathcal{F}$ is a filter. For $F \subseteq R$ finite we have $$\Z\{F\} \in \{ S \in M \mid F \subseteq S \} \neq \emptyset.$$ Since $T \supseteq \{ S \in M \mid F \subseteq S \} \fa T \in M_F$, $\emptyset \notin M_F$ (For $F = \emptyset$ this works as well).
  That $M \in M_F$ for any $F \subseteq R$ is obvious, as well as that for $T \subseteq U, T \in M_F$ we have $U \in M_F$. \\ 
\indent We only need to show that $U,T \in \mathcal{F}$ implies that $U \cap T \in \mathcal{F}$.
  Let $\hat U, \hat T \subseteq R$ be finite, such that $U \in M_{\hat U}, T \in M_{\hat T}$. $\hat U \cup \hat T \subseteq R$ is also finite and $$\{ S \in M \mid \hat U \cup \hat T \subseteq S \} \subseteq \{ S \in M \mid \hat U \subseteq S \} \subseteq U,$$
 similarly for $T$. This means that $U \cap T \in M_{\hat U \cup \hat T} \subseteq \mathcal{F}$, which finishes the proof that $\mathcal{F}$ is a filter. \\

 Now, consider an ultrafilter $\mathcal{G} \supseteq \mathcal{F}$ and define $\sim$ on $P$ as above. This is an equivalence relation: Let $f \sim g, g \sim h$ for $f,g,h \in P$. 
 This means that $\{ S \in M \mid f_S = g_S \} \in \mathcal{G}, \{ S \in M \mid g_S = h_S \} \in \mathcal{G}$. However then $$\{ S \in M \mid f_S = g_S \} \cap \{ S \in M \mid g_S = h_S \} \subseteq \{ S \in M \mid f_S = h_S \} \in \mathcal{G},$$
 since $\mathcal{G}$ is a filter.
 Reflexivity follows from the fact that $M \in \mathcal{G}$, and symmetry is obvious. \\
\indent  We now only need to show that we have a well-defined $\s$-ring structure on $P/\textsim$.
 Consider $f,f' \in P$ with $f \sim f'$. For all $S \in M$ with $f_S = f'_S$  we have $\sigma(f)_S = \sigma(f')_S$. 
 Then $\{ S \in M \mid \s(f)_S = \s(f')_S \} \supseteq \{ S \in M \mid f_S = f'_S \} \in \mathcal{G}$, by assumption, and since $\mathcal{G}$ is a filter, this means that $$\{ S \in M \mid \s(f)_S = \s(f')_S \} \in \mathcal{G}.$$
 Hence $\s(f) \sim \s(f')$. That $+$ and $\cdot$ are also well-defined can be proven in an analogous fashion.
\end{proof}
\end{lem}

We can now turn our attention to the generalization of Proposition \ref{mixedintersectionprimesfinite}. 


\begin{theorem}\label{intersectionprimes}
Let $R$ be a $\s$-ring and $F \subseteq R$ be a subset of $R$. Then, 
\begin{align*} \{F\}_m = \bigcap_{\substack{F \subseteq \p \si R \\ \p \text{ prime}}} \p. \end{align*}
In particular, every radical, mixed $\s$-ideal of $R$ is the intersection of prime $\s$-ideals.
\begin{bew}
It suffices to show that every radical, mixed $\s$-ideal of $R$ is the intersection of prime $\s$-ideals.
Indeed, since prime $\s$-ideals are radical and mixed, it is clear that $\{F\}_m \subseteq \p$ for every prime $\p \si R$ with $F \subseteq \p$. Together with the fact that every radical, mixed $\s$-ideal of $R$ is the intersection of prime $\s$-ideals, this gives the representation 
\begin{align*} \{F\}_m = \bigcap_{\substack{F \subseteq \p \si R \\ \p \text{ prime}}} \p. \end{align*}
Now, by the same argument as in the beginning of the proof of Proposition \ref{mixedintersectionprimesfinite}, it is enough to prove in the case that $R$ is well-mixed and reduced, that the intersection of all prime $\s$-ideals of $R$ is $[0]$.
Let $0 \neq f \in R$. We will construct a prime $\s$-ideal $\q$ of $R$, which does not contain $f$: 

Let $P/\mathcal{G}$ be the difference ring as in Lemma \ref{lemmafilters}. Consider the mapping $\varphi: R \rightarrow P/\mathcal{G}, g \mapsto (g_S)_{S \in M}$ with $(g_S) = g \fa S \in M$ with $g \in S$ and $(g_S) = 0$ for $g \notin S$. 
It is, in fact, $\{ S \in M \mid g \in S \} \in M_{\{g\}}$ (with $M_{\{g\}}$ as in Lemma \ref{lemmafilters}). It follows from this that the image of the mapping onto $P/\mathcal{G}$ is independent of the $(g_S)$ for $g \notin S$, as any other choice of these would be in the same equivalence class as the image described above.
It follows that $\varphi$ is a well-defined morphism of $\s$-rings. \\
\indent From Proposition \ref{mixedintersectionprimesfinite} we know that for every $S \in M$, there exists a prime $\s$-ideal $\p_S \si S$, such that $f \notin \p_S$. 
We define $\p \subseteq P/\mathcal{G}$ as the set of all equivalence classes of elements $(g_S)_{S \in M}$, such that $\{ S \in M \mid g_S \in \p_S \} \in \mathcal{G}$. 
For $[(g_S)_{S \in M}]_{\sim}, [(h_s)_{S \in M}]_{\sim} \in \p$ we have $$ \mathcal{G} \ni \{ S \in M \mid  g_S \in \p_S \} \cap  \{ S \in M \mid  h_S \in \p_S \} \subseteq \{ S \in M \mid  g_S + h_S \in \p_S \} \in \mathcal{G},$$
since $\mathcal{G}$ is a filter. Similar arguments for $\s(g), gh$ for $h \in P/\mathcal{G}$ show that $\p$ is indeed a $\s$-ideal. $\p$ is also prime since $\mathcal{G}$ is an ultrafilter:
Let $g,h \in P$ with $\{ S \in M \mid g_Sh_S \in \p_S \} \in \mathcal{G}$. If $[g]_\sim \notin \p$, then $V:= \{ S \in M \mid g_S \in \p_S \} \notin \mathcal{G}$. Since $\mathcal{G}$ is an ultrafilter, 
this means that $M \setminus V \in \mathcal{G}$. However, $$\mathcal{G} \ni (M \setminus V) \cap \{ S \in M \mid g_S h_S \in \p_S \} \subseteq \{ S \in M \mid h_S \in \p_S \} \in \mathcal{G},$$
which means that $[h]_\sim \in \p$. The preimage of a prime $\s$-ideal, $\q := \varphi^{-1}(\p) \si R$ is also prime (see Lemma \ref{bijmapping}). By construction, $[\varphi(f)]_\sim \notin \p$, which means that $f \notin \varphi^{-1}(\p)$, as desired. 

\end{bew}
\end{theorem}

\subsection{An Analog of the Cohn Topology}

\begin{defn}
Let $R$ be a $\s$-ring. We denote the set of all prime $\s$-ideals of $R$ by $\sSpec(R):= \{ \p \si R \mid \p \text{ prime }\}$. Similarly, we denote the set of $\s$-prime ideals by $\Spec^\s(R):= \{ \p \si R \mid \p ~ \s\text{-prime }\} \subseteq \sSpec(R)$.
\index{$\sSpec$} \index{$\Spec^\s$}
\end{defn}


\begin{rem}
As is the case with $\Spec^\s(R)$, it can be the case that \\ $\sSpec(R)= \emptyset$. For example, let $R$ be a $\s$-ring, and consider the $\s$-ring $R \oplus R$ with $\s( (r,s)):= (\s(s),\s(r))$. 
We will show that this ring has no prime $\s$-ideals. Let $\p \unlhd R$ prime. Then $0 = (1,0)(0,1) \in \p$, which means that either $(1,0) \in \p$ or $(0,1) \in \p$. Then $R \oplus 0 \subseteq \p$ or $0 \oplus R \subseteq \p$. If we assume that $\p$ is a $\s$-ideal then
 this implies that $R \oplus R \subseteq \p$, which by definition cannot be.
\end{rem}

In algebraic geometry, one usually considers $\Spec(R)$ as a topological space with a topology called the Zariski topology. This has an analog for $\Spec^\s(R)$, usually called the Cohn topology. Here we will develop a further analog of both,
 which we will define on $\sSpec(R)$ and will be closely related to radical, mixed $\s$-ideals, as we shall see by its many properties.

\begin{defn}
Let $R$ be a $\s$-ring and $F \subseteq R$ be a subset of $R$. We set $\Vm (F):= \{ \p \in \sSpec(R) \mid F \subseteq \p \}$. 
If $F$ has only one element $f$, we write $\Vm(f)$ for $\Vm(F)$. \index{$\Vm(F)$}
\end{defn}

\begin{lem}\label{topologywelldef}
Let $R$ be a $\s$-ring. Then we have:
\begin{enumerate}[(a)]
\item $\Vm((0)) = \sSpec(R)$, and $\Vm(R) = \emptyset$.
\item For any two ideals $\a,\b \unlhd R$ we have $\Vm(\a) \cup \Vm(\b) = \Vm(\a \cap \b).$
\item For any family of ideals $(\a_i)_{i \in I}$ for an index set $I$, we have $$\bigcap_{i \in I} \Vm(\a_i) = \Vm(\sum_{i \in I} \a_i).$$ \label{vmintersectionideals}
\end{enumerate}
\begin{bew} $~$
\begin{enumerate}[(a)]
\item We have $(0) \subseteq \p \fa \p \in \sSpec(R)$, as well as $R~\not\subseteq~\p $ for all $\p~\in~\sSpec(R)$.
\item Let $\a, \b \unlhd R$ be two ideals in $R$. Then $\Vm(\a) \cup \Vm(\b) \subseteq \Vm(\a \cap \b)$, since for $\p \si R$ prime $\a \subseteq \p$, it follows that $\a \cap \b \subseteq \p$, and similarly for $\b$.
On the other hand, let $\p \si R$ prime with $\a \cap \b \subseteq \p$, and $\a \not\subseteq \p$ (otherwise $\p \in \Vm(\a)$ and we are done). Then there exists an $f \in \a$, $f \notin \p$. 
For any $g \in \b$, it follows that $fg \in \a \cap \b \subseteq \p$. Since $\p$ is prime, this implies that $g \in \p$. Hence, $\b \subseteq \p$, which concludes the proof.
\item Let $(\a_i)_{i \in I}$ be a family  of ideals of $R$. Then $$\p \in \bigcap_{i \in I} \Vm(\a_i) \Leftrightarrow \a_i \subseteq \p \fa i \in I \Leftrightarrow \p \in \Vm(\sum_{i \in I} \a_i).$$
\end{enumerate}
\end{bew}
\end{lem}


\begin{rem}\label{vmsequal}
Since for a $\s$-ring $R$ any prime $\s$-ideal of $R$ is radical and mixed, then for any $F \subseteq R$, and any prime $\s$-ideal $\p \si R$ with $F \subseteq \p$ we have
$(F) \subseteq [F] \subseteq \{ F \}_m \subseteq \p$. In particular, this means that $$\Vm(F) = \Vm((F)) = \Vm([F]) = \Vm(\{F\}_m).$$ 
\end{rem}

\begin{defn}\label{deftop}
Let $R$ be a $\s$-ring. We define a topology on $\sSpec(R)$ by letting $A \subseteq \sSpec(R)$ be closed if $A = \Vm(\a)$ for an ideal $\a \unlhd R$, or equivalently,
 by defining a set to be open, if it is a complement of such a $\Vm(\a)$. This is a well-defined topology thanks to Lemma \ref{topologywelldef}.
For $f \in R$ we set $$\sD(f):= \sSpec(R) \setminus \Vm(f).$$ $\sD(f)$ is the complement of a closed set, and hence, open. 
We call the sets of the form $\sD(f) \subseteq \sSpec(R)$ \emph{basic open subsets} of $\sSpec(R)$. \index{basic open subsets}
\end{defn}

From here on, if not explicitly stated otherwise, when referring to topological concepts on $\sSpec(R)$ we will be referring to the topology just defined.

\begin{rem}
From its definition it is clear that $\sSpec(R) \subseteq \Spec(R):= \{ I \unlhd R \mid I \text{ prime} \}$. Since Lemma \ref{topologywelldef} does not require the ideals to be difference ideals, 
it is easy to conclude that in fact the topology on $\sSpec(R)$ is just the topology induced by restriction of the Zariski topology to $\sSpec(R)$. The same argument can be made to see that the Cohn topology in turn,
defined on $\Spec^\s(R) = \{ \p \si R \mid $ $\p$ $ \s$-prime $\} \subseteq \sSpec(R)$, is also the restriction of the topology defined on $\sSpec(R)$. 
\end{rem}

\begin{defn}
Let $X$ be a topological space.
\begin{enumerate}[(a)]
\item  We say that $X$ is \emph{irreducible}, if $X = X_1 \cup X_2$ with $X_1, X_2$ closed implies that $X = X_1$ or $X = X_2$. 
$X_1 \subseteq X$ is called \emph{irreducible}, if it is an irreducible topological space with the topology induced by the restriction to $X_1$.\index{irreducible topological space}
\item Let $Y \subseteq X$ be closed. We say that a point $f \in Y$ is a \emph{generic point} of $Y$, if $\overline{\{  f \} } = Y$, where for $A \subseteq X$, $\overline{A}$ denotes the closure of $A$. \index{generic point}
\end{enumerate}
\end{defn}

\begin{prop}
Let $R$ be a $\s$-ring. We have:
\begin{enumerate}[(a)]
\item \label{vmbijection} The mapping 
$$\{ \a \si R \mid \a\text{ mixed, radical }\} \rightarrow \{ A \subseteq \sSpec(R) \mid A \text{ closed }\}, \a \mapsto \Vm(\a)$$
 is bijective and order-reversing.
\item \label{irred=prime} Let $F \subseteq R$ be a subset of $R$. Then $\Vm(F)$ is irreducible, if and only if $\{F\}_m$ is prime.
\item $\sSpec(R)$ is quasi-compact.
\item The basic open sets $\{ \sD(f) \mid f \in R \}$ form a basis for the topology on $\sSpec(R)$.
\item Every irreducible closed subset $Y$ of $\sSpec(R)$ has a unique generic point $y$.
\end{enumerate}
\begin{bew} $~$
\begin{enumerate}[(a)]
\item \label{orderreversingbij} That the mapping is order-reversing is obvious. The injectivity follows from the fact that by Theorem \ref{intersectionprimes} $\a = \bigcap_{\a \subseteq \p \in \sSpec(R)} \p$. By Remark \ref{vmsequal} we obtain the surjectivity,
 since $\Vm(\a) = \Vm(\{\a\}_m)$.
\item Since $\Vm(F) = \Vm(\{F\}_m)$, we can assume without loss of generality, that $F \si R$ is a radical, mixed $\s$-ideal.
For the first implication, ``$\Leftarrow$'', let $F \si R$ be prime, and $\Vm(F) = \Vm(\a) \cup \Vm(\b)$ with radical, mixed $\s$-ideals $\a, \b$. Assume that $\Vm(F) \not\subseteq \Vm(\a)$. Then by (\ref{orderreversingbij}), $\a \not \subseteq F$, so there exists an $a \in \a$, with $a \notin F$.
For any $b \in \b$ we then have $ab \in \p \fa \p \in \V(F) = \Vm(\a) \cup \Vm(\b)$, and with Theorem \ref{intersectionprimes} we get $ab \in F = \bigcap_{\p \in \Vm(F)}\p$. By assumption, $F$ is prime and $a \notin F$, which implies
 that  $b \in F$. However, this means that $\b \subseteq F$, and thus $\Vm(F) \subseteq \Vm(\b)$, which shows the irreducibility. \\
\indent Now, for the other implication, ``$\Rightarrow$'', assume that $\Vm(F)$ is irreducible, and let $a,b \in R$ with $ab \in F$. Consider $F \subseteq \p \in \Vm(F)$. Then $ab~\in~\p$, 
which means that $a \in \p$ or $b \in \p$, since $\p$ is prime. This implies that $\p \in \Vm(\{a\}_m) \cup \Vm(\{b\}_m)$, which in turn implies that $$\Vm(F) \subseteq \Vm(\{a\}_{m}) \cup \Vm(\{b\}_{m}).$$
Now, by assumption, $\Vm(F)$ is irreducible, and thus it has to be that $\Vm(F) \subseteq \Vm(\{a\}_{m})$ or $\Vm(F) \subseteq \Vm(\{b\}_m)$. By the bijectivity of the mapping in (\ref{orderreversingbij}) this means that $a \in F$ or $b \in F$.
\item Let $\Vm(\a_i)_{i \in I}$ be a family of closed sets, $\a_i \si R$ mixed, radical for all $i \in I$, satisfying that 
$\bigcap_{i \in J} \Vm( a_i) \neq \emptyset$ for every finite $J \subseteq I$. By  going to the complement of open sets, quasi-compactness is equivalent to the implication that $\bigcap_{i \in I} \Vm(a_i) \neq \emptyset$.
By Lemma \ref{topologywelldef} we see that $$\bigcap_{i \in I} \Vm( \a_i) = \Vm ( \sum_{i \in I} \a_i).$$ Assume that $ \Vm ( \sum_{i \in I} \a_i) = \emptyset$. 
By Theorem \ref{intersectionprimes} this means that \\ $\{\sum_{i \in I}~\a_i\}_m~=~R$. In particular, $1 \in \{ \sum_{i \in I} \a_i \}_m$. By the construction in Lemma \ref{lemshuffling} (and with the notation used there), this means that there has to be an $n \in \NE$,
so that $1 \in (\sum_{i \in I} \a_i )^{\{n\}}$. In particular, this means that $1$ can be written as a $\s$-polynomial in finitely many elements from finitely many $\a_i$. This implies that there exists a $J \subseteq I$ finite, such that
$1 \in (\sum_{i \in J} \a_i)^{\{n\}}$. This in turn means that $$\Vm(\sum_{i \in J} \a_i) = \cap_{i \in J} \Vm(\a_i) = \emptyset,$$ a contradiction. 
\item For an open subset $U \subseteq \sSpec(R)$ there exists by definition an $\a \si R$, such that $U = \sSpec(R) \setminus \Vm(\a)$. We can then write $U$ as a union of basic open sets as follows: $$U = \bigcup_{a \in \a} \sD(a).$$
\item By (\ref{irred=prime}), an irreducible closed subset $A$ of $\sSpec(R)$ has the form $A = \Vm(\p)$, for $\p \si R$ prime. This prime $\s$-ideal $\p$ is the unique generic point of $A$.
To see this, consider the closure of $\p$: $$\overline{\{\p\}} = \bigcap_{\{\p\} \subseteq \Vm(F)}\Vm(F).$$ From the definition of $\Vm(F)$ we know that $\{\p\} \subseteq \V_m(F)$ if and only if $F \subseteq \p$. By Lemma \ref{topologywelldef} (\ref{vmintersectionideals}) and the obvious fact that we can restrict the intersection to $\s$-ideals, we thus get
\[ \overline{\{\p\}} = \bigcap_{F \subseteq \p}\Vm(F) = \Vm(\sum_{F \subseteq \p, F \si R} F) = \Vm(\p) = A. \]
\end{enumerate}
\end{bew}
\end{prop}

\clearpage

%\documentclass[12pt,a4paper,BCOR15mm,twoside,DIV12]{article}
\documentclass{article}
\usepackage[paper=a4paper,left=20mm,right=20mm,top=25mm,bottom=25mm]{geometry}
\usepackage[english]{babel}
\usepackage[utf8]{inputenc}
\usepackage{amsmath}
\usepackage{color}
\usepackage{enumerate}
\usepackage{amssymb}
\usepackage{amsfonts}
\usepackage{amsthm}
\usepackage{hyperref}
\usepackage{makeidx}
\usepackage{graphicx, float,epsfig}
\usepackage[nottoc,numbib]{tocbibind}


\newcommand{\properideal}{%
  \mathrel{\ooalign{$\lneq$\cr\raise.22ex\hbox{$\lhd$}\cr}}}

\def\P{\mathcal{P}}
\def\I{\mathbb{I}}
\def\R{\mathbb{R}} 
\def\E{\mathcal{E}} 
\def\NE{\mathbb{N}_{\geq1}} 
\def\N{\mathbb{N}} 
\def\Z{\mathbb{Z}} 
\def\Q{\mathbb{Q}} 
\def\F{\mathbb{F}}
\def\Vm{\mathcal{V}_m}
\def\V{\mathcal{V}}
\def\VV{\mathbb{V}}
\def\C{\mathbb{C}}
\def\U{\mathcal{U}}
\def\a{\mathfrak{a}}
\def\b{\mathfrak{b}}
\def\p{\mathfrak{p}}
\def\q{\mathfrak{q}}
\def\s{\sigma}
\def\si{\unlhd_{\sigma}}
\def\GL{\text{GL}}
\def\supp{\text{Supp}}
\def\id{\text{id}}
\def\n{\underline{n}}
\def\Spec{\text{Spec}}
\def\sSpec{\sigma\text{-Spec}}
\def\diag{\text{diag}}
\def\End{\text{End}}
\def\Hom{\text{Hom}}
\def\fa{\text{ for all }}
\def\Tr{\text{Tr}}
\def\Id{\text{Id}}
\def\Sym{\text{Sym}}
\def\H{\mathcal{H}}
\def\wt{\text{wt}}
\def\Perf{\text{Perf}}


\renewcommand{\labelenumi}{\alph{enumi})}
%\renewcommand{\P}{\textfrak{P}}
\newcommand{\cupdot}{\mathop{\mathaccent\cdot\cup}}
\newcommand{\textsim}{\mathord{\sim}}
\newenvironment{bew}{\begin{proof}[Proof]}{\end{proof}}
\theoremstyle{plain}
\newtheorem{Satz}{Satz}[section]
\newtheorem{theorem}[Satz]{Theorem}
\newtheorem{ex}[Satz]{Example}
\newtheorem{cor}[Satz]{Corollary}
\newtheorem{algorithm}[Satz]{Algorithm}
\newtheorem{prop}[Satz]{Proposition}
\newtheorem{lem}[Satz]{Lemma}
\newtheorem{defn}[Satz]{Definition}
\theoremstyle{definition}
\newtheorem{rem}[Satz]{Remark}


\makeindex
\title{Mixed Ideals in Difference Algebra}
\author{Andr\'{e}s Goens}
\date{\today}
\begin{document}
\setlength{\parindent}{1.5em}
\section{An alternative proof of the theorem about the decomposition of radical, mixed $\s$-ideals in prime $\s$-ideals}
\begin{defn}
Let $R$ be a difference ring and $F \subseteq R$ a subset of $F$. We denote by $\{F\}_\text{mixed}$ the mixed closure of $F$, i.e. the smallest, mixed difference ideal, with respect to inclusion, which contains $F$.
\end{defn}

\begin{lem}\label{maxmixed=prime}
Let $R$ be a difference ring and let $U \subset R$ be a multiplicatively closed set of $R$. Then a mixed difference ideal $\a \subset R$ maximal in the class of mixed difference
ideals not meeting $U$, i.e. such that $\a \cap U = \emptyset$, is prime. 
\begin{bew}
Let $\a \subset R$ be a maximal mixed difference ideal not meeting $U$, and let $f, g \in R \setminus U$. We will show that assuming $fg \in \a$ yields a contradiction. \\
\indent Assume thus, that $fg \in \a$. The $\s$-ideals $\{\a \cup \{f\}\}_\text{mixed}$ and $\{\a \cup \{g\}\}_\text{mixed}$ are mixed, so that by maximality of $\a$,
they have to meet $U$. This implies that there exist $u_1 \in U \cap \{\a \cup \{f\}\}_\text{mixed}$ and $u_2 \in U \cap \{\a \cup \{g\}\}_\text{mixed}$.
By the construction of $\{\a \cup \{f\}\}_\text{mixed}$ with shuffling, there exists an $n_1 \in \NE$ such that $u_1 \in (\a \cup \{f\})^{\{n_1\}}$, and similarly $n_2 \in \NE$ for $u_2$.
\\ We will show by induction that an element of $(\a \cup \{f\})^{\{n_1\}}$ is always of the form \begin{align}\label{generalformaf} \sum_{i=1}^m r_i \s^{k_i}(f^{l_i}) + a\text{ with }m \in \N, r_i \in R, k_i \in \N, l_i \in \NE\text{ for }i = 1,\ldots,m\text{ and }a \in \a. \end{align}
For $n_1 = 0$, an element of $(\a \cup \{f\})^{\{0\}} = [\a \cup \{f\}]$ the assertion is obvious. Now, assume the assertion holds for $n-1 \in \NE$. Then consider the set 
$$B := ((\a \cup \{f\})^{\{n-1\}})' = \{ h \s(h') \mid hh' \in (\a \cup \{f\})^{\{n-1\}} \}.$$ Any $b \in B$ can be written as 
$$ b = (\sum_{i=1}^m r_i \s^{k_i}(f^{l_i}) + a)\s((\sum_{i=1}^m \tilde r_i \s^{\tilde k_i}(f^{\tilde l_i}) + \tilde a))$$
with $a, \tilde a \in \a, r_i, \tilde r_i \in R \fa i = 1,\ldots,m$ (we can always have the same number of summands $m$ by adding $0$ enough times, if necessary).
 We can thus always write an element $c \in  (\a \cup \{f\})^{\{n\}} = [B]$ as follows:
\[ c = \sum_{j=1}^s \hat r_j \s^{t_j} \left( (\sum_{i=1}^{m} r_{i,j} \s^{k_{i,j}}(f^{l_{i,j}}) + a_j)\s((\sum_{i=1}^{m} \tilde r_{i,j} \s^{\tilde k_{i,j}}(f^{l_{i,j}}) + \tilde a_j)) \right) \]
with $r_j, r_{i,j}, \tilde r_{i,j} \in R, a_j \in \a \fa i = 1,\ldots, m, j = 1, \ldots, s$. Multiplying out and collecting the terms yields the form in (\ref{generalformaf}). \\
\indent With this we know thus that we can write 
\begin{align}\label{formu1u2} u_1 = \sum_{i=1}^m r_i \s^{k_i}(f^{l_i}) + a, ~ u_2 = \sum_{i=1}^m \tilde r_i \s^{\tilde k_i}(g^{\tilde l_i}) + \tilde a \end{align}
again with $a, \tilde a \in \a, r_i, \tilde r_i \in R \fa i = 1,\ldots,m$. 
Assume without loss of generality that $k \geq \tilde k$ and consider a product of the form 
\[ \s^{k}(f^{l}) \s^{\tilde k}(g^{\tilde l}) = \underbrace{\s^{\tilde k}(\s^{k - \tilde k}(f) g)}_{\in \a, \text{ since } \a \text{ is mixed}} \s^{\tilde k}(\s^{k - \tilde k}(f^{l-1}) g^{\tilde l - 1}) \in \a \]
This implies however, by multiplying out (\ref{formu1u2}), that $u_1 u_2 \in \a$, a contradiction.
\end{bew}
\end{lem}

\begin{theorem}
Let $R$ be a differece ring and $F \subseteq R$ be a subset of $R$. Then
\[ \{ F \}_m = \bigcap_{F \subseteq \p \si R, \p \text{ prime }} \p.\]
\begin{proof}
The inclusion ``$\subseteq$'' is obvious, since prime difference ideals are mixed. For the inclusion ``$\supseteq$'', let $g \in R, g \notin \{ F \}_m$, and consider the multiplicatively closed set $U = \{ g^k \mid k \in \NE \} \subset R$. 
$\{ F \}_m \cap U = \emptyset$ since $\{ F \}_m$ is radical. By Lemma \ref{maxmixed=prime}, any maximal mixed difference ideal $\p$ that is disjoint with $U$ is a prime difference ideal. In particular, since $\{F\}_m$ is a mixed difference ideal disjoint from $U$, there exisits a prime difference ideal $\p \supseteq \{F\}_m$ such that $g \notin \p$,
which implies that $g \notin \bigcap_{\a \subseteq \p \si R, \p \text{ prime }} \p$. By taking the contraposition of this we get the desired inclusion.
\end{proof}
\end{theorem}
\end{document}

\clearpage 

\section{Difference Varities}
In this section we will introduce difference varieties. We will do so in a way that they correspond with the topology on $\sSpec(R)$, which we defined in the previous section. It will be again based on M. Wibmer's lecture notes \cite{wibmer}, 
where it is worked out for the analogous case of perfect $\s$-ideals.

\begin{defn}
Let $A$ be a $\s$-ring. If $A$ is (algebraically) an integral domain, we call $A$ an \emph{integral $\s$-ring}. If additionally the endomorphism $\s$ on $A$ is injective, then we call $A$ a \emph{$\s$-domain}. \index{integral $\s$-ring} \index{$\s$-domain}
\end{defn}

\begin{rem}\label{sdomain=field}
Let $A$ be a $\s$-domain. Then $k:=\operatorname{Quot}(A)$ is a $\s$-field: For $\frac{r}{s} \in k$ we can define $\s(\frac{r}{s}):= \frac{\s(r)}{\s(s)}$. Since $\s$ is injective, it holds that $\s(s) \neq 0$ for $s \neq 0$, which implies that $\s$ is well defined on $k$.
By this argument we see that in general for an integral $\s$-ring $A$, $\operatorname{Quot}(A)$ is a $\s$-field (in this natural way) if and only if $A$ is a $\s$-domain.
\end{rem}

Our main purpose, in a first instance at least, is to investigate the properties of solutions to difference equations. 
We will start with an integral $\s$-field $k$ and look for solutions (zeros) of some $\s$-polynomial $p$ over $k$, i.e. $p \in k\{y_1, \ldots, y_n \}$. In general, rather, it will be a set of $\s$-polynomials $F \subseteq k\{y_1, \ldots, y_n \}$ that we will study. 
For this we want to define $\s$-varieties; we cannot mimic the usual approach from algebraic geometry, where we would take the algebraic closure of $k$. The next remark shows why.

\begin{rem}\label{incompatibleextensions}
 Consider the constant $\s$-field $\Q$ and $K = \Q(\sqrt{2})$, with $\s (\sqrt{2}) = \sqrt{2}$; $L = \Q(\sqrt{2}), \s(\sqrt{2}) = - \sqrt{2}$. 
Both $K$ and $L$ are $\s$-field extensions of $\Q$, but there cannot be a further extension $\Q \leq M$ of $\s$-fields, such that $K,L \leq M$ are both (isomorphic to) $\s$-subfields of $M$. 
To see this, assume there was such an $M$. Then the set $\{ a \in M \mid a^2 - 2 = 0 \}$ has exactly two elements, which we will call $\sqrt{2}, -\sqrt{2}$ (since $\sqrt{2} + (- \sqrt{2}) = 0$).
But $\sqrt{2} \in K$ has to be mapped to one of these two in any embedding, and the same for $\sqrt{2} \in L$, which already yields the contradiction,
 since in $M$ either $\s(\sqrt{2}) = \sqrt{2}$ or $\s(\sqrt{2}) = -\sqrt{2}$.
\end{rem}

To avoid this problem, we will define $\s$-varieties as functors. For this we will need a few category-theoretic definitions:

\begin{defn}
Let $k$ be a $\s$-field. The category of all $\s$-ring extensions $A$ of $k$ we denote by $\sringk$, where the morphisms are defined as follows: For $B,C \in \sringk$ we say that a morphism of $\s$-rings $\varphi: B \rightarrow C$ is a morphism of $\s$-ring extensions of $k$, if and only if, $\varphi_{|k} = \id_k$.
The subcategory which arises from restricting the object class to integral $\s$-rings, the category of integral $\s$-ring extensions of $A$, we denote by $\sintk$. \index{$\sintk$} \index{$\sringk$}
\end{defn}

Now we are ready to define $\s$-varieties of $\s$-rings, with mixed $\s$-ideals in mind:

\begin{defn}\label{defnVV}
Let $k$ be a $\s$-field and $B \in \sintk$ an integral $\s$-overring of $k$. Further let $F \subseteq k\{y_1, \ldots, y_n\}$ be a set of $\s$-polynomials over $k$. 
Then we define $\VV_B(F):= \{ b \in B^n \mid f(b) = 0 \fa f \in F \}$. A functor $X: \sintk \rightarrow \Set$, for which there exists a set $F \subseteq k\{y_1, \ldots, y_n \}$ such that $X(B) = \VV_B(F)$ for all $B \in \sintk$ we denote as a \emph{$\s$-variety over $k$}, or a \emph{$k$-$\s$-variety}.
Here, $\Set$ denotes the usual category of sets with mappings as morphisms. We also write $X := \VV(F)$ as a short notation for this functor. \index{$\s$-variety} \index{$\s$-variety over $k$} \index{$k$-$\s$-variety}
\end{defn}

\begin{defn}
Let $k$ be a $\s$-field and $X: \sintk \rightarrow \Set$ be a $k$-$\s$-variety. We say a subfunctor $Y \subseteq X$ is a \emph{$\s$-subvariety} of $X$, if $Y$ is a $k$-$\s$-variety itself. \index{$\s$-subvariety}
\end{defn}

 \begin{rem}
Let $k$ be a $\s$-field and $X$ be a $k$-$\s$-variety. Not every subfunctor of $X$ is a $\s$-subvariety. Consider the functor $X = \VV(0)$, for $\{0\} \subset k\{y_1\}$.
For $B \in \sintk$ we denote by $B^* = \{ b \in B \mid b \text{ invertible } \}$ the set of units of $B$. Then for $B \in \sintk$, $B \mapsto B^*$ is a subfunctor $Y$ of $X$ (since $B^* \subset B \fa B \in \sintk$ and morphisms of rings always map units to units). $Y$ is not a $\s$-variety, however:
there exists no $F \subseteq k\{y_1\}$ such that $\VV_B(F) = B^* \fa B \in \sintk$.  Indeed, assume there was such an $F$, and let $0 \neq f \in F$. Then $f(b) = 0 \fa b \in B^*$ and $\fa B \in \sintk$. In particular,
for $B = k\langle y_1 \rangle = \operatorname{Quot}(k\{y_1\}) \in \sintk$ it holds that $f(y_1) = f = 0, ~ y_1 \in k \langle y_1 \rangle^*$, a contradiction.  
\end{rem}

\begin{defn}\label{defnI}
Let $X = \VV(F)$ be a $\s$-variety over the $\s$-field $k$, $F \subseteq k\{y_1,\ldots,y_n\}$. Then we set $$\I(X):= \{ f \in k\{y_1,\ldots,y_n\} \mid f(b) = 0 \fa b \in \VV_B(F), ~ B \in \sintk \}.$$ \index{ $\I(X)$}
\end{defn}

\begin{ex}\label{A^n}
Let $k$ be a $\s$-field and consider the set $\{ 0 \} = F \subseteq k\{y_1,\ldots,y_n\}$. Then the $\s$-variety $X$ defined by $F$, $X(B) := \VV_B(F) \fa B \in \sintk$ is called the affine $n$-space, and is denoted by $\mathbb{A}^n_k$, 
or simply $\mathbb{A}^n$, whenever $k$ is clear from the context. Then for every $G \subseteq A\{y_1,\ldots,y_n\}$ the $\s$-variety given by $Y: B \mapsto \VV_B(G)$ is a $\s$-subvariety of $\mathbb{A}^n$, 
and we write $Y \subseteq \mathbb{A}^n$.
\end{ex}

We note that $0$ is in any (radical, mixed, difference) ideal, so it is not surprising that every $\s$-variety is a $\s$-subvariety of $\VV(0)$. This ``intuition'' will be made more concrete later on.

Since we have this functorial definition, we have in principle a whole proper class of solutions for most systems of difference equations. 
It is obvious we want to have some sort of equivalence relation between solutions to group them up in a reasonable manner.

\begin{defn}\label{equivsols}
Let $k$ be a $\s$-field, $B,C \in \sintk$. Further let $F \subseteq k\{y_1,\ldots,y_n\}$ be a system of difference equations and $b \in B^n, c \in C^n$ be solutions of $F$, i.e. $b \in \VV_B(F), c \in \VV_C(F)$.
We say that $b$ and $c$ are equivalent if the mapping $b \mapsto c$ is a well-defined isomorphism between the integral $\s$-rings $k\{b\}$ and $k\{c\}$  (as elements of $\sintk$). \index{equivalent solutions}
\end{defn}

%% \begin{lem}
%% Let $k$ be a $\s$-field, $B,B' \in \sintk$, and let $b \in B^n; b' \in B'^n$ be equivalent solutions for a system of difference equations $F \subseteq A\{y_1,\ldots,y_n\}$
%% \end{lem}

\begin{rem}
The usual approach, for example in \cite{cohn}, Chapter 4 or \cite{levin}, Section 2.6, is to restrict the definition of $\s$-varieties to $\s$-fields instead of allowing any integral $\s$-rings. With this concept,
two solutions $a,b$ of a system of difference equation over a difference field $k$ are said to be equivalent if the $\s$-field extensions $k\langle a \rangle$ and $k\langle b \rangle$ are isomorphic as $\s$-field extensions of $k$ via $a \mapsto b$.
This is in accordance with Definition \ref{equivsols}, i.e., solutions in difference field extensions are equivalent if and only if they are equivalent as solutions in integral $\s$-overrings in the sense of Definition \ref{equivsols}.
\begin{bew}
Assume there exist $\s$-field extensions $k \leq A,B$, and elements $a \in A^n$, $b \in B^n$ such that $a$ and $b$ are equivalent as solutions in the sense of Definition \ref{equivsols}. Since $A,B$ are $\s$-fields, it means that $k\{a\}$ and $k\{b\}$ are $\s$-domains, 
and $k\langle a \rangle, k\langle b \rangle$ have the ``canonical'' difference structure induced by $k\{a\}, k\{b\}$ (see Remark \ref{sdomain=field}). Let $\varphi: k\{a\} \rightarrow k\{b\}, a \mapsto b$ be an isomorphism of integral $\s$-ring extensions of $k$.
Then we can define $\tilde \varphi: k \langle a \rangle \rightarrow k\langle b \rangle, \frac{x}{y} \mapsto \frac{\varphi(x)}{\varphi{(y)}}$. This is a well-defined isomorphism of $\s$-field extensions of $k$, since:
\begin{align*}
\tilde \varphi \left(\s \left(\frac{x}{y}\right)\right) = \tilde \varphi \left( \frac{\s \left(x\right)}{\s \left(y\right)}\right) = \frac{ \varphi \left(\s \left(x\right)\right)}{ \varphi \left(\s \left(y\right)\right)} =  \frac{\s \left(\varphi \left(x\right)\right)}{\s \left(\varphi \left(y\right)\right)} = \s \left( \tilde \varphi \left(\frac{x}{y}\right)\right)
\end{align*}
The inverse implication is obvious.
\end{bew}
\end{rem}

\begin{ex}
In the two $\s$-field extensions of $\Q$ in Remark \ref{incompatibleextensions} we have two solutions of the (algebraic) polynomial $y^2-2$, which represent two different solutions in the difference algebraic sense,
since the $\s$-fields $\Q(\sqrt{2}), \s(\sqrt{2}) = \sqrt{2}$ and $\Q(\sqrt{2}), \s(\sqrt{2}) = -\sqrt{2}$ are not isomorphic. 
\end{ex}

\begin{ex}
Let $k$ be a $\s$-field. The $\s$-variety $X$ given by $\s(y) \in k\{y\}$, i.e. $X(B) = \VV_B(\s(y)) \fa B \in \sintk$ has a single point in any $\s$-field extension of $k$, namely $0$. However, in general integral $\s$-rings,
this is not necessarily the case: Take, for example, $B:= k\{y\}/[\s(y)] \in \sintk$. In $B$ we have $0 \neq $ ker$(\s) = [y] \si B$, which means that in particular, $[y + [\s(y)]] \subseteq \VV_B(\s(y))$.
\end{ex}

It is not a coincidence that in the previous example we found more solutions on the $\s$-ring $B = k\{y\}/[\s(y)]$. The $\s$-ideal $[\s(y)]$ is radical and mixed, i.e., $[\s(y)] = \{ [\s(y)] \}_m$.
In fact, the ring $B$ as we chose it plays an analogous role to that of the coordinate ring of an affine variety in the usual (algebraic) case.

The next proposition shows why our definition of $\s$-variety is ``the right one'' for mixed ideals:

\begin{prop}\label{I=F_m}
Let $k$ be a $\s$-field and $X = \VV(F) \subseteq \mathbb{A}^n$ be a difference variety over $k$. Then $\I(X) = \{F\}_m \si k\{y_1,\ldots,y_n\}$. 
\begin{bew}
We will first show that $\I(X)$ is a radical, mixed $\s$-ideal.
Let $f, g \in \I(X)$, $h \in k\{y_1,\ldots,y_n\}$. Then, for every $B \in \sintk$, $b \in \VV_B(F)$, we have $f(b) = g(b) = 0$.
It follows that $(f + g)(b) = f(b) + g(b) = 0$ as well as $(fh)(b) = f(b)h(b) = 0 \cdot h(b) = 0$ and $\s(f)(b) = \s(f(b)) = \s(0) = 0$, so that $\I(X)$ is a $\s$-ideal.
It further follows that $h(b)^n = 0$ implies $h(b) = 0$, since $B$ is an integral domain, and this means that $h^n \in \I(X)$ implies that $ h \in \I(X)$. \\
\indent It only remains to show that $\I(X)$ is mixed. Let now $f,g \in k\{y_1,\ldots,y_n\}$ be such that $fg \in \I(X)$. This means that for all  $B \in \sintk$, $b \in \VV_B(F)$ it holds
 $(fg)(b) = f(b) g(b) = 0$. Since $B$ is an integral domain,
this implies that $f(b) = 0$ or $g(b) = 0$. But that also implies that $\s(f(b)) = \s(0) = 0$, or $\s(g(b)) = 0$, so that in any case $(f\s(g))(b) = 0$, from which it follows that $f\s(g) \in \I(X)$.
Note that it does not always have to be the same case, $f(b) = 0$ or $g(b) = 0$, as it depends on $B$. In particular, $\I(X)$ does not have to be prime in general. We thus see that $\I(X)$ is radical and mixed, hence $\{F\}_m \subseteq k\{y_1,\ldots,y_n\}$. \\
\indent For the other inclusion, let $f \in \I(X)$. We will show that $f \in \{F\}_m$. Let $F \subseteq \p \si k\{y_1,\ldots,y_n\}$ be a prime $\s$-ideal.
Then, consider $B:= k\{y_1,\ldots,y_n\}/\p$: this is an integral $\s$-ring. Since $F \subseteq \p$, we know that $y + \p \in \VV_B(F)$. By assumption we have $f \in \I(\VV(F))$, which means by definition that $f(y + \p) = 0$, which
in turn means that $f \in \p$. But since this holds for any prime $\p \si R$, Theorem \ref{intersectionprimes} implies that $f \in \{F\}_m$.
\end{bew}
\end{prop}

From this we immediately get a further result on radical, mixed ideals, which is analogous to the case for radical ideals in algebraic geometry.
\begin{cor}\label{prod=cap}
Let $\a, \b \si k\{y\}$ be two radical, mixed difference ideals. Then it holds that $\a \cap \b = \{ \a \b \}_m$.
\begin{bew}
We can assume that $\a, \b \neq \{0\}$, as the assertion is obvious otherwise. Since $\a, \b$ are radical and mixed, we know from Proposition \ref{I=F_m} that $\a = \I(\VV(\a)), \b = \I(\VV(\b))$, and $\{ \a \b \}_m = \I( \VV( \a \b ))$.
For any $B \in \sintk$, it holds that:
\begin{align*} \VV_B( \a \b) = \VV_B( \a) \cup \VV_B( \b) \end{align*}
The inclusion ``$\supseteq$'' is obvious. For ``$\subseteq$'', let $p \in \VV_B(\a\b)$ and assume there exists an $f \in \a$ such that $f(p) \neq 0$.
Then, from the definition of $\VV_B(\a\b)$ it follows that $f(p)g(p) = 0 \fa g \in \b$. This means, however, that $p \in \VV_B(\b)$ (since $B$ is an integral domain). The other case is completely analogous.
Since this holds for any $B$, the $\s$-varieties are also equal: $\VV( \a \b) = \VV( \a) \cup \VV( \b)$. Now,
\begin{align*} \I(\VV(\a \b)) = \I(\VV(\a) \cup \VV(\b)) \\ = \{ f \in k\{y\} \mid f(p) = 0 \fa p \in \VV_B(\a) \cup \VV_B(\b), ~ B \in \sintk \} \end{align*}
And $f(p) = 0 \fa p \in \VV_B(\a) \cup \VV_B(\b), ~ B \in \sintk$, is equivalent to \\
$ \underbrace{f(p) = 0 \fa p \in \VV_B(\a), ~ B \in \sintk}_{\Leftrightarrow f \in \I(\VV(\a))}$ and $ \underbrace{f(p) = 0 \fa p \in \VV_B(\b), ~ B \in \sintk}_{\Leftrightarrow f \in \I(\VV(\b))}$. \\
Hence, $\{\a\b\}_m = \I(\VV( \a \b)) = \I(\VV(\a)) \cap \I(\VV(\b)) = \{\a\}_m \cap \{\b\}_m = \a \cap \b$.
\end{bew}
\end{cor}

\begin{defn}
Let $k$ be a $\s$-field and let $X$ be a $\s$-variety over $k$. Further let $F \subseteq k\{y_1, \ldots, y_n\}$ be a system of difference equations over $k$ with $X(B) = \VV_B(F) \fa B \in \sintk$.
Then we consider the $\s$-ring $k\{y_1, \ldots, y_n\}/\{F\}_m = k\{y_1, \ldots, y_n\}/\I(X) =: k\{X\}$ and call it the \emph{coordinate ring} of $X$. Since $\{F\}_m$ is a radical, mixed $\s$-ideal, $k\{X\}$ is reduced and well-mixed. \index{coordinate ring}
\end{defn}

\begin{rem}
Let $k$ be a $\s$-field and $X$ a $k$-$\s$-variety. Further let $b \in X(B), ~ B \in \sintk, f + \I(X) \in k\{X\}$. Then the value of $f(b) \in k$ is independent of the representative $f$,
since for $f' + \I(X) = f + \I(X)$, we know that $f - f' \in \I(X)$, and thus by definition, $(f - f')(b) = 0$. By abuse of notation,
we will sometimes use the representative $f$ to refer to its equivalence class $f + \I(X)$ and we will simply write $f(b)$ to mean the well-defined value of evaluating $b$ on any representative of the class.
\end{rem}

We can now clarify what we meant after Example \ref{A^n}.

 \begin{lem}\label{bijsubvarsideals}
Let $k$ be a $\s$-field. Then the maps $X \mapsto \I(X)$ and $\a \mapsto \VV(\a)$ define inclusion-reversing bijections between the set of all $\s$-subvarieties of $\mathbb{A}^n$ and the radical, mixed ideals of $k\{y_1,\ldots,y_n\}$.
\begin{bew}
From Proposition \ref{I=F_m} we know that $\I(\VV(\a)) = \a$ for all $\a \si k\{y_1,\ldots,y_n\}$ radial, mixed. Conversely, for a $\s$-variety $X = \VV(F) \subseteq \mathbb{A}^n$ we know  $\VV(\I(X)) = \VV(\I(\VV(F))) \subseteq \VV(F) = X$,
 since $F \subseteq \I(X)$. On the other hand it is clear from the definitions of $\VV$ and $ \I$, that $X \subseteq \VV(\I(X))$, so that $X = \VV(\I(X))$. This proves the bijectivity of both mappings. That both mappings are inclusion-reversing follows directly from the definitions.
\end{bew}
\end{lem}

Note that since every $\s$-variety (as defined in this thesis) is a $\s$-subvariety of $\mathbb{A}^n$ for an $n \in \NE$, it is no restriction to consider $\mathbb{A}^n$ instead of an arbitrary $\s$-variety, as we can see in the following corollary:
\begin{cor}
  Let $X$ be a $\s$-variety over the $\s$-field $k$. Then there is a bijection between the radical, mixed $\s$-ideals of $k\{X\}$ and the $\s$-subvarieties of $X$ via
 $$X \supseteq Y \mapsto \{f \in k\{X\} \mid f(b) = 0 \fa b \in Y(B), \fa B \in \sintk \} =: \I_{k\{X\}}(Y)$$
\begin{bew}
If we identify the radical, mixed ideals of $k\{X\}$ with the radical, mixed ideals of $k\{y\}$ which contain $\I(X)$ (see Proposition \ref{bijideals}), then this is just the restriction of the mapping described in Lemma \ref{bijsubvarsideals}.
\end{bew}
\end{cor}

A further very interesting bijection can also help us better understand equivalence classes of solutions: 
\begin{prop}\label{bijsols}
Let $X = \VV(F)$ be a $\s$-variety over the $\s$-field $k$. The equivalence classes of solutions of $F$ are in bijection with the $\s$-spectrum of the coordinate ring $\sSpec(k\{X\})$
\begin{bew}
Let $B \in \sintk$, $b \in B^n$ be a solution of $F \subseteq k\{y_1,\ldots,y_n\}$, i.e. $f(b) = 0 \fa f \in F$. Consider the mapping $$\varphi: k\{y_1,\ldots,y_n\} \rightarrow B, y \mapsto b.$$
Then $F \subseteq $ ker$( \varphi) \si R$.
Since (forgetting the difference structure for a moment), $B$ is an integral domain, the ideal ker$(\varphi)$ has to be prime. It follows from this that $\{F\}_m = \I(X) \subseteq $ker$(\varphi)$. 
In particular, this implies that the mapping $\varphi$ factors over $\I(X)$, and it induces a morphism of $\s$-rings $\tilde \varphi: k\{X\} \rightarrow B$. By the same argument as above, the kernel of this induced
morphism, $\p_b := $ker$(\tilde \varphi) \si k\{X\}$ is a prime $\s$-ideal of $k\{X\}$. The kernel of the mapping constructed this way is always the same for equivalent solutions. To see this, let $b' \in B^n$ such that $k\{b\} \cong k\{b'\}$ via $\iota: b \mapsto b'$.
then it holds for the mapping $\varphi': k\{y_1, \ldots, y_n\} \rightarrow B', y \mapsto b$ that $\varphi' = \iota \circ \varphi$ (which is well-defined since Im$(\varphi)\subseteq k\{b\}$). In particular, since $\iota$ is an isomorphism, ker$(\varphi) = $ker$(\varphi')$. 
We define the mapping $\Psi$ from the equivalence classes of solutions of $F$ to $\sSpec(k\{X\})$ via $b \mapsto \p_b$.\\ 

\indent On the other hand, for $\p \in \sSpec(k\{X\})$, which we identify with $\I(X) \subseteq \tilde \p \in \sSpec(k\{y_1,\ldots,y_n\})$ (see Proposition \ref{bijideals}), consider the integral $\s$-ring $B(\p):= k\{y_1,\ldots,y_n\}/ \tilde \p$.
Since $\tilde \p$ is a prime $\s$-ideal, $B(\p)$ is an integral $\s$-ring. Set $b(\p) := \bar y \in B(\p)$, as the image of $y$ in $B(\p)$. Then, because $F \subseteq \I(X) \subseteq \tilde \p$ we know that $b(\p)$ is a solution of $F$. 
We define $\Psi^{-1}(\p)$ as the equivalence class of $b(\p)$. Then $\Psi$ and $\Psi^{-1}$ are inverses of each other, and hence, are both bijections.
\end{bew}
\end{prop}

From Proposition \ref{bijsols} we see that it is a good idea to concentrate on $\sSpec(k\{X\})$ for a $\s$-variety $X$ over a $\s$-field $k$.
 From here on, we will speak of the ``topology on/of X'' to refer to the topology on $\sSpec(k\{X\})$, as in Definition \ref{deftop}. 
We will also use the convention $x \in X$ to mean $x \in \sSpec(k\{X\})$, or $T \subseteq X$ closed to speak of a closed subset of $\sSpec(k\{X\})$, and so forth.

\subsection{Morphisms of Difference Varieties}

So far we have only studied difference varieties themselves, but not really a way to relate them with each other; we have yet to properly define the category of difference varieties over a fixed $\s$-field $k$: 
we still have to define what the morphisms in this category shall be.

\begin{defn}\label{spolynomialmaps}
Let $k$ be a $\s$-field, $X \subseteq \mathbb{A}^n,Y \subseteq \mathbb{A}^m$ $\s$-varieties over $k$. Then, a morphism of functors $f: X \rightarrow Y$ is called a \emph{morphism of $\s$-varieties over $k$} or \emph{$\s$-polynomial map} if 
there exist $\s$-polynomials $f_1,\ldots,f_m \in k\{y_1,\ldots,y_n\}$ such that $f(b) = (f_1(b),\ldots,f_m(b))$ for all $b \in X(B), B \in \sintk$. \index{morphism of $\s$-varieties} \index{$\s$-polynomial map}
\index{morphism of $\s$-varieties} \index{$\s$-polynomial map}
\end{defn}

\begin{ex}
For two $\s$-varieties $X \subseteq Y = \mathbb{A}^n_k$, over the $\s$-field $k$, the inclusion mapping $\iota: X \hookrightarrow Y$ is a morphism of $\s$-varieties over $k$, since we can choose $f_1 = y_1, f_2 = y_2, \ldots, f_n = y_n$.
Similarly, for $m \geq n$ and $X \subseteq \mathbb{A}^m_k, Y \subseteq \mathbb{A}^n_k$ the ``projection onto $\mathbb{A}^n$'' is also a morphism of $\s$-varieties over $k$ (with the same choice of $f_i$ as the example above).
\end{ex}

\begin{rem}\label{dualmor}
Let $f: X \rightarrow Y$ be a morphism of $\s$-varieties over the $\s$-field $k$, $X \subseteq \mathbb{A}^n, Y \subseteq \mathbb{A}^m$. Then by definition there exist $f_1, \ldots, f_m \in k\{y_1,\ldots,y_n\}$ such 
that $f(b) = (f_1(b),\ldots,f_m(b))$ for all $b \in B, ~ B \in \sintk$. Modulo $\I(X)$, these $f_i$ are unique:
 If there is $f_1', \ldots, f_m' \in k\{y_1,\ldots,y_n\}$ such that $f_i(b) = f'_i(b) \fa b \in B, ~ B \in \sintk,$ and for all $i \in \underline{m}$,
then it follows that $(f_i - f_i')(b) = 0 \fa b \in B, ~ B \in \sintk$, which implies that $f_i - f_i' \in \I(X)$ by definition, for all $i \in \underline{m}$. \\
\indent Now, consider the mapping \[ \phi: k\{z_1,\ldots,z_m \} \rightarrow k\{X\}, ~ z_i \mapsto f_i + \I(X) =: \overline{f_i} \]
This mapping factors over $\I(Y)$, since for $h \in \I(Y) \subseteq k\{z_1,\ldots,z_m\}$, $b \in X(B), ~ B \in \sintk$, we have that 
\[ (\phi(h))(b) = h(\overline f_1(b), \ldots, \overline f_m(b)) = h(f(b)) \]
But since $\phi$ is a morphism of $\s$-varieties over $k$, it follows that $f(b) \in Y(B)$, which implies that $h(f(b)) = 0$, by choice of $h$, hence $h \in $ ker$(\phi)$.
Altogether, this yields a mapping 
\[ f^* : k\{Y\} \rightarrow k\{X\}, ~ z_i + \I(Y) \mapsto y_i + \I(X) \]
This mapping is a morphism of integral $\s$-rings over $k$, and is called the \emph{dual mapping} or \emph{dual morphism} to $f$ \index{dual morphism}. It holds that
\[ f^*(h)(b) = h(f(b)) \fa h \in k\{Y\}, b \in X(B), ~ B \in \sintk. \]
From the definition it follows that for morphisms $X \xrightarrow{f} Y \xrightarrow{g} Z$ of $\s$-varieties over $k$, it holds that $ (f \circ g)^* = g^* \circ f^*$. 
We thus get a contravariant functor $-^*$ from the category of difference varieties over $k$ to $\sringk$.
\end{rem}

\begin{prop}\label{dualisequiv}
Let $k$ be a $\s$-field. Then $-^*$ as defined in Remark \ref{dualmor} is an anti-equivalence between the category of $\s$-varieties over $k$ and the subcategory of $\sringk$ which arises by restricting the object class to reduced, well-mixed, finitely $\s$-generated $\s$-overrings of $k$. 
In particular, a morphism $f: X \rightarrow Y$ of $\s$-varieties over $k$ is an isomorphism if and only $f^*: k\{Y\} \rightarrow k\{X\}$ is an isomorphism.
\begin{bew}
Since for a $\s$-variety $X$ over $k$, $\I(X)$ is radical and mixed, $k\{X\}$ is always a reduced and well-mixed $\s$-overring of $k$, 
and finitely $\s$-generated since $\s$-varieties are defined only for equations with finitely many difference variables. From this it follows that the functor $-^*$ from Remark \ref{dualmor} is well defined. \\
It suffices to show that it is surjective on the skeleton of the categories and bijective on morphisms. 
Let $B$ be a finitely $\s$-generated, well-mixed and reduced $\s$-overring of $k$. We can then write $B \cong k\{y_1,\ldots,y_n\}/\a$, for an $\a \si k\{y_1,\ldots,y_n\}$ radical and mixed. The $\s$-variety $X = \VV(\a) \subseteq \mathbb{A}^n$
is then a preimage of the isomorphism class of $B$, since $\I(X) = \I(\VV(\a)) = \a$, because of Proposition \ref{I=F_m}. Thus, $B \cong k\{X\}$. \\
\indent Now, for the morphisms: First, let $X,Y$ be $\s$-varieties over $k$ and $f,g \in \Hom(X,Y)$ with $f^* = g^*$. Then we know that for every $h \in k\{X\}$, and every $b \in B, ~ B \in \sintk$ it holds that:
\[ h(f(b)) = f^*(h(b)) = g^*(h(b)) = h(g(b)). \]
In particular, $f(b) = g(b) \fa b \in B, ~ B \in \sintk$, which implies that $f = g$, and $-^*$ is injective. 
On the other hand, consider $\varphi: k\{Y\} \rightarrow k\{X\}$ a morphism of $\s$-overrings of $k$. There exist $n,m \in \NE$ such that $X \subseteq \mathbb{A}^n, Y \subseteq \mathbb{A}^m$,
 which means that $k\{X\} = k\{z_1,\ldots,z_n\}/\I(X), k\{Y\} = k\{y_1,\ldots,y_m\}/\I(Y)$. We will construct a preimage of $\varphi$: Choose $f_1,\ldots,f_m \in k\{z_1,\ldots,z_n\}$ such that $\varphi(y_i + \I(Y)) = f_i + \I(X) \fa i \in \underline{m}$.
Then we define a morphism $f: X \rightarrow Y$ of $\s$-varieties over $k$ as follows: $f(b) := (f_1(b),\ldots,f_m(b)) \fa b \in B, ~ B \in \sintk$. This is well-defined: Let $h \in \I(Y)$. Then, by definition, $h(y_1 + \I(Y),\ldots,y_n + I(Y)) = 0 + \I(Y)$.
This implies that $h(f_1 + \I(X),\ldots,f_m + \I(X)) = 0 + \I(X)$, since $\varphi$ is a morphism of $\s$-overrings of $k$. But this in turn implies that $h(f(b)) = 0 \fa b \in X(B), ~ B \in \sintk$, which means that $f$ maps indeed onto $Y$ and $f^* = \varphi$ by construction.
\end{bew}
\end{prop}

This gives us a pretty good idea about the importance of the coordinate ring in difference algebra.
Having defined a category for $\s$-varieties, we can now see how this new category-theoretic language helps us better understand the topological aspects of difference varieties.

\begin{lem}\label{inducedcont}
Let $R,S,T$ be $\s$-rings, and $\varphi: R \rightarrow S, \psi: S \rightarrow T$ morphisms of $\s$-rings. Then the mapping $$\tilde \varphi: \sSpec(S) \rightarrow \sSpec(R), \p \mapsto \varphi^{-1}(\p)$$ 
induced by $\varphi$ is continuous. 
In fact, it holds that $\widetilde{ \psi \circ \varphi} = \tilde \varphi \circ \tilde \psi$, and in particular, $R \mapsto \sSpec(R)$ with $\psi \mapsto\tilde \psi$ is a contravariant functor from the category of $\s$-rings to \Top, the category of topological spaces.
\begin{bew}
Let $A = \V(F) \subseteq \sSpec(R)$ be closed. We have to show that $\tilde \varphi^{-1}(A) \subseteq \sSpec(S)$ is closed.
But 
\begin{align*} \tilde \varphi^{-1}(A) = \tilde \varphi^{-1}(\V(F)) = \{ \p \in \sSpec(S) \mid F \subseteq \varphi^{-1}(\p) \} \\ = \{\p \in \sSpec(S) \mid \varphi(F) \subseteq \p \} = \V(\varphi(F)). \end{align*}
That $\widetilde{ \psi \circ \varphi} = \tilde \varphi \circ \tilde \psi$ is immediately clear from definition.
\end{bew}
\end{lem}

We see thus how radical, mixed $\s$-ideals and the definition of $\s$-varieties as functors from $\sintk$, for an integral $\s$-ring $A$, as well as the topology on $\sSpec(A\{X\})$ all fit together well. 
These are all in analogous relations to the case for perfect $\s$-ideals, where $\s$-varieties are defined from the category of $\s$-overfields of a $\s$-field $k$, and a topology called the Cohn topology is defined on $\Spec^\s(k\{X\})$ of $\s$-prime ideals (see Ch. 1 \& 2 of \cite{wibmer}).
We will try to shed some light on the choice of the category $\sintk$ here:

\begin{defn}
Let $k$ be a $\s$-field. 
\begin{enumerate}[(a)]
\item We denote by $\s\text{\catname{-VarField}}_k$ the category which has functors of the form $B \mapsto \VV_B(F)$ as objects, where $B$ is a finitely $\s$-generated $\s$-field extension of $k$,
 and as morphisms $\s$-polynomial maps defined in a fashion analogous to Definition \ref{spolynomialmaps}. We define $\I_{\operatorname{Field}}(X)$ for $X \in \s\text{\catname{-VarField}}_k$ and $\VV_{\operatorname{Field}}(F)$ analogous to Definitions \ref{defnVV} and \ref{defnI}.
\item Similarly, we denote by $\s\text{\catname{-VarDomain}}_k$ the category which has functors of the form $B \mapsto \VV_B(F)$ as objects, where $B$ is a finitely $\s$-generated $\s$-domain extension of $k$,
 and as morphisms $\s$-polynomial maps defined in a fashion analogous to Definition \ref{spolynomialmaps}. Again we define $\I_{\operatorname{Domain}}(X)$ for $X \in \s\text{\catname{-VarDomain}}_k$ and $\VV_{\operatorname{Domain}}(F)$ analogous to Definitions \ref{defnVV} and \ref{defnI}
\item Finally,  we denote by $\s\text{\catname{-VarRing}}_k$ the category which has functors of the form $B \mapsto \VV_B(F)$ as objects, where $B \supseteq k$ is a perfectly $\s$-reduced, finitely $\s$-generated ring over $k$,
 and as morphisms $\s$-polynomial maps defined in a fashion analogous to Definition \ref{spolynomialmaps}. We also define $\I_{\operatorname{Ring}}(X)$ for $X \in \s\text{\catname{-VarRing}}_k$ and $\VV_{\operatorname{Ring}}(F)$ analogous to Definitions \ref{defnVV} and \ref{defnI}
\end{enumerate}
In all three cases $F \subseteq k\{y\}$ denotes a set of $\s$-polynomials on finitely many difference variables $y = y_1, \ldots, y_2$.
\end{defn}

\begin{prop}
Let $k$ be a $\s$-field. The three categories $\s\text{\catname{-VarField}}_k$, $\s\text{\catname{-VarDomain}}_k$ and $\s\text{\catname{-VarRing}}_k$ are equivalent.

\begin{bew}
Similar to Proposition \ref{dualisequiv}, the category $\s\text{\catname{-VarField}}_k$ is anti-equivalent to the category of perfectly $\s$-reduced $\s$-overrings of $k$ which are finitely $\s$-generated over $k$ (see \cite{wibmer}, p. 30).
It suffices to show that the other two categories are also anti-equivalent to it. From the proof of Proposition \ref{dualisequiv} we can see that it is enough to show that $\I_{\operatorname{Domain}}(\VV_{\operatorname{Domain}}(\a)) = \{\a\}$, and $\I_{\operatorname{Ring}}(\VV_{\operatorname{Ring}}(\a)) = \{\a\}$,
 where $\a \si k\{y\}$ is a $\s$-ideal and $\{ \a \}$ its perfect closure. 

Let first $X  = \VV_{\operatorname{Ring}}(F) \in \s\text{\catname{-VarRing}}_k$ be a $\s$-variety in this sense of perfectly reduced $\s$-fields. 
We first show that \begin{align*} \I_{\operatorname{Ring}}(X) =  \{ f \in k\{y\} \mid f(b) = 0 \fa b \in \VV_B(F), \\ B \supseteq k\text{ perfectly }\s\text{-reduced and finitely }\s\text{-generated over }k \} \end{align*}
 is a perfect $\s$-ideal. Similar to Proposition \ref{I=F_m}, we know that $\I_{\operatorname{Ring}}(X)$ is a difference ideal. Let $f \in k\{y\}$ with $\s^{i_1}(f) \cdots \s^{i_r}(f) \in \I(X)$. This means that for all $b \in V_B(F)$,
 $B$ perfectly $\s$-reduced and finitely $\s$-generated over $k$: $\s^{i_1}(f)(b) \cdots \s^{i_r}(f)(b) = 0$. Since $B$ is perfectly $\s$-reduced, this means that $f(b) = 0$ for all such $b$,
 which in turn, by definition, means that $f \in \I_{\operatorname{Ring}}(X)$. Since every $\s$-domain is perfectly $\s$-reduced, the argument works the same for $X \in \s\text{\catname{-VarDomain}}_k$ with $\I_{\operatorname{Domain}}$ instead of $\I_{\operatorname{Ring}}$.

For the other inclusion we shall consider first $X \in \s\text{\catname{-VarDomain}}_k$.
For $F \subseteq k\{y\}$, it holds that $\{F\} \subseteq \I_{\operatorname{Domain}}(\VV_{\operatorname{Domain}}(F))$. To show this, let $f \in \I_{\operatorname{Domain}}(\VV_{\operatorname{Domain}}(F))$.
It holds that $\{F\}$ is the intersection of all $\s$-prime ideals of $k\{y\}$ which contain $F$ (see for example Proposition 1.2.22 of \cite{wibmer}), so it is enough to show that $f \in \p$ for each $\s$-prime $\p \si k\{y\}$ with $F \subseteq \p$.
We define the $\s$-domain $B:= k\{y\}/\p =: k\{a\}$, with $a := y + \p \in k\{y\}/\p$. Since $F \subseteq \p$, it holds that $a \in \VV_B(F)$, which, by definition of $\I_{\operatorname{Domain}}(\VV_{\operatorname{Domain}}(F))$ means that $f(a) = 0$. Hence, $f \in \p$, for all $\s$ prime $\p$
with $F \subseteq \p$, which in turn implies that $f \in \{F\}$. Since every $\s$-domain $B$ is perfectly $\s$-reduced, this works for the category $\s\text{\catname{-VarRing}}_k$ as well, with $\I_{\operatorname{Ring}}, \VV_{\operatorname{Ring}}$ instead of $\I_{\operatorname{Domain}}, \VV_{\operatorname{Domain}}$.
\end{bew}

\end{prop}
%% \begin{rem}
%% Let $f: X \rightarrow Y$ be a morphism of $\s$-varieties over the integral $\s$-ring $A$. Then the morphism $f^*: A\{y\} \rightarrow A\{x\}$ of $-s$-overrings of $A$ induces a continuous function
%% \[ \tilde{(f^*)}: \sSpec(A\{X\}) \rightarrow \sSpec(A\{Y\}), M \mapsto (f^*)^{-1}(M) \]
%% On the other hand FIXME: finish!
%% as in Lemma \ref{inducedcont}
%% \end{rem}

%% \begin{ex}
%% 2.4, here:more points? + 2.2.5

%% \end{ex}

%% section 2.3 not necesarry!

\clearpage 
\section{Difference Kernels}
%%blah blah de entrada...
%% \begin{theorem}

%% Let $k$ be a $\s$-field and $a$ 
%% \end{theorem}
In this section we will see a further approach to investigating difference ideals. We will take a look at the properties a difference ideal has by looking at its intersection with the variables up to a finite power of sigma.

\begin{defn}
Let $k$ be a difference field and $d \in \N$. Then we define $k\{y\}[d]:= k[y,\s(y),\ldots,\s^d(y)]$ and we set $k\{y\}[-1] := k$. Let $\a \si k\{y\}$ be a difference ideal in the full $\s$-polynomial ring $k\{y\}$. 
We set $\a[d] := \a \cap k\{y\}[d]$.
\end{defn}


\begin{defn}
Let $\a \unlhd k\{y\}[d], d \geq 1$ be an ideal of $k\{y\}[d]$. Then $\a$ is called a \emph{difference kernel of length $d$}, if $\s(\a[d-1]) \subseteq \p$. The difference kernel is called \emph{prime}, if additionally $\a$ is a prime ideal of $k\{y\}[d]$.
Finally, $\a$ is called a \emph{reflexive difference kernel of length $d$} if $\s^{-1}(\a) = \a[d-1]$. \index{difference kernel} \index{prime difference kernel} \index{reflexive difference kernel}
\end{defn}\index{difference kernels}

It is worth noting that this notation differs from that of the standard literature (in particular \cite{cohn} and \cite{levin}). In the standard literature, the quotient ring obtained by factoring out 
what we here defined as a prime, reflexive kernel is called a kernel, and the rest of the concepts are not treated. We chose to change the notation here to be able to study other $\s$-ideals with this methodology.

\begin{rem}
It is easy to see that a reflexive kernel is always a kernel, but the converse is not necesarilly true: $\a$ being a kernel of length $d$ only guarantees the inclusion $\a[d-1] \subseteq \s^{-1}(\a)$.
\end{rem}

\begin{ex}
Let $\p \si k\{y\}$ be a prime $\s$-ideal, $d \geq 1$. Then $\p[d] \unlhd k\{y\}[d]$ is a prime kernel: since $\p$ is a $\s$-ideal we have $\s(\p[d-1]) \subseteq \p$, 
and since $\p$ is prime we know that $\p[d]$ has to be prime as well. If $\p$ is also reflexive (i.e. a $\s$-prime ideal), then $\p[d]$ is a reflexive, prime kernel. 
\end{ex}

Inspired by the former example we define the following:
\begin{defn}
Let $\a \unlhd k\{y\}[d]$ be a kernel of length $d$, and $\a' \si k\{y\}$ be a $\s$-ideal. Then we call $\a'$ a realization of $\a$, if $\a \subseteq \a'$, 
and we call the realization regular if it also holds that $\a'[d] = \a$. Similarly, if $\a$ is a reflexive/prime kernel, then we define a realization to be a reflexive/prime realization if the $\s$-ideal $\a'$ is reflexive/prime.
\end{defn}\index{(regular) realization}

The former example and the acompanying definition are actually much more general than it would seem to be at first. 
It is in fact the case that prime,reflexive kernels are always of the form $\p[d]$ for a $\s$-prime ideal $\p$.
For prime kernels which are not reflexive in general it is not that simple, as we will also see later. 

\begin{rem}\label{sigmawelldeffker}
Let $\p \subseteq k\{y\}$ be a difference kernel of length $d$. Then, $\s$ induces a well-defined mapping 
\begin{align*}
\s: k[y,\ldots,\s^{d-1}(y)]/\p[d-1] \rightarrow k[y,\ldots,\s^{d}(y)]/\p
\end{align*}
If $\p$ is a reflexive kernel, then this mapping is injective. 
We set $a := \bar y = y + \p \in k\{y\}[d]/\p$. If $\p$ is a prime, reflexive kernel, then we can extend $\s$ to the quotient fields:
\[ \s: k(a,\ldots,\s^{d-1}(a)) \cong \text{Quot}(k[y,\ldots,\s^{d-1}(y)]/\p[d-1]) \rightarrow k(a,\ldots,\s^d(a)) = k(\p) \]
\end{rem}

Even in the case of prime difference kernels we can work with the properties of field extensions, though we cannot properly extend $\s$ to the fields.
 In particular, we get a nice way of defining some sort of ``difference degree'' of prime kernels:
\begin{defn}
Let $\p \subseteq k\{y\}$ be a prime difference kernel of length $d$, and let $a:= y + \p \in k(p)$; we define the difference dimension of $\p$ as follows:
\begin{align*} \sdim(\p) := \trdeg(k(\p)/k(\p[d-1]) = \trdeg((k\{y\}[d]/\p)/(k\{y\}[d-1]/\p[d-1])) \\  = \trdeg(k(a,\ldots,\s^{d}(a))/k(a,\ldots,\s^{d-1}(a))) \end{align*}
\end{defn}\index{$\s$-dimension of a $\s$-kernel}

If we want to show that prime, reflexive difference kernels are the intersection of $\s$-prime ideals with $k\{y\}[d]$, and similar cases for other types of kernels,
it is reasonable to consider some sort of extension, or \emph{prolongation} of a kernel, which would be the intesrection with $k\{y\}[d+1]$.  \index{prolongation (of a difference kernel)}
This motivates the following definition: 

\begin{defn}
Let $\a \subseteq k\{y\}$ be a difference kernel of length $d$, and $\a' \supset \a$ be a further difference kernel, of length $d+1$.
Then we call $\a'$ a \emph{prolongation} of $\a$, if $\a'[d] = \a$. Similarly, for a prime/reflexive kernel we say the prolongation is prime/reflexive if, $\a'$ is also prime/reflexive.
For a prime prolongation, if it holds further that $\sdim(\a) = \sdim(a')$, then we call the prolongation \emph{generic}.
\end{defn}\index{generic prolongation}

\begin{defn}
Let $\p \subseteq k\{y\}$ be a prime difference kernel of length $d$, and let $\p'$ be a realization of $\p$. We call $\p'$ a \emph{principal realization} of $\p$, if $\p'[i+1]$ is a generic prolongation of $\p'[i]$ for all $i \geq d$,
and we say the realization is \emph{reflexive} if the prolongations are reflexive. \index{principal realization} \index{reflexive realization}
\end{defn}

As mentioned above, the case for reflexive, prime kernels is well understood. It can be summarized by the following Theorem, which can be found as Corollary 5.2.8 of \cite{wibmer}:
\begin{theorem}
Let $\q \subseteq k\{y\}$ be a prime, reflexive kernel of length $d$. Then there exists a reflexive, principal realization of $\q$. 
\end{theorem}

The case is unfortunately not as simple in the case of prime kernels that are not necessarily reflexive, as shown by the following example.

\begin{ex}\label{counterexker}
Consider the difference polynomials $f_1 := \s(y_2) + 1, f_2:= \s(y_1)y_2 + y_1y_2 + \s(y_1) + y_1 + y_2 \in k[y_1,y_2,\s(y_1),\s(y_2)] =: R$.
A calculation with \cite{M2} quickly identifies that the ideal $(f_1,f_2) \unlhd R$ is prime, and in fact, $f_1,f_2$ is a Gr\"{o}bner Basis of $(f_1,f_2)$ with respect to the
term ordering $y_1 < y_2 < \s(y_1) < \s(y_2)$. This means in particular, that $(f_1,f_2)[0] = \{0\}$, and thus $(f_1,f_2)$ is a prime kernel of length $1$ of $k\{y_1,y_2\}$. However,
\[ f_1 \cdot (\s^2(y_1) + \s(y_1) + 1 ) - \s(f_2) = 1 \]
Which means that any $\s$-ideal of $k\{y_1,y_2\}$ containing $f_1, f_2$ is already the whole ring, and is not prime, by definition.
\end{ex}

As the example shows, prime kernels are not always the intersection of a prime difference ideal with $k\{y\}[d]$. 
If we want to try and classify those prime kernels that are in fact such intersections, we can try and find necessary conditions to be able to find a prolongation of a kernel. 
We will develop a condition that is necessary but not sufficient. First, however, we need a Lemma to help prove this.

\begin{lem}\label{primeoverp1}
Let R be an integral domain and $I \unlhd R[y]$ be an ideal of $R[y] = R[y_1,\ldots,y_n]$ which satisfies that $I \cap R = \{ 0 \}$.
Then there exists a prime ideal $P$ with $I \subseteq P \subseteq R[y] $ such that $P \cap R = \{0\}$.
\begin{proof}
We can assume without loss of generality that I is radical:
Namely, if $f \in \sqrt{I} \cap R$, then there exists an $m \in \N$ such that $f^m \in I \cap R = \{0\}$, and since $R$ is an integral domain this already means that $f = 0$.
We then note that for two sets $A,B \subseteq R[y]$ it holds that $\sqrt{A}\sqrt{B} \subseteq \sqrt{AB}$: Consider $f \in \sqrt{A}, g \in \sqrt{B}$. Then there exist $m, \tilde m \in \N$ such that $f^m \in (A), g^{\tilde m} \in (B)$;
 assume without loss of generality that $m > \tilde m$, then $(fg)^m \in (A)(B)$, which implies $fg \in \sqrt{(A)(B)} = \sqrt{AB}$.
Now, for the proof, consider the set of all radical ideals $J$ contaning $I$ which satisify $J \cap R = \{0\}$. This set is not empty and is inductively ordered by inclusion.
By Zorn's lemma this means that there is a maximal element $P$ of this set. This ideal $P$ is prime, then: assume there exist $f,g \notin P$ with $fg \in P$. 
Then the radical ideals $\sqrt{P \cup \{f\}}$, $\sqrt{P \cup \{f\}}$ strictly include $P$, and by the maximality of $P$ it means there exist $t_1, t_2 \in R\backslash\{0\}$ such that
$t_1 \in \sqrt{P \cup \{f\}}$, $t_2 \in \sqrt{P \cup \{g\}}$. But in particular, because $R$ is free of zero divisors, this implies that
 \[0 \neq t_1t_2 \in \sqrt{P \cup \{f\}}\sqrt{P \cup \{g\}} \subseteq \sqrt{ \underbrace{(P \cup \{f\})(P \cup \{g\})}_{=P\text{, since }fg \in P}} = P\]
A contradiction.
\end{proof}
\end{lem}


%% \begin{lem}\label{idealstill0}
%% Let $R \subseteq S$ be two rings, and let $I \unlhd R[y]$ be an ideal in the polynomial ring $R[y]$ with $I \cap R = 0$. 
%% Then it holds for the ideal $(I) \unlhd S[y]$, that $(I) \cap S = 0$.
%% \begin{proof}
%% Since $S$ is an $R$ module, we know that $S \cong R \otimes_R S$, from which it easily follows that $S[y] \cong R[y] \otimes_R S$, and similarly, that $I \otimes_R S \cong (I) \unlhd S[y]$.
%% Together, these two imply that as well $R[y]/I \otimes_R S \cong S[y]/(I)$. FIXME: check this!
%% The condition $R \cap I = 0$ is equivalent to the mapping $R \rightarrow R[y]/I, r \mapsto r + I$ being injective. We can express this as the exactness of the following sequence:
%% \[ 0 \rightarrow R \rightarrow R[y]/I \]
%% Since the tensor product functor is exact, we know that tensoring over $R$ with $S$ yields an exact sequence:
%% \[ 0 \rightarrow R \otimes_R S \cong S \rightarrow R[y]/I \otimes_R S \cong S[y]/(I) \]
%% This, in turn, is equivalent to $S \cap (I) = 0$ by going the arguments above backwards.

%% \end{proof}
%% \end{lem}

 
\begin{prop}
Let $\a \subseteq k[y,\ldots,\s^d(y)]$ be a prime difference kernel of length $d$ and let $k[y,\ldots,\s^d(y)]/\a =: k[a,\s(a),\ldots,\s^d(a)]$. Consider the mapping 
\[ \s: k[a,\ldots,\s^{d-1}(a)] \rightarrow k[a,\ldots,\s^d(a)]. \]
Assume that for $(\operatorname{ker}(\s)) \subseteq k[a,\ldots,\s^d(a)]$ it holds that $(\operatorname{ker}(\s)) \cap k[a,\ldots,\s^{d-1}(a)] = \operatorname{ker}(\s)$. 
Then there exists a prime difference kernel $\tilde \a \subseteq k[y,\ldots,\s^{d+1}(y)]$ of length $d+1$ with $\tilde \a \cap k[y,\ldots,\s^d(y)] = \a$
\begin{bew}
Consider the surjective mapping 
\[ k[a,\ldots,\s^{d-1}(a)][\s^d(y)] \rightarrow k[a,\ldots,\s^d(a)], \s^d(y) \mapsto \s^d(a) \]
Let $\p_1$ be the kernel of this mapping. Since $k[a,\ldots,\s^d(a)]$ is a domain, we know by the fundamental theorem on homomorphisms that $\p_1$ has to be prime. 
First, we will show that our assumption on $\operatorname{ker}(\s)$ is equivalent to:
\begin{equation}\label{skercapsk} \s(\p_1) \cap \s(k)[\s(a),\ldots,\s^d(a)] = 0 \end{equation}
To see this consider the following commutative diagram:
\[
\begin{xy}
 \xymatrix{
      k[a,\ldots,\s^{d-1}(a)][\s^d(y)] \ar[rr]^\s \ar@{->>}[rd]^\s  &     &  k[a,\ldots,\s^d(a)][\s^{d+1}(y)]   \\
      &  \s(k)[\s(a),\ldots,\s^d(a)][\s^{d+1}(y)] \ar@^{(->}[ur] &  }
\end{xy}
\]

We can factor out $\p_1$ and its image, $\s(\p_1)$. Since $k[a,\ldots,\s^{d-1}(a)][\s^d(y)] / \p_1 \cong k[a,\ldots, \s^d(a)]$,
we get a surjectve mapping $k[a,\ldots, \s^d(a)] \twoheadrightarrow \s(k)[\s(a),\ldots,\s^d(a)][\s^{d+1}(y)]/\s(\p_1)$.
If we factor out the kernel, we get an isomorphism:
\[ k[a,\ldots, \s^d(a)]/(\operatorname{ker}(\s)) \cong \s(k)[\s(a),\ldots,\s^d(a)][\s^{d+1}(y)]/\s(\p_1)\]
On the other hand, we have the canonical embedding $k[a,\ldots,\s^{d-1}(a)] \hookrightarrow k[a,\ldots,\s^{d}(a)]$.
This mapping stays injective after factoring out the kernel of $\s$ on both sides, 
i.e. $k[a,\ldots,\s^{d-1}(a)]/\operatorname{ker}(\s) \hookrightarrow k[a,\ldots,\s^{d}(a)]/(\operatorname{ker}(\s))$ is injective, if and only if our assumption on $\ker(\s)$ holds.
In this case, by composition with the isomorphism above, we get an embedding:
\[ k[a,\ldots,\s^{d-1}(a)]/\operatorname{ker}(\s) \hookrightarrow \s(k)[\s(a),\ldots,\s^d(a)][\s^{d+1}(y)]/\s(\p_1) \]
This is injective if and only if Equation \ref{skercapsk} holds.
%Now, by Lemma \ref{idealstill0} 
Equation \ref{skercapsk} implies \textcolor{red}{(WHY??)} that
\begin{equation}\label{skercapk}
(\s(\p_1)) \cap k[a,\s(a),\ldots,\s^d(a)][\s^{d+1}(y)] = 0
\end{equation}
We will now construct a prime difference kernel $\tilde \a$ using Equation \ref{skercapsk}:

By Lemma \ref{primeoverp1} there exists a minimal prime ideal $\p_2 \supset (\s(\p_1))$ of $k[a,\ldots,\s^d(a)][\s^{d+1}(y)]$ containing $\s(p_1)$ with $\p_2 \cap k[a,\ldots,\s^d(a)] = \{0\}$. 
We thus get a well-defined maping
\[ \s: k[a,\ldots,\s^{d-1}(a)][\s^d(y)]/\p_1 \rightarrow k[a,\ldots, \s^d(a)][\s^{d+1}(y)]/\p_2 \]
We define $R_2:= k[a,\ldots,\s^d(a)][\s^{d+1}(y)]/\p_2 =: k[a,\ldots,\s^d(a),\s^{d+1}(a)]$, which is an integer domain since $\p_2$ is prime. Since $\p_2 \cap k[a,\ldots,\s^d(a)] = 0$ we can use this notation unambiguosly:
this guarantees namely that for $a, \ldots, \s^d(a)$ we have the same residue classes modulo $\p_2$ as we had modulo $\a$.
The kernel $\tilde \a$ of the natural epimorphism $k[y,\ldots,\s^{d+1}(y)] \rightarrow R_2$ is thus a prime ideal.
Further we have $\a \subseteq \tilde \a$ by construction (as $\a = 0 \subset R_2$). In fact, it holds that $\tilde \a[d] = \p$ since: 
\begin{align*}
\tilde \a[d] = \{ f \in k[y,\ldots,\s^d(y)] \mid f(a) = 0 \} = \ker( k\{y\}[d] \rightarrow k[a,\ldots,\s^{d}(a)]) = \a
\end{align*}
where the first equality uses the fact that $\p_2 \cap k[a,\ldots,\s^d(a)] = 0$, as noted by the use of the notation explained above, and last equality holds by definition of $a \in k[a,\ldots,\s^d(a)]$. This means, that $\tilde \a$ is a prolongation of $\a$. 
\end{bew}
\end{prop}

The condition $(\operatorname{ker}(\s)) \cap k[a,\ldots,\s^{d-1}(a)] = \operatorname{ker}(\s)$ is thus necessary for finding principal realizations of prime kernels. 
That this the case is not surprising once you consider the case for reflexive prime kernels, since the triviality of $\operatorname{ker}(\s)$ is implied by the prime difference kernel being reflexive.


\begin{ex}
Let $k$ be a $\s$-field and consider the $\s$-polynomial ring $k\{y_1,y_2\}$. 
The ideal $$\a = (\s^2(y_2)+1,\s^2(y_1)y_2 + \s(y_1)y_2 + \s^2(y_1) + \s(y_1) + y_2) \unlhd k\{y_1,y_2\}[2]$$ is a prime kernel,
as can be verified by a calculation using \cite{M2}, for which it holds that if we consider the factor ring 
$k\{y\}[2]/\a =: k\{a,\s(a),\s^2(a)\}$, the kernel of $\s: k\{a,\s(a)\} \rightarrow k\{a,\s(a),\s^2(a)\}$
satisfies that $$(ker(\s)) \cap k\{a,\s(a)\} = ker(\s).$$ However, similarly to the case in Example \ref{counterexker}, any difference ideal of $k\{y\}$ containing 
$\a$ has to contain $1$ and thus be already the whole ring $k\{y\}$.
\end{ex}

We see then, that the condition $(\operatorname{ker}(\s)) \cap k[a,\ldots,\s^{d-1}(a)] = \operatorname{ker}(\s)$, while sufficient for finding a single generic prolongation,
is not sufficient for finding a principal realization. This motivates the following conjecture:

\begin{conj}
Let $k$ be a $\s$-field and $\a \subseteq k\{y\}[d]$ a prime $\s$-kernel of length $d$.
And let $k\{y\}[d]/\a \cong k\{a\}$, i.e., let $a_1 = y_1 + \a ,\ldots,a_n = y_n + \a$.
For $r \geq 1$ consider the mapping 
\[ \s^r : k[a,\s(a),\ldots,s^{d-r}(a)] \rightarrow k[a,\ldots,\s^{d}], f \mapsto \s^r(f) \]
Assume that
\begin{equation}\label{conditionkernels} (\operatorname{ker}(\s),\ldots,\operatorname{ker}(\s^r)) \cap k[a,\ldots,\s^{d-r}(a)] = \operatorname{ker}(\s^r) \end{equation}
Then there exists a principal realization of $\a$.
\end{conj}

Since in $k\{y\}$ it holds that $\operatorname{ker}(\s^r) \subseteq \operatorname{ker}(\s^{r+1})$, this condition,
Equation \ref{conditionkernels}, is a necessary condition for the existence of a principal realization. 



        %% @Misc{M2,
        %%   author = {Grayson, Daniel R. and Stillman, Michael E.},
        %%   title = {Macaulay2, a software system for research 
        %%            in algebraic geometry},
        %%   howpublished = {Available at 
        %%       \href{http://www.math.uiuc.edu/Macaulay2/}%
        %%            {http://www.math.uiuc.edu/Macaulay2/}}
        %% }


\subsection{A Geometric Application of Difference Kernels}

There is an application of Difference Kernels that is of geometric nature. We will just quote a result for the case of reflexive prime kernels here,
and do a partial generalization for the case that arises for some non-necesarilly-reflexive prime $\s$-ideals.

\begin{theorem}\label{di=d(i+1)+e}
Let $k$ be a $\s$-field and let $\p \in k\{y\} = k\{y_1,\ldots,y_n\}$ be a prime $\s$-ideal of $k\{y\}$. For $i \in \N$ set $$d_i := \dim(k\{y\}[i]/\p[i]).$$
Then there exists integers $d, e \in \N$ such that $d_i = d(i+1) + e$ for $i \gg 0$. Moreover, $d = \s\operatorname{-trdeg}(k\{y\}/\p^*)$.
\begin{bew}
See Theorem 5.1 of \cite{wibmer}.
\end{bew}
\end{theorem}

\begin{defn}
Let $k$ be a $\s$-field and $\p \in k\{y_1,\ldots,y_n\}$ be a prime $\s$-ideal. Further let $d, e \in \N$ as in Theorem \ref{di=d(i+1)+e}. We call $d$ the $\s$-dimension of $\p$, 
or the $\s$-dimension of the irreducible $\s$-variety $X:= \VV(\p)$ and denote it by $\s$-$\dim(\p)$ and $\s$-$\dim(X)$ respectively.
\end{defn}

Recall from the introductory Chapter, Theorem \ref{irredcomp}:
\begin{reptheorem}{irredcomp}
Let $k$ be a $\s$-field and $f \in k\{y_1,\ldots,y_n\}, f \notin k$ an irreducible $\s$-polynomial such that $\operatorname{Eord}(f) = \operatorname{Ord}(f)$. Then $\VV_{\operatorname{Field}}(f) \subset \mathbb{A}_k^n$ has an irreducible component $X$ such that $\s$-$\dim(X) = n-1$ and $\s$-$\operatorname{deg}(X) = \operatorname{Ord}(f)$.
\end{reptheorem}

\begin{cor}\label{corfinal}
Let $k$ be a $\s$-field and $f \in k\{y_1,\ldots,y_n\}, f \notin k$ an irreducible $\s$-polynomial. 
Then $\VV_m(f) \subset \mathbb{A}_k^n$ has an irreducible closed subset $X$ such that $\s$-$\dim(X) = n-1$ and $\s$-$\operatorname{deg}(X) = \operatorname{Ord}(f)$.
\begin{bew}
Consider the $\s$-polynomial ring $k\{\s^d(y)\} =: k\{z\}$, where $d$ is maximal, such that $f \in k\{\s^d(y)\}$ (i.e., $d = \operatorname{Ord}(f) - \operatorname{Eord}(f)$). We let $f'\in k\{z\}$ be the $\s$-polynomial corresponding to $f$. 
Since $f$ is irreducible, $f'$ has to be irreducible as well. Additionally, by definition of $d$, it has to hold that $\operatorname{Eord}_z(f') = \operatorname{Ord}_z(f')$, where the order is considered with respect to $z$,
as indicated by the subscript. 

By Theorem \ref{irredcomp} we know that the $\s$-variety $ \VV_{\operatorname{Field}}(f')$ has an irreducbile component $X' = \VV(\p)$ of $\s$-dimension $n-1$ and $\s$-$\operatorname{deg} = \operatorname{Ord}_z(f')$. In other words, there exists a $\s$-prime difference ideal $\p \si k\{z\}$ minimal over $\{f'\} \si k\{z\}$ with
$\s$-$\dim(\p) = n-1$ and $\s$-$\operatorname{deg}(\p) = \operatorname{Ord}_z(f')$. 

Now, we assert that the ideal $(\p)$ generated by $\p$ in $k\{y\}$ is a prime difference ideal. 
To prove this, let $f = \sum_{i=1}^r f_i p_i \in (\p)$, with $p_i \in \p; f_i \in k\{y\}$ for all $i$. Then, since $\p$ is a difference ideal, it has to be that $\s(p_i) \in \p$ for all $i$.
This, in turn, implies that $\s(f) =  \sum_{i=1}^r \s(f_i) \s(p_i) \in (\p)$. Thus, $(\p)$ is a difference ideal in $k\{y\}$. 

Now we can show the primality of $(\p)$. As rings, it holds that $$k\{y\} \cong k[y,\ldots,\s^{d-1}(y)] \otimes_k k\{z\}.$$
From this it follows that 
\begin{equation}\label{isomoduloideal} (k[y,\ldots,\s^{d-1}(y)] \otimes_k k\{z\}) / (\p) \cong k\{z\} / \p \otimes_k k[y,\ldots,\s^{d-1}(y)]. \end{equation}
This last statement is proven in detail in the proof of Proposition \ref{idealzeroabove} (NOTE: I would split that off as a lemma and make a reference to the lemma, not to the proof of the proposition).
Since $\p$ is prime, $k\{z\}/\p$ is an integral domain, and thus $(k\{z\}/\p) \otimes_k k[y,\ldots,\s^{d-1}(y)]$ as well.
Equation (\ref{isomoduloideal}) then implies that $(\p) \si k\{y\}$ is prime as well. 

We only have to compare the dimension polynomials of $\p$ (over $k\{z\}$) and $(\p)$ over $k\{y\}$.
For this, we first consider the isomorphism
\begin{align*} \varphi: k\{y\} \xrightarrow{\sim} k[y,\ldots,\s^{d-1}(y)] \otimes_k k\{z\}, \\ (y,\ldots,\s^{d-1}(y)) \mapsto (y,\ldots,\s^{d-1}(y)); \s^d(y) \mapsto z. \end{align*}
For any $f \in k\{\s^{d}(y)\}$ we have the property that $\operatorname{Ord}_y(f) = \operatorname{Ord}_z(\varphi(f)) + d$ (where $\operatorname{Ord}_z$ is only defined on the subset $\varphi(k\{\s^d(y)\}) = k\{z\}$ of $k\{z\}~\otimes_k~k[y,\ldots,\s^{d-1}(y)]).$
In particular, this means that $\varphi$ commutes with the opreation of restricting to the $i$-th power of $\s$, in the following sense:
For a subset $F \subseteq k\{y\}$ and $i > d$ it holds that
\begin{align*} \varphi(F[i]) = \varphi(k\{y\}[i] \cap F) = \varphi(k\{y\}[i]) \cap \varphi(F) \\
 = (k[y,\ldots,\s^{d-1}(y)] \otimes_k k\{z\}[i-d]) \cap \varphi(F) .\end{align*}
If we apply this to $\p$, we get that for $i\geq 0$, $\p[i] = \varphi((\p)[i+d])$, and thus 
$$ k\{y\}[i+d]/(\p)[i+d] \cong \underbrace{k[y,\ldots,\s^{d-1}(y)]}_{ \text{dim} = d \cdot n} \otimes_k k\{z\}[i]/\p[i]$$
This implies that 
$$\operatorname{dim}(k\{y\}[i+d]/(\p)[i+d]) = \operatorname{dim}(k\{z\}[i]/\p[i]) + nd.$$
By definition of the $\s$-$\operatorname{deg}$ and $\s$-$\operatorname{dim}$ we can conclude that 
\begin{align*}
(i+d+1) \s\text{-}\operatorname{dim}( (\p) ) = \omega_{(\p)}(i+d) = \omega_\p(i) + dn \\
= (i+1) \underbrace{\s\text{-}\operatorname{dim}(\p)}_{ = n - 1} + \underbrace{\s\text{-}\operatorname{deg}(\p)}_{\operatorname{Ord}(f')} + dn
\end{align*}
It follows that 
$$\s\text{-}\operatorname{dim}((\p))= \s\text{-}\operatorname{dim}(\p) = n-1 $$
as well as
$$\s\text{-}\operatorname{deg}((\p))= \s\text{-}\operatorname{deg}(\p) + \d = \operatorname{Ord}(f') + d = \operatorname{Ord}(f)$$
Since $f \in (\p)$ and $(\p)$ is a prime difference ideal, the inclusion $\{f\}_m \subseteq (\p)$ is obvious, which in terms of the $\s$-m-varieties means that $\VV_m(\p) \subset \VV_m(f)$. 

%% We only have to show the minimality.
%% For this, assume there exists a prime difference ideal $\p' \si k\{y\}$ with $$\{f\}_m \subseteq \p' \subseteq (\p).$$
%% Then the ideal $k\{z\} \cap \p' \si k\{z\}$ is a prime, difference ideal and $f' \in k\{z\}.$
%% Since by assumption $\p' \subseteq (\p)$, we know that $$\p' \cap k\{z\} \subseteq (\p) \cap k\{z\} = \p.$$
%% Why is $\{f'\} \subseteq \p' \cap k\{z\}$???
%% The minimality of $ uao $
\end{bew}
\end{cor}

This corollary shows an application of the theory developed for mixed difference ideals, where in this generalization we do not need any restriction on the order of the difference polynomial.
It is not a complete generalization however, as it remains unclear if the irreducible closed subset given by the theorem is in fact maximal, in other words, an irreducible component of $\VV_m(f)$.
The following simple example is not covered by the assertion of Theorem \ref{irredcomp}, but is a simple application of Corollary \ref{corfinal}.

\begin{ex}
Consider the irreducible $\s$-polynomial $$(\s^2(y_1 y_2 y_3) + 1) \in \Q\{y_1,y_2,y_3\}.$$
\end{ex}

%% \begin{lem}
%% Let $k$ be a $\s$-field and $f \in k\{y_1,\ldots,y_n\}, f \notin k$ an irreducible $\s$-polynomial.  
%% Then there exists a prime difference ideal $\q \si k\{y_1,\ldots,y_n\}$, minimal containing $f$, and a $d \in \N$, such that 
%% $$\dim(k\{y_1,\ldots,y_n\}[i]/\q[i]) = \left\{ \begin{array}{lr} i, i < \operatorname{Ord}(f) \\ d(i-\operatorname{Ord}(f)) +\ operatorname{Ord}(f),
%%  i \geq \operatorname{Ord}(f) \end{array} \right.$$
%% \end{lem}


\clearpage

\section{Final Remarks}
\input{final}
\clearpage

\begin{appendices}
\lstset{ %
  backgroundcolor=\color{white},   % choose the background color; you must add \usepackage{color} or \usepackage{xcolor}
  basicstyle=\footnotesize,        % the size of the fonts that are used for the code
  breakatwhitespace=false,         % sets if automatic breaks should only happen at whitespace
  breaklines=true,                 % sets automatic line breaking
  captionpos=b,                    % sets the caption-position to bottom
  frame=single,                    % adds a frame around the code
  keepspaces=true,                 % keeps spaces in text, useful for keeping indentation of code (possibly needs columns=flexible)
  keywordstyle=\color{blue},       % keyword style
  language=Octave,                 % the language of the code
  rulecolor=\color{black},         % if not set, the frame-color may be changed on line-breaks within not-black text (e.g. comments (green here))
  showspaces=false,                % show spaces everywhere adding particular underscores; it overrides 'showstringspaces'
  showstringspaces=false,          % underline spaces within strings only
  showtabs=false,                  % show tabs within strings adding particular underscores
  stepnumber=2,                    % the step between two line-numbers. If it's 1, each line will be numbered
  stringstyle=\color{mymauve},     % string literal style
  tabsize=2,                       % sets default tabsize to 2 spaces
}




\section{Code for the Examples in Macaulay2}\label{appendixcode}

In this appendix we give the code used for calculating two examples in the thesis with its corresponding output. We do this with brief commentary inbetween to make it easier to understand.

\subsection{Example \ref{counterexker}}
We consider the difference polynomials $f_1 := \s(y_2) + 1, f_2:= \s(y_1)y_2 + y_1y_2 + \s(y_1) + y_1 + y_2 \in k[y_1,y_2,\s(y_1),\s(y_2)] =: R$.

For the calculations we considered the rational numbers $\Q$ as a constant $\s$-field. Since these calculations do not depend on the field, there should not much be of a difference. We begin by defining the ring $R$:

\begin{lstlisting}
i1 : R = QQ[s2y_1,s2y_2,sy_1,sy_2,y_1,y_2]

o1 = R

o1 : PolynomialRing

\end{lstlisting}

The definition is that of a simple polynomial ring, and we just take the names of the variables in a way we can easily interpret them as powers of $\s$ applied on a variable. We add one more power of $\s$ as in the example, so that we can calculate the relation stated later, which includes $\s^2$.
The output, which can be recognized from the lines begining with an ``o'', indicates that a polynomial ring has been succesfully constructed. We continue and define the ideal.

\begin{lstlisting}
i2 : I = ideal(sy_2+1,sy_1*y_2 + y_1*y_2 + sy_1 + y_1 + y_2)

o2 = ideal (sy  + 1, sy y  + y y  + sy  + y  + y )
              2        1 2    1 2     1    1    2

o2 : Ideal of R

\end{lstlisting}

Macaulay conveniently has a function to test if the ideal is prime: 

\begin{lstlisting}

i3 : isPrime(I)

o3 = true

\end{lstlisting}

This, however, does not necesarilly mean that the ideal is a prime kernel. To see this, we use Gr\"{o}bner bases.

\begin{lstlisting}

i4 : groebnerBasis(I)

o4 = | sy_2+1 sy_1y_2+y_1y_2+sy_1+y_1+y_2 |

             1       2
o4 : Matrix R  <--- R

i5 : y_1 < sy_1

o5 = true

\end{lstlisting}

It tells us that $f_1, f_2$ is a Gr\"{o}bner basis of $I$.
The variable ordering is per default reverse lexicographical with respect to the input.  
We tested it here as an example to make sure it was as intended.
Using elimination theory (see for example Theorem 2 in Chapter 3 of \cite{cox}),
we thus see that $(f_1,f_2)[0] = \{0\}$.

Finally, we use macaulay to test the relation given in the example.

\begin{lstlisting}

i6 : (sy_2+1)*(s2y_1+ sy_1 + 1)- (s2y_1*sy_2 + sy_1*sy_2 + s2y_1 + sy_1 + sy_2)

o6 = 1

o6 : R


\end{lstlisting}

\subsection{Example \ref{secondexamplem2}}

This second example is a bit more complex, since we have to check the condition that
$$(ker(\s)) \cap k\{a,\s(a)\} = ker(\s).$$

We begin again by defining the ring we will work with

\begin{lstlisting}

i1 :  D = QQ[s2y_1,s2y_2,sy_1,sy_2,y_1,y_2]

o1 = D

o1 : PolynomialRing


\end{lstlisting}

This time, we want to consider a quotient ring, so we begin by defining the subring $DM1 = k[y_1,y_2,\s(y_1),\s(y_2)]$.

\begin{lstlisting}

i2 :  (DM1,i) = selectVariables(new List from 2..5,D)

o2 = (DM1, map(D,DM1,{sy , sy , y , y }))
                        1    2   1   2

o2 : Sequence

\end{lstlisting}

We return to use our original ring and define the ideal we want to factor out.

\begin{lstlisting}

i3 : use D

o3 = D

o3 : PolynomialRing

i4 :  I = ideal(s2y_2+1,s2y_1*y_2 + sy_1*y_2 + s2y_1 + sy_1 + y_2)

o4 = ideal (s2y  + 1, s2y y  + sy y  + s2y  + sy  + y )
               2         1 2     1 2      1     1    2

o4 : Ideal of D

i5 :  groebnerBasis I

o5 = | s2y_2+1 s2y_1y_2+sy_1y_2+s2y_1+sy_1+y_2 |

             1       2
o5 : Matrix D  <--- D


\end{lstlisting}

From the Gr\"{o}bner basis we again see that $I$ is indeed a prime kernel.
Now we can define the quotient ring $D/I$.

\begin{lstlisting}

i6 : F2 = D/I

o6 = F2

o6 : QuotientRing

\end{lstlisting}

Now we proceed to define the mapping $\s$. Since $I \cap k\{y\}[1] = \{ 0 \}$ 
we know that $k[a,\s(a)] \cong k[y,\s(y)]$ and can use the latter for defining $\s$.

\begin{lstlisting}


i7 : F1 = DM1

o7 = DM1

o7 : PolynomialRing

i8 : use F1

o8 = DM1

o8 : PolynomialRing

i9 :  sigma = map(F2,F1,{y_1 => F2_(symbol sy_1),y_2 => F2_(symbol sy_2), sy_1 => F2_(symbol s2y_1), sy_2 => F2_(symbol s2y_2)})

o9 = map(F2,DM1,{s2y , -1, sy , sy })
                    1        1    2

o9 : RingMap F2 <--- DM1

\end{lstlisting}

With this output Macaulay2 tells us that it successfully defined the mapping. We can look at its kernel now.

\begin{lstlisting}

i10 :  groebnerBasis kernel(sigma)

o10 = | sy_2+1 |

                1         1
o10 : Matrix DM1  <--- DM1


\end{lstlisting}


We see that $\ker(\s) = (\s(y_2) + 1) \unlhd k[y_1,y_2, \s(y_1), \s(y_2)]$. 
From this we should be able to already deduce that the condition holds, but we can ask Macaulay to explicitly
calculate it too

\begin{lstlisting}

i11 : use F2

o11 = F2

o11 : QuotientRing

i12 : J = ideal(sy_2 + 1)

o12 = ideal(sy  + 1)
              2

o12 : Ideal of F2

i13 : groebnerBasis(J)

o13 = | sy_2+1 |

               1        1
o13 : Matrix F2  <--- F2

\end{lstlisting}
 
Which, using elimination theory again, gives the desired result, namely

$$ (\ker(\s)) \cap k[y_1,y_2,\s(y_1),\s(y_2)] = (\s(y_2) +1) = \ker(\s) .$$

\end{appendices}

\clearpage 
\begin{thebibliography}{9}
\bibitem{wibmer} Wibmer, Michael \emph{Algebraic Difference Equations (Lecture Notes)}, Available online: \url{http://www.algebra.rwth-aachen.de/de/Mitarbeiter/Wibmer/Algebraic\%20difference\%20equations.pdf}
\bibitem{lang} Lang, Serge, \emph{Algebra}, Revised Third Edition, Springer, 2005
\bibitem{eisenbud} Eisenbud, David \emph{Commutative Algebra with a View Toward Algebraic Geometry}, Springer, 1995
\bibitem{hartshorne} Hartshorne, Robin \emph{Algebraic Geometry}, Springer, 1977
\bibitem{cohn} Cohn,  Richard \emph{Difference Algebra}, Interscience Publishers, 1965
\bibitem{levin} Levin, Alexander \emph{Difference Algebra}, Springer, 2008
\bibitem{hrushovski} Hrushovski, Ehud \emph{The Elementary Theory of the Frobenius Automorphism}, arXiv:math/0406514 
\bibitem{bourbaki} Bourbaki, Nicolas \emph{Commutative Algebra}, Hermann, 1972
\bibitem{M2} Grayson, Daniel R. and Stillman, Michael E., Macaulay2, a software system for research in algebraic geometry, Available at \href{http://www.math.uiuc.edu/Macaulay2/}{http://www.math.uiuc.edu/Macaulay2/}
\bibitem{levinmixed} Levin, Alexander, \emph{On the ascending chain condition for mixed difference ideals}, 	arXiv:1207.4721
\bibitem{cox} Cox, Little and O'Shea, \emph{ Ideals, Varieties and Algorithms}, Second Edition, Springer, 1997
\end{thebibliography}


\clearpage
\section*{Conventions on notation for this thesis} 
We will use the following conventions throughout the thesis:
\begin{itemize}
\item $\N := \{0,1,\ldots \}$
\item $\NE := \{1,2,\ldots \}$
\item $ \n := \{1,2,\ldots, n\}$
\item $ \n_0 := n \cup \{0\}$
\item $A \setminus B = \{ a \in A \mid a \notin B \}$
\item Let $R$ be a ring, $a_1,\ldots,a_n \in R$. Then $(a_1,\ldots,a_n)$ denotes the smallest ideal of $R$ containing $a_1,\ldots,a_n$.
\item Let $R$ be a difference ring, $a_1,\ldots,a_n \in R$. Then $[a_1,\ldots,a_n]$ denotes the smallest difference ideal of $R$ containing $a_1,\ldots,a_n$.
\item Let $R$ be a difference ring. Then $A \si R$ means that A is a difference ideal of $R$.
\item The variable $y$ in (difference) polynomial rings will, in general, mean $n$ variables $y_1, \ldots, y_n$, i.e., we will write for example $k[y]$ for $k[y_1,\ldots,y_n]$.
\end{itemize}
%% \clearpage
%% \tableofcontents
\clearpage


\clearpage
\printindex
\end{document}

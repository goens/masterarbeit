


\documentclass[12pt,a4paper,BCOR15mm,twoside,DIV12]{article}
%\documentclass{article}
%\usepackage[paper=a4paper,left=20mm,right=20mm,top=25mm,bottom=25mm]{geometry}
\usepackage[english]{babel}
\usepackage[utf8]{inputenc}
\usepackage{amsmath}
\usepackage{color}
\usepackage{enumerate}
\usepackage[titletoc]{appendix}
\usepackage{amssymb}
\usepackage{amsfonts}
\usepackage{amsthm}
\usepackage{hyperref}
\usepackage{makeidx}
\usepackage{graphicx, float,epsfig}
\usepackage[nottoc,numbib]{tocbibind}
\usepackage{listings}

\newcommand{\properideal}{%
  \mathrel{\ooalign{$\lneq$\cr\raise.22ex\hbox{$\lhd$}\cr}}}

\def\P{\mathcal{P}}
\def\I{\mathbb{I}}
\def\R{\mathbb{R}} 
\def\E{\mathcal{E}} 
\def\NE{\mathbb{N}_{\geq1}} 
\def\N{\mathbb{N}} 
\def\Z{\mathbb{Z}} 
\def\Q{\mathbb{Q}} 
\def\F{\mathbb{F}}
\def\Vm{\mathcal{V}_m}
\def\V{\mathcal{V}}
\def\VV{\mathbb{V}}
\def\C{\mathbb{C}}
\def\U{\mathcal{U}}
\def\a{\mathfrak{a}}
\def\b{\mathfrak{b}}
\def\p{\mathfrak{p}}
\def\q{\mathfrak{q}}
\def\s{\sigma}
\def\si{\unlhd_{\sigma}}
\def\sD{\s\text{-}\operatorname{D}}
\def\GL{\text{GL}}
\def\supp{\text{Supp}}
\def\id{\text{id}}
\def\n{\underline{n}}
\def\Spec{\operatorname{Spec}}
\def\sSpec{\sigma\operatorname{-Spec}}
\def\diag{\text{diag}}
\def\End{\text{End}}
\def\Hom{\text{Hom}}
\def\fa{\text{ for all }}
\def\Tr{\text{Tr}}
\def\Id{\text{Id}}
\def\Sym{\text{Sym}}
\def\H{\mathcal{H}}
\def\wt{\text{wt}}
\def\Perf{\text{Perf}}
\def\sdim{\sigma\operatorname{-dim}}
\def\trdeg{\operatorname{trdeg}}

\renewcommand{\labelenumi}{\alph{enumi})}
%\renewcommand{\P}{\textfrak{P}}
\newcommand{\cupdot}{\mathop{\mathaccent\cdot\cup}}
\newcommand{\textsim}{\mathord{\sim}}
\newcommand{\catname}[1]{{\normalfont\textbf{#1}}}
\newcommand{\Set}{\catname{Set}}
\newcommand{\Top}{\catname{Top}}
\newcommand{\sintk}{\s\text{\catname{-int}}_k}
\newcommand{\sringk}{\s\text{\catname{-ring}}_k}
\newenvironment{bew}{\begin{proof}[Proof]}{\end{proof}}
\theoremstyle{plain}
\newtheorem{Satz}{Satz}[section]
\newtheorem{theorem}[Satz]{Theorem}
\newtheorem{ex}[Satz]{Example}
\newtheorem{cor}[Satz]{Corollary}
\newtheorem{algorithm}[Satz]{Algorithm}
\newtheorem{prop}[Satz]{Proposition}
\newtheorem{conj}[Satz]{Conjecture}
\newtheorem{lem}[Satz]{Lemma}
\newtheorem{defn}[Satz]{Definition}
\theoremstyle{definition}
\newtheorem{rem}[Satz]{Remark}

%%begin restatement stuff
\makeatletter
\newtheorem*{rep@theorem}{\rep@title}
\newcommand{\newreptheorem}[2]{%
\newenvironment{rep#1}[1]{%
 \def\rep@title{#2 \ref{##1}}%
 \begin{rep@theorem}}%
 {\end{rep@theorem}}}
\makeatother


\newreptheorem{theorem}{Theorem}
%%end restatement stuff

\usepackage[arrow, matrix, curve]{xy}

\makeindex
\title{Geometric Aspects of Mixed Difference Ideals}
\author{Andr\'{e}s Goens}
\date{\today}
\begin{document}
\setlength{\parindent}{1.5em}

\maketitle

\clearpage
\section*{Introduction}
In this thesis we will develop the basis for studying geometric aspects of difference algebraic structures based on mixed difference ideals. This will be put in contrast to the usual theory which is based on perfect difference ideals, a subclass of mixed difference ideals. Section 1 shall serve as an introduction to difference algebra, presenting most basic definitions and some of the more known results concerning perfect ideals and their geometric aspects.
The rest of the thesis will then deal with the results presented in Section 1 and, to the extent possible, generalize them to the case of mixed difference ideals. 
Section 2 begins with a more in-depth study of difference ideals, with a particular emphasis on mixed difference ideals. It also touches on some geometrical subjects, defining a topological space similar to the Zariski topology on a special class of ideals of a difference ring.
Section 3 continues with a more geometric nature, presenting a definition of mixed difference varieties as well as some basic properties of these. 
Finally, Section 4 deals with a concrete method for studying difference ideals, namely through so-called difference kernels. In this section the difference to perfect difference ideals becomes more visible with a result that cannot be generalized in a simple way. A conjecture is presented of a possible generalization. Also, a result of more geometric nature is presented which gives an idea of why the theory with mixed ideals might be useful,
as a partial generalization of a theorem is achieved which has less restrictive conditions than the original.

\clearpage
\section*{Acknowledgments}
I would like to thank Michael Wibmer, for his patience, ideas and support throughout the work on this thesis. Writing this thesis would certainly not have been possible if it was not for his mentoring.
In the time spent working on this thesis, I did not only learn about difference algebra, but also how to do mathematics in general, and I have Michael Wibmer to thank for this.
I would also like to thank Julia Hartmann for her patience and help in writing this thesis.
Finally, I would like to thank Daniel Rettstadt, Annette Maier and Stefan Ernst for always offering advice the numerous times I had smaller and not-so-small problems and/or questions.

\clearpage

\tableofcontents

\clearpage 


\section{Difference Algebra}



Difference algebra is a small branch of mathematics, whose origin is closely related to that of differential algebra: a larger field with which it bears great similarity. It is also a relatively new field.
It could be argued that difference algebra was born as a branch of mathematics around the 1930s through a series of articles published by J. Ritt between 1929 and 1939. However, it was not until the 1950s thanks to R. Cohn 
that difference algebra reached levels of development comparable to those of differential algebra.  Since then it has enjoyed a satisfactory growth thanks to a large number of mathematicians, and although it remains a small field today,
it still has many important results and a mature structure. A more detailed historical review of the origin of difference algebra can be found in the preface of \cite{levin}, where this one is based off.

\subsection{Introduction to Difference Algebra} 

To give a first idea of the object of study of difference algebra, we will start off with a few examples of difference equations. Probably one of the best known examples is the Fibonacci sequence: $1,1,2,3,5,8,13,\ldots$, which can be seen as a solution of the following recursive equation:
\begin{align*}
a_0 = 1,  a_1 = 1 \\ a_n = a_{n-1} + a_{n-2}, n\geq 2
\end{align*}

Another example that probably any mathematician of physicist will know is the functional equation of the Gamma function:

\begin{align*}
\Gamma(x+1) = x \Gamma(x)
\end{align*}

A classical result in complex analysis states that any function which satisfies this equation is a multiple of the $\Gamma$ function,
which is considered a generalization of the factorial:
\begin{align*}
\Gamma(x) = \int_0^\infty{\frac{t^x}{t} e^{-t} dt}
\end{align*}

These are two notable examples of difference equations, as we will soon see. In difference algebra, however, we do not seek to find 'explicit' solutions of these equations,
 as are the numbers in the Fibonacci sequence or the integral representation of the $\Gamma$ function. We will seek to study the structure of these equations and tackle problems like the existence of solutions in a more abstract sense.

\subsection{Basics of Difference Algebra}\label{fundamentos}
\begin{defn}
Let  $R$ be a ring (in this thesis all rings will be commutative and unital), and let
 $\sigma: R \rightarrow R$ be an endomorphism of rings in $R$. Then we call the tuple $(R,\sigma)$ a \emph{difference ring}, or $\sigma$\emph{-ring}. \index{$\s$-ring}
By abuse of notation we will say that $R$ is a $\sigma$-ring  to refer to the pair, and if $R'$ is a further $\sigma$-ring, we will also use the symbol $\sigma$ for the endomorphism on $R'$; This should not lead to confusion, since it can be inferred from the context which endomorphism is meant. 
\end{defn}

\begin{defn}
Let $R, R'$ be  $\sigma$-rings and let $\varphi: R \rightarrow R'$ be a morphism of rings. We say that $\varphi$ is a \emph{morphism of $\sigma$-rings}  if \index{moprhism of $\sigma$-rings}
\begin{align*}
\sigma(\varphi(r)) = \varphi(\sigma(r)) \fa r \in R
\end{align*}
\end{defn}

\begin{ex} A few of the more notable examples of $\s$-rings are the following:

\begin{itemize}
\item Every ring $R$ is a $\sigma$-ring with $\sigma = \Id_R$. We call this a \emph{constant $\s$-ring}.  \index{constant $\s$-ring}
\item The field of meromorphic functions $\C \rightarrow \C$, which we will denote by $\mathcal{M}$,
is a $\sigma$-ring with $\sigma(f)(x) = f(x+1)$ for every $x \in \C, f \in \mathcal{M}$.
\item The sequences of integers, which we will denote by $\text{Seq}(\Z)$, build a $\sigma$-ring with the operation of shifting its terms to the left:
\begin{align*} \sigma: (a_n)_{n \in \N} \mapsto (a_{n+1})_{n \in \N}. \end{align*}
\end{itemize}
\end{ex}

\begin{defn}
Let $R$ be a $\sigma$-ring. If $R$ is also a field, we call the pair $(R,\sigma)$ a $\sigma$\emph{-field}. \index{$\s$-field} 
If $k$ is a $\sigma$-field, $A$ a $k$-algebra which (as a ring) is a $\sigma$-ring and it it holds that 
$\sigma(ra) = \sigma(r) \sigma(a), r \in k, a \in A$, then we call $A$ a  $k$-$\sigma$\emph{-algebra}. \index{$k$-$\sigma$-algebra}
\end{defn}

\begin{ex}
An example of special importance is that of $\sigma$\emph{-polynomial-rings}. \index{$\sigma$-polynomial-rings}
Let $k$ be a $\s$-field. We consider the polynomial ring in infinitely many variables $R:= k[y_1,\sigma(y_1),\sigma^2(y_1),\ldots]$,
 where $y_1,\sigma(y_1),\sigma^2(y_1),\ldots$ are, for the moment at least, simply the names of (algebraically independent) variables.
This ring we can turn into a $k$-$\sigma$-algebra by defining:
\begin{align*} 
\sigma:  R \rightarrow R, y_1 \mapsto \sigma(y_1), \sigma^{n-1}(y_1) \mapsto \sigma^{n}(y_1) \fa n > 1 
\end{align*}
and extending this mapping $k$-linearly in the obvious way. We denote this $\sigma$-polynomial-ring by $k\{y_1\}$. In an analogous fashion we can define $\sigma$-polynomial-rings $k\{y_1, \ldots, y_n \} =: k\{y\}$ in many variables.
We will use the shorthand notation $y = y_1,\ldots,y_n$ throughout the thesis.
\end{ex}

\begin{defn} $\phantom{}$
\begin{itemize}
\item Given a $\sigma$-ring $S$ and a subring $R \leq S$, we say that $R$ is a $\sigma$\emph{-subring} \index{$\s$-subring} of $S$ if $(R,\sigma|_{R})$ is a $\sigma$-ring,
i.e. , if the image of $\sigma|_{R}$ is contained in $R$.
\item Let $R$ be a $\s$-ring. A $\sigma$\emph{-ideal} \index{$\s$-ideal} is an ideal $I \unlhd R$ which is also a $\sigma$-subring of $R$; we denote this by $I \si R$. In this case, there exists a canonical $\sigma$-ring structure on the quotient ring $R/I$:
\begin{align*} \sigma: R/I \rightarrow R/I, a + I \mapsto \sigma(a) + I. \end{align*}
\end{itemize}
\end{defn}

\begin{defn}
Let $R$ be a $\s$-ring and let  $I \si R$ be a $\s$-ideal of $R$. For elements $a_1, \ldots, a_k \in R$ we denote by $[a_1, \ldots, a_k] \si R$ the $\s$-ideal minimal (with respect to inclusion) of $R$ which contains $a_1,\ldots,a_k$. 
In fact, it holds that \[[a_1,\ldots,a_k] = \{ \sum_{i=1}^n \s^{j_i}(x_i) \mid n \geq 0, j_i \geq 0, x_i \in \{a_1,\ldots,a_k\}, i=1,\ldots,n \}. \] If there exist $b_1,\ldots,b_r \in I$ such that $I = [b_1,\ldots,b_r]$,
 we say that $I$ is \emph{finitely $\s$-generated} as a $\s$-ideal \index{finitely $\s$-generated $\s$-ideal}
\end{defn}

\begin{defn}
A $k$-$\sigma$-algebra  $A$ is \emph{finitely $\sigma$-generated} if there exist elements $f_1, \ldots, f_n$ such that $$A = k[f_1,f_2,\ldots,f_n,\sigma(f_1),\ldots,\sigma(f_n),\sigma^2(f_1),\ldots].$$
\end{defn}

\begin{rem}\label{epipoli}
If $A$ is a $k$-$\sigma$-algebra, $\sigma$-generated by $f_1, \ldots, f_n$, then we have a canonical epimorphism of $k$-$\sigma$-algebras from the $\sigma$-polynomial-ring $k\{y_1, \ldots, y_n \}$ to $A$: $y_i \mapsto f_i, i = 1, \ldots, n$. We thus see that $\sigma$-polynomial-rings are free objects in the category of finitely generated $k$-$\sigma$-algebras. We write  $k\{f_1, \ldots, f_n\}$ to denote the  $\sigma$-algebra generated by $f_1, \ldots, f_n$.
\end{rem}

\begin{defn}
If the kernel $I$ of the epimorphism mentioned on Remark \ref{epipoli} is finitely $\s$-generated as well, say, by $r_1, \ldots, r_m$, then we call the algebra $A$ \emph{finitely $\sigma$-presented}. \index{finitely $\sigma$-presented}
\end{defn}

\begin{rem}
In the conditions of the former definition, by the fundamental theorem on homomorphisms we have $A \cong k\{y_1, \ldots, y_n\}/[r_1,\ldots,r_m]$. Note as well that the concepts ``finitely $\sigma$-generated'' and ``finitely $\sigma$-presented'' are truly different, contrary to the case of polynomial rings over fields where we can argue with the Hilbert basis theorem, as shows the next example:
\end{rem}

\begin{ex}
Let $k$ be a $\sigma$-field and let $I \si k\{y\} $ be the $\sigma$-ideal $\s$-generated by $y\s(y), y\s^2(y), y\s^3(y), \ldots$, i.e., $I = [y \s^i(y) \mid i\geq 1]$. Then the $\sigma$-ring $R := k\{y\}/I$ (where $\s (r + I) := \s(r) + I)$ is 
finitely $\sigma$-generated, $R = k\{ y + I \}$, but not finitely $\sigma$-presented, since $I$ is not finitely $\sigma$-generated.
\end{ex}


Just as the usual approach with algebraic equations is to consider them as solutions of a polynomial, difference equations can be expressed as the search for solutions of $\s$-polynomials. To see what is meant with this, we will return to the examples of the introduction,
 as we now have the necessary concepts to formulate these in the language of difference algebra. 

\begin{ex}
The Fibonacci sequence is a solution to the $\s$-polynomial $\sigma^2(y) + \sigma(y) - y$ in the $\sigma$-ring  $\text{Seq}(\Z)$; This is precisely the recurrence relation $a_{n+2} + a_{n+1} = a_n$.
In the same way, the $\Gamma$ function is the solution to the $\s$-polynomial $\sigma(y) - zy$, where $z \in \C(z)$ denotes the identity function $z \mapsto z$ in the $\s$-ring $\mathcal{M} \supset \C(z).
\end{ex}

\clearpage 
\section{Difference Ideals}

In this section we will study difference ideals more closely. A special emphasis will be given to certain difference ideals which mimic in many ways the properties of and relationship between radical and prime ideals on (general) rings.
Much of the following is based on Section 1.2 of the lecture notes of M. Wibmer \cite{wibmer}, where most of it is worked out for an analogous case. \\

\subsection{Difference Ideals}

\begin{rem}
It is easy to see from the definitions that $\s$-prime ideals are perfect, and that perfect $\s$-ideals are mixed, radical and reflexive. Prime $\s$-ideals are also mixed, but not necessarily perfect. Note that there is a difference between a prime $\s$-ideal, and a $\s$-prime ideal:
the former does not necessarily have to be reflexive, as is the case with the latter. Both, prime $\s$-ideals and $\s$-prime ideals will be very important throughout this thesis. It is for that reason that the distinction between both is of utmost importance.
\end{rem}

The above properties behave well with respect to morphisms of $\s$-rings in the sense of the following Lemma (see Exercise 1.2.7 of \cite{wibmer}):
\begin{lem}\label{bijmapping}
Let $\varphi: R \rightarrow S$ be a morphism of $\s$-rings and $\a \si S$ a $\s$-ideal of $S$. Then $\varphi^{-1}(\a) \si R$ is a $\s$-ideal of $R$. Similarly, if $\a$ is a mixed $\s$-ideal, then so is $\varphi^{-1}(\a)$. The same is true for perfect and for reflexive $\s$-ideals.
\begin{bew}
Since $\a \unlhd S$ is an ideal, so is $\b := \varphi^{-1}(\a) \unlhd R$. Let $b \in \b$. Then $\varphi(b) =: a \in \a$ by definition. Since $\a \si S$ is a $\s$-ideal, $\s(a) \in \a$, and since $\varphi$ is a morphism of $\s$-rings
it follows that $\sigma(a) = \sigma(\varphi(b)) = \varphi (\s (b)) \in \a$. Hence, $\s(b) \in \b$ which implies that $\b$ is a $\s$-ideal. \\
\indent Now let $\a$ be mixed and $fg \in \b$. This means by definition of $\b$, 
that $\varphi(fg) = \varphi(f) \varphi(g) \in \a$. Since $\a$ is mixed, this in turn implies that $$\varphi(f) \s( \varphi(g)) = \varphi(f) \varphi(\s(g)) = \varphi(f\s(g)) \in \a,$$ which yields $f\s(g) \in \b$, so that $\b$ is also mixed. 
The proof for perfect and for reflexive difference ideals is analogous.
\end{bew}
\end{lem}

\begin{rem}
Let $R$ be a $\s$-ring and $\a \si R$ a $\s$-ideal. We can define a canonical $\s$-ring structure on the quotient ring $R/\a$ via $\s(r+\a):= \s(r) + \a$. 
This is well defined and in particular makes the canonical epimorphism $\tau: R \twoheadrightarrow R/\a$ a morphism of $\s$-rings.
\end{rem}

\begin{prop}\label{bijideals}
Let $R$ be a $\s$-ring and $\a \si R$ a $\s$-ideal. The canonical epimorphism $\tau: R \twoheadrightarrow R/\a$ induces, in the sense of Lemma \ref{bijmapping}, a bijection between the sets $\{ \b \si R/\a \}$ and $\{ \a \si \b \si R \}$. The same holds true if we restrict both sets to prime, radical and mixed, $\s$-prime or perfect $\s$-ideals.
\begin{bew}
See Proposition 1.2.8 of \cite{wibmer}.
\end{bew}
\end{prop}

\begin{rem}\label{wmwelldef}
Let $R$ be a $\s$-ring, and $F \subseteq R$ be a subset of $R$. Any intersection of mixed, radical $\s$-ideals containing $F$ is also a mixed, radical $\s$-ideal, which of course contains $F$. 
This means that there is a smallest (with respect to inclusion) mixed, radical $\s$-ideal $\a$ containing $F$; namely, the intersection of all such $\s$-ideals:
\begin{align*} \a = \bigcap_{\substack{ \b \si R, \\ \b \text{ radical and mixed}}} \b. \end{align*}
\begin{proof}
Let $I$ be an index set and $\a_i \si R \fa i \in I$ be mixed, radical $\s$-ideals. Further let $\b := \bigcap_{i \in I} \a_i$ be the intersection of these. Obviously, $\b$ is (algebraically) an ideal of $R$. We will show that it is also a $\s$-ideal, radical and mixed.
If $a \in \a_i \fa i \in I$, then $\s(a) \in \a_i \fa i \in I$, since each $\a_i$ is a $\s$-ideal.
It follows that $\s(a) \in \b$. \\
\indent Similarly, if $aa' \in \a_i \fa i \in I$, then $a \s(a') \in \a_i \fa i \in I$, since each $\a_i$ is mixed, which implies that $a \s(a') \in \b$.  \\
\indent Finally, if $a \in \sqrt(\b)$ there exists an $n \in \N$ such that $a^n \in \b$. This means that $a^n \in \a_i \fa i \in I$, which implies that $a \in \sqrt(\a_i) \fa i \in I$. Since every $\a_i, i \in I$ is radical, this means that $a \in \a_i \fa i \in I$,
and thus $a \in \b$.
\end{proof}
\end{rem}

\begin{defn}
The $\s$-ideal $\a$ from Remark \ref{wmwelldef} is called the radical, mixed closure of $F$, and we will denote it by $\{F\}_{m}$.
\end{defn}


\begin{lem}\label{sqrtmixed}
Let $R$ be a $\s$-ring and $\a \si R$ be a mixed $\s$-ideal. Then the radical of $\a$, $\sqrt{\a}$, is also mixed.
\begin{bew}
Let $f,g \in R$ be such that $fg \in \sqrt \a$. By definition there exists an $n \in \NE$ such that $f^n g^n = (fg)^n \in \a$. Since $\a$ is mixed, this implies that $f^n \s(g^n) = f^n \s(g)^n = (f\s(g))^n \in \a$. 
But this in turn implies that $f\s(g) \in \sqrt \a$, which is what we wanted to show.
\end{bew}
\end{lem}


\begin{ex}\label{nombasisex}
Let $k$ be a constant $\s$-field and let $R:= k\{y_1,y_2\}$. Consider the difference ideal $\a:= [y_1y_2] \si R$. We can inductively define a chain \begin{align*}\a^{\{0\}}:= \a,~ \a^{\{m+1\}}:= [\{ f \s(g) \mid f,g \in \a^{\{m\}}\}] \\ = [\s^k(y_1)\s^l(y_2) \mid k,l = 0,1,\ldots,m+1], \fa m \in \NE.\end{align*}
This is an infinite properly ascending chain of difference ideals. 
%% \begin{bew}
%% We will show that $\a^{\{m\}} = [\s^k(y_1)\s^l(y_2) \mid k,l = 0,1,\ldots,m]$ by induction. The assertion that the $\a^{\{m\}}$ form an infinite chain of properly ascending $\s$-ideals is obvious from this.\\
%% \indent The inclusion $ [\s^k(y_1)\s^l(y_2) \mid k,l = 0,1,\ldots,m] \subseteq \a^{\{m\}}$ follows from an induction on $m$ directly from the definition of $\a^{\{m\}}$. \\ 
%% \indent  To show the other inclusion, ``$\supseteq$'', we do it also inductively. For $m=0$ it holds that $\a^{\{0\}} = \a = [y_1y_2]$. Let thus $m \in \N$ and $a \in \a^{\{m+1\}}$.
%% By definition, there exist $k \in \NE, f_i, g_i, r_i \in R$ with $f_ig_i \in \a^{\{m\}}, i = 1,\ldots,k$ such that $a = \sum_{i=1}^k r_i f_i \s(g_i)$. 

%% FIXME: FINISH THIS!!!!!!!

%%  To prove that the chain is properly ascending, it is enough to show that $y_1\s^m(y_2) \in \a^{\{m\}}$, but $y_1\s^{m+1}(y_2) \notin \a^{\{m\}}$, since $\a^{\{m\}} \subseteq \a^{\{m+1\}}$ is obvious by definition.
%% This can be shown by induction: For $m = 0$, it is clear that $y_1y_2 \in [y_1y_2]$, but $y_1 \s(y_2) \notin [y_1y_2]$. Now, if $y_1 \s^m(y_2) \in \a^{\{m\}}$, then by definition, $y_1 \s^{m+1}(y_2) \in \a^{\{m+1\}}$.
%% On the other hand, if $\y_1 \s^{m+1}(y_2) \notin \a^{\{m\}}$, then it is clear that $\y_1 \s^{m+2}(y_2) \notin \{ f\s(g) \mid f,g \in \a^{\{m\}}\} =: M$. We only need to show that $y_1\s^{m+2}(y_2) \notin [M]$.

%% \end{bew}
\end{ex}


To try to find $\{\a\}_m$ for a $\s$-ideal $\a \si R$ in a difference ring $R$, it might be tempting to consider $\a':= \{ f\s(g) \mid fg \in \a \}$, or to ensure that is a difference ideal rather, $[\a']$. The example above shows that this is not enough,
as the ideal $[\a']$ does not have to be mixed in general. However, by iteratively repeating this process and taking the union of $\s$-ideals obtained this way, we do get a mixed $\s$-ideal, as we will see in the following Lemma:
\begin{lem}\label{lemshuffling}
Let $R$ be a $\s$-ring and $F \subseteq R$. Further let $F' := \{f\s(g) \mid fg \in F \}$, and set $F^{\{1\}}:= [F']$, $F^{\{n\}}:= [{F^{\{n-1\}}}'] \fa m \in \NE$. Then
\begin{align} \{F\}_m = \sqrt{\bigcup_{n=1}^{\infty} F^{\{n\}}}. \end{align}
This way of obtaining $\{F\}_m$ is called a shuffling processes and has an analog for perfect $\s$-ideals (see for example \cite{levin}, p. 121f.) \index{Shuffling process}
\begin{proof}
Let $\a:= \bigcup_{n=1}^{\infty} F^{\{n\}}$. It is obvious from the construction that $F \subseteq \a$. It also holds that $\a$ is a mixed $\s$-ideal, since for any $f,g \in \a$ there exists an $m \in \NE$ such that $f,g \in F^{\{m\}}$.
And hence $f + g, \s(f) \in F^{\{m\}} \subseteq \a$, as well as $fh \in F^{\{m\}} \subseteq \a$ for any $h \in R$. Furthermore, for $f, g \in R$ with $fg \in \a$ there also exists an $m \in \NE$ such that $fg \in F^{\{m\}}$. 
Then we have $f\s(g) \in F^{\{m+1\}} \subseteq \a$. \\
\indent On the other hand, by induction on the iterative steps $F^{\{n\}}$ it follows that for every mixed $\s$-ideal $\b$ which contains $F$, $F^{\{n\}} \subseteq \b$. Hence, $\a$ is the smallest mixed $\s$-ideal containing $F$. \\
\indent By Lemma \ref{sqrtmixed} we know that $\sqrt{\a}$ is mixed. This actually shows that $\sqrt a$ is indeed the smallest mixed, radical $\s$-ideal of $R$ containing $F$, since every such ideal has to contain $\a$, and thus $\sqrt{\a}$ as well.

% To see this, assume there exists a radical, mixed $\s$-ideal $\b \supseteq F$
%such that $\b \subsetneqq \sqrt{\a}$. Then we have $F \subseteq \a \cap \b \subsetneqq \a$, a contradiction to the minimality of $\a$ (since the intersection of mixed $\s$-ideals is mixed). 
\end{proof}
\end{lem}


\begin{ex} %%not *radical*! 
Let $k$ be a $\s$-field, and consider $R = k\{y_1\}$. Then the $\s$-ideal $[y_1] \si R$ is mixed, hence equal to its mixed closure.
The mixed closure of $[y_1] \cdot [y_1]$ is $[ y_1 \s^i(y_1) \mid i \in \N ] \not \ni y_1$.
One could have expected, perhaps, to get an analog of the statement in algebraic geometry that $\sqrt{F_1  F_2 } = \sqrt{F_1} \cap \sqrt{F_2}$, but this example shows it is not in general so for mixed ideals.
It is however very noteworthy that the ideal $[ y_1 \s^i(y_1) \mid i \in \N ] \si R$ is not radical. For radical, mixed difference ideals we will in fact get such a statement later (Corollary \ref{prod=cap}).
\end{ex}

A very important result in commutative algebra is the fact that every radical ideal is the intersection of prime ideals. This has an analogue for perfect $\s$-ideals, as well as for mixed $\s$-ideals. 
We will prove the latter, but for this we need a few additional tools. We will first prove a weaker version of the statement, for which we need a few results from commutative algebra:

\begin{lem}\label{commalg}
Let $R$ be a ring. 
\begin{enumerate}[(a)]
\item If $S \geq R$ is an overring of $R$, and $\p$ is a minimal prime ideal of $R$, then there exists a minimal prime ideal $\q$ of $S$ such that $\p = \q \cap R$.
\item Every radical ideal of $R$ is the intersection of prime ideals. If $R$ is Noetherian, then every radical ideal of $R$ is the intersection of finitely many prime ideals.
\item If $R$ is Noetherian and $\p \unlhd R$ is a minimal prime ideal of $R$, then there exists an element $a \in R$ such that $\p$ is the annihilator ideal of $a$, i.e. $\p = \text{Ann}(a) = \{ r \in R \mid ra = 0 \}$.
\end{enumerate}
\begin{bew} $~$
\begin{enumerate}[(a)]
\item See Remark 2.9 of \cite{hrushovski}
\item See \cite{bourbaki} Ch. 2, \S 2.6, Corollary 2 to Proposition 13 and Ch. 2, \S 4.3, Corollary 3 to Proposition 14.
\item This is a special case of Theorem 3.1 of \cite{eisenbud} for $R$ as an $R$-module.
\end{enumerate}
\end{bew}
\end{lem}


The following is adapted from the proof given in \cite{wibmer} of Proposition 1.2.28 (here the upcoming Theorem \ref{intersectionprimes}). It has been worked out in more detail and divided further in a way to try and make this more detailed version of the proof easier to read.

\begin{defn}
Let $R$ be a difference ring. We say $R$ is \emph{finitely $\s$-generated over $\Z$} if there exists a finite set $A \subseteq R$ so that every $f \in R$ can be written as a finite $\Z$-linear combination of $\s$-powers of elements in $A$. In other words,
for every $f \in R$ there exists an $n \in \NE$ so that $f \in \Z[A,\sigma(A),\ldots,\s^n(A)]$. \\ 
\indent For any subset $F \subseteq R$ we denote by $$\Z\{F\} = \{ f \in R \mid \text{ there exists an } n \in \N: f \in \Z[F, \s(F), \ldots, \s^n(F)] \}$$ the set of all elements $\s$-generated by $F$ over $\Z$.
\end{defn}\index{finitely $\s$-generated over $\Z$}

\begin{prop}\label{mixedintersectionprimesfinite}
Let $R$ be a $\s$-ring finitely $\s$-generated over $\Z$. Then, every radical, mixed $\s$-ideal of $R$ is the intersection of prime $\s$-ideals.
\begin{bew}
Let $\a \si R$ be a mixed, radical $\s$-ideal. By Proposition \ref{bijideals} there is a bijection between the prime $\s$-ideals of $R$ containing $\a$ and those of $R/\a$. Hence, we can assume without loss of generality that $\a = [0] \si R$,
 by replacing $R$ with $R/\a$. This means that we only have to show that the zero ideal $[0]$ of a well-mixed, reduced $\s$-ring $R$ is the intersection of all its prime $\s$-ideals. Note that this does not change the fact
that $R$ is finitely $\s$-generated over $\Z$. \\
\indent Let thus $f \in R$ be such that $f \in \q \fa \q \si R$ prime. We assert that $f$ then has to be $0$. Assume this is not the case, i.e., $f \neq 0$. Then by assumption on $R$ there is an $n \in \N$ such that $f \in \Z[A,\s(A),\ldots,\s^n(A)]$.
We now use the special case for (algebraic) ideals: since $\Z[A,\s(A),\ldots,\s^n(A)]$ is Noetherian and reduced, $(0) \unlhd \Z[A,\ldots,\s^n(A)]$ is the intersection of all prime ideals of $R$. In particular, there exist prime ideals which do not contain $f$.
Let $\q_0 \unlhd \Z[A,\ldots,\s^n(A)]$ be a minimal such prime ideal, i.e., with $f \notin \q_0$. Since $f \in \Z[A,\s(A),\ldots,\s^n(A)] \subset \Z[A,\s(A),\ldots,\s^{n+1}(A)]$, again by Lemma \ref{commalg}, we can find a minimal prime ideal $\q_1 \unlhd \Z[A,\s(A),\ldots,\s^{n+1}(A)]$
such that $\q_1 \cap \Z[A,\s(A),\ldots,\s^{n}(A)] = \q_0$. \\
\indent Inductively we find a chain of minimal prime ideals $\q_i, i \in \N$, $\q_i \unlhd \Z[A,\s(A),\ldots,\s^{n+i}(A)]$, with $\q_{i+1} \cap \Z[A,\s(A),\ldots,\s^{n+i}(A)] = \q_i$ for all $i \in \N$.
Then $\q := \bigcup_{i=0}^{\infty} \q_i$ is a prime ideal of $R$, with $f \notin \q$. In fact, $\q$ is a $\s$-ideal of $R$: Let $a \in \q$. We want to show that $\s(a) \in \q$. By construction of $\q$ there exists an $i \in \N$ such,
that $a \in \q_{i-1} \subseteq \Z[A,\s(A),\ldots,\s^{n+i-1}(A)]$, which implies that $\s(a) \in \Z[A,\s(A),\ldots,\s^{n+i}(A)]$. Lemma \ref{commalg} states then, that there is an $h \in \Z[A,\s(A),\ldots,\s^{n+i}(A)]$ such that $ \q_i = \text{Ann}(h)$.
It follows that $ah = 0$, and since $R$ is well-mixed, this implies that $\s(a)h = 0$, hence, $\s(a) \in \q_i \subseteq \q$. This means that $\q$ is a prime $\s$-ideal of $R$ not containing $f$, which contradicts the assumption on $f$, so that $f = 0$ has to follow.
\end{bew}
\end{prop}

For the general case we need yet another tool, the concept of filters:

\begin{defn}\index{filter}\index{ultrafilter}
Let $U$ be a set, and let $F \subseteq \text{Pot}(U)$, where $\text{Pot}(U)$ denotes the power set on $U$. Then $F$ is called a \emph{filter} if it satisfies the following axioms: 
\begin{itemize}
\item  $U \in F$ and $\emptyset \notin F$.
\item If $V,W \subseteq U$ with $V \subseteq W \text{ and }V  \in F $ it holds that $W \in F$.
\item For $V_1, \ldots, V_n \in F$ it holds that \[ \bigcap_{i = 1}^n V_i \in F. \]
\end{itemize}
A filter $F$ is called an \emph{ultrafilter}, if for any $V \subseteq U$ it holds that $V \in F$ or $U \setminus V \in F$. Note that the first and third axioms together imply that at most one of $V$ and $U \setminus V$ can be in $F$.
\end{defn}



\begin{rem}
Let $U$ be a set. Then, the set of filters on $U$ is inductively ordered by inclusion. By Zorn's lemma, for every filter $F$ on $U$ there must exist a maximal filter $G$ with respect to inclusion such that $F \subseteq G$.
The maximality of the filter implies that $G$ will be an ultrafilter, since we could otherwise find a new filter $G'$ where $G$ is properly included by adding one of the sets which contradict the ultrafilter property and considering the smallest filter containing this set.
\end{rem}

The reason why this concept is useful in our context is the following:

\begin{lem}\label{lemmafilters}
Let $R$ be a $\s$-ring, and let $M$ be the set of all $\s$-subrings of $R$ which are finitely $\s$-generated over $\Z$. For any fixed subset $F \subseteq R$, consider the set $M_F:= \{ T \subseteq M \mid \{S \in M \mid F \subseteq S \} \subseteq T \} \subseteq \text{Pot}(M)$. 
Then, \[ \mathcal{F}:= \bigcup_{ F \subseteq R \text{ finite} } M_F \]
 defines a filter on $M$. If $\mathcal{G}$ is an ultrafilter containing $\mathcal{F}$, and $P:= \prod_{S \in M} S$ with component-wise operations,
 then the ultrafilter $\mathcal{G}$ defines an equivalence relation on $P$ via $(g_S)_{S \in M} \sim (h_S)_{S \in M} : \Leftrightarrow \{ S \in M \mid g_S = h_S \} \in \mathcal{G}$. 
The set of equivalence classes $P/\mathcal{G}:= P/\textsim$ has a natural $\s$-ring structure and is called an \emph{ultraproduct}. \index{ultraproduct} %%fixme: of what?
\begin{proof}
Let us first show that $\mathcal{F}$ is a filter. For $F \subseteq R$ finite we have $\Z\{F\} \in \{ S \in M \mid F \subseteq S \} \neq \emptyset$, and since $T \supseteq \{ S \in M \mid F \subseteq S \} \fa T \in M_F$, $\emptyset \notin M_F$ (Note that $\Z\{\emptyset\} = (0)$, so it also holds for $F = \emptyset$).
  That $M \in M_F$ for any $F \subseteq R$ is obvious, as well as that for $T \subseteq U, T \in M_F$ it holds that $U \in M_F$. \\ 
\indent We only need to show that $U,T \in \mathcal{F}$ implies that $U \cap T \in \mathcal{F}$.
  Let $\hat U, \hat T \subseteq R$ be finite, such that $U \in M_{\hat U}, T \in M_{\hat T}$. $\hat U \cup \hat T \subseteq R$ is also finite and it holds that  $\{ S \in M \mid \hat U \cup \hat T \subseteq S \} \subseteq \{ S \in M \mid \hat U \subseteq S \} \subseteq U$,
 and similarly for $T$. This means that $U \cap T \in M_{\hat U \cup \hat T} \subseteq \mathcal{F}$, which finishes the proof that $\mathcal{F}$ is a filter. \\

 Now, consider an ultrafilter $\mathcal{G} \supseteq \mathcal{F}$ and define $\sim$ on $P$ as above. This is an equivalence relation: Let $f \sim g, g \sim h$ for $f,g,h \in P$. 
 This means that $\{ S \in M \mid f_S = g_S \} \in \mathcal{G}, \{ S \in M \mid g_S = h_S \} \in \mathcal{G}$. But then $$\{ S \in M \mid f_S = g_S \} \cap \{ S \in M \mid g_S = h_S \} \subseteq \{ S \in M \mid f_S = h_S \} \in \mathcal{G},$$
 since $\mathcal{G}$ is a filter.
 Reflexivity follows from the fact that $M \in \mathcal{G}$, and symmetry is obvious. \\
\indent  We now only need to show that we have a well-defined $\s$-ring structure on $P/\textsim$.
 Consider $f,f' \in P$ with $f \sim f'$. We have that for all $S \in M$ with $f_S = f'_S$ it holds that $\sigma(f)_S = \sigma(f')_S$. 
 But then $\{ S \in M \mid \s(f)_S = \s(f')_S \} \supseteq \{ S \in M \mid f_S = f'_S \} \in \mathcal{G}$ by assumption, and since $\mathcal{G}$ is a filter, this means that $\{ S \in M \mid \s(f)_S = \s(f')_S \} \in \mathcal{G}$,
 hence $\s(f) \sim \s(f')$. That $+$ and $\cdot$ are also well-defined can be proven in an analogous fashion.
\end{proof}
\end{lem}

We can now turn our attention to the generalization of Proposition \ref{mixedintersectionprimesfinite}. 


\begin{theorem}\label{intersectionprimes}
Let $R$ be a $\s$-ring and $F \subseteq R$ be a subset of $R$. Then, 
\begin{align*} \{F\}_m = \bigcap_{\substack{F \subseteq \p \si R \\ \p \text{ prime}}} \p \end{align*}
In particular, every radical, mixed $\s$-ideal of $R$ is the intersection of prime $\s$-ideals.
\begin{bew}
It suffices to show that every radical, mixed $\s$-ideal of $R$ is the intersection of prime $\s$-ideals.
Indeed, since prime $\s$-ideals are radical and mixed, it is clear that $\{F\}_m \subseteq \p$ for every prime $\p \si R$ with $F \subseteq \p$, which together with the fact that every radical, mixed $\s$-ideal of $R$ is the intersection of prime $\s$-ideals gives the representation 
\begin{align*} \{F\}_m = \bigcap_{\substack{F \subseteq \p \si R \\ \p \text{ prime}}} \p. \end{align*}
Now, by the same argument as in the beginning of the proof of Proposition \ref{mixedintersectionprimesfinite}, it is enough to prove in the case that $R$ is well-mixed and reduced, that the intersection of all prime $\s$-ideals of $R$ is $[0]$.
Let $0 \neq f \in R$. We will construct a prime $\s$-ideal $\q$ of $R$ which does not contain $f$: 

Let $P/\mathcal{G}$ be the difference ring as in Lemma \ref{lemmafilters}. Consider the mapping $\varphi: R \rightarrow P/\mathcal{G}, g \mapsto (g_S)_{S \in M}$ with $(g_S) = g \fa S \in M$ with $g \in S$ and $(g_S) = 0$ for $g \notin S$. 
It is in fact $\{ S \in M \mid g \in S \} \in M_{\{g\}}$ (with $M_{\{g\}}$ as in Lemma \ref{lemmafilters}). It follows from this that the image of the mapping onto $P/\mathcal{G}$ is in fact independent of the $(g_S)$ for $g \notin S$, as any other choice of these would be in the same $\sim$ class as the image described above.
It follows that $\varphi$ is a well-defined morphism of $\s$-rings. \\
\indent From Proposition \ref{mixedintersectionprimesfinite} we know that for every $S \in M$, there exists a prime $\s$-ideal $\p_S \si S$ such that $f \notin \p_S$. 
We define $\p \subseteq P/\mathcal{G}$ as the set of all equivalence classes of elements $(g_S)_{S \in M}$ such that $\{ S \in M \mid g_S \in \p_S \} \in \mathcal{G}$. 
For $[(g_S)_{S \in M}]_{\sim}, [(h_s)_{S \in M}]_{\sim} \in \p$ we have $$ \mathcal{G} \ni \{ S \in M \mid  g_S \in \p_S \} \cap  \{ S \in M \mid  h_S \in \p_S \} \subseteq \{ S \in M \mid  g_S + h_S \in \p_S \} \in \mathcal{G},$$
since $\mathcal{G}$ is a filter. Similar arguments for $\s(g), gh$ for $h \in P/\mathcal{G}$ show that $\p$ is indeed a $\s$-ideal. $\p$ is also prime since $\mathcal{G}$ is an ultrafilter:
Let $g,h \in P$ with $\{ S \in M \mid g_Sh_S \in \p_S \} \in \mathcal{G}$. If $[g]_\sim \notin \p$, then $V:= \{ S \in M \mid g_S \in \p_S \} \notin \mathcal{G}$. Since $\mathcal{G}$ is an ultrafilter, 
this means that $M \setminus V \in \mathcal{G}$. But $$\mathcal{G} \ni (M \setminus V) \cap \{ S \in M \mid g_S h_S \in \p_S \} \subseteq \{ S \in M \mid h_S \in \p_S \} \in \mathcal{G},$$
which means that $[h]_\sim \in \p$. The preimage of a prime $\s$-ideal, $\q := \varphi^{-1}(\p) \si R$ is also prime. By construction, $[\varphi(f)]_\sim \notin \p$, which means that $f \notin \varphi^{-1}(\p)$, as desired. 

\end{bew}
\end{theorem}

\subsection{An Analog of the Cohn Topology}

\begin{defn}
Let $R$ be a $\s$-ring. We denote the set of all prime $\s$-ideals of $R$ by $\sSpec(R):= \{ \p \si R \mid \p \text{ prime }\}$. Similarly, we denote the set of $\s$-prime ideals by $\Spec^\s(R):= \{ \p \si R \mid \p ~ \s\text{-prime }\} \subseteq \sSpec(R)$.
\index{$\sSpec$} \index{$\Spec^\s$}
\end{defn}


\begin{rem}
As is the case with $\Spec^\s(R)$, it can be the case that $\sSpec(R)= \emptyset$. For example, let $R$ be a $\s$-ring, and consider the $\s$-ring $R \oplus R$, with $\s( (r,s)):= (\s(s),\s(r))$. 
We will show that this ring has no prime $\s$-ideals. Let $\p \unlhd R$ prime. Then $0 = (1,0)(0,1) \in \p$, which means that either $(1,0) \in \p$ or $(0,1) \in \p$. But then $R \oplus 0 \subseteq \p$ or $0 \oplus R \subseteq \p$. If we assume that $\p$ is a $\s$-ideal then
 this implies that $R \oplus R \subseteq \p$, which cannot be, by definition.
\end{rem}

In algebraic geometry, one usually considers $\Spec(R)$ as a topological space with a topology called the Zariski topology. This has an analog for $\Spec^\s(R)$, usually called the Cohn topology. Here we will develop a further analog of both,
 which we will define on $\sSpec(R)$, and will be closely related to radical, mixed $\s$-ideals, as we shall see by its many properties.

\begin{defn}
Let $R$ be a $\s$-ring and $F \subseteq R$ be a subset of $R$. We set $\Vm (F):= \{ \p \in \sSpec(R) \mid F \subseteq \p \}$. 
If $F$ has only one element $f$, we write $\Vm(f)$ for $\Vm(F)$. \index{$\Vm(F)$}
\end{defn}

\begin{lem}\label{topologywelldef}
Let $R$ be a $\s$-ring. Then we have:
\begin{enumerate}[(a)]
\item $\Vm((0)) = \sSpec(R)$, and $\Vm(R) = \emptyset$.
\item For any two ideals $\a,\b \unlhd R$ we have $\Vm(\a) \cup \V(\b) = \Vm(\a \cap \b).$
\item For any family of ideals $(\a_i)_{i \in I}$ for an index set $I$, we have $$\bigcap_{i \in I} \Vm(\a_i) = \Vm(\sum_{i \in I} \a_i).$$ \label{vmintersectionideals}
\end{enumerate}
\begin{bew} $~$
\begin{enumerate}[(a)]
\item We have $(0) \subseteq \p \fa \p \in \sSpec(R)$, as well as $R \not\subseteq \p \fa \p \in \sSpec(R)$.
\item Let $\a, \b \unlhd R$ be two ideals in $R$. Then $\Vm(\a) \cup \Vm(\b) \subseteq \Vm(\a \cap \b)$, since for $\p \si R$ prime, $\a \subseteq \p$ it follows that $\a \cap \b \subseteq \p$, and similarly for $\b$.
On the other hand, let $\p \si R$ prime with $\a \cap \b \subseteq \p$, and $\a \not\subseteq \p$ (otherwise $\p \in \Vm(\a)$ and we are done). Then there exists an $f \in \a$, $f \notin \p$. 
For any $g \in \b$, it follows that $fg \in \a \cap \b \subseteq \p$. Since $\p$ is prime, this implies that $g \in \p$. Hence, $\b \subseteq \p$, which concludes the proof.
\item Let $(\a_i)_{i \in I}$ be a family  of ideals of $R$. Then $$\p \in \bigcap_{i \in I} \Vm(\a_i) \Leftrightarrow \a_i \subseteq \p \fa i \in I \Leftrightarrow \p \in \Vm(\sum_{i \in I} \a_i).$$
\end{enumerate}
\end{bew}
\end{lem}


\begin{rem}\label{vmsequal}
Since for a $\s$-ring $R$ any prime $\s$-ideal of $R$ is radical and mixed, it holds that for any $F \subseteq R$, and any prime $\s$-ideal $\p \si R$ with $F \subseteq \p$ we have
$(F) \subseteq [F] \subseteq \{ F \}_m \subseteq \p$. In particular, this means that $\Vm(F) = \Vm((F)) = \Vm([F]) = \Vm(\{F\}_m)$. 
\end{rem}

\begin{defn}\label{deftop}
Let $R$ be a $\s$-ring. We define a topology on $\sSpec(R)$ by setting $A \subseteq \sSpec(R)$ closed if $A = \Vm(\a)$ for an ideal $\a \unlhd R$, or equivalently,
 by defining a set to be open, if it is a complement of such a $\Vm(\a)$. This is a well-defined topology thanks to Lemma \ref{topologywelldef}.
For $f \in R$ we set $$\sD(f):= \sSpec(R) \setminus \Vm(f).$$ $\sD(f)$ is the complement of a closed set, and hence, open. 
We call the sets of the form $\sD(f) \subseteq \sSpec(R)$ \emph{basic open subsets} of $\sSpec(R)$. \index{basic open subsets}
\end{defn}

From here on, if not explicitly stated otherwise, when referring to topological concepts on $\sSpec(R)$ we will be referring to the topology just defined.

\begin{rem}
From its definition it is clear that $\sSpec(R) \subseteq \Spec(R):= \{ I \unlhd R \mid I \text{ prime} \}$. Since Lemma \ref{topologywelldef} does not require the ideals to be $\s$-ideals, 
it is easy to conclude that in fact the topology on $\sSpec(R)$ is just the topology induced by restriction of the Zariski topology to $\sSpec(R)$. The same argument can be made to see that the Cohn topology in turn,
defined on $\Spec^\s(R) = \{ \p \si R \mid $ $\p$ $ \s$-prime $\} \subseteq \sSpec(R)$, is also the restriction of the topology defined on $\sSpec(R)$. 
\end{rem}

\begin{defn}
Let $X$ be a topological space.
\begin{enumerate}[(a)]
\item  We say that $X$ is \emph{irreducible} if $X = X_1 \cup X_2$ with $X_1, X_2$ closed implies that $X = X_1$ or $X = X_2$. 
$X_1 \subseteq X$ is called \emph{irreducible} if it is an irreducible topological space with the topology induced by the restriction to $X_1$.\index{irreducible topological space}
\item Let $Y \subseteq X$ be closed. We say that a point $f \in Y$ is a \emph{generic point} of $Y$, if $\overline{\{  f \} } = Y$, where for $A \subseteq X$, $\overline{A}$ denotes the closure of $A$. \index{generic point}
\end{enumerate}
\end{defn}

\begin{prop}
Let $R$ be a $\s$-ring. We have:
\begin{enumerate}[(a)]
\item \label{vmbijection} The mapping 
$$\{ \a \si R \mid \a\text{ mixed and radical }\} \rightarrow \{ A \subseteq \sSpec(R) \mid A \text{ closed }\}, \a \mapsto \Vm(\a)$$
 is bijective and order-reversing.
\item \label{irred=prime} For $F \subseteq R$ it holds that $\Vm(F)$ is irreducible if and only if $\{F\}_m$ is prime.
\item $\sSpec(R)$ is quasi-compact.
\item The basic open sets $\{ \sD(f) \mid f \in R \}$ form a basis for the topology on $\sSpec(R)$.
\item Every irreducible closed subset $Y$ of $\sSpec(R)$ has a unique generic point $y$.
\end{enumerate}
\clearpage %%FIXME: remove, just so a page is not left basically completely blank
\begin{bew} $~$
\begin{enumerate}[(a)]
\item \label{orderreversingbij} That the mapping is order-reversing is obvious. The injectivity follows from the fact that by Theorem \ref{intersectionprimes} $\a = \bigcap_{\a \subseteq \p \in \sSpec(R)} \p$. By Remark \ref{vmsequal} we obtain the surjectivity,
 since $\Vm(\a) = \Vm(\{\a\}_m)$.
\item Since $\Vm(F) = \Vm(\{F\}_m)$, we can assume without loss of generality, that $F \si R$ is a radical, mixed $\s$-ideal.
For the first implication, ``$\Leftarrow$'', let $F \si R$ be prime, and $\Vm(F) = \Vm(\a) \cup \Vm(\b)$ with radical, mixed $\s$-ideals $\a, \b$. Assume that $\Vm(F) \not\subseteq \Vm(\a)$. Then by (\ref{orderreversingbij}), $\a \not \subseteq F$, so there exists an $a \in \a$, with $a \notin F$.
For any $b \in \b$ we then have $ab \in \p \fa \p \in \V(F) = \Vm(\a) \cup \Vm(\b)$, and with Theorem \ref{intersectionprimes} we get $ab \in F = \bigcap_{\p \in \Vm(F)}\p$. By assumption, $F$ is prime and $a \notin F$, which implies
 that  $b \in F$. But this means that $\b \subseteq F$, and thus $\Vm(F) \subseteq \Vm(\b)$, which shows the irreducibility. \\
\indent Now, for the other implication, ``$\Rightarrow$'', assume that $\Vm(F)$ is irreducible, and let $a,b \in R$ with $ab \in F$. Consider $F \subseteq \p \in \Vm(F)$. Then $ab \in \p$, 
which means that $a \in \p$ or $b \in \p$, since $\p$ is prime. This implies that $\p \in \Vm(\{a\}_m) \cup \Vm(\{b\}_m)$, which in turn implies that $\Vm(F) \subseteq \Vm(\{a\}_{m}) \cup \Vm(\{b\}_{m})$.
Now, by assumption, $\Vm(F)$ is irreducible, and thus it has to be that $\Vm(F) \subseteq \Vm(\{a\}_{m})$ or $\Vm(F) \subseteq \Vm(\{b\}_m)$. By the bijectivity of the mapping in (\ref{orderreversingbij}) this means that $a \in F$ or $b \in F$.
\item Let $\Vm(\a_i)_{i \in I}$ be a family of closed sets, $\a_i \si R$ mixed, radical for all $i \in I$, satisfying that 
$\bigcap_{i \in J} \Vm( a_i) \neq \emptyset$ for every finite $J \subseteq I$. By  going to the complement of open sets, quasi-compactness is equivalent to the implication that $\bigcap_{i \in I} \Vm(a_i) \neq \emptyset$.
By Lemma \ref{topologywelldef} we see that $\bigcap_{i \in I} \Vm( \a_i) = \Vm ( \sum_{i \in I} \a_i)$. Assume that $ \Vm ( \sum_{i \in I} \a_i) = \emptyset$. 
By Theorem \ref{intersectionprimes} this means that $\{ \sum_{i \in I} \a_i \}_m = R$. In particular, $1 \in \{ \sum_{i \in I} \a_i \}_m$. By the construction in Lemma \ref{lemshuffling} (and with the notation used there), this means that there has to be an $n \in \NE$,
so that $1 \in (\sum_{i \in I} \a_i )^{\{n\}}$. In particular, this means that $1$ can be written as a finite $R$-linear combination $\sum_{k=1}^l r_k a_k$ with $a_k \in (\sum_{i \in I} \a_i )^{\{n\}}, r_k \in R, k = 1,\ldots,l$. In particular, there exists a $J \subseteq I$ finite,
such that $a_k \in (\sum_{i \in J} \a_i )^{\{n\}}$ for all $k \in \{1, \ldots, l \}$. But this implies that $1 \in (\sum_{i \in J} \a_i)^{\{n\}}$, meaning that $\Vm(\sum_{i \in J} \a_i) = \cap_{i \in J} \Vm(\a_i) = \emptyset$, a contradiction. 
\item For an open subset $U \subseteq \sSpec(R)$ there exists by definition an $\a \si R$ such that $U = \sSpec(R) \setminus \Vm(\a)$. We can then write $U$ as a union of basic open sets as follows: $$U = \bigcup_{a \in \a} \sD(a).$$
\item By (\ref{irred=prime}), an irreducible closed subset $A$ of $\sSpec(R)$ has the form $A = \Vm(\p)$, for $\p \si R$ prime. This prime $\s$-ideal $\p$ is the unique generic point of $A$.
To see this, consider the closure of $\p$: $$\overline{\{\p\}} = \bigcap_{\{\p\} \subseteq \Vm(F)}\Vm(F).$$ From the definition of $\Vm(F)$ it holds that $\{\p\} \subseteq \V_m(F)$ if and only if $F \subseteq \p$. By Lemma \ref{topologywelldef} (\ref{vmintersectionideals}), and the obvious fact that we can restrict the intersection to $\s$-ideals, we get thus
\[ \overline{\{\p\}} = \bigcap_{F \subseteq \p}\Vm(F) = \Vm(\sum_{F \subseteq \p, F \si R} F) = \Vm(\p) = A \]
\end{enumerate}
\end{bew}
\end{prop}

\clearpage

\subsection{An alternative proof of Theorem \ref{intersectionprimes}}

There is an alternative proof of Theorem \ref{intersectionprimes} without ultrafilters, just using difference algebraic arguments. We will develop it here additionally.

\begin{defn}
Let $R$ be a difference ring and $\a \si R$ be a mixed $\s$-ideal of $R$. Further let $f \in R \setminus \a$. We set $$(\a:f):= \{ g \in R \mid gf \in \a \} \subseteq R$$
\end{defn}

\begin{lem}\label{a:f}
Let $R$ be a $\s$-ring, $\a \si R$ a mixed $\s$-ideal, $f \notin \a$. Then $(\a:f)$ is a mixed $\s$-ideal of $R$ which contains $\a$. If $\a$ is radical(prime), then $(\a:f)$ is radical(prime) as well.
\begin{bew}
We will only prove the difference-algebraic statements, since the rest of the assertion is standard in commutative algebra.
Let $a \in (\a:f), r \in R$. 

Since $\a$ is mixed and $af \in \a$ it follows that $\s(a)f \in \a$, which in turn means that $\s(a) \in (\a:f)$. This shows that $(\a:f)$ is a difference ideal. 
To see that $(\a:f)$ is mixed, let $g,h \in R$ with $gh \in (\a:f)$. Then we know that $(gh)f = (gf)h  \in \a$. Since $\a$ is mixed, this implies that $(gf)\s(h) = (g\s(h))f \in \a$, from which $g\s(h) \in (\a:f)$ follows immediately by definition. \\

%% Let $a,b \in (\a:f), r \in R$. Then $af, bf \in \a$ and thus $(a + b)f = af + bf \in \a$, which means that $a + b \in (\a:f)$. Similarly $$(ra)f = r\underbrace{(af)}_{\in \a} \in \a,$$
%% which means that $ra \in (\a:f)$. Also, since $\a$ is mixed and $af \in \a$ it follows that $\s(a)f \in \a$, which in turn means that $\s(a) \in (\a:f)$. We have shown that $(\a:f)$ is a difference ideal. 
%% That $\a \subseteq (\a:f)$ is an obvious conclusion from the fact that since $\a$ is an ideal, $fa' \in \a \fa a' \in \a$.  Finally, to see that $(\a:f)$ is mixed,
%% let $g,h \in R$ with $gh \in (\a:f)$. Then it holds that $(gh)f = (gf)h  \in \a$. Since $\a$ is mixed, this implies that $(gf)\s(h) = (g\s(h))f \in \a$, from which $g\s(h) \in (\a:f)$ follows immediately by definition. \\
%% \indent Assume now that $\a$ is radical, and let $g \in R$ such that $g^n \in (\a:f)$ for an $n \in \NE$. Then $g^nf \in \a$. This also implies (since $\a$ is an ideal), that $g^nf \cdot f^{n-1} = (gf)^n \in \a$.
%% Since $\a$ is radical, it follows from this that $gf \in \a$, and thus $g \in (a:f)$.\\ 
%% \indent Finally, assume that $\a$ is prime and let $g, h \in R$ with $gh \in (\a:f)$. This means that $(gh)f = g(hf) \in \a$. Since $\a$ is prime, it has to hold that $g \in \a$ or $hf \in \a$. Since $\a \subseteq (\a:f)$, it follows that $g \in (\a:f)$ or $h \in (\a:f)$.

\end{bew}
\end{lem}

\begin{lem}\label{maxmixed=prime}
Let $R$ be a difference ring and let $\emptyset \neq U \subset R$ be a multiplicatively closed subset of $R$. Then a mixed difference ideal $\a \subset R$ which is maximal (with respect to inclusion) in the class of mixed difference
ideals not meeting $U$, i.e. such that $\a \cap U = \emptyset$, is prime. 
\begin{bew}
Let $\a \subset R$ be a maximal mixed difference ideal not meeting $U$, and assume that $\a$ is not prime. Then there exist $f, g \in R$ such that $f,g \notin \a$ but $fg \in \a$.
$\a$ is a proper subset of the $\s$-ideal $(\a:f)$, since $g \in (\a:f)$ by definition, but by assumption $g \notin \a$. Then, since by Lemma \ref{a:f} $(\a:f)$ is mixed, by the maximality of $\a$, there exists an $u \in U \cap (\a:f)$. \\

We distinguish two cases: First, assume that $f \in U$.  We know that $uf \in \a$, but by assumption $f \in U$ and $U$ is multiplicatively closed, which implies that $fu \in U \cap \a$, a contradiction.

Now consider the case that $f \notin U$. Since $(\a:f)~\supsetneqq~\a$, this again implies that there exists an $u \in U \cap (\a:f)$. By definition of $(\a:f)$ we know that that $uf \in \a$, as well as $u,f \notin \a$. We can now apply the arguments as in the first case by considering $f' = u, g' = f$.
\end{bew}
\end{lem}

We are now ready to give an alternative proof of Theorem \ref{intersectionprimes}.

\begin{reptheorem}{intersectionprimes}
Let $R$ be a difference ring and $F \subseteq R$ be a subset of $R$. Then
\[ \{ F \}_m = \bigcap_{F \subseteq \p \si R, \p \text{ prime }} \p.\]
\begin{proof}
The inclusion ``$\subseteq$'' is obvious, since prime difference ideals are radical and mixed (see Remark \ref{rempropideals}). For the inclusion ``$\supseteq$'', let $g \in R, ~ g \notin \{ F \}_m$, and consider the multiplicatively closed set $U = \{ g^k \mid k \in \NE \} \subset R$. 
$\{ F \}_m \cap U = \emptyset$ since $\{ F \}_m$ is radical. By Lemma \ref{maxmixed=prime}, any maximal mixed difference ideal $\p$ that is disjoint with $U$ is a prime difference ideal. We can always find a maximal mixed difference ideal over $\{F\}_m$ by Zorn's Lemma since
the union of any ascending chain of mixed difference ideals is always a mixed difference ideal. In particular, since $\{F\}_m$ is a mixed difference ideal disjoint from $U$, this implies that there exists a prime difference ideal $\p \supseteq \{F\}_m$ such that $g \notin \p$,
which in turn implies that $g \notin \bigcap_{\a \subseteq \p \si R, \p \text{ prime }} \p$. By taking the contraposition of this we get the desired inclusion.
\end{proof}
\end{reptheorem}


\clearpage 

\section{Difference Varities}
In this section we will introduce a new concept of difference varieties. We will do so in a way that they correspond with the topology on $\sSpec(R)$ which we defined in the previous section. This section will again be based on M. Wibmer's lecture notes \cite{wibmer}, 
where this is worked out for the analogous case of perfect $\s$-ideals.

\subsection{Mixed Difference Varieties}


\begin{defn}
Let $A$ be a $\s$-ring. If $A$ is an integral domain, we call $A$ an \emph{integral $\s$-ring}. If, additionally, the endomorphism $\s$ on $A$ is injective, then we call $A$ a \emph{$\s$-domain}. \index{integral $\s$-ring} \index{$\s$-domain}
\end{defn}

\begin{rem}\label{sdomain=field}
Let $A$ be a $\s$-domain. Then its field of fractions $\operatorname{Quot}(A) =: k$ is a $\s$-field: For $\frac{r}{s} \in k$ we define $\s(\frac{r}{s}):= \frac{\s(r)}{\s(s)}$. Since $\s$ is injective, we have $\s(s) \neq 0$ for $s \neq 0$, which implies that $\s$ is well defined on $k$.
By this argument, we see that - in general - for an integral $\s$-ring $A$, $\operatorname{Quot}(A)$ is a $\s$-field (in this natural way), if and only if $A$ is a $\s$-domain.
\end{rem}

Our main purpose, in a first instance at least, is to investigate the properties of solutions to difference equations. 
We will start with a $\s$-field $k$ and look for solutions (zeros) of some $\s$-polynomial $p$ over $k$, i.e. $p \in k\{y_1, \ldots, y_n \}$. In general it will be a set of $\s$-polynomials $F \subseteq k\{y_1, \ldots, y_n \}$ that we will study. 
For this, we want to define a new concept of $\s$-varieties. However, we cannot mimic the usual approach from algebraic geometry, where one would work over the algebraic closure of $k$. The next remark shows why.

\begin{rem}\label{incompatibleextensions}
 Consider the constant $\s$-field $\Q$ and $K = \Q(\sqrt{2})$, with \\ $\s (\sqrt{2}) = \sqrt{2}$; $L = \Q(\sqrt{2}), \s(\sqrt{2}) = - \sqrt{2}$. 
Both $K$ and $L$ are $\s$-field extensions of $\Q$, but there cannot be a further extension $\Q \leq M$ of $\s$-fields, such that $K,L \leq M$ are both (isomorphic to) $\s$-subfields of $M$. 
To see this, assume there was such an $M$. Then the set $\{ a \in M \mid a^2 - 2 = 0 \}$ has exactly two elements, which we will call $\sqrt{2}, -\sqrt{2}$ (since $\sqrt{2} + (- \sqrt{2}) = 0$).
But $\sqrt{2} \in K$ has to be mapped to one of these two in any embedding. The same holds true for $\sqrt{2} \in L$, which already yields the contradiction,
 since in $M$ either $\s(\sqrt{2}) = \sqrt{2}$ or $\s(\sqrt{2}) = -\sqrt{2}$.
\end{rem}

To avoid this problem, we will define our new, mixed $\s$-varieties as functors. For this, we will need a few category-theoretic definitions:

\begin{defn}
Let $k$ be a $\s$-field. The category of all $\s$-ring extensions $A$ of $k$ (i.e. such that $k \subseteq A$ is a sub $\s$-ring of $A$) is denoted by $\sringk$. The morphisms are defined as follows: For $B,C \in \sringk$ we say that a morphism of $\s$-rings $\varphi: B \rightarrow C$ is a morphism of $\s$-ring extensions of $k$, if and only if $\varphi_{|k} = \id_k$.
The subcategory which arises from restricting the object class to integral $\s$-rings, the category of \emph{integral $\s$-ring extensions of $k$}, we denote by $\sintk$. \index{$\sintk$} \index{$\sringk$} \index{integral $\s$-ring extensions}
\end{defn}

Because $k$ is a field, any non-trivial morphism of $\s$ rings of $k$ to a $k$-$\s$-algebra is injective. This means that the category $\sringk$ is equivalent to that of $k$-$\s$-algebras, the only difference is if $k$ is a subset or just isomorphic to one. Now we are ready to define the new concept of $\s$-varieties of $\s$-rings, with mixed $\s$-ideals in mind:

\begin{defn}\label{defnVV}
Let $k$ be a $\s$-field and $B \in \sintk$ an integral $\s$-overring of $k$. Further let $F \subseteq k\{y_1, \ldots, y_n\}$ be a set of $\s$-polynomials over $k$. 
Then we define $\VV_B(F):= \{ b \in B^n \mid f(b) = 0 \fa f \in F \}$. A functor \\ $X: \sintk \rightarrow \Set$, for which there exists a set $F \subseteq k\{y_1, \ldots, y_n \}$, such that $X(B) = \VV_B(F)$ for all $B \in \sintk$, we denote as a \emph{$\s$-variety over $k$}, or a \emph{$k$-$\s$-m-variety}.
Here, $\Set$ denotes the usual category of sets with mappings as morphisms. We also write $X := \VV(F)$ as a short notation for this functor. \index{$\s$-variety} \index{$\s$-variety over $k$} \index{$k$-$\s$-m-variety}
\end{defn}

We can compare this definition to the definition of difference varieties in Section 1. The differencie lies in the category considered for the codomain of the functor, where instead of $\s$-fields over $k$ we take $\sintk$.
This begs the question what would happen if we considered yet another category instead, for example only allowing the ring to be perfectly $\s$-reduced, or a $\s$-domain. It can be shown that the category of difference varieties
yielded by those definitions is equivalent to the standard definition of difference varities. We will prove this at the end of this Section when we have the necessary tools to do it.

\begin{defn}
Let $k$ be a $\s$-field and $X: \sintk \rightarrow \Set$ be a $k$-$\s$-m-variety. We say a subfunctor $Y \subseteq X$ is a \emph{$\s$-m-subvariety} of $X$, if $Y$ is a $k$-$\s$-m-variety itself. \index{$\s$-m-subvariety}
\end{defn}

 \begin{rem}
Let $k$ be a $\s$-field and $X$ be a $k$-$\s$-m-variety. Not every subfunctor of $X$ is a $\s$-m-subvariety. Consider the functor $X = \VV(0)$, for $\{0\} \subset k\{y_1\}$.
For $B \in \sintk$ we denote by $B^* = \{ b \in B \mid b \text{ invertible } \}$ the set of units of $B$. Then for $B \in \sintk$, $B \mapsto B^*$ is a subfunctor $Y$ of $X$ (since $B^* \subset B \fa B \in \sintk$ and morphisms of rings always map units to units). However, $Y$ is not a $\s$-m-variety:
there exists no $F \subseteq k\{y_1\}$, such that $\VV_B(F) = B^* \fa B \in \sintk$.  Indeed, assume there was such an $F$, and let $0 \neq f \in F$. Then $f(b) = 0 \fa b \in B^*$ and $\fa B \in \sintk$. In particular,
for $B = k\langle y_1 \rangle = \operatorname{Quot}(k\{y_1\}) \in \sintk$ we get the that \\$f(y_1) = f = 0, ~ y_1 \in k \langle y_1 \rangle^*$, a contradiction.  
\end{rem}

\begin{defn}\label{defnI}
Let $X = \VV(F)$ be a $\s$-m-variety over the $\s$-field $k$, \\$F \subseteq k\{y_1,\ldots,y_n\}$. Then we set $$\I_m(X):= \{ f \in k\{y_1,\ldots,y_n\} \mid f(b) = 0 \fa b \in \VV_B(F), ~ B \in \sintk \}.$$ \index{ $\I_m(X)$}
\end{defn}

\begin{ex}\label{A^n}
Let $k$ be a $\s$-field and let $\{ 0 \} =: F \subseteq k\{y_1,\ldots,y_n\}$. Then a $\s$-m-variety $X$ is defined by $F$, $X(B) := \VV_B(F) \fa B \in \sintk$ with the property that $X(B) = B^n \fa B \in \sintk$.
For every $G \subseteq A\{y_1,\ldots,y_n\}$ the $\s$-m-variety given by $Y: B \mapsto \VV_B(G)$ is a $\s$-m-subvariety of $X$. 
\end{ex}

\begin{defn}\label{defA^n}
The $\s$-m-variety $X$ defined in Example \ref{A^n} is called the \emph{affine $n$-space}, and is denoted by $\mathbb{A}^n_k$, 
or simply $\mathbb{A}^n$, whenever $k$ is clear from the context.  \index{affine $n$-space} \index{$\mathbb{A}^n_k$}
Since for every $G \subseteq A\{y_1,\ldots,y_n\}$ the $\s$-m-variety given by $Y: B \mapsto \VV_B(G)$ is a $\s$-m-subvariety of $X$, 
we write $Y \subseteq \mathbb{A}^n$ as a shorthand for the former.
\end{defn}

We note that $0$ is in any (radical, mixed, difference) ideal, so it is not surprising that every $\s$-m-variety is a $\s$-m-subvariety of $\VV(0)$. This ``intuition'' will be made more concrete later on.

Since we have this functorial definition, we have, in principle, a whole proper class of solutions for most systems of difference equations. 
We want to have some sort of equivalence relation between solutions to group them up in a reasonable manner.

\begin{defn}\label{equivsols}
Let $k$ be a $\s$-field, $B,C \in \sintk$. Further let \\ $F \subseteq k\{y_1,\ldots,y_n\}$ be a system of difference equations and $b \in B^n, c \in C^n$ be solutions of $F$, i.e. $b \in \VV_B(F), c \in \VV_C(F)$.
We say that $b$ and $c$ are \emph{equivalent}\index{equivalent solutions} if the mapping $b \mapsto c$ is a well-defined isomorphism between the integral $\s$-rings $k\{b\}$ and $k\{c\}$  (as elements of $\sintk$). \index{equivalent solutions}
\end{defn}

%% \begin{lem}
%% Let $k$ be a $\s$-field, $B,B' \in \sintk$, and let $b \in B^n; b' \in B'^n$ be equivalent solutions for a system of difference equations $F \subseteq A\{y_1,\ldots,y_n\}$
%% \end{lem}

\begin{rem}
Recall the definition of equivalence of solutions for $\s$-varieties in Section 1, where two solutions $a,b$ of a system of difference equation over a difference field $k$ are said to be equivalent if the $\s$-field extensions $k\langle a \rangle$ and $k\langle b \rangle$ are isomorphic as $\s$-field extensions of $k$ via $a \mapsto b$.
This is in accordance with Definition \ref{equivsols}, i.e. solutions in difference field extensions are equivalent if and only if they are equivalent as solutions in integral $\s$-overrings in the sense of Definition~\ref{equivsols}.
\begin{bew}
Assume that there exist $\s$-field extensions $k \leq A,B$, and elements $a \in A^n$, $b \in B^n$ such that $a$ and $b$ are equivalent as solutions in the sense of Definition~\ref{equivsols}. Since $A,B$ are $\s$-fields, it means that $k\{a\}$ and $k\{b\}$ are $\s$-domains, 
and $k\langle a \rangle, k\langle b \rangle$ have the ``canonical'' difference structure induced by $k\{a\}, k\{b\}$ (see Remark \ref{sdomain=field}). Let $\varphi: k\{a\} \rightarrow k\{b\}, a \mapsto b$ be an isomorphism of integral $\s$-ring extensions of $k$.
Then we can define $$\tilde \varphi: k \langle a \rangle \rightarrow k\langle b \rangle, \frac{x}{y} \mapsto \frac{\varphi(x)}{\varphi{(y)}}.$$ This is a well-defined isomorphism of $\s$-field extensions of $k$, since:
\begin{align*}
\tilde \varphi \left(\s \left(\frac{x}{y}\right)\right) = \tilde \varphi \left( \frac{\s \left(x\right)}{\s \left(y\right)}\right) = \frac{ \varphi \left(\s \left(x\right)\right)}{ \varphi \left(\s \left(y\right)\right)} =  \frac{\s \left(\varphi \left(x\right)\right)}{\s \left(\varphi \left(y\right)\right)} = \s \left( \tilde \varphi \left(\frac{x}{y}\right)\right).
\end{align*}
The converse implication is obvious.
\end{bew}
\end{rem}

\begin{ex}
In the two $\s$-field extensions of $\Q$ in Remark \ref{incompatibleextensions} we have two solutions of the (algebraic) polynomial $y^2-2$, which represent two non-equivalent solutions in the difference algebraic sense,
since the $\s$-fields $\Q(\sqrt{2}),$ \\ $\s(\sqrt{2}) = \sqrt{2}$ and $\Q(\sqrt{2}), \s(\sqrt{2}) = -\sqrt{2}$ are not isomorphic. 
\end{ex}

\begin{ex}
Let $k$ be a $\s$-field. The $\s$-m-variety $X$ given by $\s(y)~\in~k\{y\}$, i.e. $X(B) = \VV_B(\s(y)) = \{ b \in B^n \mid \s(b) = 0 \} \fa B \in \sintk$, has a single point in any $\s$-field extension of $k$, namely $0$. However, in general integral $\s$-rings,
this is not necessarily the case: Take, for example, $B:= k\{y\}/[\s(y)] \in \sintk$. In $B$ we have $0 \neq $ ker$(\s) = [y+ [\s(y)]] \si B$, which means that in particular, $[y + [\s(y)]] \subseteq \VV_B(\s(y))$.
\end{ex}

It is not a coincidence that in the previous example we found more solutions on the $\s$-ring $B = k\{y\}/[\s(y)]$. The $\s$-ideal $[\s(y)]$ is radical and mixed, i.e. $[\s(y)] = \{ [\s(y)] \}_m$.
In fact, the ring $B$, as we chose it, plays an analogous role to that of the coordinate ring of an affine variety in the usual (algebraic) case.

The next proposition shows why our definition of $\s$-m-variety is ``the right one'' for mixed ideals:

\begin{prop}\label{I=F_m}
Let $k$ be a $\s$-field and $X = \VV(F) \subseteq \mathbb{A}^n$ be a mixed difference variety over $k$. Then $\I_m(X) = \{F\}_m \si k\{y_1,\ldots,y_n\}$. 
\begin{bew}
We will first show that $\I_m(X)$ is a radical, mixed $\s$-ideal.
Let \\ $f,~g~\in~\I_m(X)$, $h \in k\{y_1,\ldots,y_n\}$. Then, for every $B \in \sintk$, $b \in \VV_B(F)$, we have $f(b) = g(b) = 0$.
It follows that $(f + g)(b) = f(b) + g(b) = 0$ as well as $(fh)(b) = f(b)h(b) = 0 \cdot h(b) = 0$ and $\s(f)(b) = \s(f(b)) = \s(0) = 0$, so that $\I_m(X)$ is a $\s$-ideal.
It further follows that $h(b)^n = 0$ implies $h(b) = 0$, since $B$ is an integral domain, and this means that $h^n \in \I_m(X)$ implies that $ h \in \I_m(X)$. \\
\indent It only remains to show that $\I_m(X)$ is mixed. Let now $f,g \in k\{y_1,\ldots,y_n\}$ be such that $fg \in \I_m(X)$. This means that for all  $B \in \sintk$, $b \in \VV_B(F)$ we have
 $(fg)(b) = f(b) g(b) = 0$. Since $B$ is an integral domain,
this implies that $f(b) = 0$ or $g(b) = 0$. This also implies that $\s(f(b)) = \s(0) = 0$, or $\s(g(b)) = 0$, so that in any case $(f\s(g))(b) = 0$, from which it follows that $f\s(g) \in \I_m(X)$. We thus see that $\I_m(X)$ is radical and mixed, hence $\{F\}_m \subseteq k\{y_1,\ldots,y_n\}$. \\
\indent For the other inclusion, let $f \in \I_m(X)$. We will show that $f \in \{F\}_m$. Let $F \subseteq \p \si k\{y_1,\ldots,y_n\}$ be a prime $\s$-ideal.
Then, consider $B:= k\{y_1,\ldots,y_n\}/\p$. This is an integral $\s$-ring. Since $F \subseteq \p$, we know that $y + \p \in \VV_B(F)$. By assumption, we have $f \in \I_m(\VV(F))$, which means by definition that $f(y + \p) = 0$, which
in turn means that $f \in \p$. Since this holds for any prime $\p \si R$, Theorem \ref{intersectionprimes} implies that $f \in \{F\}_m$.
\end{bew}
\end{prop}
Note that, in the notation of the proof above,  it does not always have to be the same case, $f(b) = 0$ or $g(b) = 0$, as it depends on $B$ (and $b$, of course). In particular, $\I_m(X)$ does not have to be prime in general. \\

From this, we immediately get a further result on radical, mixed ideals, which is analogous to the case of radical ideals in algebraic geometry.
\begin{cor}\label{prod=cap}
Let $\a, \b \si k\{y\}$ be two radical, mixed difference ideals. Then $\a \cap \b = \{ \a \b \}_m$.
\begin{bew}
We can assume that $\a, \b \neq \{0\}$, as the assertion is obvious otherwise. Since $\a, \b$ are radical and mixed, we know from Proposition \ref{I=F_m} that $\a = \I_m(\VV(\a)), \b = \I_m(\VV(\b))$, and $\{ \a \b \}_m = \I_m( \VV( \a \b ))$.
For any $B \in \sintk$, we have:
\begin{align*} \VV_B( \a \b) = \VV_B( \a) \cup \VV_B( \b). \end{align*}
The inclusion ``$\supseteq$'' is obvious. For ``$\subseteq$'', let $p \in \VV_B(\a\b)$ and assume there exists an $f \in \a$, such that $f(p) \neq 0$.
Then, from the definition of $\VV_B(\a\b)$ it follows that $f(p)g(p) = 0 \fa g \in \b$. This means, however, that $p \in \VV_B(\b)$ (since $B$ is an integral domain). The other case is completely analogous.
Since this is true for any $B$, the $\s$-m-varieties are also equal: $\VV( \a \b) = \\ \VV( \a) \cup \VV( \b)$. Now,
\begin{flalign*} & \phantom{ = \{ f \in k\{y\} \mid f(p) =}\I_m(\VV(\a \b)) = \I_m(\VV(\a) \cup \VV(\b)) \\ & = \{ f \in k\{y\} \mid f(p) = 0 \fa p \in \VV_B(\a) \cup \VV_B(\b), ~ B \in \sintk \}, & \end{flalign*} 
and $f(p) = 0 \fa p \in \VV_B(\a) \cup \VV_B(\b), ~ B \in \sintk$, is equivalent to \\
\begin{align*}  \underbrace{f(p) = 0 \fa p \in \VV_B(\a), ~ B \in \sintk}_{\Leftrightarrow f \in \I_m(\VV(\a))} \text{ and }\\  \underbrace{f(p) = 0 \fa p \in \VV_B(\b), ~ B \in \sintk}_{\Leftrightarrow f \in \I_m(\VV(\b))}. \end{align*} \\
Hence, $\{\a\b\}_m = \I_m(\VV( \a \b)) = \I_m(\VV(\a)) \cap \I_m(\VV(\b)) = \{\a\}_m \cap \{\b\}_m = \a \cap \b$.
\end{bew}
\end{cor}

\begin{defn}
Let $k$ be a $\s$-field and let $X$ be a $\s$-m-variety over $k$. Further let $F \subseteq k\{y_1, \ldots, y_n\}$ be a system of difference equations over $k$ with $X(B) = \VV_B(F) \fa B \in \sintk$.
Then we consider the $\s$-ring $$k\{y_1, \ldots, y_n\}/\{F\}_m = k\{y_1, \ldots, y_n\}/\I_m(X) =: k\{X\}$$ and call it the \emph{coordinate ring} of $X$. Since $\{F\}_m$ is a radical, mixed $\s$-ideal, $k\{X\}$ is reduced and well-mixed. \index{coordinate ring}
\end{defn}

\begin{rem}
Let $k$ be a $\s$-field and $X$ a $k$-$\s$-m-variety. Further let $b~\in~X(B), ~ B \in \sintk, f + \I_m(X) \in k\{X\}$. Then the value of $f(b) \in k$ is independent of the representative $f$,
since for $f' + \I_m(X) = f + \I_m(X)$, we know that $f - f' \in \I_m(X)$, and thus by definition, $(f - f')(b) = 0$. By abuse of notation,
we will sometimes use the representative $f$ to refer to its equivalence class $f + \I_m(X)$ and we will simply write $f(b)$ to mean the well-defined value of evaluating $b$ on any representative of the class.
\end{rem}

We can now clarify what we meant after Definition \ref{defA^n}.

 \begin{lem}\label{bijsubvarsideals}
Let $k$ be a $\s$-field. Then the maps $X \mapsto \I_m(X)$ and $\a \mapsto \VV(\a)$ define inclusion-reversing bijections between the set of all $\s$-m-subvarieties of $\mathbb{A}^n$ and the radical, mixed $\s$-ideals of $k\{y_1,\ldots,y_n\}$.
\begin{bew}
From Proposition \ref{I=F_m} we know that $\I_m(\VV(\a)) = \a$ is true for all $\a~\si~k\{y_1,\ldots,y_n\}$ radical, mixed. Conversely, let $X~=~\VV(F)~\subseteq~\mathbb{A}^n$ be a $\s$-m-variety. Then we know  $\VV(\I_m(X)) = \VV(\I_m(\VV(F))) \subseteq \VV(F) = X$,
 since $F \subseteq \I_m(X)$. On the other hand, it is clear from the definitions of $\VV$ and $ \I_m$ that $X \subseteq \VV(\I_m(X))$, so that $X = \VV(\I_m(X))$. This proves the bijectivity of both mappings. That both mappings are inclusion-reversing follows directly from the definitions.
\end{bew}
\end{lem}

Note that since every $\s$-m-variety is a $\s$-m-subvariety of $\mathbb{A}^n$ for an $n~\in~\NE$, it is no restriction to consider $\mathbb{A}^n$ instead of an arbitrary $\s$-m-variety, as we can see in the following corollary:
\begin{cor}
  Let $X$ be a $\s$-m-variety over the $\s$-field $k$. Then there is a bijection between the radical, mixed $\s$-ideals of $k\{X\}$ and the $\s$-m-subvarieties of $X$ via
 \begin{align*} X \supseteq Y \mapsto \{f \in k\{X\} \mid f(b) = 0 \fa b \in Y(B), \fa B \in \sintk \} \\ =: \I_{k\{X\}}(Y). \end{align*}
\begin{bew}
If we identify the radical, mixed $\s$-ideals of $k\{X\}$ with the radical, mixed $\s$-ideals of $k\{y\}$ which contain $\I_m(X)$ (see Proposition \ref{bijideals}), then this is just the restriction of the mapping described in Lemma \ref{bijsubvarsideals}.
\end{bew}
\end{cor}

A further very interesting bijection can also help us to better understand equivalence classes of solutions: 
\begin{prop}\label{bijsols}
Let $X = \VV(F)$ be a $\s$-m-variety over the $\s$-field $k$. The equivalence classes of solutions of $F$ are in bijection with the $\s$-spectrum of the coordinate ring $\sSpec(k\{X\})$.
\begin{bew}
Let $B \in \sintk$, $b \in B^n$ be a solution of $F \subseteq k\{y_1,\ldots,y_n\}$, i.e. $f(b) = 0 \fa f \in F$. Consider the mapping $$\varphi: k\{y_1,\ldots,y_n\} \rightarrow B, y \mapsto b.$$
Then $F \subseteq $ ker$( \varphi) \si R$.
Since (forgetting the difference structure for a moment), $B$ is an integral domain, the ideal ker$(\varphi)$ has to be prime. It follows from this that $\{F\}_m = \I_m(X) \subseteq~$ker$(\varphi)$. 
In particular, this implies that the mapping $\varphi$ factors over $\I_m(X)$, and it induces a morphism of $\s$-rings $\tilde \varphi: k\{X\} \rightarrow B$. By the same argument as above, the kernel of this induced
morphism, $\p_b := $ker$(\tilde \varphi) \si k\{X\}$ is a prime $\s$-ideal of $k\{X\}$. The kernel of the mapping constructed this way is always the same for equivalent solutions. To see this, let $b' \in B'^n$, such that $k\{b\} \cong k\{b'\}$ via $\iota: b \mapsto b'$.
Then, for the mapping $\varphi': k\{y_1, \ldots, y_n\} \rightarrow B', y \mapsto b$, we get that $\varphi' = \iota \circ \varphi$ (which is well-defined, since Im$(\varphi)\subseteq k\{b\}$). In particular, since $\iota$ is an isomorphism, ker$(\varphi) = $ker$(\varphi')$. 
We define the mapping $\Psi$ from the equivalence classes of solutions of $F$ to $\sSpec(k\{X\})$, via $b \mapsto \p_b$.\\ 

\indent On the other hand, for $\p \in \sSpec(k\{X\})$, which we identify with \\$\I_m(X) \subseteq \tilde \p \in \sSpec(k\{y_1,\ldots,y_n\})$ (see Proposition \ref{bijideals}), consider the integral $\s$-ring $B(\p):= k\{y_1,\ldots,y_n\}/ \tilde \p$.
Since $\tilde \p$ is a prime $\s$-ideal, $B(\p)$ is an integral $\s$-ring. Set $b(\p) := \bar y \in B(\p)$, as the image of $y$ in $B(\p)$. Then, because $F \subseteq \I_m(X) \subseteq \tilde \p$, we know that $b(\p)$ is a solution of $F$. 
We define $\Psi^{-1}(\p)$ as the equivalence class of $b(\p)$. Then $\Psi$ and $\Psi^{-1}$ are inverses of each other, and hence, are both bijections.
\end{bew}
\end{prop}

From Proposition \ref{bijsols} we see that it is a good idea to concentrate on $\sSpec(k\{X\})$ for a $\s$-m-variety $X$ over a $\s$-field $k$.
 From here on, we will speak of the ``topology on/of X'' to refer to the topology on $\sSpec(k\{X\})$, as in Definition \ref{deftop}. 
We will also use the notation $x \in X$ to mean \\ $x~\in~\sSpec(k\{X\})$, or $T \subseteq X$ closed to speak of a closed subset of $\sSpec(k\{X\})$, and so forth.

\subsection{Morphisms of Mixed Difference Varieties}

So far we have only studied mixed difference varieties themselves, but not a way to relate them with each other; we have yet to properly define the category of mixed difference varieties over a fixed $\s$-field $k$. 
We still have to define what the morphisms in this category shall be.

\begin{defn}\label{spolynomialmaps}
Let $k$ be a $\s$-field, $X \subseteq \mathbb{A}^n,Y \subseteq \mathbb{A}^m$ $\s$-m-varieties over $k$. Then, a morphism of functors $f: X \rightarrow Y$ is called a \emph{morphism of $\s$-m-varieties over $k$} or a \emph{$\s$-polynomial map},
if there exist $\s$-polynomials $f_1,\ldots,f_m \in k\{y_1,\ldots,y_n\}$, such that $f(b) = (f_1(b),\ldots,f_m(b))$ for all \\ $b~\in~X(B), B~\in~\sintk$. \index{morphism of $\s$-m-varieties} \index{$\s$-polynomial map}
\index{morphism of $\s$-m-varieties} \index{$\s$-polynomial map}
\end{defn}

\begin{ex}
For two $\s$-m-varieties $X \subseteq Y = \mathbb{A}^n_k$, over the $\s$-field $k$, the inclusion mapping $\iota: X \hookrightarrow Y$ is a morphism of $\s$-m-varieties over $k$, since we can choose $f_1 = y_1, f_2 = y_2, \ldots, f_n = y_n$.
Similarly, for $m \geq n$ and $X \subseteq \mathbb{A}^m_k, Y \subseteq \mathbb{A}^n_k$ the ``projection onto $\mathbb{A}^n$'' is also a morphism of $\s$-m-varieties over $k$ (with the same choice of $f_i$ as the example above).
\end{ex}

\begin{rem}\label{dualmor}
Let $f: X \rightarrow Y$ be a morphism of $\s$-m-varieties over the \\ $\s$-field $k$, $X \subseteq \mathbb{A}^n, Y \subseteq \mathbb{A}^m$. Then, by definition, there exist difference polynomials $f_1, \ldots, f_m \in k\{y_1,\ldots,y_n\}$, such 
that $f(b) = (f_1(b),\ldots,f_m(b))$ for all $b \in B, ~ B \in \sintk$. Modulo $\I_m(X)$, these $f_i$ are unique:
 If there is $f_1', \ldots, f_m' \in k\{y_1,\ldots,y_n\}$ such that $f_i(b) = f'_i(b) \fa b \in B$, $B~\in~\sintk,$ and for all $i \in \underline{m}$,
then it follows that $(f_i - f_i')(b) = 0 \fa b \in B,$ $B~\in~\sintk$, which implies that $f_i - f_i' \in \I_m(X)$ by definition, for all $i \in \underline{m}$. \\
\indent Now, consider the mapping \[ \phi: k\{z_1,\ldots,z_m \} \rightarrow k\{X\}, ~ z_i \mapsto f_i + \I_m(X) =: \overline{f_i}. \]
This mapping factors over $\I_m(Y)$, since for $h \in \I_m(Y) \subseteq k\{z_1,\ldots,z_m\}$, $b \in X(B), ~ B \in \sintk$, we have that 
\[ (\phi(h))(b) = h(\overline f_1(b), \ldots, \overline f_m(b)) = h(f(b)). \]
Since $\phi$ is a morphism of $\s$-m-varieties over $k$, it follows that $f(b) \in Y(B)$, which implies that $h(f(b)) = 0$, by choice of $h$, hence $h \in $ ker$(\phi)$.
Altogether, this yields a mapping 
\[ f^* : k\{Y\} \rightarrow k\{X\}, ~ z_i + \I_m(Y) \mapsto y_i + \I_m(X). \]
This mapping is a morphism of integral $\s$-rings over $k$, and is called the \emph{dual mapping} or \emph{dual morphism} to $f$\index{dual morphism}. We have
\[ f^*(h)(b) = h(f(b)) \fa h \in k\{Y\}, b \in X(B), ~ B \in \sintk. \]
From the definition it follows that for morphisms $X \xrightarrow{f} Y \xrightarrow{g} Z$ of $\s$-m-varieties over $k$, we get $ (f \circ g)^* = g^* \circ f^*$. 
We thus get a contravariant functor $-^*$ from the category of mixed difference varieties over $k$ to $\sringk$.
\end{rem}

\begin{prop}\label{dualisequiv}
Let $k$ be a $\s$-field. Then $-^*$, as defined in Remark~\ref{dualmor}, is an anti-equivalence between the category of $\s$-m-varieties over $k$ and the subcategory of $\sringk$, which arises by restricting the object class to reduced, well-mixed, finitely $\s$-generated $\s$-overrings of $k$. 
In particular, a morphism $f: X \rightarrow Y$ of $\s$-m-varieties over $k$ is an isomorphism if and only if $f^*:~k\{Y\}~\rightarrow~k\{X\}$ is an isomorphism.
\begin{bew}
Since for a $\s$-m-variety $X$ over $k$, $\I_m(X)$ is radical and mixed, $k\{X\}$ is always a reduced and well-mixed $\s$-overring of $k$, 
and finitely $\s$-generated since $\s$-m-varieties are defined only for equations with finitely many difference variables. From this it follows that the functor $-^*$ from Remark \ref{dualmor} is well defined. 

It suffices to show that it is surjective on the skeleton of the categories and bijective on morphisms. 
Let $B$ be a finitely $\s$-generated, well-mixed and reduced $\s$-overring of $k$. We can then write $B \cong k\{y_1,\ldots,y_n\}/\a$, for an $\a \si k\{y_1,\ldots,y_n\}$ radical and mixed. The $\s$-m-variety $X = \VV(\a) \subseteq \mathbb{A}^n$
is then a preimage of the isomorphism class of $B$, since $\I_m(X) = \I_m(\VV(\a)) = \a$, because of Proposition \ref{I=F_m}. Thus, $B \cong k\{X\}$. \\
\indent Now, for the morphisms: First, let $X,Y$ be $\s$-m-varieties over $k$ and $f,g \in \Hom(X,Y)$ with $f^* = g^*$. Then we know that for every $h \in k\{X\}$, and every $b \in B, ~ B \in \sintk$:
\[ h(f(b)) = (f^*(h))(b) = (g^*(h))(b) = h(g(b)). \]
In particular, $f(b) = g(b) \fa b \in B, ~ B \in \sintk$, which implies that $f = g$, and $-^*$ is injective. 
On the other hand, let $\varphi: k\{Y\} \rightarrow k\{X\}$ be a morphism of $\s$-overrings of $k$. There exist $n,m \in \NE$ such that $X \subseteq \mathbb{A}^n, Y \subseteq \mathbb{A}^m$,
 which means that $k\{X\} = k\{z_1,\ldots,z_n\}/\I_m(X)$ and $k\{Y\} = k\{y_1,\ldots,y_m\}/\I_m(Y)$. We will construct a preimage of $\varphi$: Choose $f_1,\ldots,f_m \in k\{z_1,\ldots,z_n\}$ such that $\varphi(y_i + \I_m(Y)) = f_i + \I_m(X) \fa i \in \underline{m}$.
Then we define a morphism $f: X \rightarrow Y$ of $\s$-m-varieties over $k$ as follows: $f(b) := (f_1(b),\ldots,f_m(b)) \fa b \in B, ~ B \in \sintk$. This is well-defined: Let $h \in \I_m(Y)$. Then, by definition, $h(y_1 + \I_m(Y),\ldots,y_n + \I_m(Y)) = 0 + \I_m(Y)$.
This implies that $h(f_1 + \I_m(X),\ldots,f_m + \I_m(X)) = 0 + \I_m(X)$, since $\varphi$ is a morphism of $\s$-overrings of $k$. This in turn implies that $h(f(b)) = 0 \fa b \in X(B), ~ B \in \sintk$, which means that $f$ indeed maps onto $Y$ and $f^* = \varphi$, by construction.
\end{bew}
\end{prop}

This gives us a pretty good idea about the importance of the coordinate ring in difference algebra.
Having defined a category for $\s$-m-varieties, we can now see how this new category-theoretic language helps us to better understand the topological aspects of mixed difference varieties.

\begin{lem}\label{inducedcont}
Let $R,S,T$ be $\s$-rings, and $\varphi: R \rightarrow S,~ \psi: S \rightarrow T$ morphisms of $\s$-rings. Then the mapping $$\tilde \varphi: \sSpec(S) \rightarrow \sSpec(R), \p \mapsto \varphi^{-1}(\p)$$ 
induced by $\varphi$ is continuous. 
In fact, we have $\widetilde{ \psi \circ \varphi} = \tilde \varphi \circ \tilde \psi$, and in particular, $R \mapsto \sSpec(R)$ with $\psi \mapsto\tilde \psi$ is a contravariant functor from the category of $\s$-rings to \Top, the category of topological spaces.
\begin{bew}
Let $A = \V_m(F) \subseteq \sSpec(R)$ be closed. We have to show that $\tilde \varphi^{-1}(A) \subseteq \sSpec(S)$ is closed.
We have, 
\begin{align*} \tilde \varphi^{-1}(A) = \tilde \varphi^{-1}(\V_m(F)) = \{ \p \in \sSpec(S) \mid F \subseteq \varphi^{-1}(\p) \} \\ = \{\p \in \sSpec(S) \mid \varphi(F) \subseteq \p \} = \V_m(\varphi(F)). \end{align*}
The equality $\widetilde{ \psi \circ \varphi} = \tilde \varphi \circ \tilde \psi$ is immediately clear from the definitions.
\end{bew}
\end{lem}

We thus see how radical, mixed $\s$-ideals and the definition of $\s$-m-varieties as functors from $\sintk$, for a $\s$-field $k$, as well as the topology defined on $\sSpec(A\{X\})$ all fit together. 
These are all in analogous relations to the case for perfect $\s$-ideals and the Cohn topology outlined in Section 1. We will try to shed some light on the choice of the category $\sintk$ here:

\begin{defn}
Let $k$ be a $\s$-field. 
\begin{enumerate}[(a)]
\item We denote by $\s\text{\catname{-VarField}}_k$ the category of $\s$-varieties (as defined in Section 1). It has functors of the form $B \mapsto \VV_B(F)$ as objects, where $B$ is a $\s$-field extension of $k$.
As morphisms it has $\s$-polynomial maps defined in a fashion analogous to Definition \ref{spolynomialmaps}. 
\item Similarly, we denote by $\s\text{\catname{-VarDomain}}_k$ the category which has functors of the form $B \mapsto \VV_B(F)$ as objects, where $B$ is a $\s$-domain extension of $k$,
 and as morphisms $\s$-polynomial maps defined in a fashion analogous to Definition \ref{spolynomialmaps}. We define $\I_{\operatorname{Domain}}(X)$ for $X \in \s\text{\catname{-VarDomain}}_k$ and $\VV_{\operatorname{Domain}}(F)$ analogous to Definitions \ref{defnVV} and \ref{defnI}.
\item Finally,  we denote by $\s\text{\catname{-VarRing}}_k$ the category which has functors of the form $B \mapsto \VV_B(F)$ as objects, where $B \supseteq k$ is a perfectly $\s$-reduced ring over $k$,
 and as morphisms $\s$-polynomial maps defined in a fashion analogous to Definition \ref{spolynomialmaps}. We also define $\I_{\operatorname{Ring}}(X)$ for $X \in \s\text{\catname{-VarRing}}_k$ and $\VV_{\operatorname{Ring}}(F)$ analogous to Definitions \ref{defnVV} and \ref{defnI}.
\end{enumerate}
In all three cases $F \subseteq k\{y\}$ denotes a set of $\s$-polynomials on finitely many difference variables $y = (y_1, \ldots, y_n)$.
\end{defn}

\begin{prop}
Let $k$ be a $\s$-field. The three categories $\s\text{\catname{-VarField}}_k$, $\s\text{\catname{-VarDomain}}_k$ and $\s\text{\catname{-VarRing}}_k$ are equivalent.

\begin{bew}
Similar to Proposition \ref{dualisequiv}, the category of $\s$-varieties $\s\text{\catname{-VarField}}_k$ is anti-equivalent to the category of perfectly $\s$-reduced $\s$-overrings of $k$ which are finitely $\s$-generated over $k$ (see \cite{wibmer}, Theorem 2.1.21).
It suffices to show that the other two categories are also anti-equivalent to it. From the proof of Proposition \ref{dualisequiv} we can see that it is enough to show that $\I_{\operatorname{Domain}}(\VV_{\operatorname{Domain}}(\a)) = \{\a\}$, and $\I_{\operatorname{Ring}}(\VV_{\operatorname{Ring}}(\a)) = \{\a\}$,
 where $\a \si k\{y\}$ is a $\s$-ideal and $\{ \a \}$ its perfect closure. 

Let first $X  = \VV_{\operatorname{Ring}}(F) \in \s\text{\catname{-VarRing}}_k$ be a $\s$-m-variety in this sense of perfectly reduced $\s$-rings. 
We first show that \begin{align*} \I_{\operatorname{Ring}}(X) =  \{ f \in k\{y\} \mid f(b) = 0 \fa b \in \VV_B(F), \\ B \supseteq k\text{ perfectly }\s\text{-reduced} \} \end{align*}
 is a perfect $\s$-ideal. Similar to Proposition \ref{I=F_m}, we know that $\I_{\operatorname{Ring}}(X)$ is a difference ideal. Let $f \in k\{y\}$ with $\s^{i_1}(f) \cdots \s^{i_r}(f) \in \I(X)$. This means that for all $b \in \VV_B(F)$,
 $B$ perfectly $\s$-reduced: $\s^{i_1}(f)(b) \cdots \s^{i_r}(f)(b) = 0$. Since $B$ is perfectly $\s$-reduced, this means that $f(b) = 0$ for all such $b$,
 which in turn, by definition, means that $f \in \I_{\operatorname{Ring}}(X)$. Since every $\s$-domain is perfectly $\s$-reduced, the argument also works for $X \in \s\text{\catname{-VarDomain}}_k$, with $\I_{\operatorname{Domain}}$ instead of $\I_{\operatorname{Ring}}$.
This implies that $$ \{ F \} \subseteq \I_{\operatorname{Domain}}(\VV_{\operatorname{Domain}}(F))\text{, as well as } \{ F \} \subseteq \I_{\operatorname{Ring}}(\VV_{\operatorname{Ring}}(F)).$$
For the other inclusion ``$\supseteq$'', we shall first consider $X \in \s\text{\catname{-VarDomain}}_k$.
For $F \subseteq k\{y\}$, we have $\{F\} \supseteq \I_{\operatorname{Domain}}(\VV_{\operatorname{Domain}}(F))$. To show this, let \\ $f \in \I_{\operatorname{Domain}}(\VV_{\operatorname{Domain}}(F))$.
We know that $\{F\}$ is the intersection of all $\s$-prime ideals of $k\{y\}$ which contain $F$ (see Theorem \ref{intersectionperfect}), so it is enough to show that $f \in \p$ for each $\s$-prime $\p \si k\{y\}$ with $F \subseteq \p$.
We define the $\s$-domain $B:= k\{y\}/\p =: k\{a\}$, with $a := y + \p \in k\{y\}/\p$. Since $F \subseteq \p$, we get $a \in \VV_B(F)$, which, by definition of $\I_{\operatorname{Domain}}(\VV_{\operatorname{Domain}}(F))$, means that $f(a) = 0$. Hence, $f \in \p$ for all $\s$-prime $\p$
with $F \subseteq \p$, which in turn implies that $f \in \{F\}$. Since every $\s$-domain $B$ is perfectly $\s$-reduced, this works for the category $\s\text{\catname{-VarRing}}_k$ as well, with $\I_{\operatorname{Ring}}, \VV_{\operatorname{Ring}}$ instead of $\I_{\operatorname{Domain}}, \VV_{\operatorname{Domain}}$.
\end{bew}

\end{prop}

While there are many analogies with the Cohn topology, a very important aspect still remains open. Perfect difference ideals of $k\{y\}$ satisfy the ascending chain condition (see Chapter 3, Theorem IV of \cite{cohn}), which means that the Cohn topology yields a Noetherian topological space.
In fact, there is a generalization of the Hilbert basis theorem for this case; see Chapter 3, Theorem V of \cite{cohn}. The methods used to prove this in \cite{cohn} do not seem to work as simply for the case of radical, mixed difference ideals.
In fact, as far as the author knows, it is still not known if radical, mixed difference ideals satisfy the ascending chain condition. Mixed difference ideals do not satisfy it, as was shown by Levin in \cite{levinmixed}.

%% \begin{rem}
%% Let $f: X \rightarrow Y$ be a morphism of $\s$-m-varieties over the integral $\s$-ring $A$. Then the morphism $f^*: A\{y\} \rightarrow A\{x\}$ of $-s$-overrings of $A$ induces a continuous function
%% \[ \tilde{(f^*)}: \sSpec(A\{X\}) \rightarrow \sSpec(A\{Y\}), M \mapsto (f^*)^{-1}(M) \]
%% On the other hand FIXME: finish!
%% as in Lemma \ref{inducedcont}
%% \end{rem}

%% \begin{ex}
%% 2.4, here:more points? + 2.2.5

%% \end{ex}

%% section 2.3 not necesarry!

\clearpage 
\section{Difference Kernels}
%\documentclass[12pt,a4paper,BCOR15mm,twoside,DIV12]{article}
\documentclass{article}
%\usepackage[paper=a4paper,left=20mm,right=20mm,top=25mm,bottom=25mm]{geometry}
\usepackage[english]{babel}
\usepackage[utf8]{inputenc}
\usepackage{amsmath}
\usepackage{color}
\usepackage{amssymb}
\usepackage{amsfonts}
\usepackage{amsthm}
\usepackage{hyperref}
\usepackage{makeidx}
\usepackage{graphicx, float,epsfig}
\usepackage[nottoc,numbib]{tocbibind}


\newcommand{\properideal}{%
  \mathrel{\ooalign{$\lneq$\cr\raise.22ex\hbox{$\lhd$}\cr}}}

\def\P{\mathcal{P}}
\def\I{\mathbb{I}}
\def\R{\mathbb{R}} 
\def\E{\mathcal{E}} 
\def\NE{\mathbb{N}_{\geq1}} 
\def\N{\mathbb{N}} 
\def\Z{\mathbb{Z}} 
\def\Q{\mathbb{Q}} 
\def\F{\mathbb{F}}
\def\Vm{\mathcal{V}_m}
\def\V{\mathcal{V}}
\def\VV{\mathbb{V}}
\def\C{\mathbb{C}}
\def\U{\mathcal{U}}
\def\a{\mathfrak{a}}
\def\b{\mathfrak{b}}
\def\p{\mathfrak{p}}
\def\q{\mathfrak{q}}
\def\s{\sigma}
\def\si{\unlhd_{\sigma}}
\def\GL{\text{GL}}
\def\supp{\text{Supp}}
\def\id{\text{id}}
\def\n{\underline{n}}
\def\Spec{\text{Spec}}
\def\sSpec{\sigma\text{-Spec}}
\def\diag{\text{diag}}
\def\End{\text{End}}
\def\Hom{\text{Hom}}
\def\fa{\text{ for all }}
\def\Tr{\text{Tr}}
\def\Id{\text{Id}}
\def\ker{\text{ker}}
\def\H{\mathcal{H}}
\def\trdeg{\text{trdeg}}
\def\sdim{\sigma\text{-dim}}


\renewcommand{\labelenumi}{\alph{enumi})}
%\renewcommand{\P}{\textfrak{P}}
\newcommand{\cupdot}{\mathop{\mathaccent\cdot\cup}}
\newenvironment{bew}{\begin{proof}[Proof]}{\end{proof}}
\theoremstyle{definition}
\newtheorem{Satz}{Satz}[section]
\newtheorem{theorem}[Satz]{Theorem}
\newtheorem{ex}[Satz]{Example}
\newtheorem{cor}[Satz]{Corollary}
\newtheorem{algorithm}[Satz]{Algorithm}
\newtheorem{prop}[Satz]{Proposition}
\newtheorem{rem}[Satz]{Remark}
\newtheorem{defn}[Satz]{Definition}
\newtheorem{lem}[Satz]{Lemma}


\makeindex
\title{Difference Kernels}
\author{Andr\'{e}s Goens}
\date{\today}

\begin{document}
\section{Difference Kernels}

Not corrected.
%% blah blah de entrada...
%% \begin{theorem}

%% Let $k$ be a $\s$-field and $a$ 
%% \end{theorem}

\begin{defn}
Let $\p \unlhd k\{y\}[d], d \geq 1$ be a prime ideal of $k\{y\}[d]$. Then $\p$ is called a weak difference kernel of length $d$, if $\s(\p[d-1]) \subseteq \p$. It is called a difference kernel of length $d$, if $\s^{-1}(\p) = \p[d-1]$.
\end{defn}\index{(weak) kernels}

\begin{rem}
It is easy to see that a kernel is always a weak kernel, but the converse is not necesarilly true: a weak kernel only guarantees the inculsion $\p[d-1] \subseteq \s^{-1}(\p)$.
\end{rem}

\begin{ex}
Let $\p \si k\{y\}$ be a prime $\s$-ideal, $d \geq 1$. Then $\p[d] \unlhd k\{y\}[d]$ is a weak kernel: since $\p$ is a $\s$-ideal we have $\s(\p[d-1]) \subseteq \p$, 
and since $\p$ is prime we know that $\p[d]$ has to be prime as well. If $\p$ is also reflexive (i.e. a $\s$-prime ideal), then $\p[d]$ is a kernel. 
\end{ex}

Inspired by the former example we define the following:
\begin{defn}
Let $\p \unlhd k\{y\}[d]$ be a weak kernel of length $d$, and $\p' \si k\{y\}$ be a prime $\s$-ideal. Then we call $\p'$ a realization of $p$, iff $\p \subseteq \p'$, 
and we call the realization regular if it also holds that $\p'[d] = \p$. Similarly, if $\p$ is a kernel, then we require a realization to be a $\s$-prime ideal.
\end{defn}\index{(regular) realization}

The former example and the acompanying definition is actually much more general than it would seem to be at first. 
It is in fact the case, that weak kernels are exactly of the form $\p[d]$ for prime $\s$ ideals,
and kernels for $\s$-prime ones, i.e., we can always find a regular realization of (weak) kernels. 
To prove this, however, we still need to develop more of a framework around difference kernels.

\begin{rem}\label{sigmawelldeffker}
Let $\p \subseteq k\{y\}$ be a weak difference kernel of length $d$. Then, $\s$ induces a well-defined mapping 
\begin{align*}
\s: k[y,\ldots,\s^{d-1}(y)]/\p[d-1] \rightarrow k[y,\ldots,\s^{d}(y)]/\p
\end{align*}
If $\p$ is a difference kernel, then this mapping is injective. 
If we set $a := \bar y = y + \p \in k(\p) = k\{y\}[d]/\p$, then we can extend $\s$ to the quotient fields:
\[ \s: k(a,\ldots,s^{d-1}(a)) \cong \text{Quot}(k[y,\ldots,\s^{d-1}(y)]/\p[d-1]) \rightarrow k(a,\ldots,\s^d(a)) = k(\p) \]
\end{rem}

Even in the case of weak difference kernels we can work with the properties of field extensions, though we cannot properly extend $\s$ to the fields.
 In particular, we get a nice way of defining some sort of ``difference degree'' of (weak) kernels:
\begin{defn}
Let $\p \subseteq k\{y\}$ be a (weak) difference kernel of length $d$, and let $a:= y + \p \in k(p)$; we define the difference dimension of $\p$ as follows:
\begin{align*} \sdim(\p) := \text{trdeg}(k(\p)/k(p[d-1]) = \text{trdeg}((k\{y\}[d]/\p)/(k\{y\}[d-1]/p[d-1])) \\  = \text{trdeg}(k(a,\ldots,\s^{d}(a))/k(a,\ldots,\s^{d-1}(a))) \end{align*}
\end{defn}\index{$\s$-dimension of a $\s$-kernel}

If we want to show that (weak) difference kernels are the intersection of (prime $\s$-ideals) $\s$-prime ideals with $k\{y\}[d]$,
it is reasonable to consider some sort of extension, or \emph{prolongation} of a (weak) kernel, which would be the intesrection with $k\{y\}[d+1]$. 
This motivates the following definition:

\begin{defn}
Let $\p \subseteq k\{y\}$ be a (weak) difference kernel of length $d$, and $\p' \supset \p$ be a further (weak) difference kernel, of length $d+1$.
Then we call $\p'$ a \emph{prolongation} of $\p$, iff $\p'[d] = \p$. If it holds further that $\sdim(\p) = \sdim(p')$, then we call the prolongation \emph{generic}.
\end{defn}\index{prolongation, generic}

\begin{defn}
Let $\p \subseteq k\{y\}$ be a (weak) difference kernel of length $d$, and let $\p'$ be a realization of $\p$. We call $\p'$ a principal realization of $\p$, iff $\p'[i+1]$ is a generic prolongation of $\p'[i]$ for all $i \geq d$.
\end{defn}

\begin{ex}
Consider the difference polynomials $f_1 := \s(y_2) + 1, f_2:= \s(y_1)y_2 + y_1y_2 + \s(y_1) + y_1 + y_2 \in k[y_1,y_2,\s(y_1),\s(y_2)] =: R$.
A calculation with \cite{M2} quickly identifies that the ideal $(f_1,f_2) \unlhd R$ is prime, and in fact, $f_1,f_2$ is a Gr\"{o}bner Basis of $(f_1,f_2)$ with respect to the
term ordering $y_1 < y_2 < \s(y_1) < \s(y_2)$. This means in particular, that $(f_1,f_2)[0] = \{0\}$, and thus $(f_1,f_2)$ is a weak kernel of length $1$ of $k\{y_1,y_2\}$. However,
\[ f_1 \cdot (\s^2(y_1) + \s(y_1) + 1 ) - \s(f_2) = 1 \]
Which means that any $\s$-ideal of $k\{y_1,y_2\}$ containing $f_1, f_2$ is already the whole ring, and is not prime, by definition.
\end{ex}

We see thus, that we need a stronger concept for weak kernels. 

\begin{defn}
Let $\p \unlhd k\{y\}[d]$ be a weak kernel of length $d$. We define 
\[\p^* := ([\p]^*)[d] = \{ f \in k\{y\} \mid \text{ there exists an } n \in \N: \s^n(f) \in [\p] \} \cap k\{y\}[d] \]
Then $\p^*$ is an ideal in $k\{y\}[d]$. We say that $\p$ is \emph{regular} if $\p^* \unlhd k\{y\}[d]$ is prime. \index{regular weak kernel}
\end{defn}

\begin{rem}\label{regular->kernel}
From the definitions it is easily clear that for a weak kernel $\p \unlhd k\{y\}[d]$, $\p^*$ is a kernel if and only if $\p$ is regular. 
\end{rem}

The definition of $\p^*$ for a weak kernel $\p$ of length $d$, while clear from the theoretical standpoint, makes use of the whole difference polynomial ring $k\{y\}$. 
We would want a way of locally obtaining $\p^*$, i.e., only in $k\{y\}[d]$. Luckily, there is a nice algorithmical way of doing this using Gröbner bases.
To do this, we fix a lexicographical ordering on the variables $y = y_1, \ldots, y_n$, and from this we define a lexicographical ordering on $k\{y\}$, namely:
\[ \s^{m_1}(y_{i_1}) < \s^{m_2}(y_{i_2})\text{ if and only if }m_1 > m_2\text{ or }m_1 = m_2\text{ and }y_{i_1} < y_{i_2} \]
 By a lexicograhpical order we mean that for two variables $a, b$:
\[ a^{k_1} < b^{k_2}\text{ if and only if }a < b\text{ or }a = b\text{ and }k_1 < k_2 \]
We will use this term ordering for the remainder of this section.
\begin{rem}\label{remorder}
A polynomial $f \in k\{y\}$ and its leading term $LT(f)$ have the same order. Further, if $k$ is inversive it holds that if $f$ is a power of $\s$, i.e. $f = \s^m(f')$ for an $f \in k\{y\}$,
if and only if it is the case with $LT(f)$, and then it holds that $LT(f) = s^k(LT(f')) = LT(\s^k(f'))$.
\begin{bew}
That Ord$(f) \leq $Ord($LT(f)$) is obvious. Assume then that Ord$(f) < $Ord($LT(f)$). Then there exists a monomial $m$ of $f$
with Ord($f$) = Ord($m) <$ Ord($LT(f)$). But then $m > LT(f)$, a contradiction. If $f =  \s^k(f')$, then every monomial of $f$ is as well an $k$-th power of $\s$, in particular, $LT(f)$ as well.
On the other hand let $LT(f) = \s^k(g)$ for a monomial $g \in k\{y\}$, and assume $f$ cannot be written as $\s^k(f')$ for an $f' \in k\{y\}$. 
Then there must be a monomial $m$ of $f$ which cannot be written as $\s^k(m')$ for a monomial $m' \in k\{y\}$. In particular, since $k$ is inversive, $m'$ has a variable $\s^j(y_i)$ with $j < k$, which is not the case
for $LT(f)$ by assumption. This means, however, that $m' > LT(f)$, again a contradiction. 
The last equalities all hold by the definition of the term ordering.
\end{bew}
\end{rem}

\begin{lem}
Let $k$ be a $\s$-field, and $k^*$ its inversive closure. Further let $\p \subset k\{y\}[d]$ be a weak kernel of length $d$, and $f_1, \ldots, f_k$ be a Gröbner basis for $\p$. 
Then the ideal $\p^*$ is given by 
\[ \p^* = \underbrace{(\s^{-i}(f_j) \mid 0 \leq i \leq d, 1 \leq j \leq k)}_{\unlhd k^*\{y\}[d]} \cap k\{y\}[d] \]
where we by this notation ignore the $\s^{-i}(f_j)$ which do not exist. 
In particual, if $k$ is inversive, it holds that
\[ \p^* = (\s^{-i}(f_j) \mid 0 \leq i \leq d, 1 \leq j \leq k) \unlhd k\{y\}[d] \]
\begin{bew}
We consider first the special case where $k$ is inversive. One inclusion is obvious, as the elements $\s^{-i}(f_j)$, if they exist, have to be in $\p^*$. For the other inclusion let $f \in \p^*$.
It is enough to show that there exists a $\s^{-i}(f_j)$ such that $LT(\s^{-i}(f_j))$ divides $LT(f)$ (see for example \cite{cox}, p.75). 
By definition of $\p^*$ there exists an $n \in \N$ such that $\s^n(f) \in [\p]$. This means that $\s^n(LT(f)) \in LT([\p][d+n])$. 
Since $f_j, 0 \leq j \leq k$ is a Gröbner basis of $\p$, it is clear that $LT([\p][d+n])$ is generated by $LT(\s^i(f_j)), 0 \leq i \leq n, 0 \leq j \leq k$ as an ideal in $k\{y}[d+n]$.
Hence, there exists a $\s^i(f_j)$ and a monomial $m \in k\{y\}[d+n]$ such that $\s^n(LT(f)) = LT(\s^i(f_j)) h$. Since these are monomials and $k$ is inversive, it follows that $LT(\s^i(f_j
By Remark \ref{remorder}
\end{bew}
\end{lem}

\begin{lem}\label{primeoverp1}
Let R be an integral domain and $I \unlhd R[y]$ be an ideal of $R[y] = R[y_1,\ldots,y_n]$ which satisfies that $I \cap R = \{ 0 \}$.
Then there exists a prime ideal $P, I \subseteq P \subseteq R[y] $ with $P \cap R = \{0\}$.
\begin{proof}
We can assume without loss of generality that I is radical:
Namely, if $f \in \sqrt{I} \cap R$, then there exists an $m \in \N$ such that $f^m \in I \cap R = \{0\}$, and since $R$ is an integral domain this already means that $f = 0$.
We then note that for two sets $A,B \subseteq R[y]$ it holds that $\sqrt{A}\sqrt{B} \subseteq \sqrt{AB}$: Consider $f \in \sqrt{A}, g \in \sqrt{B}$. Then there exist $m, \tilde m \in \N$ such that $f^m \in (A), g^{\tilde m} \in (B)$;
 assume without loss of generality that $m > \tilde m$, then $(fg)^m \in (A)(B)$, which implies $fg \in \sqrt{(A)(B)} = \sqrt{AB}$.
Now, for the proof, consider the set of all radical ideals $J$ contaning $I$ which satisify $J \cap R = \{0\}$. This set is not empty and is inductively ordered by inclusion.
By Zorn's lemma this means that there is a maximal element $P$ of this set. This ideal $P$ is prime, then: assume there exist $f,g \notin P$ with $fg \in P$. 
Then the radical ideals $\sqrt{P \cup \{f\}}$, $\sqrt{P \cup \{f\}}$ strictly include $P$, and by the maximality of $P$ it means there exist $t_1, t_2 \in R\backslash\{0\}$ such that
$t_1 \in \sqrt{P \cup \{f\}}$, $t_2 \in \sqrt{P \cup \{g\}}$. But in particular, because $R$ is free of zero divisors, this implies that
 \[0 \neq t_1t_2 \in \sqrt{P \cup \{f\}}\sqrt{P \cup \{g\}} \subseteq \sqrt{ \underbrace{(P \cup \{f\})(P \cup \{g\})}_{=P\text{, since }fg \in P}} = P\]
A contradiction.
\end{proof}
\end{lem}

\begin{prop}\label{genericprol}
Let $\p \subseteq k\{y\}$ be a (regular weak) difference kernel of length $d$. Then there exists a prolongation of $\p$. If $\p$ is a kernel, then there exists a prologation that is generic. 
\begin{bew}
Consider the canonical epimorphism of rings $k[y,\ldots,\s^{d}(y)] \rightarrow k[y,\ldots,\s^{d}(y)]/\p$ and let $a = y + \p$ be the image of $y$ under it. 
Then the we have a subring $R_1: = k[a,\ldots,\s^{d-1}(a)] \subseteq k[a,\ldots,\s^{d}(a)] = k[y,\ldots,\s^{d}(y)]/\p =: R$. Now, consider the univariate polynomial ring over $R_1$ on the free variable $\s^d(y)$:
\[ R_1[\s^d(y)] = k[a,\ldots,\s^{d-1}(a)][s^d(y)] = k[a,\ldots,\s^{d-1}(a),s^d(y)] \]
We have a natural morphism of rings \[ R_1[s^d(y)] \rightarrow R = k[a,\ldots,\s^{d}(a)], \s^d(y) \mapsto \s^d(a) \]
Let $\p_1$ be the kernel of this morphism. Since $R$ is an integer domain (as $\p$ is prime), so is $\p_1$ also prime. 
Now we consider the univariate polynomial ring $R[\s^{d+1}(y)] \supset R = k[a,\ldots, \s^d(a)]$. Here we get a natural definition of $\s: R_1[\s^d(y)] \rightarrow R[\s^{d+1}(y)]$ by mapping $\s( \s^d(y))  := \s^{d+1}(y)$.
By definition of $\p_1$ we have that $\s(\p_1) \cap R = \{0\}$. 

We first consider the case where $\p$ is a kernel. In this case, $\s$ from Remark \ref{sigmawelldeffker} is injective and we can go over to the quotient fields:
We let $\tilde \p_1$ be the prime ideal which is the kernel of the morphism of rings
\[ \text{Quot}(R_1)[s^d(y)] \rightarrow \text{Quot}(R) = k(a,\ldots,\s^{d}(a)), \s^d(y) \mapsto \s^d(a) \]
And let $\tilde \p_2$ be a minimal prime ideal in 
\[k(a,\ldots,\s^d(a))[\s^{d+1}(y)] = \text{Quot}(k[a,\ldots,\s^d(a)])[\s^{d+1}(y)] \] containing $\s(\tilde \p_1)$.
In particular, $\s$ is injective and thus an isomorphism to the prime ideal $\s(\tilde \p_1) \subseteq \s(k)(\s(a),\ldots,\s^d(a))[\s^{d+1}(y)]$.
It holds that the mapping
\[ \s: k(a,\ldots,\s^{d-1}(a))[\s^d(y)]/\tilde \p_1 \rightarrow k(a,\ldots, \s^d(a))[\s^{d+1}(y)]/\tilde \p_2 \]
is well-defined and injective as well. By abuse of notation (and because of the minimality of $\tilde \p_2$ again) 
we will also call $\s^{d+1}(a)$ the image of $\s^{d+1}(y)$ in $k(\tilde \p_2) = \text{Quot}(k(a,\ldots, \s^d(a))[\s^{d+1}(y)]/\tilde \p_2)$.
Then the kernel $\q$ of \[ k[y,\ldots,\s^{d+1}(y)] \rightarrow k(\tilde p_2), \s^{i}(y) \mapsto \s^{i}(a) \] is a prolongation of $\p$, as above, and we have:
\begin{align*}
\sdim(\p) = \trdeg( k(a,\ldots,\s^d(a)) / k(a,\ldots,\s^{d-1}(a))) = \text{dim}(k(a,\ldots,\s^{d-1}(a))[\s^d(y)]/\tilde \p_1) \\ = \text{dim}(\s(k)(\s(a),\ldots,\s^d(a))[\s^{d+1}(y)]/\s(\tilde \p_1)) 
= \text{dim}(k(a,\ldots,\s^{d}(a))[\s^{d+1}(y)]/\tilde \p_2) \\ = \trdeg(k(a,\ldots,\s^{d+1}(a))/k(a,\ldots,\s^d(a))) = \sdim(\q)
\end{align*}

Now back to the general case: If $\p$ is a regular weak kernel, by the proof above, there existis a prolongation of $\p^*$, since by Remark \ref{regular->kernel}, $\p^*$ is a kernel. 
In particular, since $\p \subseteq \p^*$ we know that \[ (\s(\tilde \p_1)) \subseteq (\s(\tilde \p_1)) \subset \tilde p_2 \],
and since $\{0\} = \tilde p_2 \cap k(a,\ldots,\s^d(a)) \supseteq k[a,\ldots,\s^d(a)] = R$ it yields that $(\p_1) \cap R = \{0\}$.
By Lemma \ref{primeoverp1} there exists a minimal prime ideal $\p_2 \supset (\s(\p_1))$ of $R[\s^{d+1}(y)]$ containing $\s(p_1)$ with $\p_2 \cap R = \{0\}$. 
We thus get a well-defined maping
\[ \s: k[a,\ldots,\s^{d-1}(a)][\s^d(y)]/\p_1 \rightarrow k[a,\ldots, \s^d(a)][\s^{d+1}(y)]/\p_2 \]
We define $R_2:= k[a,\ldots,\s^d(a)][\s^{d+1}(y)]/\p_2 =: k[a,\ldots,\s^d(a),\s^{d+1}(a)]$, which is an integer domain since $\p_2$ was prime. Since $\p_2 \cap R = \{0\}$ we can use this notation unambiguosly:
this guarantees namely that for $a, \ldots, \s^d(a)$ we have the same residue classes modulo $\p_2$ as we had modulo $\p$.
The kernel $\p'$ of the natural epimorphism $k[y,\ldots,\s^{d+1}(y)] \rightarrow R_2$ is thus a prime ideal.
Further we have $\p \subseteq \p'$ by construction (as $\p = 0 \subset R_2$). In fact, it holds that $\p'[d] = \p$ since: 
\begin{align*}
\p'[d] = \{ f \in k[y,\ldots,\s^d(y)] \mid f(a) = 0 \} = \ker( k\{y\}[d] \rightarrow k[a,\ldots,\s^{d}(a)]) = \p
\end{align*}
where the first equality uses the fact that $\p_2 \cap R = \{0\}$, as noted by the use of the notation explained above, and last equality holds by definition of $a \in R$. This means, that $\p'$ is a prolongation of $\p$. 
We only have to show that $\p'$ is regular. FINISH-ME!
\end{bew}
\end{prop}

\begin{defn}
Let $k$ be a $\s$-field and  $\a \si k\{y\}$ be a $\s$-ideal. 
Further let $d_1, d_2 \in \Z$ with $-1 \leq d_1 < d_2$ We define $\a[d_1, d_2]:= \a[d_2] \backslash \a[d_1] = \{ r \in a[d_2] \mid r \notin a[d_1]$.
Note that this is not an ideal, as, for example, $0 \notin \a[d_1,d_2]$, but every element in $r \in a[d_1,d_2]$ has the property that $d_1 < $Ord$(r) \leq d_2$.
\end{defn}

\begin{prop}
Let $\p \subseteq k\{y\}$ be a (weak) kernel of length $d$. Then there exists a realization of $\p$. If $\p$ is a kernel, then this realization is principal.
In particular, weak kernels are exactly of the form $\p[d]$ for a prime $\s$-ideal $\p \si k\{y\}$, and similarly kernels for $\s$-prime ideals.
\begin{bew}
By Lemma \ref{genericprol} there exists a chain of prolongations $\p_{d+1} \supset \p$, $\p_{d+2} \supset \p_{d+1}, \ldots$, which are generic if $\p$ is a kernel.
Consider \[ \p':= \bigcup_{i \geq d} \p_i \]
This is a prime $\s$-ideal, since for $f \in \p'$ there exists an $i \in \N$ such that $f \in \p_i$. Then $\s(f) \in \s(p_{i+1}) \subset \p'$.
Since all $p_i$ are prime, any product $fg \in \p'$ must already be in a $\p_i$ for $i \in \N$ and thus $f \in \p_i$ or $g \in \p_i$. Further,
 if $\p$ is a kernel, then $\s^{-1}(\p_i) = \p_{i-1} \subset \p'$, so $\p'$ is reflexive. By construction is $\p'[d] = \p$ and if $\p$ is a kernel then the
 prolongations are all generic and thus $\p'$ is principal. Since for a prime $\s$-ideal $\p$ we have that $\p[d]$ is a weak kernel, and a kernel if $\p$ is also reflexive, 
the final assertion follows.
\end{bew}
\end{prop}



%% \begin{lem}
%% Let $\p \si k\{y\}$ be a prime $\s$-ideal and let $\p' := \p^* \supseteq \p$ be its reflexive closure (which is $\s$-prime). Then there exists an $n \in \N$ such that 
%% $\p[d_1,d_2] = \p'[d_1,d_2] \fa n \leq d_1 < d_2$.
%% \begin{bew}
%% We know that there exists an $n \in \N$ such that $\s^{-n}(\p) = \p^* = \p'$ is $\s$-prime. Let $d > n$. Then 
%% we have $\s^{-n}(\p[d]) = \s^{-n}(\p \cap k\{y\}[d]) = \s^{-n}(\p)\cap\s^{-n}(k\{y\}[d]) = \p'[d-n]$. 
%% Since $\p'[d-n]$ is a kernel of length $d-n$, we know that there exist unique generic prolongations of $\p'[n-d]$
%% \end{bew}
%% \end{lem}



        %% @Misc{M2,
        %%   author = {Grayson, Daniel R. and Stillman, Michael E.},
        %%   title = {Macaulay2, a software system for research 
        %%            in algebraic geometry},
        %%   howpublished = {Available at 
        %%       \href{http://www.math.uiuc.edu/Macaulay2/}%
        %%            {http://www.math.uiuc.edu/Macaulay2/}}
        %% }

\end{document}


\subsection{A Geometric Application of Difference Kernels}

There is an application of Difference Kernels that is of geometric nature. We will just quote a result for the case of reflexive prime kernels here,
and do a partial generalization for the case that arises for some non-necesarilly-reflexive prime $\s$-ideals.

\begin{theorem}\label{di=d(i+1)+e}
Let $k$ be a $\s$-field and let $\p \in k\{y\} = k\{y_1,\ldots,y_n\}$ be a prime $\s$-ideal of $k\{y\}$. For $i \in \N$ set $$d_i := \dim(k\{y\}[i]/\p[i]).$$
Then there exists integers $d, e \in \N$ such that $d_i = d(i+1) + e$ for $i \gg 0$. Moreover, $d = \s\operatorname{-trdeg}(k\{y\}/\p^*)$.
\begin{bew}
See Theorem 5.1 of \cite{wibmer}.
\end{bew}
\end{theorem}

\begin{defn}
Let $k$ be a $\s$-field and $\p \in k\{y_1,\ldots,y_n\}$ be a prime $\s$-ideal. Further let $d, e \in \N$ as in Theorem \ref{di=d(i+1)+e}. We call $d$ the $\s$-dimension of $\p$, 
or the $\s$-dimension of the irreducible $\s$-variety $X:= \VV(\p)$ and denote it by $\s$-$\dim(\p)$ and $\s$-$\dim(X)$ respectively.
\end{defn}

Recall from the introductory Chapter, Theorem \ref{irredcomp}:
\begin{restatement}{irredcomp}
Let $k$ be a $\s$-field and $f \in k\{y_1,\ldots,y_n\}, f \notin k$ an irreducible $\s$-polynomial such that $\operatorname{Eord}(f) = \operatorname{Ord}(f)$. Then $\VV_{\operatorname{Field}}(f) \subset \mathbb{A}_k^n$ has an irreducible component $X$ such that $\s$-$\dim(X) = n-1$ and $\s$-$\operatorname{deg}(X) = \operatorname{Ord}(f)$.
\end{restatement}

\begin{cor}
Let $k$ be a $\s$-field and $f \in k\{y_1,\ldots,y_n\}, f \notin k$ an irreducible $\s$-polynomial. Assume that $f$ can be written as $f = \s^r(f')$ with $f' \in k\{y_1,\ldots,y_n\}$, such that $\operatorname{Ord}(f') = \operatorname{Eord}(f')$  with $f = \s^k(f')$. 
Then $\VV(f) \subset \mathbb{A}_k^n$ has an irreducible component $X$ such that $\s$-$\dim(X) = n-1$.
\begin{bew}
By Theorem \ref{irredcomp} we know that the $\s$-variety (in the sense of $\s\text{\catname{-VarField}}_k$) $ \VV_{\operatorname{Field}}(f')$ has an irreducbile component (also in the sense of $\s\text{\catname{-VarField}}_k$) $X' = \VV(\p')$ of $\s$-dimension $n-1$.
This means that there exists a minimal (with respect to inclusion) $\s$-prime $\s$-ideal $\p' \subset k\{y_1,\ldots,y_n\}$ with $f' \in \p'$, such that $\s$-$\dim(\p') = n-1$. 
Since $f = \s^r(f')$ it holds that $f \in \p'$ as well. Let $\p \subset k\{y_1,\ldots,y_n\}$ be a minimal (again, with respect to inclusion) prime $\s$-ideal with $f \in \p, \p \subseteq \p'$ (such a $\s$-ideal exists due to Zorn's Lemma). 

We will now show that in this case it always holds that $\p^* = \p'$. Since $\p \subseteq \p'$ and $\p'$ is radical, it follows that $\p^* \subseteq \p'$. On the other hand, assume $\p^* \subsetneqq \p'$. Since $f \in \p$, it follows that $f' \in \p^*$,
and since $\p$ is a prime $\s$-ideal, it follows that $\p^*$ is $\s$-prime. But then $\p^* \subsetneqq \p'$ is a contradiction to the minimality of $\p'$. We know thus, that $\p^* = \p'$. Hence, it follows that $\s$-$\dim(\p) = n -1$ from Theorem \ref{di=d(i+1)+e}.

\end{bew}
\end{cor}

%% \begin{lem}
%% Let $k$ be a $\s$-field and $f \in k\{y_1,\ldots,y_n\}, f \notin k$ an irreducible $\s$-polynomial.  
%% Then there exists a prime difference ideal $\q \si k\{y_1,\ldots,y_n\}$, minimal containing $f$, and a $d \in \N$, such that 
%% $$\dim(k\{y_1,\ldots,y_n\}[i]/\q[i]) = \left\{ \begin{array}{lr} i, i < \operatorname{Ord}(f) \\ d(i-\operatorname{Ord}(f)) +\ operatorname{Ord}(f),
%%  i \geq \operatorname{Ord}(f) \end{array} \right.$$
%% \end{lem}


\clearpage

\section{Final Remarks}
\input{final}
\clearpage

\begin{appendices}
\lstset{ %
  backgroundcolor=\color{white},   % choose the background color; you must add \usepackage{color} or \usepackage{xcolor}
  basicstyle=\footnotesize,        % the size of the fonts that are used for the code
  breakatwhitespace=false,         % sets if automatic breaks should only happen at whitespace
  breaklines=true,                 % sets automatic line breaking
  captionpos=b,                    % sets the caption-position to bottom
  frame=single,                    % adds a frame around the code
  keepspaces=true,                 % keeps spaces in text, useful for keeping indentation of code (possibly needs columns=flexible)
  keywordstyle=\color{blue},       % keyword style
  language=Octave,                 % the language of the code
  rulecolor=\color{black},         % if not set, the frame-color may be changed on line-breaks within not-black text (e.g. comments (green here))
  showspaces=false,                % show spaces everywhere adding particular underscores; it overrides 'showstringspaces'
  showstringspaces=false,          % underline spaces within strings only
  showtabs=false,                  % show tabs within strings adding particular underscores
  stepnumber=2,                    % the step between two line-numbers. If it's 1, each line will be numbered
  stringstyle=\color{mymauve},     % string literal style
  tabsize=2,                       % sets default tabsize to 2 spaces
}




\section{Code for the Examples in Macaulay2}\label{appendixcode}

In this appendix we give the code used for calculating two examples in the thesis with its corresponding output. We do this with brief commentary inbetween to make it easier to understand.

\subsection{Example \ref{counterexker}}
We consider the difference polynomials $f_1 := \s(y_2) + 1, f_2:= \s(y_1)y_2 + y_1y_2 + \s(y_1) + y_1 + y_2 \in k[y_1,y_2,\s(y_1),\s(y_2)] =: R$.

For the calculations we considered the rational numbers $\Q$ as a constant $\s$-field. Since these calculations do not really depend on the field, this should not be of much difference. We begin by defining the ring $R$:

\begin{lstlisting}
i1 : R = QQ[s2y_1,s2y_2,sy_1,sy_2,y_1,y_2]

o1 = R

o1 : PolynomialRing

\end{lstlisting}

The definition is that of a simple polynomial ring, and we just take the names of the variables in a way we can easily interpret them as powers of $\s$ applied on a variable. We add one more power of $\s$ as in the example, so that we can calculate the relation satted later, which includes $\s^2$.
The output, which can be recognized from the lines begining with an ``o'', indicates that a polynomial ring has been succesfully constructed. We continue and define the ideal.

\begin{lstlisting}
i2 : I = ideal(sy_2+1,sy_1*y_2 + y_1*y_2 + sy_1 + y_1 + y_2)

o2 = ideal (sy  + 1, sy y  + y y  + sy  + y  + y )
              2        1 2    1 2     1    1    2

o2 : Ideal of R

\end{lstlisting}

Macaulay conveniently has a function to test if the ideal is prime: 

\begin{lstlisting}

i3 : isPrime(I)

o3 = true

\end{lstlisting}

This, however, does not necesarilly mean that the ideal is a prime kernel. To see this, we use Gr\"{o}bner bases.

\begin{lstlisting}

i4 : groebnerBasis(I)

o4 = | sy_2+1 sy_1y_2+y_1y_2+sy_1+y_1+y_2 |

             1       2
o4 : Matrix R  <--- R

i5 : y_1 < sy_1

o5 = true

\end{lstlisting}

It tells us that $f_1, f_2$ is a Gr\"{o}bner basis of $I$.
The variable ordering is per default reverse lexicographical with respect to the input, 
we tested it here to as an example to make sure it was as intended.
Using elimination theory (see for example Theorem 2 in Chapter 3 of \cite{cox}),
we see thus that $(f_1,f_2)[0] = \{0\}$.

Finally, we use macaulay to test the relation given in the example.

\begin{lstlisting}

i6 : (sy_2+1)*(s2y_1+ sy_1 + 1)- (s2y_1*sy_2 + sy_1*sy_2 + s2y_1 + sy_1 + sy_2)

o6 = 1

o6 : R


\end{lstlisting}

\subsection{Example \ref{secondexamplem2}}

This second example is a bit more complex, since we have to check the condition that
$$(ker(\s)) \cap k\{a,\s(a)\} = ker(\s).$$

We begin again by defining the ring we will work with

\begin{lstlisting}

i1 :  D = QQ[s2y_1,s2y_2,sy_1,sy_2,y_1,y_2]

o1 = D

o1 : PolynomialRing


\end{lstlisting}

This time we want to consider a quotient ring, so we begin by defining the subring $DM1 = k[y_1,y_2,\s(y_1),\s(y_2)]$

\begin{lstlisting}

i2 :  (DM1,i) = selectVariables(new List from 2..5,D)

o2 = (DM1, map(D,DM1,{sy , sy , y , y }))
                        1    2   1   2

o2 : Sequence

\end{lstlisting}

We return to use our original ring and define the ideal we want to factor out there.

\begin{lstlisting}

i3 : use D

o3 = D

o3 : PolynomialRing

i4 :  I = ideal(s2y_2+1,s2y_1*y_2 + sy_1*y_2 + s2y_1 + sy_1 + y_2)

o4 = ideal (s2y  + 1, s2y y  + sy y  + s2y  + sy  + y )
               2         1 2     1 2      1     1    2

o4 : Ideal of D

i5 :  groebnerBasis I

o5 = | s2y_2+1 s2y_1y_2+sy_1y_2+s2y_1+sy_1+y_2 |

             1       2
o5 : Matrix D  <--- D


\end{lstlisting}

From the Gr\"{o}bner basis we see again that $I$ is indeed a prime kernel.
Now we can define the quotient ring $D/I$.

\begin{lstlisting}

i6 : F2 = D/I

o6 = F2

o6 : QuotientRing

\end{lstlisting}

Now we proceed to define the mapping $\s$. Since $I \cap k\{y\}[1] = \{ 0 \}$ 
we know that $k[a,\s(a)] \cong k[y,\s(y)]$ and can use the latter for defining $\s$.

\begin{lstlisting}


i7 : F1 = DM1

o7 = DM1

o7 : PolynomialRing

i8 : use F1

o8 = DM1

o8 : PolynomialRing

i9 :  sigma = map(F2,F1,{y_1 => F2_(symbol sy_1),y_2 => F2_(symbol sy_2), sy_1 => F2_(symbol s2y_1), sy_2 => F2_(symbol s2y_2)})

o9 = map(F2,DM1,{s2y , -1, sy , sy })
                    1        1    2

o9 : RingMap F2 <--- DM1

\end{lstlisting}

Macaulay2 tells us with this output that it successfully defined the mapping. We can look at its kernel now.

\begin{lstlisting}

i10 :  groebnerBasis kernel(sigma)

o10 = | sy_2+1 |

                1         1
o10 : Matrix DM1  <--- DM1


\end{lstlisting}


We see that $\ker(\s) = (\s(y_2) + 1) \unlhd k[y_1,y_2, \s(y_1), \s(y_2)]$. 
From this we should be able to deduce already that the condition holds, but we can ask Macaulay to explicitly
calculate it too

\begin{lstlisting}

i11 : use F2

o11 = F2

o11 : QuotientRing

i12 : J = ideal(sy_2 + 1)

o12 = ideal(sy  + 1)
              2

o12 : Ideal of F2

i13 : groebnerBasis(J)

o13 = | sy_2+1 |

               1        1
o13 : Matrix F2  <--- F2

\end{lstlisting}
 
Which, using elimination theory again, gives the desired result, namely

$$ (\ker(\s)) \cap k[y_1,y_2,\s(y_1),\s(y_2)] = (\s(y_2) +1) = \ker(\s) $$

\end{appendices}

\clearpage 
\begin{thebibliography}{9}
\bibitem{wibmer} Wibmer, Michael \emph{Algebraic Difference Equations (Lecture Notes)}, Available online: \url{http://www.algebra.rwth-aachen.de/de/Mitarbeiter/Wibmer/Algebraic\%20difference\%20equations.pdf}
\bibitem{lang} Serge Lang, \emph{Algebra}, Revised Third Edition, Springer, 2005
\bibitem{eisenbud} Eisenbud, David \emph{Commutative Algebra with a View Toward Algebraic Geometry}, Springer, 1995
\bibitem{hartshorne} Hartshorne, Robin \emph{Algebraic Geometry}, Springer, 1977
\bibitem{cohn} Cohn,  Richard \emph{Difference Algebra}, Interscience Publishers, 1965
\bibitem{levin} Levin, Alexander \emph{Difference Algebra}, Springer, 2008
\bibitem{hrushovski} Hrushovski, Ehud \emph{The Elementary Theory of the Frobenius Automorphism}, arXiv:math/0406514 
\bibitem{bourbaki} Bourbaki, Nicolas \emph{Commutative Algebra}, Hermann, 1972
\bibitem{M2} Grayson, Daniel R. and Stillman, Michael E., Macaulay2, a software system for research in algebraic geometry, Available at \href{http://www.math.uiuc.edu/Macaulay2/}{http://www.math.uiuc.edu/Macaulay2/}
\bibitem{levinmixed} Levin, Alexander \emph{On the ascending chain condition for mixed difference ideals}, 	arXiv:1207.4721
\end{thebibliography}


\clearpage
\section*{Conventions on notation for this thesis} 
We will use the following conventions throughout the thesis:
\begin{itemize}
\item $\N := \{0,1,\ldots \}$
\item $\NE := \{1,2,\ldots \}$
\item $ \n := \{1,2,\ldots, n\}$
\item $ \n_0 := n \cup \{0\}$
\item $A \setminus B = \{ a \in A \mid a \notin B \}$
\item Rings will be assumed to be commutative and unital.
\item Let $R$ be a ring, $a_1,\ldots,a_n \in R$. Then $(a_1,\ldots,a_n)$ denotes the smallest ideal of $R$ containing $a_1,\ldots,a_n$.
\item Let $R$ be a difference ring, $a_1,\ldots,a_n \in R$. Then $[a_1,\ldots,a_n]$ denotes the smallest difference ideal of $R$ containing $a_1,\ldots,a_n$.
\item Let $R$ be a difference ring. Then $A \si R$ means that A is a difference ideal of $R$.
\item The variable $y$ in (difference) polynomial rings will, in general, mean $n$ variables $y_1, \ldots, y_n$, i.e., we will write for example $k[y]$ for $k[y_1,\ldots,y_n]$.
\end{itemize}
%% \clearpage
%% \tableofcontents
\clearpage


\clearpage
\printindex
\end{document}

In this section we will introduce a new concept of difference varieties. We will do so in a way that they correspond with the topology on $\sSpec(R)$ which we defined in the previous section. This section will again be based on M. Wibmer's lecture notes \cite{wibmer}, 
where this is worked out for the analogous case of perfect $\s$-ideals.

\subsection{Mixed Difference Varieties}


\begin{defn}
Let $A$ be a $\s$-ring. If $A$ is an integral domain, we call $A$ an \emph{integral $\s$-ring}. If, additionally, the endomorphism $\s$ on $A$ is injective, then we call $A$ a \emph{$\s$-domain}. \index{integral $\s$-ring} \index{$\s$-domain}
\end{defn}

\begin{rem}\label{sdomain=field}
Let $A$ be a $\s$-domain. Then its field of fractions $\operatorname{Quot}(A) =: k$ is a $\s$-field: For $\frac{r}{s} \in k$ we define $\s(\frac{r}{s}):= \frac{\s(r)}{\s(s)}$. Since $\s$ is injective, we have $\s(s) \neq 0$ for $s \neq 0$, which implies that $\s$ is well defined on $k$.
By this argument, we see that - in general - for an integral $\s$-ring $A$, $\operatorname{Quot}(A)$ is a $\s$-field (in this natural way), if and only if $A$ is a $\s$-domain.
\end{rem}

Our main purpose, in a first instance at least, is to investigate the properties of solutions to difference equations. 
We will start with a $\s$-field $k$ and look for solutions (zeros) of some $\s$-polynomial $p$ over $k$, i.e. $p \in k\{y_1, \ldots, y_n \}$. In general it will be a set of $\s$-polynomials $F \subseteq k\{y_1, \ldots, y_n \}$ that we will study. 
For this, we want to define a new concept of $\s$-varieties. However, we cannot mimic the usual approach from algebraic geometry, where one would work over the algebraic closure of $k$. The next remark shows why.

\begin{rem}\label{incompatibleextensions}
 Consider the constant $\s$-field $\Q$ and $K = \Q(\sqrt{2})$, with \\ $\s (\sqrt{2}) = \sqrt{2}$; $L = \Q(\sqrt{2}), \s(\sqrt{2}) = - \sqrt{2}$. 
Both $K$ and $L$ are $\s$-field extensions of $\Q$, but there cannot be a further extension $\Q \leq M$ of $\s$-fields, such that $K,L \leq M$ are both (isomorphic to) $\s$-subfields of $M$. 
To see this, assume there was such an $M$. Then the set $\{ a \in M \mid a^2 - 2 = 0 \}$ has exactly two elements, which we will call $\sqrt{2}, -\sqrt{2}$ (since $\sqrt{2} + (- \sqrt{2}) = 0$).
But $\sqrt{2} \in K$ has to be mapped to one of these two in any embedding. The same holds true for $\sqrt{2} \in L$, which already yields the contradiction,
 since in $M$ either $\s(\sqrt{2}) = \sqrt{2}$ or $\s(\sqrt{2}) = -\sqrt{2}$.
\end{rem}

To avoid this problem, we will define our new, mixed $\s$-varieties as functors. For this, we will need a few category-theoretic definitions:

\begin{defn}
Let $k$ be a $\s$-field. The category of all $\s$-ring extensions $A$ of $k$ (i.e. such that $k \subseteq A$ is a sub $\s$-ring of $A$) is denoted by $\sringk$. The morphisms are defined as follows: For $B,C \in \sringk$ we say that a morphism of $\s$-rings $\varphi: B \rightarrow C$ is a morphism of $\s$-ring extensions of $k$, if and only if $\varphi_{|k} = \id_k$.
The subcategory which arises from restricting the object class to integral $\s$-rings, the category of \emph{integral $\s$-ring extensions of $k$}, we denote by $\sintk$. \index{$\sintk$} \index{$\sringk$} \index{integral $\s$-ring extensions}
\end{defn}

Because $k$ is a field, any non-trivial morphism of $\s$ rings of $k$ to a $k$-$\s$-algebra is injective. This means that the category $\sringk$ is equivalent to that of $k$-$\s$-algebras, the only difference is if $k$ is a subset or just isomorphic to one. Now we are ready to define the new concept of $\s$-varieties of $\s$-rings, with mixed $\s$-ideals in mind:

\begin{defn}\label{defnVV}
Let $k$ be a $\s$-field and $B \in \sintk$ an integral $\s$-overring of $k$. Further let $F \subseteq k\{y_1, \ldots, y_n\}$ be a set of $\s$-polynomials over $k$. 
Then we define $\VV_B(F):= \{ b \in B^n \mid f(b) = 0 \fa f \in F \}$. A functor \\ $X: \sintk \rightarrow \Set$, for which there exists a set $F \subseteq k\{y_1, \ldots, y_n \}$, such that $X(B) = \VV_B(F)$ for all $B \in \sintk$, we denote as a \emph{$\s$-variety over $k$}, or a \emph{$k$-$\s$-m-variety}.
Here, $\Set$ denotes the usual category of sets with mappings as morphisms. We also write $X := \VV(F)$ as a short notation for this functor. \index{$\s$-variety} \index{$\s$-variety over $k$} \index{$k$-$\s$-m-variety}
\end{defn}

We can compare this definition to the definition of difference varieties in Section 1. The differencie lies in the category considered for the codomain of the functor, where instead of $\s$-fields over $k$ we take $\sintk$.
This begs the question what would happen if we considered yet another category instead, for example only allowing the ring to be perfectly $\s$-reduced, or a $\s$-domain. It can be shown that the category of difference varieties
yielded by those definitions is equivalent to the standard definition of difference varities. We will prove this at the end of this Section when we have the necessary tools to do it.

\begin{defn}
Let $k$ be a $\s$-field and $X: \sintk \rightarrow \Set$ be a $k$-$\s$-m-variety. We say a subfunctor $Y \subseteq X$ is a \emph{$\s$-m-subvariety} of $X$, if $Y$ is a $k$-$\s$-m-variety itself. \index{$\s$-m-subvariety}
\end{defn}

 \begin{rem}
Let $k$ be a $\s$-field and $X$ be a $k$-$\s$-m-variety. Not every subfunctor of $X$ is a $\s$-m-subvariety. Consider the functor $X = \VV(0)$, for $\{0\} \subset k\{y_1\}$.
For $B \in \sintk$ we denote by $B^* = \{ b \in B \mid b \text{ invertible } \}$ the set of units of $B$. Then for $B \in \sintk$, $B \mapsto B^*$ is a subfunctor $Y$ of $X$ (since $B^* \subset B \fa B \in \sintk$ and morphisms of rings always map units to units). However, $Y$ is not a $\s$-m-variety:
there exists no $F \subseteq k\{y_1\}$, such that $\VV_B(F) = B^* \fa B \in \sintk$.  Indeed, assume there was such an $F$, and let $0 \neq f \in F$. Then $f(b) = 0 \fa b \in B^*$ and $\fa B \in \sintk$. In particular,
for $B = k\langle y_1 \rangle = \operatorname{Quot}(k\{y_1\}) \in \sintk$ we get the that \\$f(y_1) = f = 0, ~ y_1 \in k \langle y_1 \rangle^*$, a contradiction.  
\end{rem}

\begin{defn}\label{defnI}
Let $X = \VV(F)$ be a $\s$-m-variety over the $\s$-field $k$, \\$F \subseteq k\{y_1,\ldots,y_n\}$. Then we set $$\I_m(X):= \{ f \in k\{y_1,\ldots,y_n\} \mid f(b) = 0 \fa b \in \VV_B(F), ~ B \in \sintk \}.$$ \index{ $\I_m(X)$}
\end{defn}

\begin{ex}\label{A^n}
Let $k$ be a $\s$-field and let $\{ 0 \} =: F \subseteq k\{y_1,\ldots,y_n\}$. Then a $\s$-m-variety $X$ is defined by $F$, $X(B) := \VV_B(F) \fa B \in \sintk$ with the property that $X(B) = B^n \fa B \in \sintk$.
For every $G \subseteq A\{y_1,\ldots,y_n\}$ the $\s$-m-variety given by $Y: B \mapsto \VV_B(G)$ is a $\s$-m-subvariety of $X$. 
\end{ex}

\begin{defn}\label{defA^n}
The $\s$-m-variety $X$ defined in Example \ref{A^n} is called the \emph{affine $n$-space}, and is denoted by $\mathbb{A}^n_k$, 
or simply $\mathbb{A}^n$, whenever $k$ is clear from the context.  \index{affine $n$-space} \index{$\mathbb{A}^n_k$}
Since for every $G \subseteq A\{y_1,\ldots,y_n\}$ the $\s$-m-variety given by $Y: B \mapsto \VV_B(G)$ is a $\s$-m-subvariety of $X$, 
we write $Y \subseteq \mathbb{A}^n$ as a shorthand for the former.
\end{defn}

We note that $0$ is in any (radical, mixed, difference) ideal, so it is not surprising that every $\s$-m-variety is a $\s$-m-subvariety of $\VV(0)$. This ``intuition'' will be made more concrete later on.

Since we have this functorial definition, we have, in principle, a whole proper class of solutions for most systems of difference equations. 
We want to have some sort of equivalence relation between solutions to group them up in a reasonable manner.

\begin{defn}\label{equivsols}
Let $k$ be a $\s$-field, $B,C \in \sintk$. Further let \\ $F \subseteq k\{y_1,\ldots,y_n\}$ be a system of difference equations and $b \in B^n, c \in C^n$ be solutions of $F$, i.e. $b \in \VV_B(F), c \in \VV_C(F)$.
We say that $b$ and $c$ are \emph{equivalent}\index{equivalent solutions} if the mapping $b \mapsto c$ is a well-defined isomorphism between the integral $\s$-rings $k\{b\}$ and $k\{c\}$  (as elements of $\sintk$). \index{equivalent solutions}
\end{defn}

%% \begin{lem}
%% Let $k$ be a $\s$-field, $B,B' \in \sintk$, and let $b \in B^n; b' \in B'^n$ be equivalent solutions for a system of difference equations $F \subseteq A\{y_1,\ldots,y_n\}$
%% \end{lem}

\begin{rem}
Recall the definition of equivalence of solutions for $\s$-varieties in Section 1, where two solutions $a,b$ of a system of difference equation over a difference field $k$ are said to be equivalent if the $\s$-field extensions $k\langle a \rangle$ and $k\langle b \rangle$ are isomorphic as $\s$-field extensions of $k$ via $a \mapsto b$.
This is in accordance with Definition \ref{equivsols}, i.e. solutions in difference field extensions are equivalent if and only if they are equivalent as solutions in integral $\s$-overrings in the sense of Definition~\ref{equivsols}.
\begin{bew}
Assume that there exist $\s$-field extensions $k \leq A,B$, and elements $a \in A^n$, $b \in B^n$ such that $a$ and $b$ are equivalent as solutions in the sense of Definition~\ref{equivsols}. Since $A,B$ are $\s$-fields, it means that $k\{a\}$ and $k\{b\}$ are $\s$-domains, 
and $k\langle a \rangle, k\langle b \rangle$ have the ``canonical'' difference structure induced by $k\{a\}, k\{b\}$ (see Remark \ref{sdomain=field}). Let $\varphi: k\{a\} \rightarrow k\{b\}, a \mapsto b$ be an isomorphism of integral $\s$-ring extensions of $k$.
Then we can define $$\tilde \varphi: k \langle a \rangle \rightarrow k\langle b \rangle, \frac{x}{y} \mapsto \frac{\varphi(x)}{\varphi{(y)}}.$$ This is a well-defined isomorphism of $\s$-field extensions of $k$, since:
\begin{align*}
\tilde \varphi \left(\s \left(\frac{x}{y}\right)\right) = \tilde \varphi \left( \frac{\s \left(x\right)}{\s \left(y\right)}\right) = \frac{ \varphi \left(\s \left(x\right)\right)}{ \varphi \left(\s \left(y\right)\right)} =  \frac{\s \left(\varphi \left(x\right)\right)}{\s \left(\varphi \left(y\right)\right)} = \s \left( \tilde \varphi \left(\frac{x}{y}\right)\right).
\end{align*}
The converse implication is obvious.
\end{bew}
\end{rem}

\begin{ex}
In the two $\s$-field extensions of $\Q$ in Remark \ref{incompatibleextensions} we have two solutions of the (algebraic) polynomial $y^2-2$, which represent two non-equivalent solutions in the difference algebraic sense,
since the $\s$-fields $\Q(\sqrt{2}),$ \\ $\s(\sqrt{2}) = \sqrt{2}$ and $\Q(\sqrt{2}), \s(\sqrt{2}) = -\sqrt{2}$ are not isomorphic. 
\end{ex}

\begin{ex}
Let $k$ be a $\s$-field. The $\s$-m-variety $X$ given by $\s(y)~\in~k\{y\}$, i.e. $X(B) = \VV_B(\s(y)) = \{ b \in B^n \mid \s(b) = 0 \} \fa B \in \sintk$, has a single point in any $\s$-field extension of $k$, namely $0$. However, in general integral $\s$-rings,
this is not necessarily the case: Take, for example, $B:= k\{y\}/[\s(y)] \in \sintk$. In $B$ we have $0 \neq $ ker$(\s) = [y+ [\s(y)]] \si B$, which means that in particular, $[y + [\s(y)]] \subseteq \VV_B(\s(y))$.
\end{ex}

It is not a coincidence that in the previous example we found more solutions on the $\s$-ring $B = k\{y\}/[\s(y)]$. The $\s$-ideal $[\s(y)]$ is radical and mixed, i.e. $[\s(y)] = \{ [\s(y)] \}_m$.
In fact, the ring $B$, as we chose it, plays an analogous role to that of the coordinate ring of an affine variety in the usual (algebraic) case.

The next proposition shows why our definition of $\s$-m-variety is ``the right one'' for mixed ideals:

\begin{prop}\label{I=F_m}
Let $k$ be a $\s$-field and $X = \VV(F) \subseteq \mathbb{A}^n$ be a mixed difference variety over $k$. Then $\I_m(X) = \{F\}_m \si k\{y_1,\ldots,y_n\}$. 
\begin{bew}
We will first show that $\I_m(X)$ is a radical, mixed $\s$-ideal.
Let \\ $f,~g~\in~\I_m(X)$, $h \in k\{y_1,\ldots,y_n\}$. Then, for every $B \in \sintk$, $b \in \VV_B(F)$, we have $f(b) = g(b) = 0$.
It follows that $(f + g)(b) = f(b) + g(b) = 0$ as well as $(fh)(b) = f(b)h(b) = 0 \cdot h(b) = 0$ and $\s(f)(b) = \s(f(b)) = \s(0) = 0$, so that $\I_m(X)$ is a $\s$-ideal.
It further follows that $h(b)^n = 0$ implies $h(b) = 0$, since $B$ is an integral domain, and this means that $h^n \in \I_m(X)$ implies that $ h \in \I_m(X)$. \\
\indent It only remains to show that $\I_m(X)$ is mixed. Let now $f,g \in k\{y_1,\ldots,y_n\}$ be such that $fg \in \I_m(X)$. This means that for all  $B \in \sintk$, $b \in \VV_B(F)$ we have
 $(fg)(b) = f(b) g(b) = 0$. Since $B$ is an integral domain,
this implies that $f(b) = 0$ or $g(b) = 0$. This also implies that $\s(f(b)) = \s(0) = 0$, or $\s(g(b)) = 0$, so that in any case $(f\s(g))(b) = 0$, from which it follows that $f\s(g) \in \I_m(X)$. We thus see that $\I_m(X)$ is radical and mixed, hence $\{F\}_m \subseteq k\{y_1,\ldots,y_n\}$. \\
\indent For the other inclusion, let $f \in \I_m(X)$. We will show that $f \in \{F\}_m$. Let $F \subseteq \p \si k\{y_1,\ldots,y_n\}$ be a prime $\s$-ideal.
Then, consider $B:= k\{y_1,\ldots,y_n\}/\p$. This is an integral $\s$-ring. Since $F \subseteq \p$, we know that $y + \p \in \VV_B(F)$. By assumption, we have $f \in \I_m(\VV(F))$, which means by definition that $f(y + \p) = 0$, which
in turn means that $f \in \p$. Since this holds for any prime $\p \si R$, Theorem \ref{intersectionprimes} implies that $f \in \{F\}_m$.
\end{bew}
\end{prop}
Note that, in the notation of the proof above,  it does not always have to be the same case, $f(b) = 0$ or $g(b) = 0$, as it depends on $B$ (and $b$, of course). In particular, $\I_m(X)$ does not have to be prime in general. \\

From this, we immediately get a further result on radical, mixed ideals, which is analogous to the case of radical ideals in algebraic geometry.
\begin{cor}\label{prod=cap}
Let $\a, \b \si k\{y\}$ be two radical, mixed difference ideals. Then $\a \cap \b = \{ \a \b \}_m$.
\begin{bew}
We can assume that $\a, \b \neq \{0\}$, as the assertion is obvious otherwise. Since $\a, \b$ are radical and mixed, we know from Proposition \ref{I=F_m} that $\a = \I_m(\VV(\a)), \b = \I_m(\VV(\b))$, and $\{ \a \b \}_m = \I_m( \VV( \a \b ))$.
For any $B \in \sintk$, we have:
\begin{align*} \VV_B( \a \b) = \VV_B( \a) \cup \VV_B( \b). \end{align*}
The inclusion ``$\supseteq$'' is obvious. For ``$\subseteq$'', let $p \in \VV_B(\a\b)$ and assume there exists an $f \in \a$, such that $f(p) \neq 0$.
Then, from the definition of $\VV_B(\a\b)$ it follows that $f(p)g(p) = 0 \fa g \in \b$. This means, however, that $p \in \VV_B(\b)$ (since $B$ is an integral domain). The other case is completely analogous.
Since this is true for any $B$, the $\s$-m-varieties are also equal: $\VV( \a \b) = \\ \VV( \a) \cup \VV( \b)$. Now,
\begin{flalign*} & \phantom{ = \{ f \in k\{y\} \mid f(p) =}\I_m(\VV(\a \b)) = \I_m(\VV(\a) \cup \VV(\b)) \\ & = \{ f \in k\{y\} \mid f(p) = 0 \fa p \in \VV_B(\a) \cup \VV_B(\b), ~ B \in \sintk \}, & \end{flalign*} 
and $f(p) = 0 \fa p \in \VV_B(\a) \cup \VV_B(\b), ~ B \in \sintk$, is equivalent to \\
\begin{align*}  \underbrace{f(p) = 0 \fa p \in \VV_B(\a), ~ B \in \sintk}_{\Leftrightarrow f \in \I_m(\VV(\a))} \text{ and }\\  \underbrace{f(p) = 0 \fa p \in \VV_B(\b), ~ B \in \sintk}_{\Leftrightarrow f \in \I_m(\VV(\b))}. \end{align*} \\
Hence, $\{\a\b\}_m = \I_m(\VV( \a \b)) = \I_m(\VV(\a)) \cap \I_m(\VV(\b)) = \{\a\}_m \cap \{\b\}_m = \a \cap \b$.
\end{bew}
\end{cor}

\begin{defn}
Let $k$ be a $\s$-field and let $X$ be a $\s$-m-variety over $k$. Further let $F \subseteq k\{y_1, \ldots, y_n\}$ be a system of difference equations over $k$ with $X(B) = \VV_B(F) \fa B \in \sintk$.
Then we consider the $\s$-ring $$k\{y_1, \ldots, y_n\}/\{F\}_m = k\{y_1, \ldots, y_n\}/\I_m(X) =: k\{X\}$$ and call it the \emph{coordinate ring} of $X$. Since $\{F\}_m$ is a radical, mixed $\s$-ideal, $k\{X\}$ is reduced and well-mixed. \index{coordinate ring}
\end{defn}

\begin{rem}
Let $k$ be a $\s$-field and $X$ a $k$-$\s$-m-variety. Further let $b~\in~X(B), ~ B \in \sintk, f + \I_m(X) \in k\{X\}$. Then the value of $f(b) \in k$ is independent of the representative $f$,
since for $f' + \I_m(X) = f + \I_m(X)$, we know that $f - f' \in \I_m(X)$, and thus by definition, $(f - f')(b) = 0$. By abuse of notation,
we will sometimes use the representative $f$ to refer to its equivalence class $f + \I_m(X)$ and we will simply write $f(b)$ to mean the well-defined value of evaluating $b$ on any representative of the class.
\end{rem}

We can now clarify what we meant after Definition \ref{defA^n}.

 \begin{lem}\label{bijsubvarsideals}
Let $k$ be a $\s$-field. Then the maps $X \mapsto \I_m(X)$ and $\a \mapsto \VV(\a)$ define inclusion-reversing bijections between the set of all $\s$-m-subvarieties of $\mathbb{A}^n$ and the radical, mixed $\s$-ideals of $k\{y_1,\ldots,y_n\}$.
\begin{bew}
From Proposition \ref{I=F_m} we know that $\I_m(\VV(\a)) = \a$ is true for all $\a~\si~k\{y_1,\ldots,y_n\}$ radical, mixed. Conversely, let $X~=~\VV(F)~\subseteq~\mathbb{A}^n$ be a $\s$-m-variety. Then we know  $\VV(\I_m(X)) = \VV(\I_m(\VV(F))) \subseteq \VV(F) = X$,
 since $F \subseteq \I_m(X)$. On the other hand, it is clear from the definitions of $\VV$ and $ \I_m$ that $X \subseteq \VV(\I_m(X))$, so that $X = \VV(\I_m(X))$. This proves the bijectivity of both mappings. That both mappings are inclusion-reversing follows directly from the definitions.
\end{bew}
\end{lem}

Note that since every $\s$-m-variety is a $\s$-m-subvariety of $\mathbb{A}^n$ for an $n~\in~\NE$, it is no restriction to consider $\mathbb{A}^n$ instead of an arbitrary $\s$-m-variety, as we can see in the following corollary:
\begin{cor}
  Let $X$ be a $\s$-m-variety over the $\s$-field $k$. Then there is a bijection between the radical, mixed $\s$-ideals of $k\{X\}$ and the $\s$-m-subvarieties of $X$ via
 \begin{align*} X \supseteq Y \mapsto \{f \in k\{X\} \mid f(b) = 0 \fa b \in Y(B), \fa B \in \sintk \} \\ =: \I_{k\{X\}}(Y). \end{align*}
\begin{bew}
If we identify the radical, mixed $\s$-ideals of $k\{X\}$ with the radical, mixed $\s$-ideals of $k\{y\}$ which contain $\I_m(X)$ (see Proposition \ref{bijideals}), then this is just the restriction of the mapping described in Lemma \ref{bijsubvarsideals}.
\end{bew}
\end{cor}

A further very interesting bijection can also help us to better understand equivalence classes of solutions: 
\begin{prop}\label{bijsols}
Let $X = \VV(F)$ be a $\s$-m-variety over the $\s$-field $k$. The equivalence classes of solutions of $F$ are in bijection with the $\s$-spectrum of the coordinate ring $\sSpec(k\{X\})$.
\begin{bew}
Let $B \in \sintk$, $b \in B^n$ be a solution of $F \subseteq k\{y_1,\ldots,y_n\}$, i.e. $f(b) = 0 \fa f \in F$. Consider the mapping $$\varphi: k\{y_1,\ldots,y_n\} \rightarrow B, y \mapsto b.$$
Then $F \subseteq $ ker$( \varphi) \si R$.
Since (forgetting the difference structure for a moment), $B$ is an integral domain, the ideal ker$(\varphi)$ has to be prime. It follows from this that $\{F\}_m = \I_m(X) \subseteq~$ker$(\varphi)$. 
In particular, this implies that the mapping $\varphi$ factors over $\I_m(X)$, and it induces a morphism of $\s$-rings $\tilde \varphi: k\{X\} \rightarrow B$. By the same argument as above, the kernel of this induced
morphism, $\p_b := $ker$(\tilde \varphi) \si k\{X\}$ is a prime $\s$-ideal of $k\{X\}$. The kernel of the mapping constructed this way is always the same for equivalent solutions. To see this, let $b' \in B'^n$, such that $k\{b\} \cong k\{b'\}$ via $\iota: b \mapsto b'$.
Then, for the mapping $\varphi': k\{y_1, \ldots, y_n\} \rightarrow B', y \mapsto b$, we get that $\varphi' = \iota \circ \varphi$ (which is well-defined, since Im$(\varphi)\subseteq k\{b\}$). In particular, since $\iota$ is an isomorphism, ker$(\varphi) = $ker$(\varphi')$. 
We define the mapping $\Psi$ from the equivalence classes of solutions of $F$ to $\sSpec(k\{X\})$, via $b \mapsto \p_b$.\\ 

\indent On the other hand, for $\p \in \sSpec(k\{X\})$, which we identify with \\$\I_m(X) \subseteq \tilde \p \in \sSpec(k\{y_1,\ldots,y_n\})$ (see Proposition \ref{bijideals}), consider the integral $\s$-ring $B(\p):= k\{y_1,\ldots,y_n\}/ \tilde \p$.
Since $\tilde \p$ is a prime $\s$-ideal, $B(\p)$ is an integral $\s$-ring. Set $b(\p) := \bar y \in B(\p)$, as the image of $y$ in $B(\p)$. Then, because $F \subseteq \I_m(X) \subseteq \tilde \p$, we know that $b(\p)$ is a solution of $F$. 
We define $\Psi^{-1}(\p)$ as the equivalence class of $b(\p)$. Then $\Psi$ and $\Psi^{-1}$ are inverses of each other, and hence, are both bijections.
\end{bew}
\end{prop}

From Proposition \ref{bijsols} we see that it is a good idea to concentrate on $\sSpec(k\{X\})$ for a $\s$-m-variety $X$ over a $\s$-field $k$.
 From here on, we will speak of the ``topology on/of X'' to refer to the topology on $\sSpec(k\{X\})$, as in Definition \ref{deftop}. 
We will also use the notation $x \in X$ to mean \\ $x~\in~\sSpec(k\{X\})$, or $T \subseteq X$ closed to speak of a closed subset of $\sSpec(k\{X\})$, and so forth.

\subsection{Morphisms of Mixed Difference Varieties}

So far we have only studied mixed difference varieties themselves, but not a way to relate them with each other; we have yet to properly define the category of mixed difference varieties over a fixed $\s$-field $k$. 
We still have to define what the morphisms in this category shall be.

\begin{defn}\label{spolynomialmaps}
Let $k$ be a $\s$-field, $X \subseteq \mathbb{A}^n,Y \subseteq \mathbb{A}^m$ $\s$-m-varieties over $k$. Then, a morphism of functors $f: X \rightarrow Y$ is called a \emph{morphism of $\s$-m-varieties over $k$} or a \emph{$\s$-polynomial map},
if there exist $\s$-polynomials $f_1,\ldots,f_m \in k\{y_1,\ldots,y_n\}$, such that $f(b) = (f_1(b),\ldots,f_m(b))$ for all \\ $b~\in~X(B), B~\in~\sintk$. \index{morphism of $\s$-m-varieties} \index{$\s$-polynomial map}
\index{morphism of $\s$-m-varieties} \index{$\s$-polynomial map}
\end{defn}

\begin{ex}
For two $\s$-m-varieties $X \subseteq Y = \mathbb{A}^n_k$, over the $\s$-field $k$, the inclusion mapping $\iota: X \hookrightarrow Y$ is a morphism of $\s$-m-varieties over $k$, since we can choose $f_1 = y_1, f_2 = y_2, \ldots, f_n = y_n$.
Similarly, for $m \geq n$ and $X \subseteq \mathbb{A}^m_k, Y \subseteq \mathbb{A}^n_k$ the ``projection onto $\mathbb{A}^n$'' is also a morphism of $\s$-m-varieties over $k$ (with the same choice of $f_i$ as the example above).
\end{ex}

\begin{rem}\label{dualmor}
Let $f: X \rightarrow Y$ be a morphism of $\s$-m-varieties over the \\ $\s$-field $k$, $X \subseteq \mathbb{A}^n, Y \subseteq \mathbb{A}^m$. Then, by definition, there exist difference polynomials $f_1, \ldots, f_m \in k\{y_1,\ldots,y_n\}$, such 
that $f(b) = (f_1(b),\ldots,f_m(b))$ for all $b \in B, ~ B \in \sintk$. Modulo $\I_m(X)$, these $f_i$ are unique:
 If there is $f_1', \ldots, f_m' \in k\{y_1,\ldots,y_n\}$ such that $f_i(b) = f'_i(b) \fa b \in B$, $B~\in~\sintk,$ and for all $i \in \underline{m}$,
then it follows that $(f_i - f_i')(b) = 0 \fa b \in B,$ $B~\in~\sintk$, which implies that $f_i - f_i' \in \I_m(X)$ by definition, for all $i \in \underline{m}$. \\
\indent Now, consider the mapping \[ \phi: k\{z_1,\ldots,z_m \} \rightarrow k\{X\}, ~ z_i \mapsto f_i + \I_m(X) =: \overline{f_i}. \]
This mapping factors over $\I_m(Y)$, since for $h \in \I_m(Y) \subseteq k\{z_1,\ldots,z_m\}$, $b \in X(B), ~ B \in \sintk$, we have that 
\[ (\phi(h))(b) = h(\overline f_1(b), \ldots, \overline f_m(b)) = h(f(b)). \]
Since $\phi$ is a morphism of $\s$-m-varieties over $k$, it follows that $f(b) \in Y(B)$, which implies that $h(f(b)) = 0$, by choice of $h$, hence $h \in $ ker$(\phi)$.
Altogether, this yields a mapping 
\[ f^* : k\{Y\} \rightarrow k\{X\}, ~ z_i + \I_m(Y) \mapsto y_i + \I_m(X). \]
This mapping is a morphism of integral $\s$-rings over $k$, and is called the \emph{dual mapping} or \emph{dual morphism} to $f$\index{dual morphism}. We have
\[ f^*(h)(b) = h(f(b)) \fa h \in k\{Y\}, b \in X(B), ~ B \in \sintk. \]
From the definition it follows that for morphisms $X \xrightarrow{f} Y \xrightarrow{g} Z$ of $\s$-m-varieties over $k$, we get $ (f \circ g)^* = g^* \circ f^*$. 
We thus get a contravariant functor $-^*$ from the category of mixed difference varieties over $k$ to $\sringk$.
\end{rem}

\begin{prop}\label{dualisequiv}
Let $k$ be a $\s$-field. Then $-^*$, as defined in Remark~\ref{dualmor}, is an anti-equivalence between the category of $\s$-m-varieties over $k$ and the subcategory of $\sringk$, which arises by restricting the object class to reduced, well-mixed, finitely $\s$-generated $\s$-overrings of $k$. 
In particular, a morphism $f: X \rightarrow Y$ of $\s$-m-varieties over $k$ is an isomorphism if and only if $f^*:~k\{Y\}~\rightarrow~k\{X\}$ is an isomorphism.
\begin{bew}
Since for a $\s$-m-variety $X$ over $k$, $\I_m(X)$ is radical and mixed, $k\{X\}$ is always a reduced and well-mixed $\s$-overring of $k$, 
and finitely $\s$-generated since $\s$-m-varieties are defined only for equations with finitely many difference variables. From this it follows that the functor $-^*$ from Remark \ref{dualmor} is well defined. 

It suffices to show that it is surjective on the skeleton of the categories and bijective on morphisms. 
Let $B$ be a finitely $\s$-generated, well-mixed and reduced $\s$-overring of $k$. We can then write $B \cong k\{y_1,\ldots,y_n\}/\a$, for an $\a \si k\{y_1,\ldots,y_n\}$ radical and mixed. The $\s$-m-variety $X = \VV(\a) \subseteq \mathbb{A}^n$
is then a preimage of the isomorphism class of $B$, since $\I_m(X) = \I_m(\VV(\a)) = \a$, because of Proposition \ref{I=F_m}. Thus, $B \cong k\{X\}$. \\
\indent Now, for the morphisms: First, let $X,Y$ be $\s$-m-varieties over $k$ and $f,g \in \Hom(X,Y)$ with $f^* = g^*$. Then we know that for every $h \in k\{X\}$, and every $b \in B, ~ B \in \sintk$:
\[ h(f(b)) = (f^*(h))(b) = (g^*(h))(b) = h(g(b)). \]
In particular, $f(b) = g(b) \fa b \in B, ~ B \in \sintk$, which implies that $f = g$, and $-^*$ is injective. 
On the other hand, let $\varphi: k\{Y\} \rightarrow k\{X\}$ be a morphism of $\s$-overrings of $k$. There exist $n,m \in \NE$ such that $X \subseteq \mathbb{A}^n, Y \subseteq \mathbb{A}^m$,
 which means that $k\{X\} = k\{z_1,\ldots,z_n\}/\I_m(X)$ and $k\{Y\} = k\{y_1,\ldots,y_m\}/\I_m(Y)$. We will construct a preimage of $\varphi$: Choose $f_1,\ldots,f_m \in k\{z_1,\ldots,z_n\}$ such that $\varphi(y_i + \I_m(Y)) = f_i + \I_m(X) \fa i \in \underline{m}$.
Then we define a morphism $f: X \rightarrow Y$ of $\s$-m-varieties over $k$ as follows: $f(b) := (f_1(b),\ldots,f_m(b)) \fa b \in B, ~ B \in \sintk$. This is well-defined: Let $h \in \I_m(Y)$. Then, by definition, $h(y_1 + \I_m(Y),\ldots,y_n + \I_m(Y)) = 0 + \I_m(Y)$.
This implies that $h(f_1 + \I_m(X),\ldots,f_m + \I_m(X)) = 0 + \I_m(X)$, since $\varphi$ is a morphism of $\s$-overrings of $k$. This in turn implies that $h(f(b)) = 0 \fa b \in X(B), ~ B \in \sintk$, which means that $f$ indeed maps onto $Y$ and $f^* = \varphi$, by construction.
\end{bew}
\end{prop}

This gives us a pretty good idea about the importance of the coordinate ring in difference algebra.
Having defined a category for $\s$-m-varieties, we can now see how this new category-theoretic language helps us to better understand the topological aspects of mixed difference varieties.

\begin{lem}\label{inducedcont}
Let $R,S,T$ be $\s$-rings, and $\varphi: R \rightarrow S,~ \psi: S \rightarrow T$ morphisms of $\s$-rings. Then the mapping $$\tilde \varphi: \sSpec(S) \rightarrow \sSpec(R), \p \mapsto \varphi^{-1}(\p)$$ 
induced by $\varphi$ is continuous. 
In fact, we have $\widetilde{ \psi \circ \varphi} = \tilde \varphi \circ \tilde \psi$, and in particular, $R \mapsto \sSpec(R)$ with $\psi \mapsto\tilde \psi$ is a contravariant functor from the category of $\s$-rings to \Top, the category of topological spaces.
\begin{bew}
Let $A = \V_m(F) \subseteq \sSpec(R)$ be closed. We have to show that $\tilde \varphi^{-1}(A) \subseteq \sSpec(S)$ is closed.
We have, 
\begin{align*} \tilde \varphi^{-1}(A) = \tilde \varphi^{-1}(\V_m(F)) = \{ \p \in \sSpec(S) \mid F \subseteq \varphi^{-1}(\p) \} \\ = \{\p \in \sSpec(S) \mid \varphi(F) \subseteq \p \} = \V_m(\varphi(F)). \end{align*}
The equality $\widetilde{ \psi \circ \varphi} = \tilde \varphi \circ \tilde \psi$ is immediately clear from the definitions.
\end{bew}
\end{lem}

We thus see how radical, mixed $\s$-ideals and the definition of $\s$-m-varieties as functors from $\sintk$, for a $\s$-field $k$, as well as the topology defined on $\sSpec(A\{X\})$ all fit together. 
These are all in analogous relations to the case for perfect $\s$-ideals and the Cohn topology outlined in Section 1. We will try to shed some light on the choice of the category $\sintk$ here:

\begin{defn}
Let $k$ be a $\s$-field. 
\begin{enumerate}[(a)]
\item We denote by $\s\text{\catname{-VarField}}_k$ the category of $\s$-varieties (as defined in Section 1). It has functors of the form $B \mapsto \VV_B(F)$ as objects, where $B$ is a $\s$-field extension of $k$.
As morphisms it has $\s$-polynomial maps defined in a fashion analogous to Definition \ref{spolynomialmaps}. 
\item Similarly, we denote by $\s\text{\catname{-VarDomain}}_k$ the category which has functors of the form $B \mapsto \VV_B(F)$ as objects, where $B$ is a $\s$-domain extension of $k$,
 and as morphisms $\s$-polynomial maps defined in a fashion analogous to Definition \ref{spolynomialmaps}. We define $\I_{\operatorname{Domain}}(X)$ for $X \in \s\text{\catname{-VarDomain}}_k$ and $\VV_{\operatorname{Domain}}(F)$ analogous to Definitions \ref{defnVV} and \ref{defnI}.
\item Finally,  we denote by $\s\text{\catname{-VarRing}}_k$ the category which has functors of the form $B \mapsto \VV_B(F)$ as objects, where $B \supseteq k$ is a perfectly $\s$-reduced ring over $k$,
 and as morphisms $\s$-polynomial maps defined in a fashion analogous to Definition \ref{spolynomialmaps}. We also define $\I_{\operatorname{Ring}}(X)$ for $X \in \s\text{\catname{-VarRing}}_k$ and $\VV_{\operatorname{Ring}}(F)$ analogous to Definitions \ref{defnVV} and \ref{defnI}.
\end{enumerate}
In all three cases $F \subseteq k\{y\}$ denotes a set of $\s$-polynomials on finitely many difference variables $y = (y_1, \ldots, y_n)$.
\end{defn}

\begin{prop}
Let $k$ be a $\s$-field. The three categories $\s\text{\catname{-VarField}}_k$, $\s\text{\catname{-VarDomain}}_k$ and $\s\text{\catname{-VarRing}}_k$ are equivalent.

\begin{bew}
Similar to Proposition \ref{dualisequiv}, the category of $\s$-varieties $\s\text{\catname{-VarField}}_k$ is anti-equivalent to the category of perfectly $\s$-reduced $\s$-overrings of $k$ which are finitely $\s$-generated over $k$ (see \cite{wibmer}, Theorem 2.1.21).
It suffices to show that the other two categories are also anti-equivalent to it. From the proof of Proposition \ref{dualisequiv} we can see that it is enough to show that $\I_{\operatorname{Domain}}(\VV_{\operatorname{Domain}}(\a)) = \{\a\}$, and $\I_{\operatorname{Ring}}(\VV_{\operatorname{Ring}}(\a)) = \{\a\}$,
 where $\a \si k\{y\}$ is a $\s$-ideal and $\{ \a \}$ its perfect closure. 

Let first $X  = \VV_{\operatorname{Ring}}(F) \in \s\text{\catname{-VarRing}}_k$ be a $\s$-m-variety in this sense of perfectly reduced $\s$-rings. 
We first show that \begin{align*} \I_{\operatorname{Ring}}(X) =  \{ f \in k\{y\} \mid f(b) = 0 \fa b \in \VV_B(F), \\ B \supseteq k\text{ perfectly }\s\text{-reduced} \} \end{align*}
 is a perfect $\s$-ideal. Similar to Proposition \ref{I=F_m}, we know that $\I_{\operatorname{Ring}}(X)$ is a difference ideal. Let $f \in k\{y\}$ with $\s^{i_1}(f) \cdots \s^{i_r}(f) \in \I(X)$. This means that for all $b \in \VV_B(F)$,
 $B$ perfectly $\s$-reduced: $\s^{i_1}(f)(b) \cdots \s^{i_r}(f)(b) = 0$. Since $B$ is perfectly $\s$-reduced, this means that $f(b) = 0$ for all such $b$,
 which in turn, by definition, means that $f \in \I_{\operatorname{Ring}}(X)$. Since every $\s$-domain is perfectly $\s$-reduced, the argument also works for $X \in \s\text{\catname{-VarDomain}}_k$, with $\I_{\operatorname{Domain}}$ instead of $\I_{\operatorname{Ring}}$.
This implies that $$ \{ F \} \subseteq \I_{\operatorname{Domain}}(\VV_{\operatorname{Domain}}(F))\text{, as well as } \{ F \} \subseteq \I_{\operatorname{Ring}}(\VV_{\operatorname{Ring}}(F)).$$
For the other inclusion ``$\supseteq$'', we shall first consider $X \in \s\text{\catname{-VarDomain}}_k$.
For $F \subseteq k\{y\}$, we have $\{F\} \supseteq \I_{\operatorname{Domain}}(\VV_{\operatorname{Domain}}(F))$. To show this, let \\ $f \in \I_{\operatorname{Domain}}(\VV_{\operatorname{Domain}}(F))$.
We know that $\{F\}$ is the intersection of all $\s$-prime ideals of $k\{y\}$ which contain $F$ (see Theorem \ref{intersectionperfect}), so it is enough to show that $f \in \p$ for each $\s$-prime $\p \si k\{y\}$ with $F \subseteq \p$.
We define the $\s$-domain $B:= k\{y\}/\p =: k\{a\}$, with $a := y + \p \in k\{y\}/\p$. Since $F \subseteq \p$, we get $a \in \VV_B(F)$, which, by definition of $\I_{\operatorname{Domain}}(\VV_{\operatorname{Domain}}(F))$, means that $f(a) = 0$. Hence, $f \in \p$ for all $\s$-prime $\p$
with $F \subseteq \p$, which in turn implies that $f \in \{F\}$. Since every $\s$-domain $B$ is perfectly $\s$-reduced, this works for the category $\s\text{\catname{-VarRing}}_k$ as well, with $\I_{\operatorname{Ring}}, \VV_{\operatorname{Ring}}$ instead of $\I_{\operatorname{Domain}}, \VV_{\operatorname{Domain}}$.
\end{bew}

\end{prop}

While there are many analogies with the Cohn topology, a very important aspect still remains open. Perfect difference ideals of $k\{y\}$ satisfy the ascending chain condition (see Chapter 3, Theorem IV of \cite{cohn}), which means that the Cohn topology yields a Noetherian topological space.
In fact, there is a generalization of the Hilbert basis theorem for this case; see Chapter 3, Theorem V of \cite{cohn}. The methods used to prove this in \cite{cohn} do not seem to work as simply for the case of radical, mixed difference ideals.
In fact, as far as the author knows, it is still not known if radical, mixed difference ideals satisfy the ascending chain condition. Mixed difference ideals do not satisfy it, as was shown by Levin in \cite{levinmixed}.

%% \begin{rem}
%% Let $f: X \rightarrow Y$ be a morphism of $\s$-m-varieties over the integral $\s$-ring $A$. Then the morphism $f^*: A\{y\} \rightarrow A\{x\}$ of $-s$-overrings of $A$ induces a continuous function
%% \[ \tilde{(f^*)}: \sSpec(A\{X\}) \rightarrow \sSpec(A\{Y\}), M \mapsto (f^*)^{-1}(M) \]
%% On the other hand FIXME: finish!
%% as in Lemma \ref{inducedcont}
%% \end{rem}

%% \begin{ex}
%% 2.4, here:more points? + 2.2.5

%% \end{ex}

%% section 2.3 not necesarry!

Over a field $k$, the polynomial $x^n$ has only a single solution, namely $0$. 
There is a difference, however, between $x$ and $x^2$ for example, which we cannot establish by simply looking at the solutions of these polynomials over a field.
In particular, the solution $0$ has a different 'multiplicity' for different $n$.  A way we can distinguish $x$ and $x^2$ is by looking for solutions elsewhere, not only in fields.
Over a general commutative ring, for example, the sets of solutions of the equations $x = 0$ and $x^2 = 0$ are different. \\


On a related note, the ideals generated by the two polynomials, $(x)$ and $(x^2) \unlhd k[x]$ are different. However, the radicals of these ideals are equal: 
$$ \sqrt{ (x) } = (x) = \sqrt{(x^2)}, $$ which means we cannot distinguish the polynomials like that either.
Instead of considering the coordinate ring of the varieties, we could capture this 'multiplicity', for example, by studying the ring $k[x]/(x^n)$.
The $k$-dimension of this ring is exactly $n$, $\operatorname{dim}_k( k[x]/(x^n)) = n$, what we would like to define as the multiplicity. \\

A similar problem arises in difference algebra, where the fields and polynomial rings also have a \emph{difference operator} $\s$.
A difference operator is basically a ring(field) endomorphism, which in the polynomial ring can also be applied to the variables.
Here, instead of the degree of the polynomial, we are interested in the \emph{order} of the difference equation.
In analogy to the equation $x^n = 0$, we can consider the difference equation $\s^n(x) = 0$. This is an equation of order $n$.
However, in a difference field $k$ there is only one solution to $\s^n(x) = 0$: since $\s$ is an endomorphism of the field $k$,
it has to be injective. Thus, $\s^n(x) = 0$ implies that $x = 0$. \\

We would like to study difference equations in a way we can distinguish the difference equations $\s(x) = 0$ and $\s^2(x) = 0$. Just as in the analogy for the polynomials $x$ and $x^2$,
there seems to be a sort of 'multiplicity' of the solution $0$ in the difference equation $\s^2(x) = 0$. We would like to have a way of finding this 'multiplicity' of the solutions too.

As suggested by the analogy above, we achieve this by changing the type of structure where we look for solutions of the difference equations. Instead of looking for solutions in difference fields,
we consider integral domains with an endomorphism. This endomorphism does not have to be injective anymore, and thus, the two difference equations can have different sets of solutions.
In difference polynomial rings we are interested in so-called \emph{difference ideals}, which are ideals that are invariant under $\s$. 
As in algebraic geometry, to a set of difference equations $F$ we will define a sort of difference algebraic sets $\VV_m(F)$ that take this additional solutions into account.
This new approach will be seen to be closely related to a class of difference ideals called \emph{mixed difference ideals}.
We also achieve a way of quantifying the 'multiplicity' of the solutions to the difference equations. The following theorem does this, and it is one of the main results of this thesis

\begin{theorem*}
Let $k$ be a $\s$-field and $f \in k\{y_1,\ldots,y_n\}, f \notin k$ an irreducible $\s$-polynomial. 
Then $\VV_m(f)$ has an irreducible closed subset $X$ such that $\s$-$\dim(X) = n-1$ and $\s$-$\operatorname{deg}(X) = \operatorname{Ord}(f)$.
\end{theorem*}

The concepts of $\s$-$\operatorname{deg}$ and $\s$-$\dim$ are of course also appropriately defined as part of the thesis. \\

To prove the theorem above, we consider so-called \emph{difference kernels}.
In a difference polynomial ring, even if we only have one '\emph{difference variable}' we have an infinite number of algebraically-independent variables when considered as a polynomial ring. 
These variables come from all the powers of $\s$ applied to the difference variables. Since such a ring is not Noetherian, there is a lot of difficulty that arises when attempting to study it. 
The idea behind difference kernels is the following.
In order to study a difference ideal $\a$ in a difference polynomial ring $k[y,\s(y),\s^2(y),\ldots]$,
we study the intersection of the ideal with a finite number of variables, and call it a \emph{difference kernel} $$\a[d] := \a \cap k[y,\s(y),\ldots,\s^d(y)].$$
We can then study the properties $\a[d]$ has depending on the properties of the difference ideal $\a$. At least as interesting is the question how it works the other way around.
What properties of an ideal $I \unlhd k[y,\s(y),\ldots,\s^d(y)]$ ensure that there is a \emph{realization} of this ideal: a difference ideal $\a$ in the difference polynomial ring such that $I = \a[d]$?
 And how do the properties of $I$ affect those of $\a$?
For a class of difference ideals called \emph{$\s$-prime difference ideals} the answers to the former questions are known. 
There is a characterization of the ideals $I \unlhd k[y,\s(y),\ldots,\s^d(y)]$ where there exists a realization as a $\s$-prime difference ideal. 
For the concepts developed in this thesis, however, a larger class of difference ideals is interesting. We are interested in prime difference ideals. 

Unfortunately, a full characterization as the one just described was not found. However, some results in the direction of one were achieved. A main result of this thesis is a characterization of the difference kernels that 
can be extended from $k[y,\ldots,\s^d(y)]$ to $k[y,\ldots,\s^{d+1}(y)]$. Also, a characterization of the ideals of $k[y,\ldots,\s^d(y)]$
which have a realization as a prime difference ideal is conjectured. \\


This thesis is divided into four sections. Section 1 will introduce the subject of difference algebra, presenting the most basic definitions and some of the more known results concerning the standard theory of perfect difference ideals and their geometric aspects.
The rest of the thesis will then deal with the results presented in Section 1 and, to the extent possible, generalize them to the case of mixed difference ideals. 
Section 2 begins with a more in-depth study of difference ideals, with a particular emphasis on mixed difference ideals. It also touches on some geometrical subjects, defining a topological space similar to the Zariski topology on a special class of ideals of a difference ring.
Section 3 continues with a more geometric nature, presenting a definition of mixed difference varieties as well as some basic properties of these. 
Finally, Section 4 deals with difference kernels. The two main results and the conjecture stated above are presented here.

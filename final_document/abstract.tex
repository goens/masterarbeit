In this thesis we will develop the basis for studying geometric aspects of difference algebraic structures based on mixed difference ideals. This will be put in contrast to the usual theory which is based on perfect difference ideals, a subclass of mixed difference ideals. Section 1 shall serve as an introduction to difference algebra, presenting most basic definitions and some of the more known results concerning perfect ideals and their geometric aspects.
The rest of the thesis will then deal with the results presented in Section 1 and, to the extent possible, generalize them to the case of mixed difference ideals. 
Section 2 begins with a more in-depth study of difference ideals, with a particular emphasis on mixed difference ideals. It also touches on some geometrical subjects, defining a topological space similar to the Zariski topology on a special class of ideals of a difference ring.
Section 3 continues with a more geometric nature, presenting a definition of mixed difference varieties as well as some basic properties of these. 
Finally, Section 4 deals with a concrete method for studying difference ideals, namely through so-called difference kernels. In this section the difference to perfect difference ideals becomes more visible with a result that cannot be generalized in a simple way. A conjecture is presented of a possible generalization. Also, a result of more geometric nature is presented which gives an idea of why the theory with mixed ideals might be useful,
as a partial generalization of a theorem is achieved which has less restrictive conditions than the original.

In a field $k$, the polynomial $x^n$ has only a single solution for any $n$, namely $0$. 
There is obviously a difference, however, between the equations $x = 0$ and $x^2 = 0$ for example. In particular, the solution $0$ has a different 'multiplicity' for different $n$.
We cannot establish this difference by simply looking at the solutions of these polynomials over a field.
  A way we can distinguish $x$ and $x^2$ is by looking for solutions elsewhere, not only in fields.
Over a general $k$-algebra, for example, the sets of solutions of the equations $x = 0$ and $x^2 = 0$ could be different. \\


On a related note, the ideals generated by the two polynomials, $(x)$ and $(x^2) \unlhd k[x]$ are different. However, the radicals of these ideals are equal,
$$ \sqrt{ (x) } = (x) = \sqrt{(x^2)}, $$ which means we cannot distinguish the polynomials like that either.
Instead of considering the coordinate ring of the varieties, we could capture this 'multiplicity' by studying the ring $k[x]/(x^n)$.
The $k$-dimension of this ring is $n$, which is exactly what we would like to define as the multiplicity. \\

A similar problem arises in difference algebra, where we consider rings (fields) equipped with an endomorphism which we usually denote by $\s$ and call a \emph{difference operator}. We call the tuple of the ring (field) and the endomorphism a \emph{difference ring (field)}.
The analog concept to polynomial rings is that of so-called \emph{difference polynomial rings}. We consider difference polynomial rings as polynomial rings on infinitely many variables, a subset of which we call \emph{difference variables} and the rest represent the powers of $\s$ of a difference variable. The variables $y, \s(y), \s^2(y), \ldots$ are algebraically independent.
Here, instead of the degree of the polynomial, we are interested in the \emph{order} of the difference equation.
In analogy to the equation $x^n = 0$, we can consider the difference equation $\s^n(x) = 0$. This is a difference equation of order $n$.
However, in a difference field $k$ there is only one solution to the equation $\s^n(x) = 0$: since $\s$ is an endomorphism of the field $k$,
it has to be injective. Thus, $\s^n(x) = 0$ implies that $x = 0$. \\

We would like to study difference equations and their solutions in a way we can distinguish between $\s(x) = 0$ and $\s^2(x) = 0$. Just as in the analogy for the polynomials $x$ and $x^2$,
there seems to be a sort of 'multiplicity' of the solution $0$ in the difference equation $\s^2(x) = 0$. We would also like to have a way of finding this 'multiplicity' of the solutions we just described.

As suggested by the analogy above, we achieve this by changing the type of structure where we look for solutions of the difference equations. Instead of looking for solutions in difference fields,
we consider difference rings which are integral domains. The endomorphism $\s$ does not have to be injective anymore, and thus, the two difference equations $\s^2(y) = 0$ and $\s(y) = 0$ can have different sets of solutions.

In difference polynomial rings we are interested in so-called \emph{difference ideals}, which are the ideals that are stable under $\s$. 
As in algebraic geometry, to a set of difference equations $F$ we will define a sort of analog to algebraic sets, which we denote by $\VV_m(F)$. 
This object can be interpreted as the set of solutions to the system difference equations $F$  and takes the additional solutions discussed above into account.
We also get corresponding concepts analogous to the concepts of vanishing ideals and the Zariski topology, as well as close connection between them. 
A further analogy  we get are the concepts of $\s$-$\operatorname{deg}$ and $\s$-$\dim$, both of which are related to the concept of the Krull dimension.
This new approach will be seen to be closely related to a class of difference ideals called \emph{mixed difference ideals}, similar to the way affine algebraic varieties are related to radical ideals.
We also achieve a way of quantifying the 'multiplicity' of the solutions to the difference equations. The following theorem does this, and it is one of the main results of this thesis

\begin{theorem*}
Let $k$ be a difference field and let $f \notin k$ be an irreducible difference polynomial over $n$ difference variables. 
Then $\VV_m(f)$ has an irreducible closed subset $X$ such that $\s$-$\dim(X) = n-1$ and $\s$-$\operatorname{deg}(X) = \operatorname{Ord}(f)$.
\end{theorem*}

To prove the theorem above, we consider so-called \emph{difference kernels}.
In a difference polynomial ring, even if we only have one '\emph{difference variable}' we have an infinite number of algebraically-independent variables when considered as a polynomial ring. Such rings are not Noetherian, and thus, some difficulties arise when attempting to study them. 

The idea behind difference kernels is the following.
In order to study a difference ideal $\a$ in a difference polynomial ring $k[y,\s(y),\s^2(y),\ldots]$,
we study the intersection of the ideal with a finite number of variables, and call it a \emph{difference kernel} $$\a[d] := \a \cap k[y,\s(y),\ldots,\s^d(y)].$$
We can then study the properties that $\a[d]$ has depending on the properties of the difference ideal $\a$. At least as interesting is the question how it works the other way around.
What properties of an ideal $I \unlhd k[y,\s(y),\ldots,\s^d(y)]$ ensure that there exists a so-called \emph{realization} of this ideal, that is, a difference ideal $\a$ in the difference polynomial ring such that $I = \a[d]$?
 If that is the case, how do the properties of $I$ affect those of $\a$?
For a class of difference ideals called \emph{$\s$-prime difference ideals} the answers to the former questions are known. 
There is a characterization of the ideals $I \unlhd k[y,\s(y),\ldots,\s^d(y)]$ for which there exists a realization as a $\s$-prime difference ideal. 
For the concepts developed in this thesis, however, a larger class of difference ideals is interesting. We are interested in difference ideals that are prime. 

Unfortunately, a full characterization of the ideals of $k[y,\ldots,\s^d(y)]$ for which there exists a realization that is a prime difference ideal was not found. However, some results in the direction of one were achieved. Another main result of this thesis is a characterization of the prime difference kernels that 
can be extended from $k[y,\ldots,\s^d(y)]$ to $k[y,\ldots,\s^{d+1}(y)]$. A conjecture is given of a possible characterization of the ideals of $k[y,\ldots,\s^d(y)]$
which have a realization as a prime difference ideal. \\


This thesis is divided into four sections. Section 1 will introduce the subject of difference algebra, presenting the most basic definitions and some of the more known results concerning the standard theory of perfect difference ideals and their geometric aspects.
The rest of the thesis will then deal with the generalizations  to the case of mixed difference ideals that were achieved from the results presented in Section 1. 
Section 2 begins with a more in-depth study of difference ideals, with a particular emphasis on mixed difference ideals. It also touches on some geometrical subjects, defining a topological space similar to the Zariski topology on a special class of ideals of a difference ring.
Section 3 continues with a more geometric nature, presenting a definition of mixed difference varieties as well as some basic properties of these. 
Finally, Section 4 deals with difference kernels. The two main results and the conjecture stated above are presented here.

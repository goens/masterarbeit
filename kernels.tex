%\documentclass[12pt,a4paper,BCOR15mm,twoside,DIV12]{article}
\documentclass{article}
%\usepackage[paper=a4paper,left=20mm,right=20mm,top=25mm,bottom=25mm]{geometry}
\usepackage[english]{babel}
\usepackage[utf8]{inputenc}
\usepackage{amsmath}
\usepackage{color}
\usepackage{amssymb}
\usepackage{amsfonts}
\usepackage{amsthm}
\usepackage{hyperref}
\usepackage{makeidx}
\usepackage{graphicx, float,epsfig}
\usepackage[nottoc,numbib]{tocbibind}


\newcommand{\properideal}{%
  \mathrel{\ooalign{$\lneq$\cr\raise.22ex\hbox{$\lhd$}\cr}}}

\def\P{\mathcal{P}}
\def\I{\mathbb{I}}
\def\R{\mathbb{R}} 
\def\E{\mathcal{E}} 
\def\NE{\mathbb{N}_{\geq1}} 
\def\N{\mathbb{N}} 
\def\Z{\mathbb{Z}} 
\def\Q{\mathbb{Q}} 
\def\F{\mathbb{F}}
\def\Vm{\mathcal{V}_m}
\def\V{\mathcal{V}}
\def\VV{\mathbb{V}}
\def\C{\mathbb{C}}
\def\U{\mathcal{U}}
\def\a{\mathfrak{a}}
\def\b{\mathfrak{b}}
\def\p{\mathfrak{p}}
\def\q{\mathfrak{q}}
\def\s{\sigma}
\def\si{\unlhd_{\sigma}}
\def\GL{\text{GL}}
\def\supp{\text{Supp}}
\def\id{\text{id}}
\def\n{\underline{n}}
\def\Spec{\text{Spec}}
\def\sSpec{\sigma\text{-Spec}}
\def\diag{\text{diag}}
\def\End{\text{End}}
\def\Hom{\text{Hom}}
\def\fa{\text{ for all }}
\def\Tr{\text{Tr}}
\def\Id{\text{Id}}
\def\ker{\text{ker}}
\def\H{\mathcal{H}}
\def\trdeg{\text{trdeg}}
\def\sdim{\sigma\text{-dim}}


\renewcommand{\labelenumi}{\alph{enumi})}
%\renewcommand{\P}{\textfrak{P}}
\newcommand{\cupdot}{\mathop{\mathaccent\cdot\cup}}
\newenvironment{bew}{\begin{proof}[Proof]}{\end{proof}}
\theoremstyle{definition}
\newtheorem{Satz}{Satz}[section]
\newtheorem{theorem}[Satz]{Theorem}
\newtheorem{ex}[Satz]{Example}
\newtheorem{cor}[Satz]{Corollary}
\newtheorem{algorithm}[Satz]{Algorithm}
\newtheorem{prop}[Satz]{Proposition}
\newtheorem{rem}[Satz]{Remark}
\newtheorem{defn}[Satz]{Definition}
\newtheorem{lem}[Satz]{Lemma}


\makeindex
\title{Difference Kernels}
\author{Andr\'{e}s Goens}
\date{\today}

\begin{document}
\section{Difference Kernels}

%% blah blah de entrada...
%% \begin{theorem}

%% Let $k$ be a $\s$-field and $a$ 
%% \end{theorem}

\begin{defn}
Let $\p \unlhd k\{y\}[d], d \geq 1$ be a prime ideal of $k\{y\}[d]$. Then $\p$ is called a weak difference kernel of length $d$, if $\s(\p[d-1]) \subseteq \p$. It is called a difference kernel of length $d$, if $\s^{-1}(\p) = \p[d-1]$.
\end{defn}\index{(weak) kernels}

\begin{rem}
It is easy to see that a kernel is always a weak kernel, but the converse is not necesarilly true: a weak kernel only guarantees the inculsion $\p[d-1] \subseteq \s^{-1}(\p)$.
\end{rem}

\begin{ex}
Let $\p \si k\{y\}$ be a prime $\s$-ideal, $d \geq 1$. Then $\p[d] \unlhd k\{y\}[d]$ is a weak kernel: since $\p$ is a $\s$-ideal we have $\s(\p[d-1]) \subseteq \p$, 
and since $\p$ is prime we know that $\p[d]$ has to be prime as well. If $\p$ is also reflexive (i.e. a $\s$-prime ideal), then $\p[d]$ is a kernel. 
\end{ex}

Inspired by the former example we define the following:
\begin{defn}
Let $\p \unlhd k\{y\}[d]$ be a weak kernel of length $d$, and $\p' \si k\{y\}$ be a prime $\s$-ideal. Then we call $\p'$ a realization of $p$, iff $\p \subseteq \p'$, 
and we call the realization regular if it also holds that $\p'[d] = \p$. Similarly, if $\p$ is a kernel, then we require a realization to be a $\s$-prime ideal.
\end{defn}\index{(regular) realization}

The former example and the acompanying definition is actually much more general than it would seem to be at first. 
It is in fact the case, that weak kernels are exactly of the form $\p[d]$ for prime $\s$ ideals,
and kernels for $\s$-prime ones, i.e., we can always find a regular realization of (weak) kernels. 
To prove this, however, we still need to develop more of a framework around difference kernels.

\begin{rem}\label{sigmawelldeffker}
Let $\p \subseteq k\{y\}$ be a weak difference kernel of length $d$. Then, $\s$ induces a well-defined mapping 
\begin{align*}
\s: k[y,\ldots,\s^{d-1}(y)]/\p[d-1] \rightarrow k[y,\ldots,\s^{d}(y)]/\p
\end{align*}
If $\p$ is a difference kernel, then this mapping is injective. 
If we set $a := \bar y = y + \p \in k(\p) = k\{y\}[d]/\p$, then we can extend $\s$ to the quotient fields:
\[ \s: k(a,\ldots,s^{d-1}(a)) \cong \text{Quot}(k[y,\ldots,\s^{d-1}(y)]/\p[d-1]) \rightarrow k(a,\ldots,\s^d(a)) = k(\p) \]
\end{rem}

Even in the case of weak difference kernels we can work with the properties of field extensions, though we cannot properly extend $\s$ to the fields.
 In particular, we get a nice way of defining some sort of ``difference degree'' of (weak) kernels:
\begin{defn}
Let $\p \subseteq k\{y\}$ be a (weak) difference kernel of length $d$, and let $a:= y + \p \in k(p)$; we define the difference dimension of $\p$ as follows:
\begin{align*} \sdim(\p) := \text{trdeg}(k(\p)/k(p[d-1]) = \text{trdeg}((k\{y\}[d]/\p)/(k\{y\}[d-1]/p[d-1])) \\  = \text{trdeg}(k(a,\ldots,\s^{d}(a))/k(a,\ldots,\s^{d-1}(a))) \end{align*}
\end{defn}\index{$\s$-dimension of a $\s$-kernel}

If we want to show that (weak) difference kernels are the intersection of (prime $\s$-ideals) $\s$-prime ideals with $k\{y\}[d]$,
it is reasonable to consider some sort of extension, or \emph{prolongation} of a (weak) kernel, which would be the intesrection with $k\{y\}[d+1]$. 
This motivates the following definition:

\begin{defn}
Let $\p \subseteq k\{y\}$ be a (weak) difference kernel of length $d$, and $\p' \supset \p$ be a further (weak) difference kernel, of length $d+1$.
Then we call $\p'$ a \emph{prolongation} of $\p$, iff $\p'[d] = \p$. If it holds further that $\sdim(\p) = \sdim(p')$, then we call the prolongation \emph{generic}.
\end{defn}\index{prolongation, generic}


\begin{defn}
Let $\p \subseteq k\{y\}$ be a (weak) difference kernel of length $d$, and let $\p'$ be a realization of $\p$. We call $\p'$ a principal realization of $\p$, iff $\p'[i+1]$ is a generic prolongation of $\p'[i]$ for all $i \geq d$.
\end{defn}

\begin{lem}\label{primeoverp1}
Let R be an integral domain and $F \subseteq R[y]$ be a subset of $R[y] = R[y_1,\ldots,y_n]$ which satisfies that $I \cap R = \{ 0 \}$.
Then there exists a prime ideal $P \supset I$ with $P \cap R = \{0\}$.
\begin{proof}
We can assume without loss of generality that $I$ is an ideal: Let $f = \sum_{i=1}^kr_i x_i \in (I)$, where $r_i \in R[y], x_i \in (I), i = 1,\ldots,k$. If $f \in R$, by a degree argument it has to hold that $x_i, r_i \in R, i=1,\ldots,k$.
But that means that $r_i \in R \cap I = \{0\}, i=1,\ldots,k$, hence $f = 0$. Similarly, we can assume that I is radical:
Namely, if $f \in \sqrt{I} \cap R$, then there exists an $m \in \N$ such that $f^m \in I \cap R = \{0\}$, and since $R$ is an integral domain this already means that $f = 0$.
We then note that for two sets $A,B \subseteq R[y]$ it holds that $\sqrt{A}\sqrt{B} \subseteq \sqrt{AB}$: Consider $f \in \sqrt{A}, g \in \sqrt{B}$. Then there exist $m, \tilde m \in \N$ such that $f^m \in (A), g^{\tilde m} \in (B)$;
 assume without loss of generality that $m > \tilde m$, then $(fg)^m \in (A)(B)$, which implies $fg \in \sqrt{(A)(B)} = \sqrt{AB}$.
Now, for the proof, consider the set of all radical ideals $J$ contaning $I$ which satisify $J \cap R = \{0\}$. This set is not empty and is inductively ordered by inclusion.
By Zorn's lemma this means that there is a maximal element $P$ of this set. This ideal $P$ is prime, then: assume there exist $f,g \notin P$ with $fg \in P$. 
Then the radical ideals $\sqrt{P \cup \{f\}}$, $\sqrt{P \cup \{f\}}$ strictly include $P$, and by the maximality of $P$ it means there exist $t_1, t_2 \in R\backslash\{0\}$ such that
$t_1 \in \sqrt{P \cup \{f\}}$, $t_2 \in \sqrt{P \cup \{g\}}$. But in particular this implies that
 \[t_1t_2 \in \sqrt{P \cup \{f\}}\sqrt{P \cup \{g\}} \subseteq \sqrt{ \underbrace{(P \cup \{f\})(P \cup \{g\})}_{=P\text{, since }fg \in P}} = P\]
A contradiction.
\end{proof}
\end{lem}

\begin{lem}\label{genericprol}
Let $\p \subseteq k\{y\}$ be a (weak) difference kernel of length $d$. Then there exists a prolongation of $\p$. If $\p$ is a kernel, then there exists a prologation that is generic. 
\begin{bew}
Consider the canonical epimorphism of rings $k[y,\ldots,\s^{d}(y)] \rightarrow k[y,\ldots,\s^{d}(y)]/\p$ and let $a = y + \p$ be the image of $y$ under it. 
Then the we have a subring $R_1: = k[a,\ldots,\s^{d-1}(a)] \subseteq k[a,\ldots,\s^{d}(a)] = k[y,\ldots,\s^{d}(y)]/\p =: R$. Now, consider the univariate polynomial ring over $R_1$ on the free variable $\s^d(y)$:
\[ R_1[\s^d(y)] = k[a,\ldots,\s^{d-1}(a)][s^d(y)] = k[a,\ldots,\s^{d-1}(a),s^d(y)] \]
We have a natural morphism of rings \[ R_1[s^d(y)] \rightarrow R = k[a,\ldots,\s^{d}(a)], \s^d(y) \mapsto \s^d(a) \]
Let $\p_1$ be the kernel of this morphism. Since $R$ is an integer domain (as $\p$ is prime), so is $\p_1$ also prime. 
Now we consider the univariate polynomial ring $R[\s^{d+1}(y)] \supset R = k[a,\ldots, \s^d(a)]$. Here we get a natural definition of $\s: R_1[\s^d(y)] \rightarrow R[\s^{d+1}(y)]$ by mapping $\s( \s^d(y))  := \s^{d+1}(y)$.
By definition of $\p_1$ we have that $\s(\p_1) \cap R = \{0\}$. 
By Lemma \ref{primeoverp1} there exists a minimal prime ideal $\p_2 \supset \s(p_1)$ of $R[\s^{d+1}(y)]$ containing $\s(p_1)$ with $\p_2 \cap R = \{0\}$. 
We thus get a well-defined maping
\[ \s: k[a,\ldots,\s^{d-1}(a)][\s^d(y)]/\p_1 \rightarrow k[a,\ldots, \s^d(a)][\s^{d+1}(y)]/\p_2 \]
We define $R_2:= k[a,\ldots,\s^d(a)][\s^{d+1}(y)]/\p_2 =: k[a,\ldots,\s^d(a),\s^{d+1}(a)]$, which is an integer domain since $\p_2$ was prime. Since $\p_2 \cap R = \{0}$ we can use this notation unambiguosly:
this guarantees namely that for $a, \ldots, \s^d(a)$ we have the same residue classes modulo $\p_2$ as we had modulo $\p$.
The kernel $\p'$ of the natural epimorphism $k[y,\ldots,\s^{d+1}(y)] \rightarrow R_2$ is thus a prime ideal.
Further we have $\p \subseteq \p'$ by construction (as $\p = 0 \subset R_2$). In fact, it holds that $\p'[d] = \p$ since: 
\begin{align*}
\p'[d] = \{ f \in k[y,\ldots,\s^d(y)] \mid f(a) = 0 \} = \ker( k\{y\}[d] \rightarrow k[a,\ldots,\s^{d}(a)]) = \p
\end{align*}
where the first equality uses the fact that $\p_2 \cap R = \{0\}$, as noted by the use of the notation explained above, and last equality holds by definition of $a \in R$. This means, that $\p'$ is a prolongation of $\p$. 

Now, if additionaly, $\p$ is a kernel, then $\s$ from Remark \ref{sigmawelldeffker} is injective and we can go over to the quotient fields:
We let $\tilde \p_1$ be the prime ideal which is the kernel of the morphism of rings
\[ \text{Quot}(R_1)[s^d(y)] \rightarrow \text{Quot}(R) = k(a,\ldots,\s^{d}(a)), \s^d(y) \mapsto \s^d(a) \]
And let $\tilde \p_2$ be a minimal prime ideal in 
\[k(a,\ldots,\s^d(a))[\s^{d+1}(y)] = \text{Quot}(k[a,\ldots,\s^d(a)])[\s^{d+1}(y)] \] containing $\s(\tilde \p_1)$.
In particular, $\s$ is injective and thus an isomorphism to the prime ideal $\s(\tilde \p_1) \subseteq \s(k)(\s(a),\ldots,\s^d(a))[\s^{d+1}(y)]$.
It holds that the mapping
\[ \s: k(a,\ldots,\s^{d-1}(a))[\s^d(y)]/\tilde \p_1 \rightarrow k(a,\ldots, \s^d(a))[\s^{d+1}(y)]/\tilde \p_2 \]
is well-defined and injective as well. By abuse of notation (and because of the minimality of $\tilde \p_2$ again) 
we will also call $\s^{d+1}(a)$ the image of $\s^{d+1}(y)$ in $k(\tilde \p_2) = \text{Quot}(k(a,\ldots, \s^d(a))[\s^{d+1}(y)]/\tilde \p_2)$.
Then the kernel $\q$ of \[ k[y,\ldots,\s^{d+1}(y)] \rightarrow k(\tilde p_2), \s^{i}(y) \mapsto \s^{i}(a) \] is a prolongation of $\p$, as above, and we have:
\begin{align*}
\sdim(\p) = \trdeg( k(a,\ldots,\s^d(a)) / k(a,\ldots,\s^{d-1}(a))) = \text{dim}(k(a,\ldots,\s^{d-1}(a))[\s^d(y)]/\tilde \p_1) \\ = \text{dim}(\s(k)(\s(a),\ldots,\s^d(a))[\s^{d+1}(y)]/\s(\tilde \p_1)) 
= \text{dim}(k(a,\ldots,\s^{d}(a))[\s^{d+1}(y)]/\tilde \p_2) \\ = \trdeg(k(a,\ldots,\s^{d+1}(a))/k(a,\ldots,\s^d(a))) = \sdim(\q)
\end{align*}
\end{bew}
\end{lem}

\begin{defn}
Let $k$ be a $\s$-field and  $\a \si k\{y\}$ be a $\s$-ideal. 
Further let $d_1, d_2 \in \Z$ with $-1 \leq d_1 < d_2$ We define $\a[d_1, d_2]:= \a[d_2] \backslash \a[d_1] = \{ r \in a[d_2] \mid r \notin a[d_1]$.
Note that this is not an ideal, as, for example, $0 \notin \a[d_1,d_2]$, but every element in $r \in a[d_1,d_2]$ has the property that $d_1 < $Ord$(r) \leq d_2$.
\end{defn}

\begin{prop}
Let $\p \subseteq k\{y\}$ be a (weak) kernel of length $d$. Then there exists a realization of $\p$. If $\p$ is a kernel, then this realization is principal.
In particular, weak kernels are exactly of the form $\p[d]$ for a prime $\s$-ideal $\p \si k\{y\}$, and similarly kernels for $\s$-prime ideals.
\begin{bew}
By Lemma \ref{genericprol} there exists a chain of prolongations $\p_{d+1} \supset \p$, $\p_{d+2} \supset \p_{d+1}, \ldots$, which are generic if $\p$ is a kernel.
Consider \[ \p':= \bigcup_{i \geq d} \p_i \]
This is a prime $\s$-ideal, since for $f \in \p'$ there exists an $i \in \N$ such that $f \in \p_i$. Then $\s(f) \in \s(p_{i+1}) \subset \p'$.
Since all $p_i$ are prime, any product $fg \in \p'$ must already be in a $\p_i$ for $i \in \N$ and thus $f \in \p_i$ or $g \in \p_i$. Further,
 if $\p$ is a kernel, then $\s^{-1}(\p_i) = \p_{i-1} \subset \p'$, so $\p'$ is reflexive. By construction is $\p'[d] = \p$ and if $\p$ is a kernel then the
 prolongations are all generic and thus $\p'$ is principal. Since for a prime $\s$-ideal $\p$ we have that $\p[d]$ is a weak kernel, and a kernel if $\p$ is also reflexive, 
the final assertion follows.
\end{bew}
\end{prop}

%% \begin{lem}
%% Let $\p \si k\{y\}$ be a prime $\s$-ideal and let $\p' := \p^* \supseteq \p$ be its reflexive closure (which is $\s$-prime). Then there exists an $n \in \N$ such that 
%% $\p[d_1,d_2] = \p'[d_1,d_2] \fa n \leq d_1 < d_2$.
%% \begin{bew}
%% We know that there exists an $n \in \N$ such that $\s^{-n}(\p) = \p^* = \p'$ is $\s$-prime. Let $d > n$. Then 
%% we have $\s^{-n}(\p[d]) = \s^{-n}(\p \cap k\{y\}[d]) = \s^{-n}(\p)\cap\s^{-n}(k\{y\}[d]) = \p'[d-n]$. 
%% Since $\p'[d-n]$ is a kernel of length $d-n$, we know that there exist unique generic prolongations of $\p'[n-d]$
%% \end{bew}
%% \end{lem}

\end{document}


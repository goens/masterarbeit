%\documentclass[12pt,a4paper,BCOR15mm,twoside,DIV12]{article}
\documentclass{article}
%\usepackage[paper=a4paper,left=20mm,right=20mm,top=25mm,bottom=25mm]{geometry}
\usepackage[english]{babel}
\usepackage[utf8]{inputenc}
\usepackage{amsmath}
\usepackage{color}
\usepackage{amssymb}
\usepackage{amsfonts}
\usepackage{amsthm}
\usepackage{hyperref}
\usepackage{makeidx}
\usepackage{graphicx, float,epsfig}
\usepackage[nottoc,numbib]{tocbibind}


\newcommand{\properideal}{%
  \mathrel{\ooalign{$\lneq$\cr\raise.22ex\hbox{$\lhd$}\cr}}}

\def\P{\mathcal{P}}
\def\I{\mathbb{I}}
\def\R{\mathbb{R}} 
\def\E{\mathcal{E}} 
\def\NE{\mathbb{N}_{\geq1}} 
\def\N{\mathbb{N}} 
\def\Z{\mathbb{Z}} 
\def\Q{\mathbb{Q}} 
\def\F{\mathbb{F}}
\def\Vm{\mathcal{V}_m}
\def\V{\mathcal{V}}
\def\VV{\mathbb{V}}
\def\C{\mathbb{C}}
\def\U{\mathcal{U}}
\def\a{\mathfrak{a}}
\def\b{\mathfrak{b}}
\def\p{\mathfrak{p}}
\def\q{\mathfrak{q}}
\def\s{\sigma}
\def\si{\unlhd_{\sigma}}
\def\GL{\text{GL}}
\def\supp{\text{Supp}}
\def\id{\text{id}}
\def\n{\underline{n}}
\def\Spec{\text{Spec}}
\def\sSpec{\sigma\text{-Spec}}
\def\diag{\text{diag}}
\def\End{\text{End}}
\def\Hom{\text{Hom}}
\def\fa{\text{ for all }}
\def\Tr{\text{Tr}}
\def\Id{\text{Id}}
\def\Sym{\text{Sym}}
\def\H{\mathcal{H}}
\def\wt{\text{wt}}
\def\sdim{\sigma\text{-dim}}


\renewcommand{\labelenumi}{\alph{enumi})}
%\renewcommand{\P}{\textfrak{P}}
\newcommand{\cupdot}{\mathop{\mathaccent\cdot\cup}}
\newenvironment{bew}{\begin{proof}[Proof]}{\end{proof}}
\theoremstyle{definition}
\newtheorem{Satz}{Satz}[section]
\newtheorem{theorem}[Satz]{Theorem}
\newtheorem{ex}[Satz]{Example}
\newtheorem{cor}[Satz]{Corollary}
\newtheorem{algorithm}[Satz]{Algorithm}
\newtheorem{prop}[Satz]{Proposition}
\newtheorem{rem}[Satz]{Remark}
\newtheorem{defn}[Satz]{Definition}
\newtheorem{lem}[Satz]{Lemma}


\makeindex
\title{Difference Kernels}
\author{Andr\'{e}s Goens}
\date{\today}

\begin{document}
\section{Difference Kernels}

blah blah de entrada...
\begin{theorem}

Let $k$ be a $\s$-field and $a$ 
\end{theorem}

\begin{defn}
Let $\p \unlhd k\{y\}[d], d \geq 1$ be a prime ideal of $k\{y\}[d]$. Then $\p$ is called a weak difference kernel of length $d$, if $\s(\p[d-1]) \subseteq \p$. It is called a difference kernel of length $d$, if $\s^{-1}(\p) = \p[d-1]$.
\end{defn}\index{(weak) kernels}

\begin{rem}
It is easy to see that a kernel is always a weak kernel, but the converse is not necesarilly true: a weak kernel only guarantees the inculsion $\p[d-1] \subseteq \s^{-1}(\p)$.
\end{rem}

\begin{ex}
Let $\p \si k\{y\}$ be a prime $\s$-ideal, $d \geq 1$. Then $\p[d] \unlhd k\{y\}[d]$ is a weak kernel: since $\p$ is a $\s$-ideal we have $\s(\p[d-1]) \subseteq \p$, 
and since $\p$ is prime we know that $\p[d]$ has to be prime as well. If $\p$ is also reflexive (i.e. a $\s$-prime ideal), then $\p[d]$ is a kernel. 
\end{ex}

Inspired by the former example we define the following:
\begin{defn}
Let $\p \unlhd k\{y\}[d]$ be a weak kernel of length $d$, and $\p' \si k\{y\}$ be a prime $\s$-ideal. Then we call $\p'$ a realization of $p$, iff $\p \subseteq \p'$, 
and we call the realization regular if it also holds that $\p'[d] = \p$. Similarly, if $\p$ is a kernel, then we require a realization to be a $\s$-prime ideal.
\end{defn}\index{(regular) realization}

The former example and the acompanying definition is actually much more general than it would seem to be at first. 
It is in fact the case, that weak kernels are exactly of the form $\p[d]$ for prime $\s$ ideals,
and kernels for $\s$-prime ones, i.e., we can always find a regular realization of (weak) kernels. 
To prove this, however, we still need to develop more of a framework around difference kernels.

\begin{rem}
Let $\p \subseteq k\{y\}$ be a weak difference kernel of length $d$. Then, $\s$ induces a well-defined mapping 
\begin{align*}
\s: k[y,\ldots,\s^{d-1}(y)]/\p[d-1] \rightarrow k[y,\ldots,\s^{d}(y)]/\p
\end{align*}
If $\p$ is a difference kernel, then this mapping is injective. 
If we set $a := \bar y = y + \p \in k(\p) = k\{y\}[d]/\p$, then we can extend $\s$ to the quotient fields:
\[ \s: k(a,\ldots,s^{d-1}(a)) \cong \text{Quot}(k[y,\ldots,\s^{d-1}(y)]/\p[d-1]) \rightarrow k(a,\ldots,\s^d(a)) = k(\p) \]
\end{rem}

Even in the case of weak difference kernels we can work with the properties of field extensions, though we cannot properly extend $\s$ to the fields.
 In particular, we get a nice way of defining some sort of ``difference degree'' of (weak) kernels:
\begin{defn}
Let $\p \subseteq k\{y\}$ be a (weak) difference kernel of length $d$, and let $a:= y + \p \in k(p)$; we define the difference dimension of $\p$ as follows:
\begin{align*} \sdim(\p) := \text{trdeg}(k(\p)/k(p[d-1]) = \text{trdeg}((k\{y\}[d]/\p)/(k\{y\}[d-1]/p[d-1])) \\  = \text{trdeg}(k(a,\ldots,\s^{d}(a))/k(a,\ldots,\s^{d-1}(a))) \end{align*}
\end{defn}\index{$\s$-dimension of a $\s$-kernel}

If we want to show that (weak) difference kernels are the intersection of (prime $\s$-ideals) $\s$-prime ideals with $k\{y\}[d]$,
it is reasonable to consider some sort of extension, or \emph{prolongation} of a (weak) kernel, which would be the intesrection with $k\{y\}[d+1]$. 
This motivates the following definition:

\begin{defn}
Let $\p \subseteq k\{y\}$ be a (weak) difference kernel of length $d$, and $\p' \supset \p$ be a further (weak) difference kernel, of length $d+1$.
Then we call $\p'$ a \emph{prolongation} of $\p$, iff $\p'[d] = \p$. If it holds further that $\sdim(\p) = \sdim(p')$, then we call the prolongation \emph{generic}.
\end{defn}\index{prolongation, generic}

\begin{defn}
Let $\p \subseteq k\{y\}$ be a (weak) difference kernel of length $d$, and let $\p'$ be a realization of $\p$. We call $\p'$ a principal realization of $\p$, iff $\p'[i+1]$ is a generic prolongation of $\p'[i]$ for all $i \geq d$.
\end{defn}

\begin{lem}
Let $\p \subseteq k\{y\}$ be a (weak) difference kernel of length $d$. Then there exists a generic prolongation of $\p$. 
\end{lem}

\end{document}

\documentclass{beamer}
\setbeamertemplate{footline}[frame number]
%\usepackage[paper=a4paper,left=20mm,right=20mm,top=25mm,bottom=25mm]{geometry}
\usepackage[german]{babel}
\usepackage[utf8]{inputenc}
\usepackage{amsmath}
\usepackage{color}
\usepackage{amssymb}
\usepackage{amsfonts}
\usepackage{amsthm}
%\usepackage{hyperref}
\usepackage{graphicx, float,epsfig}
%\usepackage[nottoc,numbib]{tocbibind}

\def\P{\mathcal{P}}
\def\R{\mathbb{R}} 
\def\E{\mathcal{E}} 
\def\N{\mathbb{N}} 
\def\Z{\mathbb{Z}} 
\def\Q{\mathbb{Q}} 
\def\F{\mathbb{F}}
\def\C{\mathbb{C}}
\def\U{\mathcal{U}}
\def\GL{\text{GL}}
\def\supp{\text{Supp}}
\def\id{\text{id}}
\def\n{\underline{n}}
\def\Gram{\text{Gram}}
\def\diag{\text{diag}}
\def\End{\text{End}}
\def\Hom{\text{Hom}}
\def\fa{\text{ für alle }}
\def\Tr{\text{Tr}}
\def\Id{\text{Id}}
\def\Sym{\text{Sym}}
\def\H{\mathcal{H}}
\def\wt{\text{wt}}
\def\Perf{\text{Perf}}

\newcommand{\properideal}{%
  \mathrel{\ooalign{$\lneq$\cr\raise.22ex\hbox{$\lhd$}\cr}}}

\def\P{\mathcal{P}}
\def\I{\mathbb{I}}
\def\R{\mathbb{R}} 
\def\E{\mathcal{E}} 
\def\NE{\mathbb{N}_{\geq1}} 
\def\N{\mathbb{N}} 
\def\Z{\mathbb{Z}} 
\def\Q{\mathbb{Q}} 
\def\F{\mathbb{F}}
\def\Vm{\mathcal{V}_m}
\def\V{\mathcal{V}}
\def\VV{\mathbb{V}}
\def\C{\mathbb{C}}
\def\U{\mathcal{U}}
\def\a{\mathfrak{a}}
\def\b{\mathfrak{b}}
\def\p{\mathfrak{p}}
\def\q{\mathfrak{q}}
\def\s{\sigma}
\def\si{\unlhd_{\sigma}}
\def\sD{\s\text{-}\operatorname{D}_m}
\def\GL{\text{GL}}
\def\supp{\text{Supp}}
\def\id{\text{id}}
\def\n{\underline{n}}
\def\Spec{\operatorname{Spec}}
\def\sSpec{\operatorname{Spec}^\sigma_m}
\def\diag{\text{diag}}
\def\End{\text{End}}
\def\Hom{\text{Hom}}
\def\Tr{\text{Tr}}
\def\Id{\text{Id}}
\def\Sym{\text{Sym}}
\def\H{\mathcal{H}}
\def\wt{\text{wt}}
\def\Perf{\text{Perf}}
\def\sdim{\sigma\operatorname{-dim}}
\def\trdeg{\operatorname{trdeg}}

%%\renewcommand{\labelenumi}{\alph{enumi})}
%\renewcommand{\P}{\textfrak{P}}
\newcommand{\cupdot}{\mathop{\mathaccent\cdot\cup}}
\newcommand{\textsim}{\mathord{\sim}}
\newcommand{\catname}[1]{{\normalfont\textbf{#1}}}
\newcommand{\Set}{\catname{Set}}
\newcommand{\Top}{\catname{Top}}
\newcommand{\sintk}{\s\text{\catname{-int}}_k}
\newcommand{\sringk}{\s\text{\catname{-ring}}_k}

%\renewcommand{\P}{\textfrak{P}}
\newenvironment{bew}{\begin{proof}[Proof]}{\end{proof}}
\theoremstyle{definition}
\newtheorem{satz}{Satz}[section]
\newtheorem{conj}{Vermutung}[section]
\newtheorem{theo}[satz]{Theorem}
\newtheorem{ex}[satz]{Beispiel}
\newtheorem{cor}[satz]{Korollar}
\newtheorem{algorithm}[satz]{Algorithmus}
\newtheorem{prop}[satz]{Proposition}
\newtheorem{propudef}[satz]{Proposition und Definition}
\newtheorem{rem}[satz]{Bemerkung}
\newtheorem{defn}[satz]{Definition}
\newtheorem{lem}[satz]{Lemma}

\title{Geometrische Aspekte von gemischten Differenzenidealen}
\author{Andr\'{e}s Goens}
\date{\today}
\begin{document}

\begin{frame}
\maketitle
\end{frame}

\begin{frame}\frametitle{Überblick}
\begin{itemize}
\item Überblick Differenzenalgebra
\begin{itemize}
\item Differenzenalgebra
\item Differenzenpolynomringe
\item Differenzenvarietäten
\item Differenzenalgebraische Ergebnisse
\end{itemize}
\item Motivation 
\item Mixed Differenzenvarietäten
\begin{itemize}
\item Mixed Differenzenvarietäten
\item Mixed Differenzenideale
\item Weitere Ergebnisse
\end{itemize}
\item Differenzenkerne
\end{itemize}
\end{frame}

\begin{frame}
\begin{center}
\LARGE Überblick Differenzenalgebra
\end{center}
\end{frame}

\begin{frame}\frametitle{Differenzenalgebra}
\begin{defn}
Sei  $R$ ein kommutativer Ring mit 1 (K\"{o}rper) , und sei
 $\sigma: R \rightarrow R$ ein Ringendomorphismus.
\begin{itemize}
\item Dann heißt das Tupel $(R,\sigma)$ ein \emph{Differenzenring(k\"{o}rper)}, oder $\sigma$\emph{-Ring(K\"{o}rper)}.
\item Ein Ideal $I \unlhd R$ der stabil unter $\s$ ist, d.h. $\s(I) \subseteq I$, heißt \emph{Differenzenideal} oder $\s$-\emph{Ideal}. Symbol: $I \si R$.
\end{itemize}
\end{defn}
\end{frame}

\begin{frame}\frametitle{Differenzenpolynomringe}
\begin{defn}
Sei $k$ ein $\s$-Körper. Der Polynomring in unendlich vielen Variablen $k[y_1,\s(y_1),\s^2(y_1),\ldots]$ heißt $\s$-\emph{Polynomring} in der \emph{Differenzenvariable} $y_1$. Symbol: $k\{y_1\}.$
Es wird ein $\s$-Ring durch $$\s( \s^n(y_1)) := \s^{n+1}(y_1).$$

 Analog für ein Tupel von Variablen $y=(y_1,\ldots,y_n)$, ist $$ k\{y\} := k\{y_1,\ldots,y_n\} := k[y_1,\ldots,y_n,\s(y_1),\ldots,\s(y_n),\s^2(y_1),\ldots] $$
\end{defn}
\end{frame}

\begin{frame}\frametitle{Differenzenpolynomringe (Fortsetzung)}
Konzepte von Ordnung, Effektive Ordnung:
\begin{itemize}
\item $ f:= y_1 \s^2(y_2)~+~\s(y_1)\s(y_2)\s(y_3) \in k\{y_1,y_2,y_3\}$
$$ \operatorname{Ord}(f) = 2, \operatorname{Eord}(f) = 2 $$
\item $g := \s^3(y_1) \s^2(y_2) + \s(y_1)\s(y_2)\s(y_3)$
$$ \operatorname{Ord}(g) = 3, \operatorname{Eord}(g) = 2 $$
\end{itemize}
\end{frame}



\begin{frame}\frametitle{Differenzenvarietäten}
\begin{defn}
Sei k ein $\s$-Körper, $F \subseteq k\{y_1,\ldots,y_n\}$ ein Menge von Differenzenpolynomen. 
Für einen $\s$-Körper $K$ mit $k \subseteq K$ und $\s_{|k} = \s$ setzen wir:
$$\VV_K(F) := \{ x \in K^n \mid f(x) = 0~\forall~ f \in F \}.$$
Dies ist ein Funktor der Kategorie der $\s$-Körper wie oben ($\s$-\emph{Körpererweiterungen}) in die Kategorie der Mengen und heißt \emph{$\s$-Varietät} oder \emph{Differenzenvarietät}.
\end{defn}
\end{frame}

\begin{frame}\frametitle{Differenzenvarietäten (Fortsetzung)}
\begin{defn}
Sei $k$ ein $\s$-Körper, $X = \VV(F)$ eine $\s$-Varietät. Dann ist 
\begin{align*} \I(X) := \{ f \in k\{y_1,\ldots,y_n \} \mid f(a) = 0~\forall~a \in \VV_K(F) \\ \forall K \supseteq k ~ \s \text{-Körpererweiterung} \}. 
\end{align*}
\end{defn}

\begin{defn}
Für die Menge $F$ gibt es ein kleinstes $\s$-Ideal $\{F\} \si k\{y_1,\ldots,y_n\}$ das \emph{perfekt} ist ($\sigma^{i_1}(f) \cdots \sigma^{i_n}(f) \in \{F\} \Rightarrow f \in \{F\}$), und $F$ enthält. 
Dieses heißt der \emph{perfekte Abschluss} von $F$.
\end{defn}
\end{frame}

\begin{frame}\frametitle{Differenzenvarietäten (Fortsetzung)}
\begin{satz}
Sei $k$ ein $\s$-Körper, $F \subseteq k\{y_1,\ldots,y_n\}$.
Dann ist $\I(\VV(F)) = \{F\}$
\end{satz}
\end{frame}

\begin{frame}\frametitle{Differenzenalgebraische Ergebnisse}
\begin{defn}
Sei $\p \si R$ ein $\s$-Ideal. Dann heißt $\p$ $\s$-\emph{prim}, falls $\p$ prim ist und $\s^{-1}(\p) \subseteq \p$ gilt.
\end{defn}


\begin{satz}
Sei $R$ ein $\s$-Ring und $F \subseteq R$ eine Teilmenge. Dann gilt:
\begin{equation*} \{F\} = \bigcap_{\substack{F \subseteq \p \si R, \\ \p ~ \s\text{-Prim}}} \p. \end{equation*}
\end{satz}
\end{frame}

\begin{frame}\frametitle{Differenzenalgebraische Ergebnisse (Fortsetzung)}
\begin{itemize}
\item Die Menge aller $\s$-primen $\s$-Ideale eines $\s$-Rings $R$ bezeichnet man mit $\Spec^\s(R)$. 
Sie ist ein topologischer Raum, der noethersch ist für $R = k\{y\}$.
\item $\Spec^\s(R)$ ist quasi-kompakt.
\item Die $\s$-primen $\s$-Ideale von $R$ stehen in Bijektion zu den irreduziblen, abgeschlossenen Mengen in $\Spec^\s(R)$.
\end{itemize}
\end{frame}

%% \begin{frame}\frametitle{Differenzenalgebraische Ergebnisse (Fortsetzung)}
%% Konzepte von Differenzengrad und Differenzendimension. Ergebnis damit: 
%% \begin{satz}\label{irredcomp}
%% Sei $k$ ein $\s$-Körper und $f \in k\{y_1,\ldots,y_n\}, f \notin k$ ein irreduzibler $\s$-Polynom, so dass $\operatorname{Eord}(f) = \operatorname{Ord}(f)$. Dann hat die $\s$-Varietät $\VV(f)$ eine irreduzible Komponente $X$, so dass $\s$-$\dim(X) = n-1$ und $\s$-$\operatorname{deg}(X) = \operatorname{Ord}(f)$.
%% \end{satz}
%% \end{frame}

\begin{frame}
\begin{center}
\LARGE Motivation
\end{center}
\end{frame}

\begin{frame}\frametitle{Motivation}
Sei $k$ ein Körper. Dann haben die Polynome $y, y^2$ die Gleichen Nullstellen: 
\[ y^2 = 0 \Leftrightarrow y = 0 \]
Aber: $y^2$ hat die Nullstelle $0$ ``doppelt''. \\
Wie kann man sie unterscheiden?\\
\vspace{10pt}
Suchen der Nullstellen: \\
Sucht man nicht in einen Körper, sondern zum Beispiel in der $k$-Algebra $k[y]/(y^2)$,
dann ist 
\[ y^2 = 0 \not \Leftrightarrow y = 0 \]
%Das Polynom $y^2$ hat die Nullstellen $\bar 0, \bar y \in k[y]/(y^2)$.
Dimension:
$$\dim_k(k[y]/(y)) = 1, \dim_k(k[y]/(y^2)) = 2$$
\end{frame}

%% \begin{frame}\frametitle{Motivation (Fortsetzung)}

%% Unterscheidung durch die $k$-Algebren (vgl. Koordinatenring):
%% $$k[y]/(y) \not \cong k[y]/(y^2)$$
%% Sogar:

%% \end{frame}

\begin{frame}\frametitle{Motivation (Fortsetzung)}
$\rightarrow$ Suche Analogie für Differenzenfall \\
$\phantom{ }$ \\
Differenzengleichungen $\s(y) = 0, \s^2(y) = 0$ haben die gleichen Lösungen über Differenzenkörper,
verschiedene ``Vielfachheit''.\\
Wie kann man sie unterscheiden?
\[ \{ \s(y) \} = \{ \s^2(y) \} = (y,\s(y),\s^2(y),\ldots) \si k\{y\} \]
Gleichungen liefern gleiche Differenzenvarietät. Neuer Konzept: suche nicht nur in $\s$-Körper
\end{frame}

\begin{frame}
\begin{center}
\LARGE Mixed Differenzenvarietäten
\end{center}
\end{frame}

\begin{frame}\frametitle{Mixed Differenzenvarietäten}

\begin{defn}\label{defnVV}
Sei $k$ ein $\s$-Körper und $B$ ein $\s$-Ring, der ein Integritätsbereich ist mit $k \subseteq B$, $\s_{|k} = \s$. Weiter sei $F \subseteq k\{y_1, \ldots, y_n\}$ eine Menge von $\s$-Polynomen.
Setze $$\VV_B(F):= \{ b \in B^n \mid f(b) = 0~\forall~f \in F \}.$$
Durch $B \mapsto \VV_B(F)$ ist ein Funktor der Kategorie solcher Ringe in die Kategorie der Mengen gegeben. Wir nennen ihn eine \emph{Mixed $\s$-Varietät über $k$}, oder \emph{$k$-$\s$-m-Varietät}.
\end{defn}
$\phantom{ }$ \\
Andere Möglichkeiten: Perfekt-reduzierte $\s$-Ringe, $\s$-Integritätsberreiche $\rightarrow$ Wieder Differenzenvarietäten.
\end{frame}

\begin{frame}\frametitle{Mixed Differenzenideale}
\begin{defn} 
\begin{itemize}
\item Ein Ideal $\a \si k\{y\}$ heißt mixed, wenn $fg \in \a \Rightarrow f\s(g) \in \a$.
\item Für $F \subseteq k\{y\}$ gibt es ein kleinstes radikales, mixed $\s$-Ideal, das $F$ enthält. Bez: $\{F\}_m.$
\end{itemize}
\end{defn}
\end{frame}

\begin{frame}\frametitle{Mixed Differenzenideale (Fortsetzung)}
\begin{center} Verbindung: \\  mixed Differenzenideale $\leftrightarrow$ mixed Differenzenvarietäten. \end{center}
Mit $\I_m$ analog zu $\I$ für mixed Differenzenvarietäten: 
\begin{satz}
Sei $k$ ein $\s$-Körper, $F \subseteq k\{y_1,\ldots,y_n\}$, und $X = \VV(F)$ eine $\s$-m-Varietät. Dann:
\[ \I_m(\VV(F)) = \{ F \}_m \]
\end{satz}
\end{frame}

\begin{frame}\frametitle{Weitere Ergebnisse}
\begin{satz}\label{intersectionprimes}
Sei $R$ ein $\s$-Ring und $F \subseteq R$. Dann:
\begin{align*} \{F\}_m = \bigcap_{\substack{F \subseteq \p \si R \\ \p \text{ Prim}}} \p. \end{align*}
\end{satz}
Offen: Endlich viele $\p$ für $R = k\{y\}$?
\end{frame}

\begin{frame}\frametitle{Weitere Ergebnisse (Fortsetzung)}
Topologischer Raum: $$\sSpec(R):= \{ \p \si R \mid \p \text{ prim }\} \supseteq \Spec^\s(R)$$

\begin{itemize}
\item $\sSpec(R)$ ist Quasi-Kompakt.
\item Die primen $\s$-Ideale von $R$ stehen in Bijektion zu den irreduziblen, abgeschlossenen Mengen in $\sSpec(R)$.
\item Offen: noethersch für $R = k\{y\}$? 
\end{itemize}
\end{frame}

%% \begin{frame}\frametitle{Weitere Ergebnisse (Fortsetzung)}
%% \begin{prop}
%% Let $k$ be a $\s$-field. Then $-^*$, as defined in Remark~\ref{dualmor}, is an anti-equivalence between the category of $\s$-m-varieties over $k$ and the subcategory of $\sringk$, which arises by restricting the object class to reduced, well-mixed, finitely $\s$-generated $\s$-overrings of $k$. 
%% In particular, a morphism $f: X \rightarrow Y$ of $\s$-m-varieties over $k$ is an isomorphism if and only if $f^*:~k\{Y\}~\rightarrow~k\{X\}$ is an isomorphism.
%% \end{prop}
%% \end{frame}

\begin{frame}
\begin{center}
\LARGE Differenzenkerne
\end{center}
\end{frame}

\begin{frame}\frametitle{Differenzenkerne}
\begin{itemize}
\item Problem: $\s$-Polynomring hat unendlich viele Variablen, nicht noethersch. 
\item Idee: Untersuche $\s$-Ideale in noethersche Unterringe.
\end{itemize}
\begin{defn}
Sei $k$ ein Differenzenkörper. Wir setzen $$k\{y\}[d]:= k[y,\s(y),\ldots,\s^d(y)] \subseteq k\{y\} \text{ und }k\{y\}[-1] := k.$$
 Für ein $\s$-Ideal $\a~\si~k\{y\}$ sei $$\a[d] := \a \cap k\{y\}[d].$$
\end{defn}
\end{frame}

\begin{frame}\frametitle{Differenzenkerne (Fortsetzung)}
\begin{defn}
Sei $\a \unlhd k\{y\}[d], d \geq 1$. Dann heißt $\a$ ein \emph{Differenzenkern der Länge $d$}, wenn $\s(\a[d-1]) \subseteq \a$. Er heißt ein primer Differenzenkern, falls zusätzlich $\a$ ein primes $\s$-Ideal in $k\{y\}[d]$ ist.
Weiterhin wird $\a$ reflexiv genannt, falls $\s^{-1}(\a) = \a[d-1]$. 
\end{defn}\index{difference kernels}
\begin{ex}
Sei $\p \si k\{y\}$ ein primes $\s$-Ideal, $d \geq 1$. Dann ist $\p[d] \unlhd k\{y\}[d]$ ein primer Differenzenkern. 
Ist $\p$ auch $\s$-prim, dann ist $\p[d]$ auch reflexiv.
\end{ex}
\end{frame}

\begin{frame}\frametitle{Differenzenkerne (Fortsetzung)}
\begin{satz}
Sei $\a \unlhd k\{y\}[d]$ ein reflexiver, primer Differenzenkern. Dann existiert ein $\s$-primes Ideal $\p \si k\{y\}$
mit $\p[d] = \a$.
\end{satz}
$\phantom{ }$ \\
Frage: Gilt das auch für prime Differenzenkerne? (mit primen Differenzenidealen)
\end{frame}

\begin{frame}\frametitle{Differenzenkerne (Fortsetzung)}
\begin{prop}
Sei $\a \subseteq k[y,\ldots,\s^d(y)]$ ein primer Differenzenkern und sei $k[y,\ldots,\s^d(y)]/\a =: k[a,\s(a),\ldots,\s^d(a)]$. Betrachte die Abbildung
\[ \s: k[a,\ldots,\s^{d-1}(a)] \rightarrow k[a,\ldots,\s^d(a)]. \]
Dann existiert ein primer Differenzenkern $\a'$ der Länge $d+1$ mit $\a'[d] = \a$ genau dann, wenn das von $\operatorname{ker}(\s)$ erzeugte Ideal, $(\operatorname{ker}(\s)) \subseteq k[a,\ldots,\s^d(a)]$, Folgendes erfüllt:
 \begin{equation*} (\operatorname{ker}(\s)) \cap k[a,\ldots,\s^{d-1}(a)] = \operatorname{ker}(\s).\end{equation*} 
\end{prop}
\end{frame}


\begin{frame}\frametitle{Differenzenkerne (Fortsetzung)}
Für reflexive, prime Differenzenkerne ist einmal Fortsetzen äquivalent zu beliebig oft Fortsetzen. Ohne Reflexivität ist das i.A. nicht der Fall. \\
$\rightarrow$ Gegenbeispiel. 
\begin{conj}
Sei $k$ ein $\s$-Körper und $\a \subseteq k\{y\}[d]$ ein primes Differenzenkern.
Weiter sei $a_1 = y_1 + \a ,\ldots,a_n = y_n + \a$.
Für $r \geq 1$ betrachte die Abbildung 
\[ \s^r : k[a,\s(a),\ldots,\s^{d-r}(a)] \rightarrow k[a,\ldots,\s^{d}(a)], f \mapsto \s^r(f). \]
Dann existiert ein primes $\s$-ideal $\p$ mit $\p[d] = \a$ genau dann, wenn
\begin{equation*} (\operatorname{ker}(\s),\ldots,\operatorname{ker}(\s^r)) \cap k[a,\ldots,\s^{d-r}(a)] = \operatorname{ker}(\s^r). \end{equation*}
\end{conj}
``$\Rightarrow$'' klar.
\end{frame}

\begin{frame}
\begin{theorem}\label{di=d(i+1)+e}
Sei $k$ ein $\s$-Körper und sei $\p \si k\{y\} = k\{y_1,\ldots,y_n\}$ ein primes $\s$-Ideal. Setze $$d_i := \dim(k\{y\}[i]/\p[i]).$$
Dann existieren ganze Zahlen $d, e \in \N$, so dass $d_i = d(i+1) + e$ für $i \gg 0$. %%Moreover, $d = \s\operatorname{-trdeg}(k\{y\}/\p^*)$.
\end{theorem}

\begin{defn}
\begin{itemize}
\item Die Zahl $d$ heißt die \emph{$\s$-Dimension} von $\p$ (Bez.: $\s$-$\dim(\p)$). 
\item Die Zahl $e$ heißt \emph{$\s$-Grad} von $\p$ (Bez.: $\s$-$\operatorname{deg}(\p)$).
\end{itemize}
\end{defn}

\end{frame}

\begin{frame}\frametitle{Differenzenkerne (Fortsetzung)}

\begin{satz}\label{irredcomp}
Sei $k$ ein $\s$-Körper und $f \in k\{y_1,\ldots,y_n\}, f \notin k$ ein irreduzibles $\s$-Polynom, so dass $\operatorname{Eord}(f) = \operatorname{Ord}(f)$. Dann hat die $\s$-Varietät $\VV(f)$ eine irreduzible Komponente $X$, so dass $\s$-$\dim(X) = n-1$ und $\s$-$\operatorname{deg}(X) = \operatorname{Ord}(f)$.
\end{satz}


\begin{satz}\label{corfinal}
Sei $k$ ein $\s$-Körper und $f \in k\{y_1,\ldots,y_n\}, f \notin k$ ein irreduzibles $\s$-Polynom.
Dann hat die $\s$-m-Varietät $\VV(f)$ eine irreduzible, abgeschlossene Teilmenge $X$ so, dass $\s$-$\dim(X) = n-1$ und $\s$-$\operatorname{deg}(X) = \operatorname{Ord}(f)$.
\end{satz}
\end{frame}


\begin{frame}\frametitle{Differenzenkerne (Fortsetzung)}
\begin{ex}
Betrachte den $\s$-Polynomring $\Q\{y_1\}$, und das irreduzible $\s$-Polynom $f := \s(y_1) - 1 \in \Q\{y_1\}$.  
Weiter seinen $X:= \VV(f)$ als $\s$-Varietät, und $Y:= \VV(f)$ als $\s$-m-Varietät. Beide haben nur eine Primkomponente:
$$[y_1 -1 ] = \{ \s(y_1) - 1 \},~ [\s(y_1) -1 ] = \{ \s(y_1) - 1 \}_m$$ 
\end{ex}
\end{frame}

\begin{frame}
\LARGE Vielen dank für ihre Aufmerksamkeit!
\end{frame}

\begin{frame}[allowframebreaks]\frametitle{Referenzen}
\begin{thebibliography}{9}
\bibitem{wibmer} Wibmer, Michael \emph{Algebraic Difference Equations (Lecture Notes)}, Available online: \url{http://www.algebra.rwth-aachen.de/de/Mitarbeiter/Wibmer/Algebraic\%20difference\%20equations.pdf}
\bibitem{lang} Lang, Serge, \emph{Algebra}, Revised Third Edition, Springer, 2005
\bibitem{eisenbud} Eisenbud, David \emph{Commutative Algebra with a View Toward Algebraic Geometry}, Springer, 1995
\bibitem{hartshorne} Hartshorne, Robin \emph{Algebraic Geometry}, Springer, 1977
\bibitem{cohn} Cohn,  Richard \emph{Difference Algebra}, Interscience Publishers, 1965
\bibitem{levin} Levin, Alexander \emph{Difference Algebra}, Springer, 2008
\bibitem{hrushovski} Hrushovski, Ehud \emph{The Elementary Theory of the Frobenius Automorphism}, arXiv:math/0406514 
\bibitem{bourbaki} Bourbaki, Nicolas \emph{Commutative Algebra}, Hermann, 1972
\bibitem{M2} Grayson, Daniel R. and Stillman, Michael E., Macaulay2, a software system for research in algebraic geometry, Available at \href{http://www.math.uiuc.edu/Macaulay2/}{http://www.math.uiuc.edu/Macaulay2/}
\bibitem{levinmixed} Levin, Alexander, \emph{On the ascending chain condition for mixed difference ideals}, 	arXiv:1207.4721
\bibitem{cox} Cox, Little and O'Shea, \emph{ Ideals, Varieties and Algorithms}, Second Edition, Springer, 1997
\end{thebibliography}
\end{frame}
\end{document}

% LocalWords:  endomorphism dihedral
